\documentclass[a4paper,12pt,french,twocolumn,landscape]{article}

\usepackage[TD]{../../Style}

\pagestyle{empty}

\geometry{margin=1cm}

\setlength{\columnsep}{2cm}

% Début du document
%%%%%%%%%%%%%%%%%%%
\begin{document}

\titre{Énigmes}

\begin{probleme}
Il faut 1 minute et 25 secondes pour couper une bûche en deux. Combien de temps faut-il pour couper une bûche en 13 morceaux?
\end{probleme}

\begin{probleme}
Sept personnes se rencontrent et se serrent la main. Sachant que chaque personne a donné la main une et une seule fois à chaque autre personne, combien de poignées de mains y a-t-il eu?
\end{probleme}

\begin{probleme}
Dans une ville, 10\% des habitants sont sur liste rouge. Si on prend l'annuaire téléphonique, et on y choisit, dans les pages de cette même ville, 117 habitants au hasard, combien seront sur liste rouge?
\end{probleme}

\begin{probleme}
Quatre tapissiers font quatre tapis en quatre jours. Combien faut-il de tapissiers pour faire vingt tapis en vingt jours?
\end{probleme}

\begin{probleme}
Un père promet à son fils de lui offrir 5\euro{} pour chaque bonne réponse, mais le fils devra lui donner 8\euro{} à chaque mauvaise réponse. Au bout de 26 questions, la père et le fils ne se doivent rien. Combien le fils a-t-il donné de bonnes réponses?
\end{probleme}

\begin{probleme}
Nous avons tous les deux autant d'argent. Combien dois-je te donner pour que tu aies exactement 40\euro{} de plus que moi?
\end{probleme}

\begin{probleme}
Picsou a 2 pièces de monnaie qui font en tout 30 centimes. Étant donné que l'une des pièces n'est pas une pièce de 10 centimes, quelle est la valeur de chacune des pièces?
\end{probleme}

\begin{probleme}
Manu le professeur met 12 heures pour corriger une pile de copies du bac blanc. Élisabeth ne mettrait que 6 heures pour la même pile. Combien de temps faudrait-il pour corriger cette pile si les deux éminents professeurs unissaient leur force?
\end{probleme}

\begin{probleme}
Quatre amis visitent un musée avec seulement 3 billets d'entrée. Ils rencontrent un gardien qui veut savoir qui n'a pas payé son entrée:
\begin{itemize}
\item Ce n'est pas moi, dit Paul.
\item C'est Jean, dit Jacques.
\item C'est Pierre, dit Jean.
\item Jacques a tort, dit Pierre.
\end{itemize}
Sachant qu'un seul d'entre eux ment, quel est le resquilleur?
\end{probleme}

\newpage

\titre{Énigmes}\setcounter{probleme}{0}

\begin{probleme}
Il faut 1 minute et 25 secondes pour couper une bûche en deux. Combien de temps faut-il pour couper une bûche en 13 morceaux?
\end{probleme}

\begin{probleme}
Sept personnes se rencontrent et se serrent la main. Sachant que chaque personne a donné la main une et une seule fois à chaque autre personne, combien de poignées de mains y a-t-il eu?
\end{probleme}

\begin{probleme}
Dans une ville, 10\% des habitants sont sur liste rouge. Si on prend l'annuaire téléphonique, et on y choisit, dans les pages de cette même ville, 117 habitants au hasard, combien seront sur liste rouge?
\end{probleme}

\begin{probleme}
Quatre tapissiers font quatre tapis en quatre jours. Combien faut-il de tapissiers pour faire vingt tapis en vingt jours?
\end{probleme}

\begin{probleme}
Un père promet à son fils de lui offrir 5\euro{} pour chaque bonne réponse, mais le fils devra lui donner 8\euro{} à chaque mauvaise réponse. Au bout de 26 questions, la père et le fils ne se doivent rien. Combien le fils a-t-il donné de bonnes réponses?
\end{probleme}

\begin{probleme}
Nous avons tous les deux autant d'argent. Combien dois-je te donner pour que tu aies exactement 40\euro{} de plus que moi?
\end{probleme}

\begin{probleme}
Picsou a 2 pièces de monnaie qui font en tout 30 centimes. Étant donné que l'une des pièces n'est pas une pièce de 10 centimes, quelle est la valeur de chacune des pièces?
\end{probleme}

\begin{probleme}
Manu le professeur met 12 heures pour corriger une pile de copies du bac blanc. Élisabeth ne mettrait que 6 heures pour la même pile. Combien de temps faudrait-il pour corriger cette pile si les deux éminents professeurs unissaient leur force?
\end{probleme}

\begin{probleme}
Quatre amis visitent un musée avec seulement 3 billets d'entrée. Ils rencontrent un gardien qui veut savoir qui n'a pas payé son entrée:
\begin{itemize}
\item Ce n'est pas moi, dit Paul.
\item C'est Jean, dit Jacques.
\item C'est Pierre, dit Jean.
\item Jacques a tort, dit Pierre.
\end{itemize}
Sachant qu'un seul d'entre eux ment, quel est le resquilleur?
\end{probleme}
\end{document}
