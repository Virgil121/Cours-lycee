\documentclass[a4paper,12pt,french] {article}

\usepackage[sujet]{../../Style}

\fancyhead[L]{Pour le 26/04/2022}
\fancyhead[C]{\textbf{Devoir en temps libre}}
\fancyhead[R]{\seconde 12}

\setlist[itemize]{align=parleft,left=20pt..35pt}

\renewcommand{\baselinestretch}{1.3}

% Passage en A3
%\geometry{a3paper,landscape,bottom=7mm,twocolumn}
%\setlength{\headwidth}{39cm} %42cm-2*margin pour fancyhdr

\begin{document}

\rem{Ce devoir peut être fait par groupes d'au plus trois personnes.}

\begin{exercice}
Dresser le tableau de signes de la fonction $h:x \mapsto \frac{x-5}{2x+9}$:
\end{exercice}

\begin{exercice} \

\compo
{
\begin{enumerate}
\item Soit $f$ une fonction définie sur $[-5;7]$, décroissante sur $[-5;-1]$ et croissante sur $[1;7]$. Comparer $f(-4)$ et $f(-1.5)$.

\item Dresser le tableau de variations de la fonction $g$ représentée ci-contre.

\item Quel est le minimum de $g$ sur l'intervalle $[-3;7]$? Sur l'intervalle $[3;6]$?
\end{enumerate}
}
{
\begin{center}
\begin{tikzpicture}
\begin{axis}[
styleglobal,
width=0.9*\linewidth,
xmin=-3.5, xmax=7.5,
ymin=-2.5, ymax=4.5,
xtick distance=1,
ytick distance=1,
minor x tick num=1,
minor y tick num=1,
]
\addplot[styleplot,tension=0.3] plot coordinates{(-3,2.5) (-1,1.5) (1,-2) (3,1) (5,0.5) (6,2) (7,3)} node[pos=0.9,below right] {$\mathscr C_g$} \pointsextremites;
\end{axis}
\end{tikzpicture}
\end{center}
}

\end{exercice}

\begin{exercice}
En France, l'accès aux boîtes de nuit est réservée aux majeurs.
\begin{itemize}
\item Dans \textit{Le Macumba}, il faut avoir strictement moins de 29 ans pour entrer.
\item Dans \textit{La Playa}, il faut avoir au plus 43 ans.
\item Dans \textit{Le Millenium}, il faut avoir plus de 33 ans.
\end{itemize}
\textit{On répondra sous forme de phrase \underline{et} d'intervalle.}

\begin{enumerate}
\item Dans quel intervalle d'âge doit se situer une personne qui veut pouvoir rentrer dans \textit{Le Millenium} et \textit{La Playa}?

\item Dans quel ensemble doit se situer l'âge d'une personne qui veut pouvoir entrer dans l'une des boites de nuits parmi \textit{Le Macumba} et \textit{Le Millenium}?
\end{enumerate}
\end{exercice}

\begin{exercice} \
\begin{enumerate}
\item Représenter dans un repère les droites $d_1$ et $d_2$ d'équations respectives $y=3x-2$ et $y=-\frac 5 6 x +0,5$.
\item Quelles sont les coordonnées exactes de leur point d'intersection?
\end{enumerate}
\end{exercice}

\begin{exercice} \

\compo[0.6]
{
Gérald souhaite revendre sa maison comportant deux étages. Son ami Jérôme, aviateur de renom, a survolé cette zone puis représenté la maison dans un repère orthonormé $\Oij$, en prenant un mètre pour unité. On obtient un rectangle $ABCD$.

En sachant que le prix du m$^2$ est de 2800 \euro{}, combien Gérald peut-il espérer obtenir à la revente?
}
{
\begin{centrer}
	\begin{resizetikz}{\linewidth}
    \begin{tikzpicture}
    \repereOij[xmin=-1,xmax=14,ymin=-4,ymax=8]
    \coordinate[label=left:$A$] (A) at (2,2) {};
    \coordinate[label=below:$B$] (B) at (4,-2) {};
    \coordinate[label=right:$C$] (C) at ($(B)+(8,4)$) {};
    \coordinate[label=above:$D$] (D) at ($(A)+(8,4)$) {};
    \draw[line width=1pt,fill=white,fill opacity=0.3] (A) -- (B) -- (C) -- (D) -- (A) -- cycle; %join=round?
	%\coordinate (toit1) at ($0.4*(A)+0.4*(B)+0.1*(D)+0.1*(C)$);
	%\coordinate (toit2) at ($0.4*(C)+0.4*(D)+0.1*(A)+0.1*(B)$);  
    %\draw[line width=1pt,opacity=0.5,dotted] (A) -- (toit1) -- (B) (C) -- (toit2) -- (D) (toit1) -- (toit2); 
    \end{tikzpicture}
    \end{resizetikz}
\end{centrer}
}
\end{exercice}

\begin{exercice}
En faisant ses courses, Claire est tombée sur deux offres différentes pour un paquet de biscuits. La première est une réduction de $20\%$, alors que la seconde propose, sans surcoût, un paquet contenant $20 \%$ de biscuits en plus. Quel choix devrait faire Claire?

\textit{On pourra fixer le coût d'un paquet et le nombre de biscuits par paquet, puis calculer le coût par biscuit pour chaque offre.}
\end{exercice}

\end{document}

\begin{exercice} \
%%%%% A la base le graphique était vertical, enlever les deux options pour avoir l'original ( il faudra refaire les labels )
\compo[0.38]
{
En utilisant le repère ci-contre:
\begin{enumerate}
\item Tracer deux représentants du vecteur $\vecc {AB}$, l'un d'origine $D$, l'autre d'extrémité $C$.
\item Placer $M$, l'image de $C$ par la translation de vecteur $\vecc {DB}$.
\item Placer $N$, l'image de $C$ par la translation de vecteur $\vecc {DC}$.
\end{enumerate}
}
{
\begin{center}
\vspace{-5mm}
\begin{tikzpicture}[scale=0.48]
\node[draw=none] (liminf) at (0,0) {};
\node[draw=none] (limsup) at (20,15) {};
\draw[step=1,gray,thin] ($(liminf)+(-0.01,-0.01)$) grid ($(limsup)+(0.01,0.01)$);
%\clip (liminf) rectangle (limsup);

%\path [draw=black, line width=2pt] (0,0) -- (-1,-2) -- (0,-4);
%\draw[draw=black, fill = black, fill opacity = 0.3, line width=2pt, even odd rule] 
%        (-4,-8) rectangle (4,-4) (-3,-6.5) rectangle (-1,-5.5) (1,-6.5) rectangle (3,-5.5);
\node[color=black,circle,minimum size=1pt,fill,inner sep=2pt,fill opacity=1,label={180:\textbf{\large A}}] (A) at (2,12) {};
\node[color=black,circle,minimum size=1pt,fill,inner sep=2pt,fill opacity=1,label={180:\textbf{\large B}}] (B) at (9,4) {};
\node[color=black,circle,minimum size=1pt,fill,inner sep=2pt,fill opacity=1,label={-2:\textbf{\large C}}] (C) at (7,8) {};
\node[color=black,circle,minimum size=1pt,fill,inner sep=2pt,fill opacity=1,label={2:\textbf{\large D}}] (D) at (3,3) {};
\draw[line width=1pt,->,>=latex] (16,3) -- ++(2,3) node[pos=0.5,above left] {$\vec u$};
%\node[color=black,circle,minimum size=1pt,fill,inner sep=2pt,fill opacity=1,label={90:\textbf{\large E}}] (E) at (4,-4) {};
%\draw[line width=1pt, shorten <= -10cm, shorten >= -10cm] (A) -- (B);
\end{tikzpicture}
\end{center}
}
\begin{enumerate}
\setcounter{enumi}{3}
\item Comparer les trois caractéristiques des vecteurs $\vec u$ et $\vecc {BD}$.

\points 3
\item Quelle est la nature du quadrilatère $BCNM$? Justifier en utilisant des vecteurs:

\points 3
\end{enumerate}

\end{exercice}

\end{document}