\documentclass[17pt,xcolor={svgnames,rgb,dvipsnames}]{beamer}

\usepackage{StyleBeamer}

%\usetikzlibrary{patterns.meta}

%\selectcolormodel{cmyk}

%\usepackage{wrapfig}
%\makeatletter
%\setlength{\parskip}{1ex}
%\newcommand{\@minipagerestore}{\setlength{\parskip}{1ex}}

\setbeamertemplate{navigation symbols}{}

% Début du document
%%%%%%%%%%%%%%%%%%%
\begin{document}

\title{Exercice 6}
\frame{\titlepage}

\begin{frame}{Questions 1 et 2}
On a: \pause
\begin{itemize}
\item $\Omega = \{1;2;\ldots;14;15\}$ \pause
\item $A=\{2;4;6;8;10;12;14\}$ \pause
\item $B=\{10;11;12;13;14;15\}$ \pause
\end{itemize}

\begin{enumerate}
\item $\Pro(A)=\frac{7}{15}\simeq 0,47$ \pause
\item $\Pro(B)=\frac 6 {15} = \frac 2 5 = 0,4$
\end{enumerate}
\end{frame}

\begin{frame}{Question 3}
\begin{itemize}
\item $A=\{2;4;6;8;10;12;14\}$
\item $B=\{10;11;12;13;14;15\}$
\end{itemize}

\

On a:
\begin{itemize}
\item $\overline A=\{1;3;5;7;9;11;13;15\}$\\ \pause donc $\Pro(\overline A)=\frac 8 {15} \simeq 0,53$ \pause
\item $\overline B=\{1;2;3;4;5;6;7;8;9\}$\\ \pause donc $\Pro(\overline B)=\frac 9 {15} = \frac 3 5 = 0,6$
\end{itemize}
\end{frame}

\begin{frame}{Question 3}
\begin{itemize}
\item $A=\{2;4;6;8;10;12;14\}$
\item $B=\{10;11;12;13;14;15\}$
\end{itemize}

\

On a:
\begin{itemize}
\item $A \cap B = \{ 10;12;14 \}$\\ \pause donc $\Pro(A \cap B)=\frac 3 {15}=\frac 1 5=0,2$ \pause
\item $A \cup B = \{ 2;4;6;8;10;11;12;13;14;15 \}$\\ \pause donc $\Pro(A \cup B)=\frac{10}{15}=\frac 2 3 \simeq 0,67$
\end{itemize}
\end{frame}

\end{document}
