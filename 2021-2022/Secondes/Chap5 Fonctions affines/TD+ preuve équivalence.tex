\documentclass[a4paper,twocolumn,landscape,12pt,french]{article}

\usepackage[TD]{../../../Style}
\geometry{margin=1cm}

\setlength\columnsep{2cm}

% Début du document
%%%%%%%%%%%%%%%%%%%
\begin{document}

\newcommand{\contenu}{
\setcounter{exercice}{0}
\titre{Une caractérisation des fonctions affines}

\begin{prop}
Soit $f:\R \rightarrow \R$ une fonction. Les deux conditions suivantes sont équivalentes:
\begin{itemize}
\item $f$ est une fonction affine.
\item Quelque soient $x_1$ et $x_2 \in \R$, le nombre $\tfrac{f(x_2)-f(x_1)}{x_2-x_1}$ a toujours la même valeur.
\end{itemize}
\end{prop}

\begin{exercice}
On s'intéresse à la preuve de cette proposition. L'équivalence signifie que l'on doit prouver les deux sens de cette affirmation:
\begin{itemize}
\item (Sens direct) Si $f$ est une fonction affine, alors quelque soient [\ldots]
\item (Sens indirect) Si quelque soient [\ldots] , alors $f$ est une fonction affine.
\end{itemize}
\begin{enumerate}
\item On suppose que $f$ est une fonction affine:
$$f:x \mapsto ax+b$$
On se donne alors $x_1$ et $x_2 \in \R$.
\begin{enumerate}
\item Calculer $\tfrac{f(x_2)-f(x_1)}{x_2-x_1}$.
\item Le résultat est-il cohérent? Quel sens de l'affirmation a-t-on prouvé?
\end{enumerate}
\item On ne suppose plus que $f$ est affine, mais que quelque soient $x_1$ et $x_2 \in \R$, le nombre $\tfrac{f(x_2)-f(x_1)}{x_2-x_1}$ est toujours le même. On pose alors:
$$m:=\frac{f(x_2)-f(x_1)}{x_2-x_1}$$
Montrons que $f$ est affine.
\begin{enumerate}
\item Soit $x \in \R$. Que vaut $\frac{f(x)-f(0)}{x}$?
\item En manipulant l'égalité ainsi trouvée, donner une expression de $f(x)$ en fonction de $m$, $x$ et $f(0)$.
\item Conclure.
\end{enumerate}
\item Refaire les calculs de la question 2 en calculant cette fois $\frac{f(x)-f(k)}{x-k}$, pour $k \in \R$.
\end{enumerate}
\end{exercice}
}

\contenu

\newpage

\contenu

\end{document}
