\documentclass[a4paper,12pt,french]{article}

\usepackage{../../Style}

\pagestyle{empty}

\renewcommand\tabularxcolumn[1]{m{#1}}

\begin{document}
\begin{center}\begin{Large}Démonstration: $\frac 1 3$ n'est pas décimal \end{Large} \end{center}
\begin{itemize}
\item Il s'agit d'un schéma de preuve "par l'absurde": On suppose que $\frac 1 3$ est décimal, et on trouve une contradiction, ce qui montre que l’hypothèse de départ est fausse.
\item Si tel est le cas, on peut alors écrire $\frac 1 3 = \frac a {10^n}$ avec $a \in \Z, n \in \N$. En multipliant les deux expressions par $3 \times 10^n$, on a alors $10^n = 3a$. Alors $10^n$ est un multiple de $3$.
\item Cependant, en additionnant les chiffres de $10^n$, on obtient 1, qui n’est pas un multiple de $3$, et donc $10^n$ n'est pas un multiple de $3$. Contradiction !
\end{itemize}

\vspace{0.7cm}

\begin{center}\begin{Large}Démonstration: $\frac 1 3$ n'est pas décimal \end{Large} \end{center}
\begin{itemize}
\item Il s'agit d'un schéma de preuve "par l'absurde": On suppose que $\frac 1 3$ est décimal, et on trouve une contradiction, ce qui montre que l’hypothèse de départ est fausse.
\item Si tel est le cas, on peut alors écrire $\frac 1 3 = \frac a {10^n}$ avec $a \in \Z, n \in \N$. En multipliant les deux expressions par $3 \times 10^n$, on a alors $10^n = 3a$. Alors $10^n$ est un multiple de $3$.
\item Cependant, en additionnant les chiffres de $10^n$, on obtient 1, qui n’est pas un multiple de $3$, et donc $10^n$ n'est pas un multiple de $3$. Contradiction !
\end{itemize}

\vspace{0.7cm}

\begin{center}\begin{Large}Démonstration: $\frac 1 3$ n'est pas décimal \end{Large} \end{center}
\begin{itemize}
\item Il s'agit d'un schéma de preuve "par l'absurde": On suppose que $\frac 1 3$ est décimal, et on trouve une contradiction, ce qui montre que l’hypothèse de départ est fausse.
\item Si tel est le cas, on peut alors écrire $\frac 1 3 = \frac a {10^n}$ avec $a \in \Z, n \in \N$. En multipliant les deux expressions par $3 \times 10^n$, on a alors $10^n = 3a$. Alors $10^n$ est un multiple de $3$.
\item Cependant, en additionnant les chiffres de $10^n$, on obtient 1, qui n’est pas un multiple de $3$, et donc $10^n$ n'est pas un multiple de $3$. Contradiction !
\end{itemize}

\vspace{0.7cm}

\begin{center}\begin{Large}Démonstration: $\frac 1 3$ n'est pas décimal \end{Large} \end{center}
\begin{itemize}
\item Il s'agit d'un schéma de preuve "par l'absurde": On suppose que $\frac 1 3$ est décimal, et on trouve une contradiction, ce qui montre que l’hypothèse de départ est fausse.
\item Si tel est le cas, on peut alors écrire $\frac 1 3 = \frac a {10^n}$ avec $a \in \Z, n \in \N$. En multipliant les deux expressions par $3 \times 10^n$, on a alors $10^n = 3a$. Alors $10^n$ est un multiple de $3$.
\item Cependant, en additionnant les chiffres de $10^n$, on obtient 1, qui n’est pas un multiple de $3$, et donc $10^n$ n'est pas un multiple de $3$. Contradiction !
\end{itemize}

\end{document}
