\documentclass[12pt] {article}

\usepackage{../../../Style}

\pagestyle{empty}

\renewcommand\tabularxcolumn[1]{m{#1}}

\begin{document}

\begin{center}
Seconde 12 - Interrogation du 15/09/2021 - Sujet A
\end{center}

Nom: \hspace{7cm} Prénom:

\begin{exercice}

Compléter le tableau suivant: 

\begin{center}
\begin{tabularx}{\textwidth}{ 
  | >{\centering\arraybackslash}c 
  | >{\centering\arraybackslash}c
  | >{\centering\arraybackslash}X | }
\hline
Intervalle & Inégalité associée & Représentation\\ \hline 
$\hspace{1.5cm} \left[ -1;2 \right] \hspace{1.5cm}$ &  & \vspace{5mm} \begin{tikzpicture}[>=latex]
    \node[text=white] at (2,0) {$\textbf{\Big[}$};
    \node[text=white] at (4,0) {$\Big]$};
    \node[below=10pt,text=white] at (2,0) {$a$};
    \node[below=8pt,text=white] at (4,0) {$b$};
    \draw[ultra thick,-,draw=white] (2,0) --(4,0);
    \draw[thick,->,draw=white,color=white] (0,0) --(7,0);
    \draw[thick,->,] (0,0) --(7,0);
\end{tikzpicture}  \\ \hline
 & $\hspace{1.5cm} 5 \leq x < 9 \hspace{1.5cm}$ & \vspace{5mm} \begin{tikzpicture}[>=latex]
    \node[text=white] at (2,0) {$\textbf{\Big[}$};
    \node[text=white] at (4,0) {$\Big]$};
    \node[below=10pt,text=white] at (2,0) {$a$};
    \node[below=8pt,text=white] at (4,0) {$b$};
    \draw[ultra thick,-,draw=white] (2,0) --(4,0);
\end{tikzpicture} \\ \hline
\end{tabularx}
\end{center}
\end{exercice}

\begin{exercice}
Représenter puis donner sous forme d'intervalle l'ensemble des nombres réels qui appartiennent à $[1;4[$ et $[2;5]$.
\end{exercice}

\setcounter{exercice}{0}
\vspace{45mm}

\begin{center}
Seconde 12 - Interrogation du 15/09/2021 - Sujet B
\end{center}

Nom: \hspace{7cm} Prénom:

\begin{exercice}

Compléter le tableau suivant: 

\begin{center}
\begin{tabularx}{\textwidth}{ 
  | >{\centering\arraybackslash}c 
  | >{\centering\arraybackslash}c
  | >{\centering\arraybackslash}X | }
\hline
Intervalle & Inégalité associée & Représentation\\ \hline 
$\hspace{1.5cm} \left[ -2;1 \right] \hspace{1.5cm}$ &  & \vspace{5mm} \begin{tikzpicture}[>=latex]
    \node[text=white] at (2,0) {$\textbf{\Big[}$};
    \node[text=white] at (4,0) {$\Big]$};
    \node[below=10pt,text=white] at (2,0) {$a$};
    \node[below=8pt,text=white] at (4,0) {$b$};
    \draw[ultra thick,-,draw=white] (2,0) --(4,0);
    \draw[thick,->,] (0,0) --(7,0);
\end{tikzpicture}  \\ \hline
 & $\hspace{1.5cm} 5 < x \leq 9 \hspace{1.5cm}$ & \vspace{5mm} \begin{tikzpicture}[>=latex]
    \node[text=white] at (2,0) {$\textbf{\Big[}$};
    \node[text=white] at (4,0) {$\Big]$};
    \node[below=10pt,text=white] at (2,0) {$a$};
    \node[below=8pt,text=white] at (4,0) {$b$};
    \draw[ultra thick,-,draw=white] (2,0) --(4,0);
\end{tikzpicture} \\ \hline
\end{tabularx}
\end{center}
\end{exercice}

\begin{exercice}
Représenter puis donner sous forme d'intervalle l'ensemble des nombres réels qui appartiennent à $[2;5[$ et $[1;4]$.
\end{exercice}

\end{document}
