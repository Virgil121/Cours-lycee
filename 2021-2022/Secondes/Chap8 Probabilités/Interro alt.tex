\documentclass[a4paper,12pt,french] {article}

\usepackage[sujet]{../../Style}

\fancyhead[L]{25/05/2022}
\fancyhead[C]{\textbf{Interrogation: Probabilités - Sujet B}}
\fancyhead[R]{\seconde 12}

\renewcommand{\points}[1]{\setstretch{1.5}%
\foreach \n in{1,...,#1}%
{\noindent\strut\dotfill

}
}

\begin{document}

\rem{L'usage de la calculatrice est autorisé. Toute l'interrogation peut être faite sur le sujet.\\
Excepté dans les questions 2.2 et 2.3, des phrases ne sont pas attendues.}

Nom: \hfill Prénom: \hfill \

\setstretch{1.2}

\begin{exercice}
On place dans une urne des boules numérotées de 1 à 12, indiscernables au toucher. On tire au hasard une boule dans l'urne.

On se donne les évènements A: \og La boule tirée porte un numéro supérieur ou égal à 7 \fg et B: \og Le numéro de la boule tirée est un multiple de 4 \fg

\begin{enumerate}

\item Calculer la probabilité de A, puis de B.

\points {12}

\item Écrire $A \cap B$ sous forme d'ensemble, et en déduire la probabilité de cet évènement.

\points 6

\end{enumerate}
\end{exercice}

\newpage

\begin{exercice}
Un laboratoire veut tester, sur des souris, l'efficacité d'un vaccin. Toutes les souris ont été contaminées par le virus d'une maladie. Certaines souris ont été vaccinées, d'autres ne l'ont pas été. Certaines souris ont développé la maladie, d'autres non. Les résultats ont été consignés dans le tableau suivant:

\begin{center}\renewcommand{\arraystretch}{1.8}
    \begin{tabularx}{\linewidth}{|Y|Y|Y|Y|}
    \cline{2-4} \multicolumn{1}{c|}{}
         &Souris malades & Souris saines & Total  \\ \hline
         Souris vaccinées & 11 & 59 & 70 \\ \hline 
         Souris non vaccinées & 39 & 16 & 55 \\ \hline 
         Total & 50 & 75 & 125 \\ \hline
    \end{tabularx}
\end{center}


On sélectionne au hasard une souris, et on considère les évènements V: \og La souris sélectionnée a été vaccinée \fg et M: \og La souris sélectionnée est malade \fg .

\begin{enumerate}
\item Calculer la probabilité de sélectionner une souris ayant développé la maladie.

\points 4

\item Calculer $\Pro(V)$ et interpréter ce résultat dans le contexte de l'énoncé.

\points 4

\item Décrire par une phrase l'évènement $\overline M$.

\points 4

\item On choisit maintenant une souris au hasard \underline{parmi les souris malades}. Calculer la probabilité de sélectionner une souris vaccinée.

\points 4

\end{enumerate}
\end{exercice}
\end{document}
