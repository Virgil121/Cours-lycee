\documentclass[a4paper,12pt,french]{article}

\usepackage[cours]{../../Style}

%\selectcolormodel{cmyk}

%\usepackage{wrapfig}
%\makeatletter
%\setlength{\parskip}{1ex}
%\newcommand{\@minipagerestore}{\setlength{\parskip}{1ex}}

% Début du document
%%%%%%%%%%%%%%%%%%%
\begin{document}

\title{Probabilités}
\maketitle

\begin{programme}
\item Ensemble des issues. Evenements. Réunion, intersection, complémentaire.
\item Loi (distribution) de probabilités. Probabilité d'un évènement: somme des probabilités des issues.
\item Relation $\Pro(A \cup B)+\Pro(A \cap B)=\Pro(A)+\Pro(B)$.
\item Dénombrement à l'aide de tableaux et d'arbres.
\item Capacités
\begin{itemize}
\item Utiliser des modèles théoriques de référence (dé, pièce, tirage au sort) en comprenant que les probabilités sont définies à priori.
\item Construire un modèle à partir de fréquences observées, en distinguant nettement modèle et réalité.
\item Calculer des probabilités dans des cas simples: expérience aléatoire à deux ou trois épreuves.
\end{itemize}
\end{programme}

\section{Univers et évènements}

\subsection{Définitions de base}

\begin{defin}
On appelle expérience aléatoire une expérience dont on ne peut pas prévoir le résultat. Les issues possibles d'une expérience aléatoire, aussi appelées éventualités, constituent un ensemble appelé \textbf{l'univers}.

Si l'on répète un très grand nombre de fois la même expérience indépendamment, alors la fréquence d'une issue va s'approcher de sa probabilité.
\end{defin}

\begin{ex}
Si l'on lance un dé, alors les résultats possibles (les issues) constituent l'univers $\Omega=\{1;2;3;4;5;6\}$.
\end{ex}

\begin{defin}
Un évènement A est un ensemble d'issues, autrement dit un sous-ensemble de $\Omega$. 
\end{defin}

\begin{ex}
Dans l'univers précédent, on peut considérer l'ensemble $P$: \og Le résultat est pair \fg. On a donc $P=\{2;4;6\}$.
\end{ex}

\begin{casparts}
\begin{itemize}
\item On appelle \textbf{évènement élémentaire} tout évènement ne contenant qu'un seul élément de $\Omega$ (On les appelle des singletons).
\item L'ensemble vide, noté $\varnothing$, est \textbf{l'évènement impossible}: Il ne se réalise jamais.
\item L'ensemble $\Omega$ est \textbf{l'évènement certain}: Il est toujours réalisé.
\end{itemize}
\end{casparts}

\begin{ex}
Dans l'univers précédent, $\{1\}$ et $\{3\}$ sont des éléments élémentaires.
\end{ex}

\subsection{Opérations sur les évènements}

\begin{defins}
On se donne $E$ un ensemble et $A$,$B$ deux sous-ensembles de $E$.
\begin{itemize}
\item \textbf{L'intersection} de $A$ et $B$ notée $A\cap B$ (A inter B) désigne l'ensemble des éléments de $E$ appartenant à la fois à $A$ et à $B$.
\item \textbf{L'union} de $A$ et $B$ notée $A\cup B$ (A union B) désigne l'ensemble des éléments de $E$ appartenant à $A$ ou à $B$ ou aux deux.
\item \textbf{Le complémentaire} de $A$, noté $\overline A$ (A barre) désigne l'ensemble des éléments de $E$ qui n'appartiennent pas à $A$.
\end{itemize}

\begin{center}
\begin{tabular}{ccc}
\begin{tikzpicture}

% Bordure
\draw[fill=white,fill opacity=0.8] (-3,-2) rectangle (3,2);

% Disques
\draw[fill=DarkOrange,fill opacity=0.5] (1,0) circle (1.5);
\draw[fill=DodgerBlue!50!DeepSkyBlue,fill opacity=0.5] (-1,0) circle (1.5);
\draw (1,0) circle (1.5); % On retrace la bordure du premier disque par dessus car le fond du second l'a estompée

% Légendes
\node[circle,draw=none,fill=white,fill opacity=0.3,text opacity=1,inner sep=2pt] (A) at (1.3,0) {\textbf A};
\node[circle,draw=none,fill=white,fill opacity=0.3,text opacity=1,inner sep=2pt] (B) at (-1.3,0) {\textbf B};

% Intersection
\begin{scope}
\clip (1,0) circle (1.5); % On se restreint au disque de droite
\draw[pattern=north east lines,draw=none,opacity=0.7] (-1,0) circle (1.5); % On hachure le disque de gauche
\end{scope} % Environnement scope pour limiter la restriction (si on veut tracer d'autres trucs après, inutile ici)

\end{tikzpicture}
&
\begin{tikzpicture}

% Bordure
\draw[fill=white,fill opacity=0.8] (-3,-2) rectangle (3,2);

% Disques
\draw[fill=DarkOrange,fill opacity=0.5] (1,0) circle (1.5);
\draw[fill=DodgerBlue!50!DeepSkyBlue,fill opacity=0.5] (-1,0) circle (1.5);
\draw (1,0) circle (1.5);

% Union
\draw[pattern=north east lines,draw=none,opacity=0.7] (-1,0) circle (1.5) (1,0) circle (1.5);

% Légendes
\node[circle,draw=none,fill=white,fill opacity=0.3,text opacity=1,inner sep=2pt] (A) at (1.3,0) {\textbf A};
\node[circle,draw=none,fill=white,fill opacity=0.3,text opacity=1,inner sep=2pt] (B) at (-1.3,0) {\textbf B};

\end{tikzpicture}
&
\begin{tikzpicture}

% Bordure
\draw[fill=white,fill opacity=0.8] (-2,-2) rectangle (2,2);

% Disque
\draw[fill=DarkOrange,fill opacity=0.5] (0,0) circle (1.5);

% Légende
\node[circle,draw=none,fill=white,fill opacity=0.3,text opacity=1,inner sep=2pt] (A) at (0,0) {\textbf A};

% Complémentaire
\draw[pattern=north east lines,draw=none,opacity=0.7,even odd rule] (-2,-2) rectangle (2,2) (0,0) circle (1.5);

\end{tikzpicture} \\
$A \cap B$ & $A \cup B$ & $\overline A$
\end{tabular}
\end{center}
\end{defins}

\begin{ex}
Dans une animalerie, on sélectionne au hasard un chien. L'univers $\Omega$ est l'ensemble de tout les chiens dans cette animalerie. On se donne de plus les évènements $A$: \og Le chien sélectionné a le poil blanc \fg et $B$: \og Le chien sélectionné a le poil beige \fg. On peut donc définir les évènements suivants:
\begin{itemize}
    \item $A\cap B$: \og Le chien sélectionné a le poil blanc et beige \fg;
    \item $A\cup B$: \og Le chien sélectionné a le poil blanc, beige ou les deux à la fois \fg.
    \item $\overline A$: \og Le chien sélectionné n'a pas le poil blanc \fg.
\end{itemize}
\end{ex}

\section{Probabilités sur un univers fini}

\subsection{Généralités}

\begin{defin}
Donner la \textbf{loi de probabilité} d'une expérience aléatoire signifie donner la probabilité de chaque issue.
\end{defin}

\begin{props}
Pour obtenir la probabilité d'un évènement, on fait la somme des probabilités de chaque issue le constituant.

Pour tout évènement A d'une expérience aléatoire d'univers $\Omega$, on a:
$$0 \leq \Pro(A) \leq 1 \hspace{2cm} \Pro(\varnothing)=0 \hspace{2cm} \Pro(\Omega)=1$$
\end{props}

\begin{ex}
On lance un dé truqué. La probabilité de tomber sur chaque face est décrite dans le tableau ci-dessous:

\begin{center}
\begin{tabularx}{0.7\linewidth}{|c|*6{Y|}} \hline
Numéro & 1 & 2 & 3 & 4 & 5 & 6 \\ \hline
Probabilité associée & $0,15$ & $0,1$ & $0,2$ & $0,25$ & $0,25$ & $0,05$ \\ \hline
\end{tabularx}
\end{center}

On considère l'évènement A: \og On tombe sur un nombre inférieur à 3 \fg. Alors $\Pro(A)=0,15+0,1+0,2=0,35$.
\end{ex}

\subsection{Cas équiprobable}

\begin{ex}
On lance un dé équilibré. Soit A l'évènement \og On tombe sur un nombre pair \fg. 3 des 6 faces d'un dé contiennent un nombre pair donc $\Pro(A) = \frac 3 6 = \frac 1 2 = 0,5$.
\end{ex}

\begin{defin}
On dit qu'on est en situation d'équiprobabilité lorsque toutes les issues d'une expérience ont la même probabilité.
\end{defin}

\begin{prop}
Dans ce cas, chaque issue a pour probabilité $\frac 1 {\textrm{nombre total d'issues}}$.

La probabilité d'un évènement A est alors: $$\Pro(A)=\frac{\textrm{nombre d'éléments de A}}{\textrm{nombre d'éléments de }\Omega} = \frac{\textrm{nombre de cas favorables}}{\textrm{nombre de cas possibles}}$$
\end{prop}

\subsection{Vocabulaire et formules}

\begin{ex}
Soient A:\og On tombe sur un nombre pair\fg et B: \og On tombe sur 3 \fg. On a alors $A \cap B = \varnothing$, donc $\Pro(A \cap B) = \Pro( \varnothing ) = 0$. Ces évènements ne peuvent pas se produire en même temps.
\end{ex}

\begin{defin}
Comme $\Pro(A \cap B)=0$, ces deux événements sont dits incompatibles.
\end{defin}

\begin{proprs}
\begin{itemize}
\item Soit $A$ un évènement. Alors $\Pro( \overline A ) = 1-\Pro(A)$.
\item Soient A et B deux évènements d'une expérience aléatoire. Alors on a:
$$\Pro(A \cup B) = \Pro(A)+\Pro(B)-\Pro(A \cap B)$$
\end{itemize}
\end{proprs}

\end{document}

% Test avec schémas, joli mais pas sûr de l'utilité

\begin{proprs}
\begin{itemize}
\item Soit $A$ un évènement. Alors $\Pro( \overline A ) = 1-\Pro(A)$.
\item Soient A et B deux évènements d'une expérience aléatoire. Alors on a:
$$\Pro(A \cup B) = \Pro\tikzmark{ProA}(A)+\Pro(\tikzmark{ProB}B)-\Pro(A\tikzmark{ProInter} \cap B)$$
\end{itemize}
\hspace{0.28\linewidth}
\begin{tikzpicture}[remember picture,scale=0.5]

\draw[overlay] (0,2) -- (pic cs:ProA);

% Bordure
\draw[fill=white,fill opacity=0.8] (-3,-2) rectangle (3,2);

% Disques
\draw[fill=DarkOrange,fill opacity=0.5] (1,0) circle (1.5);
\draw[fill=DodgerBlue!50!DeepSkyBlue,fill opacity=0.5] (-1,0) circle (1.5);
\draw (1,0) circle (1.5); % On retrace la bordure du premier disque par dessus car le fond du second l'a estompée
\draw[pattern=north east lines,draw=none,opacity=0.7] (1,0) circle (1.5);

% Légendes
\node[circle,draw=none,fill=white,fill opacity=0.3,text opacity=1,inner sep=2pt] (A) at (1.3,0) {\textbf A};
\node[circle,draw=none,fill=white,fill opacity=0.3,text opacity=1,inner sep=2pt] (B) at (-1.3,0) {\textbf B};

\end{tikzpicture}
\quad
\begin{tikzpicture}[remember picture, scale=0.5]

\draw[overlay] (0,2) -- (pic cs:ProB);

% Bordure
\draw[fill=white,fill opacity=0.8] (-3,-2) rectangle (3,2);

% Disques
\draw[fill=DarkOrange,fill opacity=0.5] (1,0) circle (1.5);
\draw[fill=DodgerBlue!50!DeepSkyBlue,fill opacity=0.5] (-1,0) circle (1.5);
\draw (1,0) circle (1.5); % On retrace la bordure du premier disque par dessus car le fond du second l'a estompée
\draw[pattern=north east lines,draw=none,opacity=0.7] (-1,0) circle (1.5);

% Légendes
\node[circle,draw=none,fill=white,fill opacity=0.3,text opacity=1,inner sep=2pt] (A) at (1.3,0) {\textbf A};
\node[circle,draw=none,fill=white,fill opacity=0.3,text opacity=1,inner sep=2pt] (B) at (-1.3,0) {\textbf B};

\end{tikzpicture}
\quad
\begin{tikzpicture}[remember picture,scale=0.5]

\draw[overlay] (0,2) -- (pic cs:ProInter);

% Bordure
\draw[fill=white,fill opacity=0.8] (-3,-2) rectangle (3,2);

% Disques
\draw[fill=DarkOrange,fill opacity=0.5] (1,0) circle (1.5);
\draw[fill=DodgerBlue!50!DeepSkyBlue,fill opacity=0.5] (-1,0) circle (1.5);
\draw (1,0) circle (1.5); % On retrace la bordure du premier disque par dessus car le fond du second l'a estompée

% Légendes
\node[circle,draw=none,fill=white,fill opacity=0.3,text opacity=1,inner sep=2pt] (A) at (1.3,0) {\textbf A};
\node[circle,draw=none,fill=white,fill opacity=0.3,text opacity=1,inner sep=2pt] (B) at (-1.3,0) {\textbf B};

% Intersection
\begin{scope}
\clip (1,0) circle (1.5); % On se restreint au disque de droite
\draw[pattern=north east lines,draw=none,opacity=0.7] (-1,0) circle (1.5); % On hachure le disque de gauche
\end{scope} % Environnement scope pour limiter la restriction (si on veut tracer d'autres trucs après, inutile ici)

\end{tikzpicture}
\end{proprs}