\documentclass[a4paper,12pt,french]{article}

\usepackage[cours,eleve,NB]{../../../Style}
%\geometry{bottom=1cm}
\usetikzlibrary{decorations.markings}

\newlength{\longueurpoints}
\newcommand{\pointscompleter}[1]{%
\settowidth{\longueurpoints}{#1}%
%\leavevmode
%  \linebreak[0]%
%  \mbox{}\nobreak
%  \cleaders \hb@xt@ .44em{\hss.\hss}\hskip\longueurpoints plus 1fill minus 0pt
%  \kern\z@
\makebox[\longueurpoints]{\strut\dotfill}%
}

\pagestyle{empty}

% Début du document
%%%%%%%%%%%%%%%%%%%
\begin{document}

\title{Variations de fonctions}
\maketitle
\thispagestyle{empty}

\section{Etude des variations}

\begin{fait}%[defin]
On se donne dans cette partie $f$ une fonction définie sur un intervalle $I$.
\end{fait}

\begin{defins}
\compolignehaut
{
On dit que $f$ est \pointscompleter{croissante} sur $I$ si lorsque la variable augmente dans $I$, les images augmentent aussi. L'ordre est conservé:

\begin{center}
Pour $x,y \in I$, si $x \leq y$ alors \pointscompleter{$f(x) \leq f(y)$.}
\end{center}

Graphiquement, la courbe de $f$ \pointscompleter{"monte": \nearrow}
\begin{center}
\begin{tikzpicture}
\begin{axis}[
styleglobal,
width=0.9*\linewidth,
xmin=-0.5, xmax=2,
ymin=-0.5, ymax=2,
xtick distance=0.5,
ytick distance=0.5,
ticks=none,
declare function={f(\x)=e^(\x-1);}
]
\addplot[styleplot]{f(x)} node [pos=0.7,below right] {$\mathscr C_f$};
%\node[stylepoint,fill=blue] (A) at (0.5,{f(0.5)}) {};
%\node[stylepoint,fill=blue] (B) at (1.2,{f(1.2)}) {};
%\draw[color=blue,dashed,very thick] (0.5, 0) -- (A) node [pos=0,below] {$x$} -- (0,{f(0.5)}) node [pos=1,left] {$f(x)$};
%\draw[color=blue,dashed,very thick] (1.2, 0) -- (B) node [pos=0,below] {$y$} -- (0,{f(1.2)}) node [pos=1,left] {$f(y)$};
\end{axis}
\end{tikzpicture}
\end{center}
}
{
On dit que $f$ est \pointscompleter{décroissante} sur $I$ si lorsque la variable augmente dans $I$, les images diminuent. L'ordre n'est pas conservé:

\begin{center}
Pour $x,y \in I$, si $x \leq y$ alors \pointscompleter{$f(x) \geq f(y)$.}
\end{center}

Graphiquement,la courbe de $f$ \pointscompleter{"descend": \searrow}
\begin{center}
\begin{tikzpicture}
\begin{axis}[
styleglobal,
width=0.9*\linewidth,
xmin=-0.5, xmax=2,
ymin=-0.5, ymax=2,
xtick distance=0.5,
ytick distance=0.5,
ticks=none,
declare function={f(\x)=e^(0.5-\x);}
]
\addplot[styleplot]{f(x)} node [pos=0.9,above right] {$\mathscr C_f$};
%\node[stylepoint,fill=blue] (A) at (0.5,{f(0.5)}) {};
%\node[stylepoint,fill=blue] (B) at (1.2,{f(1.2)}) {};
%\draw[color=blue,dashed,very thick] (0.5, 0) -- (A) node [pos=0,below] {$x$} -- (0,{f(0.5)}) node [pos=1,left] {$f(x)$};
%\draw[color=blue,dashed,very thick] (1.2, 0) -- (B) node [pos=0,below] {$y$} -- (0,{f(1.2)}) node [pos=1,left] {$f(y)$};
\end{axis}
\end{tikzpicture}
\end{center}
}
\end{defins}

\begin{ex}
\compo[0.67]{
On se donne la fonction $f$, définie sur $\left[1;7\right]$ et représentée ci-contre.

\begin{itemize}
\item $f$ est \pointscompleter{croissante} sur $[1;4]$ puis \pointscompleter{décroissante} sur \ldots\ldots\ldots
\item $f$ est \pointscompleter{croissante} sur $[1;4]$ et $2 \ldots 3$ donc $f(2) \ldots f(3)$.
\item $f$ est \pointscompleter{décroissante} sur $[4;7]$ et $5 \ldots 6$ donc $f(5) \ldots f(6)$.
\end{itemize}
}
{
%\vspace{-0.5em}
\begin{center}
\begin{tikzpicture}
\begin{axis}[
styleglobal,
width=1*\linewidth,
xmin=-0.5, xmax=7.5,
ymin=-0.5, ymax=3.5,
ytick distance=1,
xtick distance=1
%scale=0.7
]
\addplot[styleplot,tension=0.7] plot coordinates {(1,1) (4,3) (7,0.5)} node [pos=0.9,above right] {$\mathscr C_f$} \pointsextremites;
\end{axis}
\end{tikzpicture}
\end{center}
}
\end{ex}

\begin{defins}

\begin{itemize}
\item On dit que $f$ est \pointscompleter{constante} sur $I$ si elle prend toujours la même valeur:
\begin{center}
Pour $x,y \in I$, on a $f(x)=f(y)$.
\end{center}
\item On dit que $f$ est \pointscompleter{monotone} sur $I$ si elle est soit croissante, soit décroissante sur I (son sens de variation ne change pas).
\end{itemize}
\end{defins}

\newpage

\setcounter{section}{2}
\section{Extrémas d'une fonction sur un intervalle}

\begin{defins}
Soit $f$ une fonction définie sur un intervalle $I$.
\begin{itemize}
\item On dit que $f$ admet un maximum $M$ en $a$ sur $I$ si \pointscompleter{pour tout $x \in I, f(x) \leq M = f(a)$.}
\item On dit que $f$ admet un minimum $m$ en $b$ sur $I$ si \pointscompleter{pour tout $x \in I, f(x) \geq m = f(b)$.}
\end{itemize}
\end{defins}

\begin{rmq}
Le maximum d'une fonction correspond au point le plus \pointscompleter{"haut"} de sa courbe représentative, et le minimum au point le plus \pointscompleter{"bas".}
\end{rmq}

\begin{ex}
\compo[0.55]
{
On reprend la fonction $g$ de l'exemple précédent.

\

Son maximum sur $I$ est \ldots \ldots, atteint en \ldots \ldots

Son minimum sur $I$ est \ldots \ldots, atteint en \ldots \ldots

Son minimum sur $[-2;0]$ est \ldots \ldots, atteint en \ldots \ldots
}
{
\begin{center}
\vspace{-2em}
\begin{tikzpicture}
\begin{axis}[
styleglobal,
width=0.9*\linewidth,
xmin=-3, xmax=6,
ymin=-1.5, ymax=4.5,
ytick distance=1,
xtick distance=1
%scale=0.7
]
\addplot[styleplot,tension=0.45] plot coordinates {(-2,1.5) (-1,4) (1,0) (2,-1) (4,0) (5,2)} node [pos=0.9,above left] {$\mathscr C_g$} \pointsextremites;
%\node[stylepoint,fill=DarkGreen] at (-1,4) {};
%\node[stylepoint,fill=DarkGreen] at (2,-1) {};
%\addplot[styleplot,dashed,color=DarkGreen] plot {4} node[pos=0.8,above] {\textbf{maximum}};
%\addplot[styleplot,dashed,color=DarkGreen] plot {-1} node[pos=0.8,below] {\textbf{minimum}};
\end{axis}
\end{tikzpicture}
\end{center}
}
\end{ex}

\vfill

\setcounter{section}{2}
\section{Extrémas d'une fonction sur un intervalle}

\begin{defins}
Soit $f$ une fonction définie sur un intervalle $I$.
\begin{itemize}
\item On dit que $f$ admet un maximum $M$ en $a$ sur $I$ si \pointscompleter{pour tout $x \in I, f(x) \leq M = f(a)$.}
\item On dit que $f$ admet un minimum $m$ en $b$ sur $I$ si \pointscompleter{pour tout $x \in I, f(x) \geq m = f(b)$.}
\end{itemize}
\end{defins}

\begin{rmq}
Le maximum d'une fonction correspond au point le plus \pointscompleter{"haut"} de sa courbe représentative, et le minimum au point le plus \pointscompleter{"bas".}
\end{rmq}

\begin{ex}
\compo[0.55]
{
On reprend la fonction $g$ de l'exemple précédent.

\

Son maximum sur $I$ est \ldots \ldots, atteint en \ldots \ldots

Son minimum sur $I$ est \ldots \ldots, atteint en \ldots \ldots

Son minimum sur $[-2;0]$ est \ldots \ldots, atteint en \ldots \ldots
}
{
\begin{center}
\vspace{-2em}
\begin{tikzpicture}
\begin{axis}[
styleglobal,
width=0.9*\linewidth,
xmin=-3, xmax=6,
ymin=-1.5, ymax=4.5,
ytick distance=1,
xtick distance=1
%scale=0.7
]
\addplot[styleplot,tension=0.45] plot coordinates {(-2,1.5) (-1,4) (1,0) (2,-1) (4,0) (5,2)} node [pos=0.9,above left] {$\mathscr C_g$} \pointsextremites;
%\node[stylepoint,fill=DarkGreen] at (-1,4) {};
%\node[stylepoint,fill=DarkGreen] at (2,-1) {};
%\addplot[styleplot,dashed,color=DarkGreen] plot {4} node[pos=0.8,above] {\textbf{maximum}};
%\addplot[styleplot,dashed,color=DarkGreen] plot {-1} node[pos=0.8,below] {\textbf{minimum}};
\end{axis}
\end{tikzpicture}
\end{center}
}
\end{ex}

\end{document}
