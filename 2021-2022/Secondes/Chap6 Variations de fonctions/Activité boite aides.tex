\documentclass[,a4paper,12pt,french]{article}

\usepackage{../../../Style}

\setlength{\parindent}{0pt}

\geometry{margin=5mm}
% Début du document
%%%%%%%%%%%%%%%%%%%
\begin{document}

\newcommand{\contenua}{
\titre{Indication}
\centering Faites un schéma de la situation. Connaissez-vous la formule du volume d'une boite ?}

\newcommand{\contenub}{
\titre{Indication}
\centering Volume d'une boite: Longueur $\times$ largeur $\times$ hauteur}

\newcommand{\contenuc}{
\titre{Indication}
\centering Si la hauteur de la boite est de 2cm, quelle est sa hauteur? Quelle est sa largeur? En déduire son volume.}

\foreach \i in {1,2} {\contenua \vfill}
\foreach \i in {1,2} {\contenub \vfill}
\foreach \i in {1,2} {\contenuc \vfill}
\newpage
\ \vfill
\begin{tikzpicture}
\draw (0,0) -- (0,10cm) -- (10cm,10cm) -- (10cm,0cm) --cycle;
\draw (10cm,0) -- (20cm,0) -- (20cm,10cm) -- (10cm,10cm) --cycle;
\begin{scope}[yshift=-10cm]
\draw (0,0) -- (0,10cm) -- (10cm,10cm) -- (10cm,0cm) --cycle;
\draw (10cm,0) -- (20cm,0) -- (20cm,10cm) -- (10cm,10cm) --cycle;
\end{scope}
\end{tikzpicture}
\vfill
\end{document}
