\documentclass[a4paper,12pt,french]{article}

\usepackage[TD]{../../Style}

% Début du document
%%%%%%%%%%%%%%%%%%%
\begin{document}

\titre{Algorithmique - Introduction}

\begin{multicols*}{2}

\subsection{Variables}

\begin{defin}
Une variable est un objet possédant plusieurs caractéristiques, dont:
\begin{itemize}
\item Un nom
\item Un type: Entier, flottant (nombre décimal), chaine de caractères, booléen (vrai ou faux) ...
\item Une valeur
\end{itemize}
L'instruction $a \gets 0$ signifie que l'on crée une variable appelée $a$ et qu'on lui affecte la valeur 0.
\end{defin}

\begin{propr}
Lors de l'exécution d'un programme, les lignes sont traitées \textbf{dans l'ordre, une par une}.
\end{propr}

\begin{exercice} \
Après exécution de l'algorithme suivant, que vaut la variable $a$ ? \dotfill

\begin{algorithme*}
    $a \gets 0$ \;
    $b \gets 1$ \;
    $a \gets 2$ \;
\end{algorithme*}

\end{exercice}

\begin{exercice} \
Après exécution de l'algorithme suivant, que vaut la variable $a$ ? \dotfill

\begin{algorithme*}
    $a \gets 2$ \;
    $b \gets 3$ \;
    $a \gets b+1$ \;
\end{algorithme*}
\end{exercice}

\begin{exercice} \
Après exécution de l'algorithme suivant, que vaut la variable $a$ ? \dotfill

\begin{algorithme*}
    $a \gets 2$ \;
    $a \gets a+1$ \;
    $a \gets a*a$ \;
\end{algorithme*}

\end{exercice}

\begin{exercice} \
Après exécution de l'algorithme suivant, que vaut la variable $a$ ? \dotfill

\begin{algorithme*}
    $a \gets 2$ \;
    $b \gets 3$ \;
    $a \gets b+1$ \;
    $b \gets 5$ \;
\end{algorithme*}

\end{exercice}

\begin{exercice} \
Après exécution de l'algorithme suivant, que vaut la variable $a$ ? \dotfill

\begin{algorithme*}
    $a \gets 2$ \;
    $b \gets 3$ \;
    $a \gets b$ \;
    $b \gets a$ \;
\end{algorithme*}

\end{exercice}

\begin{exercice}[*] \
On se donne l'algorithme suivant:

\begin{algorithme*}
    $a \gets -1$ \;
    $b \gets 4$ \;
    $a \gets a+b$ \;
    $c \gets b+1$ \;
    $b \gets c+b-a$ \;
    $a \gets 2*a$ \;
    $c \gets b-a$ \;
\end{algorithme*}

Compléter le tableau suivant, donnant la valeur de chaque variable à la \textbf{fin} de chaque ligne:

\noindent \begin{tabularx}{\linewidth}{
	|*{8}{>{\centering\arraybackslash}X|}
	} \hline
ligne & 1 & 2 & 3 & 4 & 5 & 6 & 7 \\ \hline
$a$ & $-1$ & & & & & & \\ \hline
$b$ & \diagbox[innerwidth=\linewidth, height=\line]{}{} & & & & & & \\ \hline
$c$ & \diagbox[innerwidth=\linewidth, height=\line]{}{} & & & & & & \\ \hline
\end{tabularx}
\end{exercice}

\subsection{Fonctions}

A partir de ce point, on écrira systématiquement des fonctions.

\begin{exercice}
On se donne l'algorithme suivant:

\begin{algorithme*}
	\algofonction{$f(x)$}{
		$y \gets x*x+2$ \;		
		\Retour{$y$}\;
	}
\end{algorithme*}

\begin{enumerate}
\item Que retourne-t-il avec $x=4$? \ldots $x=11$? \ldots
\end{enumerate}
\end{exercice}

\begin{exercice}
Ecrire un algorithme permettant d'obtenir l'image d'un nombre par la fonction $f:t \mapsto \frac {t^2+2t}{t^2+1}$.
\end{exercice}

\begin{exercice}
On se donne l'algorithme suivant:

\begin{algorithme*}
	\algofonction{image$(x)$}{
		$y \gets x*x+2$ \;
		$y \gets y+x$ \;		
		\Retour{$y$}\;
	}
\end{algorithme*}

\begin{enumerate}
\item Que retourne la fonction image avec $x=2$? \ldots $x=-3$? \ldots
\item Que retourne-t-il dans le cas général? \points[1] 2
\end{enumerate}
\end{exercice}

\begin{exercice}
On se donne l'algorithme suivant:

\begin{algorithme*}
	\algofonction{$h(x)$}{
		$y \gets x*x+2$ \;
		$y \gets y \text{\^{}} 2-5$ \;
		\Retour{$y$}\;
	}
\end{algorithme*}

\begin{enumerate}
\item Que retourne-t-il avec $x=0$? \ldots $x=-4$? \ldots
\item Que retourne-t-il dans le cas général? \points[1] 2
\end{enumerate}
\end{exercice}

\begin{exercice}
On se donne l'algorithme suivant, où $L$, $l$ et $h$ sont trois réels positifs.

\begin{algorithme*}
	\algofonction{fonction\_{}mystère$(L,l,h)$}{
		$V \gets L * l * h$ \;
		\Retour{$V$} \;
	}
\end{algorithme*}

\noindent Pouvez-vous interpréter ce que retourne cet algorithme? \points[1] 2

\end{exercice}

\begin{exercice}
On se donne l'algorithme suivant, où $a$ et $b$ sont deux réels strictement positifs.

\begin{algorithme*}
	\algofonction{fonction\_{}mystère\_{}2$(a,b)$}{
		$c \gets \ $sqrt$(a$\^{}$2+b$\^{}$2)$ \;
		\Retour{c} \;
	}
\end{algorithme*}

\noindent Pouvez-vous interpréter ce que retourne cet algorithme? \points[1] 2

\end{exercice}

\begin{exercice}[*]
Ecrire un algorithme qui, à partir d'un nombre $r$, donne la somme de l'aire et du périmètre d'un cercle de rayon $r$.
\end{exercice}

\subsection{Structures conditionnelles}

\begin{exercice}
On se donne l'algorithme suivant:

\begin{algorithme*}
	\algofonction{géné\_{}appréciation(note)}{
		\Si{note $\leq 8$}{
			msg $\gets$ "A retravailler." \;
		}
		\Si{$8 < $ note $\leq 12$}{
			msg $\gets$ "Ensemble correct." \;
		}
		\Si{$12 < $ note}{
			msg $\gets$ "Bon travail!" \;
		}
		\Retour{msg} \;
	}
\end{algorithme*}

\noindent Que retourne-t-il pour note $=5$? note $=12$? \newline \noindent\dotfill

\end{exercice}

\begin{exercice}
On se donne l'algorithme suivant:

\begin{algorithme*}
	\algofonction{est\_{}reçu(note)}{
		\eSi{note $\geq 10$}{
			résultat $\gets$ Vrai \;
		}
		{
			résultat $\gets$ Faux \;
		}
		\Retour{résultat} \;
	}
\end{algorithme*}

\noindent Que retourne-t-il pour note $=5$? \ldots note $=10$? \ldots

\end{exercice}

\begin{exercice} \label{abs}
On se donne l'algorithme suivant:

\begin{algorithme*}
	\algofonction{$f(x)$}{
		\eSi{$x \geq 0$}{
			$y \gets x$ \;
		}
		{
			$y \gets -x$ \;
		}
		\Retour{$y$} \;
	}
\end{algorithme*}

\noindent Après avoir fait quelques tests, pouvez-vous trouver que retourne cet algorithme pour $x$ quelconque? \points[1] 2

\end{exercice}

\begin{exercice}
Modifier l'algorithme de l'exercice \ref{abs} pour qu'il retourne le signe de $x$.
\end{exercice}

\begin{exercice}
Modifier l'algorithme de l'exercice \ref{abs} pour qu'il retourne la distance entre deux nombres $a$ et $b$.
\end{exercice}

\begin{exercice}[*]
On se donne l'algorithme suivant:

\begin{algorithme*}
	\algofonction{mystère$(A,B,C)$}{
		\eSi{$A \leq B$}{
			\eSi{$B \leq C$}{
				\Retour{$C$} \;
			}
			{
				\Retour{$B$} \;
			}
		}
		{
			\eSi{$A \leq C$}{
				\Retour{$C$} \;
			}
			{
				\Retour{$A$} \;
			}
		}
	}
\end{algorithme*}

\begin{enumerate}
\item Tester cet algorithme pour les valeurs suivantes:
\begin{itemize}
\item $A=0$, $B=1$, $C=2$: \dotfill
\item $A=-5$, $B=7$, $C=-10$: \dotfill
\item $A=3$, $B=3$, $C=2$: \dotfill
\item $A=10$, $B=3$, $C=5$: \dotfill
\item $A=0$, $B=0$, $C=0$: \dotfill
\end{itemize}
\item Que retourne-t-il de manière générale? \points[1] 2
\end{enumerate}
\end{exercice}

\begin{exercice}[*]
Ecrire un algorithme qui, à partir de deux nombres, retourne le plus grand des deux s'ils sont positifs, et le plus petit des deux sinon.
\end{exercice}

\newpage

\begin{exercice}
Que retourne l'algorithme suivant pour var$=3$? var$=-7$? On réutilisera ici l'algorithme de l'exercice \ref{abs}.

\begin{algorithme*}
	\algofonction{Procédure\_{}spéciale(var)}{
		\eSi{var$=f($var$)$}{
			res $\gets$ var$^2-6$ \;
			\eSi{res $\leq 0$}{
				\Retour{res} \;
			}
			{
				\Retour{$2*$res-5} \;
			}
		}
		{
			\Retour{$3*$var} \;
		}
	}
\end{algorithme*}
\end{exercice}

\begin{exercice}
On se donne l'algorithme suivant:

\begin{algorithme*}
	\algofonction{fonction$(A,B)$}{
		\eSi{$A > B$}{
			\eSi{$B > 0$}{
				$C \gets A+B$ \;
			}
			{
				$C \gets A-B$ \;
			}
		}
		{
			\eSi{$A > 0$}{
				$C \gets A+B$ \;
			}
			{
				$C \gets B-A$ \;
			}
		}
	\Retour{$C$} \;
	}
\end{algorithme*}
\begin{enumerate}
\item Quelle est la valeur de $C$ pour $A=15$ et $B=25$? \ldots
\item Pour $A=45$ et $B=-56$ ? \ldots
\item \begin{enumerate}
\item Démontrer que $C$ est toujours positif.
\item Est-il strictement positif? 
\end{enumerate}
\end{enumerate}
\end{exercice}

\begin{exercice}
Ecrire un algorithme qui retourne le maximum parmi deux nombres, parmi quatre nombres, puis parmi huit nombres.
\end{exercice}

\begin{exercice} On se donne les algorithmes suivants:

\compo[0.5]
{
\begin{algorithme*}
	\algofonction{calcul1$(x)$}{
		$y \gets x^3-3*x^2+7*x-5$ \;
		\Retour{$y$} \;
	}
\end{algorithme*}
}
{
\begin{algorithme*}
	\algofonction{calcul2$(x)$}{
		$y \gets x$ \;
		$y \gets y*x-3$ \;
		$y \gets y*x+7$ \;
		$y \gets y*x-5$ \;
		\Retour{$y$} \;
	}
\end{algorithme*}
}

\begin{enumerate}
\item Compléter le tableau de valeurs suivant:

\noindent\begin{tabularx}{\linewidth}{|*4{Y|}} \hline
$x$ & $-1$ & $2$ & $3$ \\ \hline
calcul1$(x)$ & & & \\ \hline
calcul2$(x)$ & & &\\ \hline
\end{tabularx}

\item Emettre et prouver une conjecture concernant ces deux algorithmes.

\end{enumerate}
\end{exercice}

\end{multicols*}

\end{document}
