\documentclass[a4paper,11pt,french,landscape]{article}

\usepackage{../../../Style}

\renewcommand\tabularxcolumn[1]{m{#1}}

\geometry{left=10mm, right=10mm, top=10mm, bottom=17mm, marginparwidth=15mm}

\setlist[itemize]{align=parleft,left=5pt..15pt}
\setlist[enumerate]{align=left,left=-3pt..15pt,leftmargin=0pt,itemindent=15pt}
\setlist[enumerate,1]{align=left,left=5pt..20pt,itemindent=*}

\pagestyle{empty}

%Sources: Indice 1Tech, Hyperbole 2nde, Indice 2nde (2019), perso, fiche Thomas

% Début du document
%%%%%%%%%%%%%%%%%%%
\begin{document}

\newcommand{\contenu}{

\setcounter{exercice}{20}

\begin{exercice}
Que retourne l'algorithme suivant pour var$=3$? \ldots var$=-7$? \ldots On réutilisera ici l'algorithme de l'exercice 16.

\begin{algorithme*}
	\algofonction{Procédure\_{}spéciale(var)}{
		\eSi{var$=f($var$)$}{
			res $\gets$ var$^2-6$ \;
			\eSi{res $\leq 0$}{
				\Retour{res} \;
			}
			{
				\Retour{$2*$res-5} \;
			}
		}
		{
			\Retour{$3*$var} \;
		}
	}
\end{algorithme*}
\end{exercice}

\begin{exercice}
On se donne l'algorithme suivant:

\begin{algorithme*}
	\algofonction{fonction$(A,B)$}{
		\eSi{$A > B$}{
			\eSi{$B > 0$}{
				$C \gets A+B$ \;
			}
			{
				$C \gets A-B$ \;
			}
		}
		{
			\eSi{$A > 0$}{
				$C \gets A+B$ \;
			}
			{
				$C \gets B-A$ \;
			}
		}
	\Retour{$C$} \;
	}
\end{algorithme*}

\begin{enumerate}
\item Quelle est la valeur de $C$ pour $A=15$ et $B=25$? \ldots
\item Pour $A=45$ et $B=-56$ ? \ldots
\item \begin{enumerate}
\item Démontrer que $C$ est toujours positif.
\item Est-il strictement positif? 
\end{enumerate}
\end{enumerate}

\end{exercice}

\begin{exercice}
Ecrire un algorithme qui retourne le maximum parmi deux nombres, parmi quatre nombres, puis parmi huit nombres.
\end{exercice}

\begin{exercice} On se donne les algorithmes suivants:


\begin{algorithme*}
	\algofonction{$f(x)$}{
		$y \gets x^3-3*x^2+7*x-5$ \;
		\Retour{$y$} \;
	}
\end{algorithme*}

\begin{algorithme*}
	\algofonction{$g(x)$}{
		$y \gets x$ \;
		$y \gets y*x-3$ \;
		$y \gets y*x+7$ \;
		$y \gets y*x-5$ \;
		\Retour{$y$} \;
	}
\end{algorithme*}

\begin{enumerate}
\item Compléter le tableau de valeurs suivant:

\noindent\begin{tabularx}{\linewidth}{|*4{Y|}} \hline
$x$ & $-1$ & $2$ & $3$ \\ \hline
$f(x)$ & & & \\ \hline
$g(x)$ & & &\\ \hline
\end{tabularx}

\item Emettre et prouver une conjecture concernant ces deux algorithmes.

\end{enumerate}
\end{exercice}

\begin{exercice}
Ecrire un ou plusieurs algorithmes qui, à partir de trois quantités, calculent le taux d'évolution entre la première et la deuxième, la deuxième et la troisième puis la première et la troisième, et retournent le couple correspondant à la plus grande évolution.
\end{exercice}
\vfill \
}

\begin{multicols*}{4}

\contenu

\columnbreak

\contenu

\end{multicols*}

\end{document}
