\documentclass[landscape,a4paper,12pt,french, twocolumn]{article}

\usepackage{../../Style}

\geometry{left=1cm,right=1cm,top=1cm}

\renewcommand\tabularxcolumn[1]{m{#1}}

\renewcommand{\baselinestretch}{1.3}

\setlength{\columnsep}{18mm}

\pagestyle{empty}

% Début du document
%%%%%%%%%%%%%%%%%%%
\begin{document}


\newcommand{\\vv}{
\setcounter{section}{4}

\section{Parité d'une fonction}

\begin{defin}
Soit $f$ une fonction définie sur un intervalle $I$ centré en $0$ ( $I=[-a;a]$, $]-a;a[$ ou $\R$). On dit que f est:
\begin{itemize}
\item \textbf{paire} lorsque pour tout $x \in I, f(-x)=f(x)$.
\item \textbf{impaire} lorsque pour tout $x \in I, f(-x)=-f(x)$.
\end{itemize}
\end{defin}

\begin{exs} \
\begin{itemize}
\item La fonction $\fonction f {[-2;2]} {\R} x {x^2-1}$ est \makebox[3cm]{\dotfill} car pour tout $x \in [-2;2]$, \dotfill
\item La fonction $\fonction g {]3;3[} {\R} x {0.5x}$ est \makebox[3cm]{\dotfill} car pour tout $x \in [-2;2]$, \dotfill
\end{itemize}
\end{exs}

\begin{proprs} \
\begin{itemize}
\item $f$ est paire si et seulement si $\mathscr C_f$ est symétrique par rapport à

\dotfill
\item $f$ est impaire si et seulement si $\mathscr C_f$ est symétrique par rapport à

\dotfill
\end{itemize}
\end{proprs}
\begin{centrer}
\begin{tikzpicture}
\begin{axis}[
styleglobal,
width=0.9*\linewidth,
xmin=-6, xmax=6,
ymin=-1.5, ymax=3,
]
\addplot[samples=101,smooth,ultra thick,domain=(-4:4),mark=none]{0.25*x^2-1} node [pos=0.9,right] {$\mathscr C_f$};
%\addlegendentry{$f(x)=x^2-1$};
\addplot[samples=101,smooth,ultra thick,domain=(-6:6),mark=none,color=blue]{0.25*x} node [pos=0.85,below right] {$\mathscr C_g$};
%\addlegendentry{$g(x)=0.5x$};

\end{axis}
\end{tikzpicture}
\end{centrer}

\begin{rmq}
Une fonction peut être ni paire ni impaire!
\end{rmq}}

\\vv

\newpage

\\vv

\end{document}
