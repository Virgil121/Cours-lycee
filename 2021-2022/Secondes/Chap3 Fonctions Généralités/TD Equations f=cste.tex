\documentclass[landscape,a4paper,12pt,french, twocolumn]{article}

\usepackage{../../Style}

\renewcommand\tabularxcolumn[1]{m{#1}}

\pagestyle{empty}

\setlength{\columnsep}{2cm}

% Début du document
%%%%%%%%%%%%%%%%%%%
\begin{document}

\titre{Equations du type $f(x)=k$}

\begin{exercice}
On se donne la fonction $f$ représentée ci-contre. Résoudre graphiquement les équations suivantes:

\compo[0.4]
{
\begin{enumerate}
\item $f(x)=3$:
\vspace{5mm}
\item $f(x)=-1$:
\vspace{5mm}
\item $f(x)=1,5$:
\end{enumerate}
}
{
\Centering{
\begin{tikzpicture}
\begin{axis}[
styleglobal,
width=0.9*\linewidth,
xmin=-6, xmax= 6,
ymin=-2, ymax=5,
xtick distance=1,
ytick distance=1,
minor x tick num=0,
minor y tick num=0,
%tick label style = {font=\scriptsize},
]
\addplot[styleplot,domain=(-5:5)] plot {-1.4+5.5*e^(-0.1*(x-2)^2)} node[pos=0.8,above right] {$\mathscr C_f$} \pointsextremites;
\end{axis}
\end{tikzpicture}}
}
\end{exercice}

\begin{exercice}
De même avec la fonction $g$ représentée ci-dessous:

\compo[0.4]
{
\begin{enumerate}
\item $g(x)=1$:
\vspace{5mm}
\item $g(x)=0$:
\vspace{5mm}
\item $g(x)=5$:
\end{enumerate}
}
{
\Centering{
\begin{tikzpicture}
\begin{axis}[
styleglobal,
width=0.9*\linewidth,
xmin=-7, xmax= 7,
ymin=-2, ymax=6,
xtick distance=1,
ytick distance=1,
minor x tick num=0,
minor y tick num=0,
%tick label style = {font=\scriptsize},
]
\addplot[styleplot] plot coordinates {(-6,1) (-4,3) (-1,-1) (1,3) (3,4) (4,2) (5,-1) (6,0)} node[pos=0.8,above right] {$\mathscr C_g$} \pointsextremites;
\end{axis}
\end{tikzpicture}}
}
\end{exercice}

\begin{exercice}
On définit maintenant les fonctions $f$ ( trait plein ) et $g$ ( pointillés ) représentées sur le graphe ci-dessous. Résoudre graphiquement les équations suivantes:

\compo[0.4]
{
\begin{enumerate}
\item $f(x)=0,5$:
\vspace{5mm}
\item $g(x)=-0.5$:
\vspace{5mm}
\item $g(x)=2$:
\end{enumerate}
}
{
\Centering{
\begin{tikzpicture}
\begin{axis}[
styleglobal,
width=0.9*\linewidth,
xmin=-6, xmax= 6,
ymin=-2, ymax=1.5,
xtick distance=1,
ytick distance=0.5,
hauteurproptick,
minor x tick num=0,
minor y tick num=0,
%tick label style = {font=\scriptsize},
]
\addplot[styleplot,domain=(-4:3)] plot coordinates {(-4,1) (-2,0) (3,0.7)} node[pos=0.75,above right] {$\mathscr C_f$} \pointsextremites;
\addplot[styleplot,densely dashed,color=DarkBlue] plot coordinates {(-4,-1.5) (-1,1) (3,-1)} node[pos=0.95,above right] {$\mathscr C_g$} \pointsextremites;
\end{axis}
\end{tikzpicture}}
}
\end{exercice}

\setcounter{exercice}{0}
\newpage

\titre{Equations du type $f(x)=k$}

\begin{exercice}
On se donne la fonction $f$ représentée ci-contre. Résoudre graphiquement les équations suivantes:

\compo[0.4]
{
\begin{enumerate}
\item $f(x)=3$:
\vspace{5mm}
\item $f(x)=-1$:
\vspace{5mm}
\item $f(x)=1,5$:
\end{enumerate}
}
{
\Centering{
\begin{tikzpicture}
\begin{axis}[
styleglobal,
width=0.9*\linewidth,
xmin=-6, xmax= 6,
ymin=-2, ymax=5,
xtick distance=1,
ytick distance=1,
minor x tick num=0,
minor y tick num=0,
%tick label style = {font=\scriptsize},
]
\addplot[styleplot,domain=(-5:5)] plot {-1.4+5.5*e^(-0.1*(x-2)^2)} node[pos=0.8,above right] {$\mathscr C_f$} \pointsextremites;
\end{axis}
\end{tikzpicture}}
}
\end{exercice}

\begin{exercice}
De même avec la fonction $g$ représentée ci-dessous:

\compo[0.4]
{
\begin{enumerate}
\item $g(x)=1$:
\vspace{5mm}
\item $g(x)=0$:
\vspace{5mm}
\item $g(x)=5$:
\end{enumerate}
}
{
\Centering{
\begin{tikzpicture}
\begin{axis}[
styleglobal,
width=0.9*\linewidth,
xmin=-7, xmax= 7,
ymin=-2, ymax=6,
xtick distance=1,
ytick distance=1,
minor x tick num=0,
minor y tick num=0,
%tick label style = {font=\scriptsize},
]
\addplot[styleplot] plot coordinates {(-6,1) (-4,3) (-1,-1) (1,3) (3,4) (4,2) (5,-1) (6,0)} node[pos=0.8,above right] {$\mathscr C_g$} \pointsextremites;
\end{axis}
\end{tikzpicture}}
}
\end{exercice}

\begin{exercice}
On définit maintenant les fonctions $f$ ( trait plein ) et $g$ ( pointillés ) représentées sur le graphe ci-dessous. Résoudre graphiquement les équations suivantes:

\compo[0.4]
{
\begin{enumerate}
\item $f(x)=0,5$:
\vspace{5mm}
\item $g(x)=-0.5$:
\vspace{5mm}
\item $g(x)=2$:
\end{enumerate}
}
{
\Centering{
\begin{tikzpicture}
\begin{axis}[
styleglobal,
width=0.9*\linewidth,
xmin=-6, xmax= 6,
ymin=-2, ymax=1.5,
xtick distance=1,
ytick distance=0.5,
minor x tick num=0,
minor y tick num=0,
hauteurproptick,
%tick label style = {font=\scriptsize},
]
\addplot[styleplot,domain=(-4:3)] plot coordinates {(-4,1) (-2,0) (3,0.7)} node[pos=0.75,above right] {$\mathscr C_f$} \pointsextremites;
\addplot[styleplot,densely dashed,color=DarkBlue] plot coordinates {(-4,-1.5) (-1,1) (3,-1)} node[pos=0.95,above right] {$\mathscr C_g$} \pointsextremites;
\end{axis}
\end{tikzpicture}}
}
\end{exercice}

\end{document}