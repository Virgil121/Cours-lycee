\documentclass[a4paper,12pt,french] {article}

\usepackage[sujet]{../../../Style}

%\pagestyle{fancy}
\setlength{\headheight}{10mm}
\fancyhead[L]{24/01/2022}
\fancyhead[C]{\textbf{DS4 : Proportions, variations, pourcentages}}
\fancyhead[R]{\seconde 12}
\fancyfoot{}

\renewcommand{\baselinestretch}{1.3}

\begin{document}

\rem{L'usage de la calculatrice est autorisé. L'usage de schémas est encouragé. La propreté et l'orthographe seront prises en compte. Des phrases de réponse sont \underline{\underline{\textbf{exigées}}}.}

\begin{exercice}
Calculer:
\vspace{2pt}

\begin{centrer}
\begin{minipage}{0.9\textwidth}
$A=\dfrac 7 3 - \dfrac 2 9$ \hfill $B=\frac {4^{-2} \times 4^{10}} {4}$ \hfill $C=(y-5)(y+5)$
\end{minipage}
\end{centrer}
\end{exercice}

\begin{exercice} \
\vspace{2mm}

\compo
{
Voici la composition d'une célèbre pâte à tartiner au bon goût de noisette.
\begin{enumerate}
\item Déterminer la proportion de lipides dans cette pâte à tartiner.
\item Déterminer la proportion d'acides gras saturés parmi les lipides.
\item Déterminer de deux manières différentes la proportion d'acides gras saturés dans cette pâte à tartiner.
\end{enumerate}
}
{
\includegraphics[width=\linewidth]{Photos DS/DS nutella NB.jpg}
}
\end{exercice}

\begin{exercice} \

\compo[0.87]
{
Le 13 novembre 2020, un bitcoin valait 13500\euro{}. Le 12 novembre 2021, son prix atteignait 56000\euro{}. Quelle sont la variation absolue et le taux d'évolution de la valeur du bitcoin entre ces deux dates?
}
{
\includegraphics[width=\linewidth]{Photos DS/bitcoin.png}
}
\end{exercice}

\begin{exercice}
Golan le restaurateur propose un menu unique à 28\euro{}. Après une distinction par un célèbre guide culinaire, il décide d'augmenter ses prix de 40$\%$. Toutefois, il décide de faire payer le prix normal à son ami Léo et lui propose donc une remise de 40$\%$ sur le nouveau prix. Quelle sera la réaction de Léo?
\end{exercice}

\begin{exercice}
Un sac à dos est en soldes. Son étiquette indique que son prix a subi une première réduction de $40\%$, puis une deuxième de $10\%$. Son prix final est affiché à $29.99$\euro{}.
\begin{enumerate}
\item Calculer les coefficients multiplicateurs associés à ces réductions.
\item \textbf{En utilisant un schéma}, calculer le taux d'évolution global du prix de ce sac. 
\item En déduire le prix du sac avant les réductions. On arrondira au centime près.
\end{enumerate}
\end{exercice}

\begin{exercice}[A faire en dernier]
\

\compo[0.4]
{
Une candidate à la présidentielle 2022 a proposé de doubler le salaire des enseignants. Cette augmentation sera échelonnée sur cinq années. Voici le compte rendu qu'en a fait TF1. Que dire de ce graphique? Expliquer leur démarche.
}
{
\includegraphics[trim=35mm 45mm 15mm 3mm, clip, width=\linewidth]{Photos DS/"Erreur JT NB".png}
}
\end{exercice}

\begin{comment}

\begin{exercice}
Vrai ou faux? Justifier.
\begin{enumerate}
\item Le nombre de spectateurs d'une représentation a augmenté de 35 $\%$ en un an. Ce nombre a donc été multiplié par $1.35$.
\item Un prix est passé de 90\euro{} à 100\euro{}. Alors il a augmenté de 10$\%$.
\end{enumerate}
\end{exercice}

\begin{exercice}[2 pts]
En 2016, en France, 1 523 442 candidats se sont présentés aux épreuves théoriques du permis de conduire et 1 053 460 ont été reçus. En 2017, il y avait 1 544 546 candidats dont 1 021 557 reçus. Killian affirme que la proportion de reçus est meilleure en 2016 qu'en 2017. A-t-il raison?
\end{exercice}

\begin{exercice}
Dans la chorale d'adultes \textit{Arpège}, on compte $64 \%$ de femmes, et $35 \%$ d'entre elles sont des sopranos. Enfin, un quart des hommes sont des ténors. Calculer la proportion en pourcentage des:

\begin{itemize}
\item Femmes sopranos
\item Hommes ténors
\item Hommes autres que ténors
\end{itemize}
\end{exercice}

\begin{exercice}[4 pts]
En France métropolitaine, en 2018, la forêt couvre environ $34 \%$ du territoire. Les trois quarts de ces forêts appartiennent à des propriétaires privés.
\begin{enumerate}
\item Quel pourcentage de la superficie de la France métropolitaine occupent les forêts privées?
\item On estime que la superficie de la France métropolitaine est $551695$ km$^2$. Quelle est, en km$^2$, la superficie des forêts privées? \textit{Arrondir à l'unité.}
\end{enumerate}
\end{exercice}

\begin{exercice}[3 pts]
Le tableau suivant donne la répartition des salariés d'une start-up.

\compo[0.46]
{
\
\begin{enumerate}
\item Quelle est la proportion des femmes dans cette startup?
\item Quelle est celle des commerciaux dans la startup?
\item Quelle est celle des commerciales parmi les commerciaux?
\end{enumerate}
}
{
\begin{center}
\begin{tabularx}{\linewidth}{
|>{\centering\arraybackslash}c
|>{\centering\arraybackslash}X
|>{\centering\arraybackslash}X
|>{\centering\arraybackslash}X|} \cline{2-4} \multicolumn{1}{c|}{}
& Hommes & Femmes & Total \\ \hline
Commerciaux & 15 & 9 & 24 \\ \hline
Développeurs web & 7 & 9 & 16 \\ \hline
Total & 22 & 18 & 40 \\ \hline
\end{tabularx}
\end{center}
}

\end{exercice}

\begin{exercice}[4 pts]
Le tableau ci-dessous donne les indices des bénéfices d'une société de 2014 à 2018.

\compo[0.7]
{
\
\begin{enumerate}
\item Quelle est l'année de référence de ce tableau?
\item \begin{enumerate}
\item En quelle année la société a-t-elle réalisé le meilleur bénéfice?
\item Quel était alors le taux d'évolution de ce bénéfice depuis 2014?
\item Depuis 2016?
\end{enumerate}
%\item Une autre société, qui prend également pour base 100 son bénéfice de 2014, a atteint en 2018 un bénéfice correspondant à l'indice 105. Peut-on en déduire que cette société a réalisé en 2018 un bénéfice plus important que la première société?
\end{enumerate}
}
{
\begin{center}
\begin{tabularx}{\linewidth}{
|>{\centering\arraybackslash}X
|>{\centering\arraybackslash}X|} \hline
Année & Indice \\ \hline
2014 & 100 \\ \hline
2015 & 101.5 \\ \hline
2016 & 98.3 \\ \hline
2017 & 103 \\ \hline
2018 & 102.9 \\ \hline
\end{tabularx}
\end{center}
}
\end{exercice}
\end{comment}
\end{document}
