\documentclass[a4paper,12pt,french]{article}

\usepackage[cours]{../../Style}

\begin{document}

\titre{Vecteurs du plan: Première approche}

\begin{FlushLeft}

\section{Translations}

\begin{defin}
Soient $A,B,M$ trois points quelconques du plan. L'image de $M$ par la translation de vecteur $\vv {AB}$ est l'unique point $N$ tel que le quadrilatère $ABNM$ soit un parallélogramme.

\begin{center}
\begin{tikzpicture}[scale=0.7]
\node[stylepoint,fill=blue,label=above:A] (A) at (0,0) {};
\node[stylepoint,fill=blue,label=above:B] (B) at (5,-2) {};
\node[stylepoint,fill=blue,label=above:M] (M) at (-2,-3) {};
\node[draw=none] (N) at (3,-5) {};
\draw[thick, -] (M) +(-17:5.385) arc (-17:-27:5.385);
\draw[thick, -] (B) +(-115:3.606) arc (-115:-132:3.606);
%\draw[dashed] (M) circle (5.3851);
%\draw (B) circle (3.6055);

\end{tikzpicture}
\end{center}
\end{defin}

\rem{Demander quelles méthodes pour placer N: deux arcs de cercle, un arc et droite parallèle, deux droites parallèles, milieu diagonales+compas}

\begin{rmqs}
Lors de cette translation:
\begin{itemize}
\item Les droites $(AB)$ et $(MN)$ sont parallèles
\item Les longueurs $AB$ et $MN$ sont égales
%\item Le glissement de $M$ vers $N$ se fait dans la même sens que de $A$ vers $B$.
\end{itemize}
\end{rmqs}

\rem{Exos 1,2,(3)}

\section{Généralités sur les vecteurs}

\begin{defin}
Un vecteur $\vv u$ peut être représenté avec une origine $A$ et une extrémité $B$. Il est caractérisé par:
\begin{itemize}
\item Sa direction ( celle de la droite $(AB)$ )
\item Son sens ( de $A$ vers $B$ )
\item Sa norme ( ou sa longueur ), notée $\Vert \vv u \Vert$. On a $\Vert \vv u \Vert = AB$.
\end{itemize}
\end{defin}

\begin{defin}
Lorsque deux vecteurs symbolisent le même déplacement, on dit qu'ils sont égaux.
\end{defin}

\begin{ex}
La figure précédente nous donne alors $\vv{AB} = \vv {MN}$, $\vv{BA} = \vv{NM}$, $\vv {AM} = \vv {BN}$.
\end{ex}

\begin{propr}
Les quatre propositions suivantes sont \textbf{équivalentes}:
\begin{itemize}
\item $\vv {AB} = \vv {MN}$
\item $N$ est l'image de $M$ par la translation de vecteur $\vv{AB}$
\item $[AN]$ et $[BM]$ se coupent en leur milieu
\item $ABNM$ est un parallélogramme
\end{itemize}
\end{propr}

\begin{casparts}
\begin{itemize}
\item Le vecteur nul, noté $\vv 0$, est obtenu à partir de la translation qui transforme $A$ et $A$, $B$ en $B$ ... On a alors $\vv 0 = \vv {AA} = \vv {BB} = \ldots$ Ce vecteur n'a ni direction, ni sens, et sa norme est nulle.
\item L'opposé du vecteur $\vv {AB}$ est le vecteur associé à la translation qui transforme $B$ en $A$. On le note $- \vv {AB}$. Ainsi, $- \vv {AB} = \vv {BA}$. Il a la même direction et la même norme que $\vv {AB}$ mais est de sens contraire.
\end{itemize}
\end{casparts}

\rem{Exos 4,5,6,7}

\section{Somme de deux vecteurs}

\rem[eleve]{Activité Chasles}

\begin{fait}[defin]
\begin{center}
\begin{tikzpicture}
\draw[stylegrille] (0,1) grid (14,8);
\clip (0,0) rectangle (13,9);
\node[stylepoint,fill=blue,label=below left:A] (A) at (2,2) {};
\node[stylepoint,fill=blue,label=above:B] (B) at (3,7) {};
\draw[stylevecteur] (A.center) -- (B.center) node[pos=0.5,left] {\large $\vv u$};
\draw[stylevecteur] (9,6) -- (13,2) node[pos=0.5,above right] {\large $\vv v$};
\node[stylepoint,fill=blue,label=below right:C] (C) at (7,3) {};
\draw[stylevecteur] (B.center) -- (C.center) node[pos=0.5,above right] {\large $\vv v$};
\draw[stylevecteur,color=red] (A.center) -- (C.center) node[pos=0.5,below] {\large $\vv u + \vv v$};
\end{tikzpicture}
\end{center}
\end{fait}

\begin{ex}
$\vv {AB} + \vv {BA} = \vv {AA} = \vv 0$
\end{ex}

\begin{rmq}
En général, la longueur de $\vv u + \vv v$ n'est pas égale à la somme de celle de $\vv u$ et de $\vv v$: $\Vert \vv u + \vv v \Vert \neq \Vert \vv u \Vert + \Vert \vv v \Vert$
\end{rmq}

\begin{methode}
Pour construire $\vv {AB} + \vv {AC}$, il suffit de construire le parallélogramme $ABDC$ et de prendre le vecteur $\vv {AD}$.

\begin{center}
\begin{tikzpicture}[scale=0.8]
\node[stylepoint,fill=blue,label=above:A] (A) at (0,0) {};
\node[stylepoint,fill=blue,label=above:B] (B) at (5,-2) {};
\node[stylepoint,fill=blue,label=above left:C] (C) at (-2,-3) {};
\node[stylepoint,fill=blue,label=right:D] (D) at (3,-5) {};
\draw[thick, -] (C) +(-17:5.385) arc (-17:-27:5.385);
\draw[thick, -] (B) +(-115:3.606) arc (-115:-132:3.606);
\draw[stylevecteur] (A.center) -- (B.center) node[pos=0.5,above] {\large $\vv u$};
\draw[stylevecteur] (A.center) -- (C.center) node[pos=0.5,above left] {\large $\vv v$};
\draw[stylevecteur,color=red] (A.center) -- (D.center) node[pos=0.5,above right] {\large $\vv u + \vv v$};
\draw[dashed,thick,-] (B) -- (D);
\draw[dashed,thick,-] (C) -- (D);

\end{tikzpicture}
\end{center}
\end{methode}

\rem{Exos 9 -> 12 (13,14)}

\section{Produit d'un vecteur par un nombre réel}

\begin{defin}
Soit $k$ un réel et $\vv u$ un vecteur. Le produit de $k$ par $\vv u$ est le vecteur noté $k \vv u$ tel que:
\begin{itemize}
\item $\vv u$ et $k \vv u$ ont la même direction
\item La longueur de $k \vv u$ est $\abs k \Vert \vv u \Vert$
\item Si $k > 0$, $\vv u$ et $k \vv u$ ont le même sens. Sinon, ils ont des sens opposés.
\end{itemize}
\end{defin}

\begin{rmq}
En particulier, $0 \times \vv u = \vv 0$.
\end{rmq}

\begin{ex}
\begin{center}
\begin{tikzpicture}[scale=0.8]
\draw[stylegrille] (-3,-9) grid (13,1);
\clip (-3,-9) rectangle (14,1);

\draw[stylevecteur] (0,0) -- (4,0) node[pos=0,left] {\large $\vv u$};
\draw[stylevecteur] (0,-2) -- (8,-2) node[pos=0,left] {\large $2 \vv u$};
\draw[stylevecteur,<-] (0,-4) -- (4,-4) node[pos=0,left] {\large $-\vv u$};
\draw[stylevecteur] (0,-6) -- (6,-6) node[pos=0,left] {\large $\frac 3 2 \vv u$};
\draw[stylevecteur,<-] (0,-8) -- (12,-8) node[pos=0,left] {\large $-3\vv u$};

\end{tikzpicture}
\end{center}

\end{ex}

\begin{app}
$I$ est le milieu de $[AB]$ se traduit par $\vv {AI} = \vv {IB}$ ou $\vv{AB} = 2\vv{AI}$ ou $\vv {AI} = \frac 1 2 \vv{AB}$.

\begin{center}
\begin{tikzpicture}
\draw[stylevecteur,line width=2pt] (0,0) -- (8,1) node[pos=0.75,above] {\large $\vv u$} node[pos=0,stylepoint,fill=blue,label=left:A] {} node[pos=1,stylepoint,fill=blue,label=right:B] {};
\draw[stylevecteur,color=red] (0,0) -- (4,0.5) node[pos=0.5,below] {\large { $0.5\vv u$} } node[pos=1,stylepoint,fill=red,label=above:I] {};
\end{tikzpicture}
\end{center}
\end{app}

\rem{Feuille en plus: maths en ligne ex3d ou original}

\end{FlushLeft}

\end{document}
