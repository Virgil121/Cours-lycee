\documentclass[a4paper,12pt,french]{article}

\usepackage[cours]{../../../Style}

\begin{document}

\titre{Vecteurs du plan: Première approche}

\begin{FlushLeft}

\section{Translations}

\begin{defin}
Soient $A,B,M$ trois points quelconques du plan. L'image de $M$ par la translation de vecteur $\vecc {AB}$ est l'unique point $N$ tel que le quadrilatère $ABNM$ soit un parallélogramme.

\begin{center}
\begin{tikzpicture}[scale=0.7]
\node[stylepoint,fill=blue,label=above:A] (A) at (0,0) {};
\node[stylepoint,fill=blue,label=above:B] (B) at (5,-2) {};
\node[stylepoint,fill=blue,label=above:M] (M) at (-2,-3) {};
\node[draw=none] (N) at (3,-5) {};
\draw[thick, -] (M) +(-17:5.385) arc (-17:-27:5.385);
\draw[thick, -] (B) +(-115:3.606) arc (-115:-132:3.606);
%\draw[dashed] (M) circle (5.3851);
%\draw (B) circle (3.6055);

\end{tikzpicture}
\end{center}
\end{defin}

\rem{Demander quelles méthodes pour placer N: deux arcs de cercle, un arc et droite parallèle, deux droites parallèles, milieu diagonales+compas}

\begin{rmqs}
Lors de cette translation:
\begin{itemize}
\item Les droites $(AB)$ et $(MN)$ sont parallèles
\item Les longueurs $AB$ et $MN$ sont égales
%\item Le glissement de $M$ vers $N$ se fait dans la même sens que de $A$ vers $B$.
\end{itemize}
\end{rmqs}

\rem{Exos 1,2,(3)}

\section{Généralités sur les vecteurs}

\begin{defin}
Un vecteur $\vec u$ peut être représenté avec une origine $A$ et une extrémité $B$. Il est caractérisé par:
\begin{itemize}
\item Sa direction ( celle de la droite $(AB)$ )
\item Son sens ( de $A$ vers $B$ )
\item Sa norme ( ou sa longueur ), notée $\Vert \vec u \Vert$. On a $\Vert \vec u \Vert = AB$.
\end{itemize}
\end{defin}

\begin{defin}
Lorsque deux vecteurs symbolisent le même déplacement, on dit qu'ils sont égaux.
\end{defin}

\begin{ex}
La figure précédente nous donne alors $\vecc{AB} = \vecc {MN}, \vecc{BA} = \vecc{NM}, \vecc {AM} = \vecc {BN}$.
\end{ex}

\begin{propr}
Les quatre propositions suivantes sont \textbf{équivalentes}:
\begin{itemize}
\item $\vecc {AB} = \vecc {MN}$
\item $N$ est l'image de $M$ par la translation de vecteur $\vecc{AB}$
\item $[AN]$ et $[BM]$ se coupent en leur milieu
\item $ABNM$ est un parallélogramme
\end{itemize}
\end{propr}

\begin{casparts}
\begin{itemize}
\item Le vecteur nul, noté $\vecc 0$, est obtenu à partir de la translation qui transforme $A$ et $A$, $B$ en $B$ ... On a alors $\vecc 0 = \vecc {AA} = \vecc {BB} = \ldots$ Ce vecteur n'a ni direction, ni sens, et sa norme est nulle.
\item L'opposé du vecteur $\vecc {AB}$ est le vecteur associé à la translation qui transforme $B$ en $A$. On le note $- \vecc {AB}$. Ainsi, $- \vecc {AB} = \vecc {BA}$. Il a la même direction et la même norme que $\vecc {AB}$ mais est de sens contraire.
\end{itemize}
\end{casparts}

\rem{Exos 4,5,6,7}

\section{Somme de deux vecteurs}

\rem[eleve]{Activité Chasles}

\begin{fait}[defin]
\begin{center}
\begin{tikzpicture}
\draw[stylegrille] (0,0) grid (13,9);
\clip (0,0) rectangle (13,9);
\node[stylepoint,fill=blue,label=below left:A] (A) at (2,2) {};
\node[stylepoint,fill=blue,label=above:B] (B) at (3,7) {};
\draw[stylevecteur] (A.center) -- (B.center) node[pos=0.5,left] {\large $\vec u$};
\draw[stylevecteur] (9,6) -- (13,2) node[pos=0.5,above right] {\large $\vec v$};
\node[stylepoint,fill=blue,label=below right:C] (C) at (7,3) {};
\draw[stylevecteur] (B.center) -- (C.center) node[pos=0.5,above right] {\large $\vec v$};
\draw[stylevecteur,color=red] (A.center) -- (C.center) node[pos=0.5,below] {\large $\vec u + \vec v$};
\end{tikzpicture}
\end{center}
\end{fait}

\begin{ex}
$\vecc {AB} + \vecc {BA} = \vecc {AA} = \vecc 0$
\end{ex}

\begin{rmq}
En général, la longueur de $\vec u + \vec v$ n'est pas égale à la somme de celle de $\vec u$ et de $\vec v$: $\Vert \vec u + \vec v \Vert \neq \Vert \vec u \Vert + \Vert \vec v \Vert$
\end{rmq}

\begin{methode}
Pour construire $\vecc {AB} + \vecc {AC}$, il suffit de construire le parallélogramme $ABDC$ et de prendre le vecteur $\vecc {AD}$.

\begin{center}
\begin{tikzpicture}[scale=0.8]
\node[stylepoint,fill=blue,label=above:A] (A) at (0,0) {};
\node[stylepoint,fill=blue,label=above:B] (B) at (5,-2) {};
\node[stylepoint,fill=blue,label=above left:C] (C) at (-2,-3) {};
\node[stylepoint,fill=blue,label=right:D] (D) at (3,-5) {};
\draw[thick, -] (C) +(-17:5.385) arc (-17:-27:5.385);
\draw[thick, -] (B) +(-115:3.606) arc (-115:-132:3.606);
\draw[stylevecteur] (A.center) -- (B.center) node[pos=0.5,above] {\large $\vec u$};
\draw[stylevecteur] (A.center) -- (C.center) node[pos=0.5,above left] {\large $\vec v$};
\draw[stylevecteur,color=red] (A.center) -- (D.center) node[pos=0.5,above right] {\large $\vec u + \vec v$};
\draw[dashed,thick,-] (B) -- (D);
\draw[dashed,thick,-] (C) -- (D);

\end{tikzpicture}
\end{center}
\end{methode}

\rem{Exos 9 -> 12 (13,14)}

\section{Produit d'un vecteur par un nombre réel}

\begin{defin}
Soit $k$ un réel et $\vec u$ un vecteur. Le produit de $k$ par $\vec u$ est le vecteur noté $k \vec u$ tel que:
\begin{itemize}
\item $\vec u$ et $k \vec u$ ont la même direction
\item La longueur de $k \vec u$ est $\abs k \Vert \vec u \Vert$
\item Si $k > 0$, $\vec u$ et $k \vec u$ ont le même sens. Sinon, ils ont des sens opposés.
\end{itemize}
\end{defin}

\begin{rmq}
En particulier, $0 \times \vec u = \vecc 0$.
\end{rmq}

\begin{ex} \saut
\begin{center}
\begin{tikzpicture}[scale=1]
%\begin{axis}[
%%axis x line=bottom,
%%axis y line = left,
%%axis lines=middle,
%width=\linewidth,
%height=0.625\linewidth,
%xmin=-3, xmax=13,
%ymin=-9, ymax=1,
%%enlargelimits={abs=0.2},
%xtick distance=1,
%ytick distance=1,
%grid=both,
%xticklabels={},
%yticklabels={},
%grid style={black,line width=1pt, draw opacity=0.5},
%ticks=none,
%axis equal,
%scale only axis,
%legend pos=north east,
%axis line style={draw=none},
%scale=0.6
%]
\draw[stylegrille] (-3,-9) grid (13,1);
\clip (-3,-9) rectangle (13,1);

\draw[stylevecteur] (0,0) -- (4,0) node[pos=0,left] {\large $\vec u$};
\draw[stylevecteur] (0,-2) -- (8,-2) node[pos=0,left] {\large $2 \vec u$};
\draw[stylevecteur,<-] (0,-4) -- (4,-4) node[pos=0,left] {\large $-\vec u$};
\draw[stylevecteur] (0,-6) -- (6,-6) node[pos=0,left] {\large $\frac 3 2 \vec u$};
\draw[stylevecteur,<-] (0,-8) -- (12,-8) node[pos=0,left] {\large $-3\vec u$};

%\end{axis}
\end{tikzpicture}
\end{center}

\end{ex}

\begin{app}
$I$ est le milieu de $[AB]$ se traduit par $\vecc {AI} = \vecc {IB}$ ou $\vecc{AB} = 2\vecc{AI}$ ou $\vecc {AI} = \frac 1 2 \vecc{AB}$.

\begin{center}
\begin{tikzpicture}
\draw[stylevecteur,line width=2pt] (0,0) -- (8,1) node[pos=0.75,above] {\large $\vec u$} node[pos=0,stylepoint,fill=blue,label=left:A] {} node[pos=1,stylepoint,fill=blue,label=right:B] {};
\draw[stylevecteur,color=red] (0,0) -- (4,0.5) node[pos=0.5,below] {\large { $0.5\vec u$} } node[pos=1,stylepoint,fill=red,label=above:I] {};
\end{tikzpicture}
\end{center}
\end{app}

\rem{Feuille en plus: maths en ligne ex3d ou original}

\end{FlushLeft}

\end{document}
