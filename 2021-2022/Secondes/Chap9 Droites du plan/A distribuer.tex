\documentclass[a4paper,12pt,french]{article}

\usepackage[cours,NB]{../../Style}

\geometry{margin=7mm}

% Début du document
%%%%%%%%%%%%%%%%%%%
\begin{document}

\begin{defin}

\compo[0.6]
{
Soit $\mathscr{D}$ une droite, et $A$ et $B$ deux points de $\mathscr{D}$.

On appelle \textbf{vecteur directeur} de $\mathscr{D}$ tout vecteur $\vv u$ non nul colinéaire à $\vv{AB}$.

Autrement dit, le vecteur $\vv u$ donne la direction de la droite $\mathscr D$.
}
{
\vspace{-7mm}
\begin{center}
\begin{tikzpicture}
\begin{axis}[
styleglobal,
width=0.9*\linewidth,
xmin=-4, xmax=7,
ymin=-2, ymax=4,
xtick distance=1,
ytick distance=1,
font=\normalsize,
minor x tick num=0,
minor y tick num=0,
]
\addplot[styleplot,DarkRed] {0.5*x-1} node [pos=0.9,above] {$\mathscr D$};
\coordinate[stylepoint,label=-45:$A$] (A) at (0,-1);
\coordinate[stylepoint,label=-45:$B$] (B) at (4,1);
%\draw[stylevecteur] (-3,1) -- (1,3) node[pos=0.5,above] {$\vv u$};
%\draw[stylevecteur] (2,2) -- (4,3) node[pos=0.5,above] {$\vv v$};
\end{axis}
\end{tikzpicture}
\end{center}
}
\end{defin}

\begin{rmq}
Un vecteur directeur n'est pas unique: Ici, $\vv{u}$ et $\vv{v}$ sont des vecteurs directeurs de la droite $(AB)$.
\end{rmq}

\begin{ex}

\compo[0.6]
{
Donner les coordonnées de \textbf{plusieurs} vecteurs directeurs des droites $\mathscr D_1$, $\mathscr D_2$, $\mathscr D_3$ et $\mathscr D_4$.

{\setstretch{1.7}
\begin{itemize}
\item $\mathscr D_1$:
\item $\mathscr D_2$:
\item $\mathscr D_3$:
\item $\mathscr D_4$:
\end{itemize}}
}
{
\vspace{-7mm}
\begin{center}
\begin{tikzpicture}
\begin{axis}[
styleglobal,
width=0.9*\linewidth,
xmin=-3, xmax=6,
ymin=-2, ymax=6,
xtick distance=1,
ytick distance=1,
font=\normalsize,
minor x tick num=0,
minor y tick num=0,
]
\addplot[styleplot] {3-x} node [pos=0.85,above right] {$\mathscr D_1$};
\addplot[styleplot,DarkRed] {4} node [pos=0.93,above] {$\mathscr D_2$};
\addplot[styleplot,DarkBlue] {3*x+1} node [pos=0.3,above left] {$\mathscr D_3$};
\draw[line width=1pt,DarkGreen] (4,-2) -- (4,6) node [pos=0.9,left] {$\mathscr D_4$};
\end{axis}
\end{tikzpicture}
\end{center}
}
\end{ex}

\vfill

\begin{exs}
\compo[0.66]
{
Soit $\mathscr D_1:2x-y+3=0$.

\begin{itemize}
\item Un vecteur directeur de $\mathscr D_1$ est:

\item Si $x=0$, l'équation devient:

\vspace{1.8cm}

Alors $A(\ldots;\ldots) \in \mathscr D_1$.
\end{itemize}

On peut donc tracer la droite $\mathscr D_1$ dans le repère ci-contre. 

\

On remarque d'ailleurs que $2x-y+3=0 \iff y=2x+3$ (On appelle cette équation l'\textbf{équation réduite} de $\mathscr D_1$). La droite $\mathscr D_1$ peut donc être identifiée à la courbe d'une fonction affine.
}
{
\vspace{-7mm}
\begin{center}
\begin{tikzpicture}
\begin{axis}[
styleglobal,
width=0.9\linewidth,
xmin=-3, xmax=4,
ymin=-2, ymax=7,
xtick distance=1,
ytick distance=1,
font=\normalsize,
minor x tick num=0,
minor y tick num=0,
]
%\addplot[styleplot,DarkBlue] {2*x+3} node [pos=0.3,above left] {$\mathscr D_1$};
%\coordinate[stylepoint,label=-45:$A$] (A) at (0,3);
%\draw[stylevecteur] (A) -- (1,5) node[midway,below right] {$\vv u$};
%\coordinate[stylepoint,label=-45:$B$] at (1,5);
%
%\draw[line width=1pt,DarkRed] (3,-2) -- (3,11) node[pos=0.65,right] {$\mathscr D_2$};
%\coordinate[stylepoint,fill=red,label=-135:$B$] (B) at (3,1);
%\draw[stylevecteur] (B) -- (3,4) node[midway,right] {$\vv v$};
%\coordinate[stylepoint,fill=red,label=0:$D$] at (3,4);
\end{axis}
\end{tikzpicture}
\end{center}
}

\vspace{3mm}

Soit $\mathscr D_2:3x-9=0$. Un vecteur directeur de $\mathscr D_2$ est:

Ici, il n'y a pas de $y$ donc il suffit de résoudre l'équation:

\vspace{2cm}

Tout point dont les coordonnées sont de la forme \makebox[3cm]{\dotfill} avec $y \in \R$ convient donc. On prend par exemple $B(\ldots; \ldots)$.
\end{exs}

\newpage

\begin{fait}
On se ramène à deux équations de droites que l'on trace.

L'unique solution du système est alors les coordonnées du point d'intersection des deux droites.
\end{fait}

\begin{ex}
\compo[0.55]
{ \setstretch{1.2}
On reprend le système précédent:$\begin{cases}
    5x-2y=4 \\
    -x+y=1
\end{cases}
$

On trace ensuite les deux droites:
$$\begin{aligned} &\mathscr D_1:5x-2y-4=0 \\ &\mathscr D_2:-x+y-1=0 \end{aligned}$$
On constate que leur point d'intersection a pour coordonnées \makebox[3cm]{\dotfill}, ce qui correspond à la solution testée dans l'exemple précédent.
}
{
\begin{center}
\begin{tikzpicture}
\begin{axis}[
styleglobal,
width=0.9\linewidth,
xmin=-3, xmax=5,
ymin=-1, ymax=5,
xtick distance=1,
ytick distance=1,
font=\normalsize,
minor x tick num=0,
minor y tick num=0,
]
%\addplot[styleplot,DarkRed] {2.5*x-2} node [pos=0.9,above] {$\mathscr D_1$};
%\addplot[styleplot,DarkBlue] {x+1} node [pos=0.9,above] {$\mathscr D_2$};

%\coordinate[stylepoint] at (2,3);
\end{axis}
\end{tikzpicture}
\end{center}
}

\end{ex}

\vfill

\begin{fait}
On se ramène à deux équations de droites que l'on trace.

L'unique solution du système est alors les coordonnées du point d'intersection des deux droites.
\end{fait}

\begin{ex}
\compo[0.55]
{ \setstretch{1.2}
On reprend le système précédent:$\begin{cases}
    5x-2y=4 \\
    -x+y=1
\end{cases}
$

On trace ensuite les deux droites:
$$\begin{aligned} &\mathscr D_1:5x-2y-4=0 \\ &\mathscr D_2:-x+y-1=0 \end{aligned}$$
On constate que leur point d'intersection a pour coordonnées \makebox[3cm]{\dotfill}, ce qui correspond à la solution testée dans l'exemple précédent.
}
{
\begin{center}
\begin{tikzpicture}
\begin{axis}[
styleglobal,
width=0.9\linewidth,
xmin=-3, xmax=5,
ymin=-1, ymax=5,
xtick distance=1,
ytick distance=1,
font=\normalsize,
minor x tick num=0,
minor y tick num=0,
]
%\addplot[styleplot,DarkRed] {2.5*x-2} node [pos=0.9,above] {$\mathscr D_1$};
%\addplot[styleplot,DarkBlue] {x+1} node [pos=0.9,above] {$\mathscr D_2$};

%\coordinate[stylepoint,fill=blue] (M) at (2,3);
\end{axis}
\end{tikzpicture}
\end{center}
}

\end{ex}

\vfill

\begin{fait}
On se ramène à deux équations de droites que l'on trace.

L'unique solution du système est alors les coordonnées du point d'intersection des deux droites.
\end{fait}

\begin{ex}
\compo[0.55]
{ \setstretch{1.2}
On reprend le système précédent:$\begin{cases}
    5x-2y=4 \\
    -x+y=1
\end{cases}
$

On trace ensuite les deux droites:
$$\begin{aligned} &\mathscr D_1:5x-2y-4=0 \\ &\mathscr D_2:-x+y-1=0 \end{aligned}$$
On constate que leur point d'intersection a pour coordonnées \makebox[3cm]{\dotfill}, ce qui correspond à la solution testée dans l'exemple précédent.
}
{
\begin{center}
\begin{tikzpicture}
\begin{axis}[
styleglobal,
width=0.9\linewidth,
xmin=-3, xmax=5,
ymin=-1, ymax=5,
xtick distance=1,
ytick distance=1,
font=\normalsize,
minor x tick num=0,
minor y tick num=0,
]
%\addplot[styleplot,DarkRed] {2.5*x-2} node [pos=0.9,above] {$\mathscr D_1$};
%\addplot[styleplot,DarkBlue] {x+1} node [pos=0.9,above] {$\mathscr D_2$};

%\coordinate[stylepoint] (M) at (2,3) {};
\end{axis}
\end{tikzpicture}
\end{center}
}

\end{ex}

\newpage

\end{document}