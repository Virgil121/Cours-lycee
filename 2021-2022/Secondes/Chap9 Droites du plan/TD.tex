\documentclass[twocolumn,landscape,a4paper,12pt,french]{article}

\usepackage[TD]{../../Style}

\pagestyle{empty}

\setlength{\columnsep}{2cm}

% Début du document
%%%%%%%%%%%%%%%%%%%
\begin{document}

\titre{Droites du plan}

\begin{exercice} \
\begin{enumerate}
    \item Soit $\mathscr{D}$ la droite passant par $A(1;-1)$, de vecteur directeur $\vv u \dbinom {-4}{-3}$.
    
    Déterminer une équation cartésienne de $\mathscr{D}$.
    
    \item Soit $\mathscr{D'}$ la droite d'équation $y=-2x+1$.
    
    Déterminer un vecteur directeur $\vv{v}$ de $\mathscr{D'}$. Était-ce prévisible?
\end{enumerate}
\end{exercice}

\begin{exercice} \
\begin{enumerate}
    \item Soit $\mathscr{D}$ la droite passant par $A(4;2)$ et de vecteur directeur $\vv u \dbinom {1}{-3}$.
    
    Déterminer une équation cartésienne de $\mathscr{D}$.
    
    \item Soit $\mathscr{D'}$ la droite d'équation cartésienne $3x-1,5y+1=0$.
    
    Le vecteur $\vv v \dbinom {-4,5}{-9}$ est-il un vecteur directeur de $\mathscr{D'}$ ?
\end{enumerate}
\end{exercice}

\compo[0.6]
{
\begin{exercice}
Déterminer une équation cartésienne de la droite $(AB)$ avec:
\begin{enumerate}
\item $A(5;1)$ et $B(7;-2)$
\item $A(-3;-2)$ et $B(1;1)$
\end{enumerate}
\end{exercice}

\begin{exercice}
Dans le repère ci-contre, tracer les droites $\mathscr D_1$ et $\mathscr D_2$ d'équations cartésiennes respectives:
\begin{enumerate}
\item $\mathscr D_1: x+3y+5=0$
\item $\mathscr D_2: 5x+4y-2=0$
\end{enumerate}
\end{exercice}
}
{
\begin{center}
\begin{tikzpicture}
\begin{axis}[
styleglobal,
width=0.9*\linewidth,
xmin=-2, xmax=8,
ymin=-6, ymax=6,
xtick distance=1,
ytick distance=1,
minor x tick num=0,
minor y tick num=0,
]
\end{axis}
\end{tikzpicture}
\end{center}
}

\begin{exercice}
Résoudre dans $\R$ les systèmes suivants:

$$
S_1:
\begin{cases}
    -x-3y=7 \\
    2x+y=4
\end{cases}
\qquad\qquad
S_2:
\begin{cases}
    3x-5y=12 \\
    4x+5y=-1
\end{cases}
$$

$$
S_3:
\begin{cases}
    2x+3y=6 \\
    x-2y=10
\end{cases}
\qquad\qquad
S_4:
\begin{cases}
    5x-2y=4 \\
    4x+2y=15
\end{cases}
$$
\end{exercice}

%%%%%%%%%%%%%%%%%%%%%%%%%%%%

\setcounter{exercice}{0}
\newpage
\titre{Droites du plan}

\begin{exercice} \
\begin{enumerate}
    \item Soit $\mathscr{D}$ la droite passant par $A(1;-1)$,de vecteur directeur $\vv u \dbinom {-4}{-3}$.
    
    Déterminer une équation cartésienne de $\mathscr{D}$.
    
    \item Soit $\mathscr{D'}$ la droite d'équation $y=-2x+1$.
    
    Déterminer un vecteur directeur $\vv{v}$ de $\mathscr{D'}$. Était-ce prévisible?
\end{enumerate}
\end{exercice}

\begin{exercice} \
\begin{enumerate}
    \item Soit $\mathscr{D}$ la droite passant par $A(4;2)$ et de vecteur directeur $\vv u \dbinom {1}{-3}$.
    
    Déterminer une équation cartésienne de $\mathscr{D}$.
    
    \item Soit $\mathscr{D'}$ la droite d'équation cartésienne $3x-1,5y+1=0$.
    
    Le vecteur $\vv v \dbinom {-4,5}{-9}$ est-il un vecteur directeur de $\mathscr{D'}$ ?
\end{enumerate}
\end{exercice}

\compo[0.6]
{
\begin{exercice}
Déterminer une équation cartésienne de la droite $(AB)$ avec:
\begin{enumerate}
\item $A(5;1)$ et $B(7;-2)$
\item $A(-3;-2)$ et $B(1;1)$
\end{enumerate}
\end{exercice}

\begin{exercice}
Dans le repère ci-contre, tracer les droites $\mathscr D_1$ et $\mathscr D_2$ d'équations cartésiennes respectives:
\begin{enumerate}
\item $\mathscr D_1: x+3y+5=0$
\item $\mathscr D_2: 5x+4y-2=0$
\end{enumerate}
\end{exercice}
}
{
\begin{center}
\begin{tikzpicture}
\begin{axis}[
styleglobal,
width=0.9*\linewidth,
xmin=-2, xmax=8,
ymin=-6, ymax=6,
xtick distance=1,
ytick distance=1,
minor x tick num=0,
minor y tick num=0,
]
\end{axis}
\end{tikzpicture}
\end{center}
}

\begin{exercice}
Résoudre dans $\R$ les systèmes suivants:

$$
S_1:
\begin{cases}
    -x-3y=7 \\
    2x+y=4
\end{cases}
\qquad\qquad
S_2:
\begin{cases}
    3x-5y=12 \\
    4x+5y=-1
\end{cases}
$$

$$
S_3:
\begin{cases}
    2x+3y=6 \\
    x-2y=10
\end{cases}
\qquad\qquad
S_4:
\begin{cases}
    5x-2y=4 \\
    4x+2y=15
\end{cases}
$$
\end{exercice}

\end{document}
