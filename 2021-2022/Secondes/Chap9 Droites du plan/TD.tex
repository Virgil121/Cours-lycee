\documentclass[twocolumn,landscape,a4paper,11pt,french]{article}

\usepackage[TD]{../../Style}

\pagestyle{empty}

\setlength{\columnsep}{2cm}

% Début du document
%%%%%%%%%%%%%%%%%%%
\begin{document}

\newcommand{\contenu}{
\titre{Droites du plan}

\begin{exercice} \
\begin{enumerate}
	\item Donner deux vecteurs directeurs de la droite $\mathscr D_1:5x-8y+13=0$.
    \item Soit $\mathscr D_2$ la droite d'équation cartésienne $3x-1,5y+1=0$.
    
    Le vecteur $\vv v \tbinom {-4,5}{-9}$ est-il un vecteur directeur de $\mathscr D_2$ ?
    
    \item Soit $\mathscr D_3$ la droite d'équation $y=-2x+1$.
    
    Déterminer un vecteur directeur $\vv{v}$ de $\mathscr D_3$. Était-ce prévisible?
\end{enumerate}
\end{exercice}

\begin{exercice}
Déterminer une équation cartésienne des droites suivantes:
\begin{enumerate}
	\item $\mathscr D_1$ passant par $A(1;-1)$ et dont un vecteur directeur est $\vv u \tbinom {-4}{-3}$.
    \item $\mathscr D_2$ passant par $A(4;2)$ et dont un vecteur directeur est $\vv u \tbinom {1}{-3}$.
    \item $\mathscr D_3$ passant par $A(-3;7)$ et dont un vecteur directeur est $\vv u \tbinom {-3}{7}$.
\end{enumerate}
\end{exercice}

\compo[0.58]
{
\begin{exercice}
Déterminer une équation cartésienne de la droite $(AB)$ avec:
\begin{enumerate}
\item $A(5;1)$ et $B(7;-2)$
\item $A(-3;-2)$ et $B(1;1)$
\item $A(-2;6)$ et $B(3;-2)$
\end{enumerate}
\end{exercice}

\begin{exercice}
Dans le repère ci-contre, tracer les droites $\mathscr D_1$ et $\mathscr D_2$ d'équations cartésiennes respectives:
\begin{enumerate}
\item $\mathscr D_1: x+3y+5=0$
\item $\mathscr D_2: 5x+4y-2=0$
\item $\mathscr D_3: 7x-42=0$
\end{enumerate}
\end{exercice}
}
{
\begin{center}
\begin{tikzpicture}
\begin{axis}[
styleglobal,
width=0.95*\linewidth,
xmin=-2, xmax=8,
ymin=-6, ymax=6,
xtick distance=1,
ytick distance=1,
minor x tick num=0,
minor y tick num=0,
font=\normalsize,
]
\end{axis}
\end{tikzpicture}
\end{center}
}

\begin{exercice}
Résoudre dans $\R$ les systèmes suivants:

$$\renewcommand\arraystretch{2.5}
\begin{array}{lll}
S_1:
\begin{cases}
	2x+3y=4 \\
	5x+3y=11
\end{cases}
&
S_2:
\begin{cases}
    3x-5y=12 \\
    4x+5y=-1
\end{cases}
&
S_3:
\begin{cases}
    -x-3y=7 \\
    2x+y=4
\end{cases}
\\
S_4:
\begin{cases}
    2x+3y=6 \\
    x-2y=10
\end{cases}
&
S_5:
\begin{cases}
    5x-2y=4 \\
    4x+2y=15
\end{cases}
&
S_6:
\begin{cases}
    7x+3y=4 \\
    4x+2y=2
\end{cases}
\end{array}
$$
\end{exercice}

\begin{exercice}
Déterminer les coordonnées du point d'intersection des droites: $$\mathscr D_1: -6x+y+17=0 \qquad \text{et} \qquad \mathscr D_2:7x+2y-4=0$$
\end{exercice}

\setcounter{exercice}{0}}

\contenu
\newpage
\contenu

\end{document}
