\documentclass[,a4paper,12pt,french]{article}

\usepackage[TD]{../../Style}
\geometry{margin=1cm}

\pagestyle{empty}

% Début du document
%%%%%%%%%%%%%%%%%%%
\begin{document}

\newcommand{\\vva}{
\compo[0.6]
{
On se donne le programme de calcul suivant:

\begin{algorithm*}[H]
Choisir un nombre \;
Le multiplier par 2 \;
Soustraire 3 \;
Donner le résultat final \;
\end{algorithm*}

\begin{enumerate}
\item Compléter le tableau de valeurs suivant:

\begin{center}
\begin{tabularx}{\linewidth}{|c|*{5}{Y|}} \hline
Valeur d'entrée & $-2$ & $-1$ & $0$ & $1$ & $2$ \\ \hline
Résultat & \rule{0pt}{1cm} & & & & \\ \hline
\end{tabularx}
\end{center}

\item Placer sur le repère ci-contre les points correspondant, avec en abscisse (axe $x$) les valeurs d'entrée et en ordonnée (axe $y$) les résultats. Relier les points.
\end{enumerate}
}
{
\begin{tikzpicture}
\begin{axis}[
styleglobal,
width=\linewidth,
xmin=-6, xmax= 6,
ymin=-5, ymax=5,
xtick distance=1,
ytick distance=1,
minor x tick num=0,
minor y tick num=0,
]

\end{axis}
\end{tikzpicture}
}}

\newcommand{\\vvb}{
\compo[0.6]
{
On définit la fonction $g:x \mapsto 2-x$.
\begin{enumerate}
\item Compléter le tableau de valeurs suivant:

\begin{center}
\begin{tabularx}{\linewidth}{|c|*{5}{Y|}} \hline
$x$ & $-2$ & $-1$ & $0$ & $1$ & $2$ \\ \hline
$g(x)$ & \rule{0pt}{1cm} & & & & \\ \hline
\end{tabularx}
\end{center}

\item Tracer la représentation graphique de $g$ sur le repère ci-contre.
\end{enumerate}
}
{
\begin{tikzpicture}
\begin{axis}[
styleglobal,
width=\linewidth,
xmin=-6, xmax= 6,
ymin=-5, ymax=5,
xtick distance=1,
ytick distance=1,
minor x tick num=0,
minor y tick num=0,
]

\end{axis}
\end{tikzpicture}
}}

\newcommand{\\vvc}{
\compo[0.6]
{
On définit la fonction $g:x \mapsto 1.5x-0.5$.
\begin{enumerate}
\item Compléter le tableau de valeurs suivant:

\begin{center}
\begin{tabularx}{\linewidth}{|c|*{5}{Y|}} \hline
$x$ & $-2$ & $-1$ & $0$ & $1$ & $2$ \\ \hline
$g(x)$ & \rule{0pt}{1cm} & & & & \\ \hline
\end{tabularx}
\end{center}

\item Tracer la représentation graphique de $g$ sur le repère ci-contre.
\end{enumerate}
}
{
\begin{tikzpicture}
\begin{axis}[
styleglobal,
width=\linewidth,
xmin=-6, xmax= 6,
ymin=-5, ymax=5,
xtick distance=1,
ytick distance=1,
minor x tick num=0,
minor y tick num=0,
]

\end{axis}
\end{tikzpicture}
}}

\newcommand{\\vvd}{
\compo[0.6]
{
On définit la fonction $g:x \mapsto -\frac 5 2 x + 0,4$.
\begin{enumerate}
\item Compléter le tableau de valeurs suivant:

\begin{center}
\begin{tabularx}{\linewidth}{|c|*{5}{Y|}} \hline
$x$ & $-2$ & $-1$ & $0$ & $1$ & $2$ \\ \hline
$g(x)$ & \rule{0pt}{1cm} & & & & \\ \hline
\end{tabularx}
\end{center}

\item Tracer la représentation graphique de $g$ sur le repère ci-contre.
\end{enumerate}
}
{
\begin{tikzpicture}
\begin{axis}[
styleglobal,
width=\linewidth,
xmin=-6, xmax= 6,
ymin=-5, ymax=5,
xtick distance=1,
ytick distance=1,
minor x tick num=0,
minor y tick num=0,
]

\end{axis}
\end{tikzpicture}
}}

\\vva

\vfill

\\vva

\vfill

\\vva

\newpage

\\vvb

\vfill

\\vvb

\vfill

\\vvb

\vfill

\\vvb

\newpage

\\vvc

\vfill

\\vvc

\vfill

\\vvc

\vfill

\\vvc

\newpage

\\vvd

\vfill

\\vvd

\vfill

\\vvd

\vfill

\\vvd

\newpage
\end{document}
