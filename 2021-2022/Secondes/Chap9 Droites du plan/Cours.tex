\documentclass[a4paper,12pt,french]{article}

\usepackage[cours]{../../Style}

% Début du document
%%%%%%%%%%%%%%%%%%%
\begin{document}

\title{Droites du plan}
\maketitle

\begin{programme}
\item Vecteur directeur d'une droite
\item Equation de droite: Equation cartésienne, \textit{réduite}
\item \textit{Pente (ou coefficient directeur) d'une droite non parallèle à l'axe des ordonnées}
\item Capacités:
\begin{itemize}
\item Déterminer une équation de droite \textit{à partir de deux points}, un point et un vecteur directeur \textit{ou un point et la pente}.
\item Déterminer la pente ou un vecteur directeur d'une droite donnée par une équation ou une représentation graphique
\item Tracer une droite connaissant son équation cartésienne \textit{ou réduite}.
\item \textit{Etablir que trois points sont alignés ou non}
\item Déterminer si deux droites sont parallèles ou sécantes
\item Résoudre un système de deux équations linéaires à deux inconnues, déterminer le point d'intersection de deux droites sécantes
\end{itemize}
\end{programme}

\section{Équations cartésiennes de droites et vecteurs directeurs}

\begin{fait}
On se place dans un repère \Oij du plan.
\end{fait}

\subsection{Vecteurs directeurs}

\begin{defin}

\compo
{
Soit $\mathscr{D}$ une droite, et $A$ et $B$ deux points de $\mathscr{D}$.

On appelle \textbf{vecteur directeur} de $\mathscr{D}$ tout vecteur $\vv u$ non nul colinéaire à $\vv{AB}$.

Autrement dit, le vecteur $\vv u$ donne la direction de la droite $\mathscr D$.
}
{
\vspace{-5mm}
\begin{center}
\begin{tikzpicture}
\begin{axis}[
styleglobal,
width=0.9*\linewidth,
xmin=-4, xmax=7,
ymin=-2, ymax=4,
xtick distance=1,
ytick distance=1,
font=\normalsize,
minor x tick num=0,
minor y tick num=0,
]
\addplot[styleplot,DarkRed] {0.5*x-1} node [pos=0.9,above] {$\mathscr D$};
\node[stylepoint,fill=blue,label=-45:$A$] (A) at (0,-1) {};
\node[stylepoint,fill=blue,label=-45:$B$] (B) at (4,1) {};
\draw[stylevecteur] (-3,1) -- (1,3) node[pos=0.5,above] {$\vv u$};
\draw[stylevecteur] (2,2) -- (4,3) node[pos=0.5,above] {$\vv v$};
\end{axis}
\end{tikzpicture}
\end{center}
}
\end{defin}

\begin{rmq}
Un vecteur directeur n'est pas unique: Ici, $\vv{u}$ et $\vv{v}$ sont des vecteurs directeurs de la droite $(AB)$.
\end{rmq}

\begin{ex}

\compo[0.6]
{
Donnons les coordonnées de \textbf{plusieurs} vecteurs directeurs des droites $\mathscr D_1$, $\mathscr D_2$, $\mathscr D_3$ et $\mathscr D_4$.

{%\setstretch{2}
\begin{itemize}
\item $\mathscr D_1$: $\vv u \tbinom 1 {-1}$ et $\vv v \tbinom 2 {-2}$
\item $\mathscr D_2$: $\vv u \tbinom 1 0$ et $\vv v \tbinom {-3} 0$
\item $\mathscr D_3$: $\vv u \tbinom 1 2$ et $\vv v \tbinom 2 4$
\item $\mathscr D_4$: $\vv u \tbinom 0 1$ et $\vv v \tbinom 0 {-4}$
\end{itemize}}
}
{
\vspace{-7mm}
\begin{center}
\begin{tikzpicture}
\begin{axis}[
styleglobal,
width=0.9*\linewidth,
xmin=-3, xmax=6,
ymin=-2, ymax=6,
xtick distance=1,
ytick distance=1,
font=\normalsize,
minor x tick num=0,
minor y tick num=0,
]
\addplot[styleplot] {3-x} node [pos=0.85,above right] {$\mathscr D_1$};
\addplot[styleplot,DarkRed] {4} node [pos=0.93,above] {$\mathscr D_2$};
\addplot[styleplot,DarkBlue] {3*x+1} node [pos=0.3,above left] {$\mathscr D_3$};
\draw[line width=1pt,DarkGreen] (4,-2) -- (4,6) node [pos=0.9,left] {$\mathscr D_4$};
\end{axis}
\end{tikzpicture}
\end{center}
}
\end{ex}

\rem{Exo 1}

\subsection{Équations cartésiennes}

\begin{prop}
Soit $\mathscr D$ une droite et $\vv u \dbinom{-b}{a}$ un vecteur directeur de $\mathscr D$. Alors $\mathscr D$ possède une équation de la forme $ax+by+c=0$, appelée équation cartésienne de $\mathscr D$.
\end{prop}

\begin{fait}
Un point $M(x;y)$ appartient à cette droite ssi $\vv{AM} \dbinom{x-x_A}{y-y_A}$ et $\vv u \dbinom {-b}{a}$ sont colinéaires.

Il faut donc que $det(\vv{AM};\vv{u})=0$. Cela donne alors: $$det(\vv{AM};\vv{u})=a(x-x_A)-(-b)(y-y_A)=0$$

En développant, l'équation peut s'écrire:

$$ax+by+(-ax_A-by_A)=0$$

En posant $c=-ax_A-by_A$, on trouve bien la forme voulue: $\boxed{ax+by+c=0}$.
\end{fait}

\begin{ex}
Un vecteur directeur de $\mathscr D:4x-8y+3=0$ est $\vv u \tbinom 8 4$. On peut en déduire un autre: $\vv v \tbinom 2 1$.
\end{ex}

\begin{cor}
Si les coordonnées $(x;y)$ d'un point $M$ vérifient l'équation $ax+by+c=0$ d'une droite $\mathscr D$, alors $M$ appartient à la droite $\mathscr D$ dont un vecteur directeur est $\vv u \dbinom {-b}{a}$.
\end{cor}

\begin{exs}
Déterminons une équation cartésienne des droites suivantes:
\begin{itemize}
    \item $\mathscr D_1$ passant par le point $A(3;1)$ et de vecteur directeur $\vv u \tbinom{-1}{5}$:
    
	On sait alors qu'une équation cartésienne de $\mathscr D_1$ est $5x+y+c=0$ où $c$ reste à trouver.
	
	De plus, en remplaçant $x$ et $y$ par les coordonnées de $A$, on obtient:
	$$\begin{aligned} 5 \times 3 + 1 + c & =0 \\ 16+c & =0 \\ c & =-16 \end{aligned}$$
	
	Une équation cartésienne de $\mathscr D_1$ est donc $5x+y-16=0$.
    
    \item $\mathscr D_2$ passant par les points $B(5;3)$ et $C(1;-3)$:
    
    $\vv{BC} \tbinom {-4} {-6}$ est un vecteur directeur de $\mathscr D_2$. On obtient alors l'équation $-6x+4y+c=0$.
    
    Pour trouver $c$, on remplace $x$ et $y$ par les coordonnées de $A$ ou $B$. Avec $A$, cela donne:
    $$\begin{aligned} -6 \times 5 +4 \times 3 +c & =0 \\ -30+12+c & =0 \\ c & =18 \end{aligned}$$
    Une équation cartésienne de $\mathscr D_2$ est donc $-6x+4y+18=0$. A noter que l'on peut simplifier cette équation pour obtenir $-3x+2y+9=0$.
\end{itemize}
\end{exs}

\progres
\rem{Exos 2,3}

\begin{methodes}
Pour tracer une droite $\mathscr D$ étant donnée son équation cartésienne, on peut:
\begin{itemize}
\item Déterminer les coordonnées de deux points appartenant à $\mathscr D$ en remplaçant $x$ ou $y$ par des valeurs spécifiques;
\item Déterminer de la même manière les coordonnées d'un point, et utiliser un vecteur directeur.
\end{itemize}
\end{methodes}

\begin{exs}
\compo[0.66]
{
Soit $\mathscr D_1:2x-y+3=0$.

\begin{itemize}
\item Un vecteur directeur de $\mathscr D_1$ est $\vv u \tbinom 1 2$.

\item Si $x=0$, l'équation devient $-y+3=0$ donc $y=3$.
Alors $A(0;3) \in \mathscr D_1$.
\end{itemize}

On peut donc tracer la droite $\mathscr D_1$ dans le repère ci-contre. On remarque d'ailleurs que $2x-y+3=0 \iff y=2x+3$ (On appelle cette équation l'\textbf{équation réduite}  de $\mathscr D_1$). La droite $\mathscr D_1$ peut donc être identifiée à la courbe d'une fonction affine.
}
{
\vspace{-7mm}
\begin{center}
\begin{tikzpicture}
\begin{axis}[
styleglobal,
width=0.9\linewidth,
xmin=-3, xmax=4,
ymin=-2, ymax=7,
xtick distance=1,
ytick distance=1,
font=\normalsize,
minor x tick num=0,
minor y tick num=0,
]
\addplot[styleplot,DarkBlue] {2*x+3} node [pos=0.3,above left] {$\mathscr D_1$};
\node[stylepoint,fill=blue,label=-45:$A$] (A) at (0,3) {};
\draw[stylevecteur] (A) -- (1,5) node[midway,below right] {$\vv u$};
%\node[stylepoint,fill=blue,label=-45:$B$] at (1,5) {};

\draw[line width=1pt,DarkRed] (3,-2) -- (3,11) node[pos=0.65,right] {$\mathscr D_2$};
\node[stylepoint,fill=red,label=-135:$B$] (B) at (3,1) {};
\draw[stylevecteur] (B) -- (3,4) node[midway,right] {$\vv v$};
%\node[stylepoint,fill=red,label=0:$D$] at (3,4) {};
\end{axis}
\end{tikzpicture}
\end{center}
}
\strut Soit $\mathscr D_2:3x-9=0$. Un vecteur directeur de $\mathscr D_2$ est $\vv v \tbinom 0 3$. Ici, il n'y a pas de $y$ donc il suffit de résoudre l'équation: $$3x-9 =0 \iff 3x=9 \iff x =3$$
Tout point dont les coordonnées sont de la forme $(3;y)$ avec $y \in \R$ convient donc. On prend par exemple $B(3;1)$.
\end{exs}

\rem{Exo 4}

\section{Systèmes de deux équations à deux inconnues}

\begin{fait}
Résoudre un système à deux inconnues, c'est trouver le ou les couples $(x;y)$ qui vérifie(nt) à la fois les deux équations.
\end{fait}

\begin{ex}
Le couple $(2;3)$ vérifie le système d'équations
$
\begin{cases}
    5x-2y=4 \\
    -x+y=1
\end{cases}
$
car %en remplaçant $x$ et $y$ par $2$ et $3$, on trouve
$
\begin{cases}
5 \times 2 - 2 \times 3 =4 \\
-2 + 3 = 1
\end{cases}
$
\end{ex}

\subsection{Résolution graphique}

\begin{fait}
On se ramène à deux équations de droites que l'on trace.

L'unique solution du système est alors les coordonnées du point d'intersection des deux droites.
\end{fait}

\begin{ex}
\compo[0.6]
{
On reprend le système précédent:
$
\begin{cases}
    5x-2y=4 \\
    -x+y=1
\end{cases}
$

On trace les deux droites:
$$\begin{aligned} &\mathscr D_1:5x-2y-4=0 \\ &\mathscr D_2:-x+y-1=0 \end{aligned}$$

Leur point d'intersection a pour coordonnées $(2;3)$, correspondant à la solution testée dans l'exemple précédent.
}
{
\vspace{-5mm}
\begin{center}
\begin{tikzpicture}
\begin{axis}[
styleglobal,
width=0.9\linewidth,
xmin=-3, xmax=5,
ymin=-1, ymax=5,
xtick distance=1,
ytick distance=1,
font=\normalsize,
minor x tick num=0,
minor y tick num=0,
]
\addplot[styleplot,DarkRed] {2.5*x-2} node [pos=0.45,right] {$\mathscr D_1$};
\addplot[styleplot,DarkBlue] {x+1} node [pos=0.3,above left] {$\mathscr D_2$};

\node[stylepoint,fill=blue] at (2,3) {};
\end{axis}
\end{tikzpicture}
\end{center}
}

\end{ex}


\subsection{Résolution algébrique}

\subsubsection{Méthode par combinaison linéaire}

\begin{fait}
Le but est de faire des opérations entre les lignes pour faire disparaitre des inconnues.
\end{fait}

\begin{ex}

Résolvons dans $\R$ le système suivant:
$
\begin{cases}
    2x-3y=9 & (L_1) \\
    2x+y=-1 & (L_2)
\end{cases}
$

On voit que les deux lignes contiennent le terme $2x$. On peut donc l'annuler en soustrayant la première ligne à la seconde ligne:

$$\renewcommand{\arraystretch}{2.5}
\begin{array}{lclcl}
\begin{cases}
    2x-3y=9 \\
    2x+y=-1
\end{cases}
&
\xLeftrightarrow{L_2 \gets L_2-L_1}
&
\begin{cases}
    2x-3y=9 \\
    4y=-10
\end{cases}
&
\xLeftrightarrow{\text{On résoud } L_2}
&
\begin{cases}
    2x-3y=9 \\
    y=-2,5
\end{cases} \\
&
\xLeftrightarrow{\text{On remplace } y}
&
\begin{cases}
    2x-7,5=9 \\
    y=-2,5
\end{cases}
&
\xLeftrightarrow{\text{On résoud } L_1}
&
\begin{cases}
    2x=1,5 \\
    y=-2,5
\end{cases} \\
&
\iff
&
\begin{cases}
    x=0,75 \\
    y=-2,5
\end{cases}
\end{array}
$$

\end{ex}

\rem{Exo 5.1,5.2}

\subsubsection{Méthode par substitution}

\begin{fait}
Le but est d'exprimer une variable en fonction d'une autre pour pouvoir l'éliminer dans une autre ligne.
\end{fait}

\begin{ex}

Résolvons dans $\R$ le système suivant:
$
\begin{cases}
    6x-7y=1 & (L_1)\\
    x-3y=2 & (L_2)
\end{cases}
$

On va isoler $x$ dans la seconde ligne pour le \og réinjecter \fg dans la première.

$$\renewcommand{\arraystretch}{2.5}
\begin{array}{lclcl}
\begin{cases}
    6x-7y=1 \\
    x-3y=2
\end{cases}
&
\xLeftrightarrow{\text{On isole } x}
&
\begin{cases}
    6x-7y=1 \\
    x=2+3y
\end{cases}
&
\xLeftrightarrow{\text{On réinjecte}}
&
\begin{cases}
    6(2+3y)-7y=1 \\
    x=2+3y
\end{cases} \\
&
\xLeftrightarrow{\text{On résoud } L_1}
&
\begin{cases}
    12+18y-7y=1 \\
    x=2+3y
\end{cases}
&
\iff
&
\begin{cases}
    11y=-11 \\
    x=2+3y
\end{cases} \\
&
\iff
&
\begin{cases}
    y=-1 \\
    x=-2+3y
\end{cases}
&
\xLeftrightarrow{\text{On remplace } y}
&
\begin{cases}
    y=-1 \\
    x=-2-3=-5
\end{cases}
\end{array}
$$
\end{ex}

\rem{Exos 5,6}

\end{document}

\section{Equations cartésiennes et réduites}

\section{Vecteur directeur d'une droite}

\section{Intersections et systèmes}
