\documentclass[a4paper,12pt,french]{article}

\usepackage[cours,eleve,NB]{../../Style}

% Début du document
%%%%%%%%%%%%%%%%%%%
\begin{document}

\title{Droites du plan}
\maketitle

\begin{programme}
\item Vecteur directeur d'une droite
\item Equation de droite: Equation cartésienne, \textit{réduite}
\item \textit{Pente (ou coefficient directeur) d'une droite non parallèle à l'axe des ordonnées}
\item Capacités:
\begin{itemize}
\item Déterminer une équation de droite à partir de deux points, un point et un vecteur directeur \textit{ou un point et la pente}.
\item Déterminer la pente ou un vecteur directeur d'une droite donnée par une équation ou une représentation graphique
\item Tracer une droite connaissant son équation cartésienne \textit{ou réduite}.
\item \textit{Etablir que trois points sont alignés ou non}
\item Déterminer si deux droites sont parallèles ou sécantes
\item Résoudre un système de deux équations linéaires à deux inconnues, déterminer le point d'intersection de deux droites sécantes
\end{itemize}
\end{programme}

\section{Équations cartésiennes de droites et vecteurs directeurs}

\begin{fait}
On se place dans un repère \Oij du plan.
\end{fait}

\subsection{Vecteurs directeurs}

\begin{defin}

\compo[0.6]
{
Soit $\mathscr{D}$ une droite, et $A$ et $B$ deux points de $\mathscr{D}$.

On appelle \textbf{vecteur directeur} de $\mathscr{D}$ tout vecteur $\vv u$ non nul colinéaire à $\vv{AB}$.

Autrement dit, le vecteur $\vv u$ donne la direction de la droite $\mathscr D$.
}
{
\vspace{-7mm}
\begin{center}
\begin{tikzpicture}
\begin{axis}[
styleglobal,
width=0.9*\linewidth,
xmin=-4, xmax=7,
ymin=-2, ymax=4,
xtick distance=1,
ytick distance=1,
font=\normalsize,
minor x tick num=0,
minor y tick num=0,
]
\addplot[styleplot,DarkRed] {0.5*x-1} node [pos=0.9,above] {$\mathscr D$};
\coordinate[stylepoint,label=-45:$A$] (A) at (0,-1);
\coordinate[stylepoint,label=-45:$B$] (B) at (4,1);
%\draw[stylevecteur] (-3,1) -- (1,3) node[pos=0.5,above] {$\vv u$};
%\draw[stylevecteur] (2,2) -- (4,3) node[pos=0.5,above] {$\vv v$};
\end{axis}
\end{tikzpicture}
\end{center}
}
\end{defin}

\begin{rmq}
Un vecteur directeur n'est pas unique: Ici, $\vv{u}$ et $\vv{v}$ sont des vecteurs directeurs de la droite $(AB)$.
\end{rmq}

\begin{ex}

\compo[0.6]
{
Donner les coordonnées de \textbf{plusieurs} vecteurs directeurs des droites $\mathscr D_1$, $\mathscr D_2$, $\mathscr D_3$ et $\mathscr D_4$.

{\setstretch{2}
\begin{itemize}
\item $\mathscr D_1$:
\item $\mathscr D_2$:
\item $\mathscr D_3$:
\item $\mathscr D_4$:
\end{itemize}}
}
{
\vspace{-5mm}
\begin{center}
\begin{tikzpicture}
\begin{axis}[
styleglobal,
width=0.9*\linewidth,
xmin=-3, xmax=6,
ymin=-2, ymax=6,
xtick distance=1,
ytick distance=1,
font=\normalsize,
minor x tick num=0,
minor y tick num=0,
]
\addplot[styleplot] {3-x} node [pos=0.85,above right] {$\mathscr D_1$};
\addplot[styleplot,DarkRed] {4} node [pos=0.93,above] {$\mathscr D_2$};
\addplot[styleplot,DarkBlue] {3*x+1} node [pos=0.3,above left] {$\mathscr D_3$};
\draw[line width=1pt,DarkGreen] (4,-2) -- (4,6) node [pos=0.9,left] {$\mathscr D_4$};
\end{axis}
\end{tikzpicture}
\end{center}
}
\end{ex}

\subsection{Équations cartésiennes}

\begin{prop}
Soit $\mathscr D$ une droite et $\vv u \tbinom{-b}{a}$ un vecteur directeur de $\mathscr D$. Alors $\mathscr D$ possède une équation de la forme $ax+by+c=0$, appelée équation cartésienne de $\mathscr D$.
\end{prop}

%\begin{fait}
%Un point $M(x;y)$ appartient à cette droite ssi $\vv{AM} \dbinom{x-x_A}{y-y_A}$ et $\vv u \dbinom {-b}{a}$ sont colinéaires.
%
%Il faut donc que $det(\vv{AM};\vv{u})=0$. Cela donne alors: $$det(\vv{AM};\vv{u})=a(x-x_A)-(-b)(y-y_A)=0$$
%
%En développant, l'équation peut s'écrire:
%
%$$ax+by+(-ax_A-by_A)=0$$
%
%En posant $c=-ax_A-by_A$, on trouve bien la forme voulue: $\boxed{ax+by+c=0}$.
%\end{fait}

\begin{cor}
Si les coordonnées $(x;y)$ d'un point $M$ vérifient l'équation $ax+by+c=0$ d'une droite $\mathscr D$, alors $M$ appartient à la droite $\mathscr D$ dont un vecteur directeur est $\vv u \tbinom {-b}{a}$.
\end{cor}

\begin{exs}
Déterminons une équation cartésienne des droites suivantes:
\begin{itemize}
    \item $\mathscr D_1$ passant par le point $A(3;1)$ et de vecteur directeur $\vv u \dbinom{-1}{5}$.
    
	\vspace{5.2cm}
    
    \item $\mathscr D_2$ passant par les points $B(5;3)$ et $C(1;-3)$.
    
    \vspace{5.2cm}
    
\end{itemize}
\end{exs}

\begin{methodes}
Pour tracer une droite $\mathscr D$ étant donnée son équation cartésienne, on peut:
\begin{itemize}
\item Déterminer les coordonnées de deux points appartenant à $\mathscr D$ en remplaçant $x$ ou $y$ par des valeurs spécifiques;
\item Déterminer de la même manière les coordonnées d'un point, et utiliser un vecteur directeur.
\end{itemize}
\end{methodes}

\begin{exs}
\compo[0.66]
{
Soit $\mathscr D_1:2x-y+3=0$.

\begin{itemize}
\item Un vecteur directeur de $\mathscr D_1$ est:

\item Si $x=0$, l'équation devient:

\vspace{1.8cm}

Alors $A(\ldots;\ldots) \in \mathscr D_1$.
\end{itemize}

On peut donc tracer la droite $\mathscr D_1$ dans le repère ci-contre. 
}
{
\vspace{-7mm}
\begin{center}
\begin{tikzpicture}
\begin{axis}[
styleglobal,
width=0.9\linewidth,
xmin=-3, xmax=4,
ymin=-2, ymax=7,
xtick distance=1,
ytick distance=1,
font=\normalsize,
minor x tick num=0,
minor y tick num=0,
]
%\addplot[styleplot,DarkBlue] {2*x+3} node [pos=0.3,above left] {$\mathscr D_1$};
%\coordinate[stylepoint,fill=blue,label=-45:$A$] (A) at (0,3);
%\draw[stylevecteur] (A) -- (1,5) node[midway,below right] {$\vv u$};
%\coordinate[stylepoint,fill=blue,label=-45:$B$] at (1,5);
%
%\draw[line width=1pt,DarkRed] (3,-2) -- (3,11) node[pos=0.65,right] {$\mathscr D_2$};
%\coordinate[stylepoint,fill=red,label=-135:$B$] (B) at (3,1);
%\draw[stylevecteur] (B) -- (3,4) node[midway,right] {$\vv v$};
%\coordinate[stylepoint,fill=red,label=0:$D$] at (3,4);
\end{axis}
\end{tikzpicture}
\end{center}
}
\strut On remarque d'ailleurs que $2x-y+3=0 \iff y=2x+3$ (On appelle cette équation l'\textbf{équation réduite} de $\mathscr D_1$). La droite $\mathscr D_1$ peut donc être identifiée à la courbe d'une fonction affine.

Soit $\mathscr D_2:3x-9=0$. Un vecteur directeur de $\mathscr D_2$ est:

Ici, il n'y a pas de $y$ donc il suffit de résoudre l'équation:

\vspace{2cm}

Tout point dont les coordonnées sont de la forme \makebox[3cm]{\dotfill} avec $y \in \R$ convient donc. On prend par exemple $B(\ldots; \ldots)$.
\end{exs}

%\begin{exs}
%\setstretch{1.2}
%\compohaut[0.66]
%{
%Soit $\mathscr D_1:2x-y+3=0$.
%
%\begin{itemize}[label=\textendash]
%\item Si $x=0$, alors l'équation devient: \vspace{1.5cm}
%
%\item Si $x=1$, alors l'équation devient: \vspace{1.5cm}
%\end{itemize}
%
%On peut donc tracer la droite $\mathscr D_1$ dans le repère ci-contre:
%}
%{
%\vspace{-5mm}
%\begin{center}
%\begin{tikzpicture}
%\begin{axis}[
%styleglobal,
%width=0.9\linewidth,
%xmin=-3, xmax=4,
%ymin=-2, ymax=7,
%xtick distance=1,
%ytick distance=1,
%font=\normalsize,
%minor x tick num=0,
%minor y tick num=0,
%]
%%\addplot[styleplot,DarkBlue] {2*x+3} node [pos=0.3,above left] {$\mathscr D_1$};
%%\coordinate[stylepoint,label=-45:$A$] at (0,3);
%%\coordinate[stylepoint,label=-45:$B$] at (1,5);
%
%%\draw[line width=1pt,DarkRed] (3,-2) -- (3,11) node[pos=0.65,right] {$\mathscr D_2$};
%%\coordinate[stylepoint,fill=red,label=0:$C$] at (3,1);
%%\coordinate[stylepoint,fill=red,label=0:$D$] at (3,4);
%\end{axis}
%\end{tikzpicture}
%\end{center}
%}
%On remarque d'ailleurs que $2x-y+3=0 \iff y=2x+3$ (On appelle cette équation l'\textbf{équation réduite} de $\mathscr D_1$). La droite $\mathscr D_1$ peut donc être identifiée à la courbe d'une fonction affine.
%
%Soit maintenant $\mathscr D_2:3x-9=0$. Ici, il n'y a pas de $y$ donc il suffit de résoudre l'équation: \vspace{3cm}
%
%Tout point dont les coordonnées sont de la forme \makebox[3cm]{\dotfill} avec $y \in \R$ convient donc. On prend par exemple $C(\ldots; \ldots)$ et $D(\ldots;\ldots)$.
%\end{exs}

\section{Systèmes de deux équations à deux inconnues}

\begin{fait}
Résoudre un système à deux inconnues , c'est trouver le ou les couples $(x;y)$ qui vérifie(nt) à la fois les deux équations.
\end{fait}

\begin{ex}
Le couple $(2;3)$ vérifie le système d'équations
$
\begin{cases}
    5x-2y=4 \\
    -x+y=1
\end{cases}
$.
En effet, en remplaçant $x$ et $y$ par $2$ et $3$, on trouve:

\vspace*{2cm}
\end{ex}

\subsection{Résolution graphique}

\begin{fait}
On se ramène à deux équations de droites que l'on trace.

L'unique solution du système est alors les coordonnées du point d'intersection des deux droites.
\end{fait}

\begin{ex}
\compo[0.55]
{ \setstretch{1.2}
On reprend le système précédent:$\begin{cases}
    5x-2y=4 \\
    -x+y=1
\end{cases}
$

On trace ensuite les deux droites:
$$\begin{aligned} &\mathscr D_1:5x-2y-4=0 \\ &\mathscr D_2:-x+y-1=0 \end{aligned}$$
On constate que leur point d'intersection a pour coordonnées \makebox[3cm]{\dotfill}, ce qui correspond à la solution testée dans l'exemple précédent.
}
{
\begin{center}
\begin{tikzpicture}
\begin{axis}[
styleglobal,
width=0.9\linewidth,
xmin=-3, xmax=5,
ymin=-1, ymax=5,
xtick distance=1,
ytick distance=1,
font=\normalsize,
minor x tick num=0,
minor y tick num=0,
]
%\addplot[styleplot,DarkRed] {2.5*x-2} node [pos=0.9,above] {$\mathscr D_1$};
%\addplot[styleplot,DarkBlue] {x+1} node [pos=0.9,above] {$\mathscr D_2$};

%\coordinate[stylepoint] (M) at (2,3);
\end{axis}
\end{tikzpicture}
\end{center}
}

\end{ex}

\newpage

\subsection{Résolution algébrique}

\subsubsection{Méthode par combinaison linéaire}

\begin{fait}
Le but est de faire des opérations entre les lignes pour faire disparaitre des inconnues.
\end{fait}

\begin{ex}

Résolvons dans $\R$ le système suivant:
$
\begin{cases}
    2x-3y=9 & (L_1) \\
    2x+y=-1 & (L_2)
\end{cases}
$

\vspace*{7.5cm}

\strut

\end{ex}

\subsubsection{Méthode par substitution}

\begin{fait}
Le but est d'exprimer une variable en fonction d'une autre pour pouvoir l'éliminer.
\end{fait}

\begin{ex}

Résolvons dans $\R$ le système suivant:
$
\begin{cases}
    6x-7y=1 & (L_1)\\
    x-3y=2 & (L_2)
\end{cases}
$

\vspace*{7.5cm}

\end{ex}

\end{document}

\section{Equations cartésiennes et réduites}

\section{Vecteur directeur d'une droite}

\section{Intersections et systèmes}
