\documentclass[a4paper,french,12pt]{article} % Overleaf ne compile pas sans ça ...
% Les deux lignes suivantes suffisent dans texmaker
\usepackage{titling}
\usepackage{anyfontsize}
\usepackage{sectsty}
\usepackage[eleve,cours]{../../../Style}
\usetikzlibrary{calc} 
\usepackage[subpreambles=false]{standalone}

\renewcommand{\progres}{}

\newcommand{\entre}{\setcounter{section}{0}\newpage}
\setcounter{tocdepth}{0}
\begin{document}
\thispagestyle{empty}
%%%%%%%%%%%%%%%%%%%%%%%%%%%%%%%%%%%%%%%%%%%%%%%%%%%%%%%%%%%%%%%
%
% Welcome to Overleaf --- just edit your LaTeX on the left,
% and we'll compile it for you on the right. If you open the
% 'Share' menu, you can invite other users to edit at the same
% time. See www.overleaf.com/learn for more info. Enjoy!
%
%%%%%%%%%%%%%%%%%%%%%%%%%%%%%%%%%%%%%%%%%%%%%%%%%%%%%%%%%%%%%%%
% How to create a beautiful cover page in LaTeX using TikZ
% Latexdraw.com
% 14:15, 03:03:2020 

\documentclass[a4paper,french,12pt]{article}
\usepackage[dvipsnames,svgnames]{xcolor}
\usepackage{tikz}
\usetikzlibrary{calc}
\usepackage{anyfontsize}
\usepackage{sectsty}
\usepackage[a4paper,margin=15mm,marginparwidth=0mm,marginparsep=0mm]{geometry}

\begin{document}

\pagestyle{empty}

\tikzset{
imgmaths/.style={
path picture={
      \node[opacity=0.5,anchor=west,rotate=-20] at (path picture bounding box.north west) {
        \includegraphics[scale=0.6]{Fondmoitie}};}
}
}

\begin{tikzpicture}[overlay,remember picture]

% Background color
\path[
top color=black!4,
bottom color=black!2,
path picture={
      \node[opacity=0.03,rotate=10] at ($(-3,-3)+(path picture bounding box.west)$) {
        \includegraphics[scale=1.2]{FondNegatifmoitie}};}
]
(current page.south west) rectangle (current page.north east);

% Rectangles
\path[
left color=Dandelion, 
right color=Dandelion!40,
imgmaths,
transform canvas ={rotate around ={45:($(current page.north west)+(0,-6)$)}}] 
($(current page.north west)+(0,-6)$) rectangle ++(9,1.5);

\path[
imgmaths,
left color=lightgray,
right color=lightgray!50,
rounded corners=0.75cm,
transform canvas ={rotate around ={45:($(current page.north west)+(.5,-10)$)}}]
($(current page.north west)+(0.5,-10)$) rectangle ++(15,1.5);

\shade[
left color=lightgray,
rounded corners=0.3cm,
transform canvas ={rotate around ={45:($(current page.north west)+(.5,-10)$)}}] ($(current page.north west)+(1.5,-9.55)$) rectangle ++(7,.6);

\shade[
imgmaths,
left color=orange!80,
right color=orange!60,
rounded corners=0.4cm,
transform canvas ={rotate around ={45:($(current page.north)+(-1.5,-3)$)}}]
($(current page.north)+(-1.5,-3)$) rectangle ++(9,0.8);

\shade[
imgmaths,
left color=red!80,
right color=red!80,
rounded corners=0.9cm,
transform canvas ={rotate around ={45:($(current page.north)+(-3,-8)$)}}] ($(current page.north)+(-3,-8)$) rectangle ++(15,1.8);

\shade[
imgmaths,
left color=orange,
right color=Dandelion,
rounded corners=0.9cm,
transform canvas ={rotate around ={45:($(current page.north west)+(4,-15.5)$)}}]
($(current page.north west)+(4,-15.5)$) rectangle ++(30,1.8);

\shade[
imgmaths,
left color=DodgerBlue,
right color=Emerald,
rounded corners=0.75cm,
transform canvas ={rotate around ={45:($(current page.north west)+(13,-10)$)}}]
($(current page.north west)+(13,-10)$) rectangle ++(15,1.5);

\shade[
imgmaths,
left color=lightgray,
rounded corners=0.3cm,
transform canvas ={rotate around ={45:($(current page.north west)+(18,-8)$)}}]
($(current page.north west)+(18,-8)$) rectangle ++(15,0.6);

\shade[
imgmaths,
left color=lightgray,
rounded corners=0.4cm,
transform canvas ={rotate around ={45:($(current page.north west)+(19,-5.65)$)}}]
($(current page.north west)+(19,-5.65)$) rectangle ++(15,0.8);

\shade[
imgmaths,
left color=DeepPink,
right color=red!80,
rounded corners=0.6cm,
transform canvas ={rotate around ={45:($(current page.north west)+(20,-9)$)}}] 
($(current page.north west)+(20,-9)$) rectangle ++(14,1.2);

\node[ultra thick,gray,
black!75,
inner sep=0pt,
anchor=east,
] (maths) at ($(current page.east)+(-1,0)$)
{
{\fontsize{30}{36} \selectfont \bfseries LGT Jean Rostand}
};

\node[ultra thick,gray,
orange,
inner sep=0pt,
anchor=north east,
] (annee) at ($(maths.south east)+(0,-0.4)$)
{
{\fontsize{25}{30}\selectfont\bfseries 1\textsuperscript{ère} ST2S}
};

\draw[ultra thick,gray] ($(annee.north west)+(0,0.2)$) -- ($(annee.north east)+(0,0.2)$);

% Title
\node[align=center] (titre) at ($(current page.center)+(0,-7)$) 
{
{\fontsize{55}{66} \selectfont\scshape\textcolor{black!90} {Mathématiques}} \\[10mm]
{\fontsize{18}{21.6} \selectfont \textcolor{orange}{ \bfseries {M. DELAUNEY}}}% \\[10mm]
%{\fontsize{14}{16.8} \selectfont \textcolor{black!75}{ \bfseries {2021 - 2022}}}
};
\end{tikzpicture}

%\vspace{16cm}
%\begin{center}
%{\fontsize{50}{60} \selectfont\scshape {Mathématiques}} \\[1cm]
%{\fontsize{16}{19.2} \selectfont \textcolor{orange}{ \bf {M. DELAUNEY}}}
%\end{center}
%
%\tableofcontents
%\selectfont\scshape\color[HTML]{912c21}{{

\end{document}
\setcounter{page}{1}
\addcontentsline{toc}{chapter}{Réels, intervalles}
\input{../Chap1 Réels Intervalles/cours}
\entre
%
\addcontentsline{toc}{chapter}{Vecteurs du plan: Première approche}
\input{../Chap2 Vecteurs/cours}
\entre
%
\addcontentsline{toc}{chapter}{Fonctions: Généralités}
\input{../Chap3 Fonctions Généralités/cours v3}
\entre
%
\addcontentsline{toc}{section}{Proportions, variations et pourcentages}
\input{../Chap4 Proportions Variations/cours}
\entre
%
\addcontentsline{toc}{section}{Fonctions affines}
\input{../Chap5 Fonctions affines/cours}
\entre
%
\addcontentsline{toc}{section}{Variations de fonctions}
\input{../Chap6 Variations de fonctions/cours}
\entre
%
\addcontentsline{toc}{section}{Repères du plan et coordonnées}
%\input{../Chap7 Repères du plan/cours}
\entre
\end{document}