\documentclass[a4paper,12pt,french]{article}

\usepackage[cours,eleve,NB]{../../../Style}

\geometry{margin=1cm,top=5mm,bottom=5mm}
% Début du document
%%%%%%%%%%%%%%%%%%%
\begin{document}

%\title{Repères du plan et coordonnées}
%\maketitle
\thispagestyle{empty}

%\section{Généralités sur les repères}

\begin{defin}
Soient $O,I,J$ trois points du plan non alignés. On pose $\vec i = \vec{OI}$ et $\vec j = \vec{OJ}$. Un repère du plan est un triplet \makebox[2cm]{\dotfill} On dit alors que :
\begin{itemize}
\item $O$ est \dotfill\hspace{6cm}\strut
\item $(OI)$ est \dotfill\hspace{6cm}\strut
\item $(OJ)$ est \dotfill\hspace{6cm}\strut
\end{itemize}
\end{defin}

\begin{propr}
Tout point $M$ du plan est repéré par un unique couple de coordonnées $(x;y)$. $x$ est \makebox[2cm]{\dotfill} de $M$ et $y$ est \makebox[2cm]{\dotfill} de $M$.
\end{propr}

\begin{exs}

\begin{tabularx}{\linewidth}{YYY}
\begin{resizetikz}{0.9\linewidth} % Repère quelconque à la main
\begin{tikzpicture}
% Cadre
\coordinate (liminf) at (-2,-2);
\coordinate (limsup) at (8,5);
\draw[line width=0.4pt,draw=black!30,densely dashed] (liminf) rectangle (limsup);
\clip (liminf)+(-0.01,-0.01) rectangle (limsup)+(0.01,0.01);

% Sous-image, changement de coordonnées
\begin{scope}[setupRepereTransformation={(2,20),(1,80)}]
% Initialisation des axes, de la grille
\draw[step=1,line width=0.4pt,draw=black!30,densely dashed] ($5*(liminf)+(-5,-5)$)+(-0.01,-0.01) grid ($5*(limsup)+(5,5)$)+(0.01,0.01);
\tkzInit[xmin=5*(-2),xmax=5*10,ymin=5*(-2),ymax=5*5]
\tkzDrawX[label={},tickup=3pt,tickdn=3pt]
\tkzDrawY[label={},ticklt=1.5pt,tickrt=1.5pt]

% Initialisation du repère
\node[label={-135:{O}}] (O) at (0,0) {};
\draw[->,>=latex,line width=1pt,cap=round,color=black!30!red] (0,0) -- (1,0) node[pos=0.5,below] {$\vec i$};
\draw[->,>=latex,line width=1pt,cap=round,color=black!30!red] (0,0) -- (0,1) node[pos=0.5,left] {$\vec j$};

%Contenu
\node[stylepoint,label=45:{$M$},fill=blue,semithick] (M) at (1,3) {}; 
\end{scope}
\end{tikzpicture}
\end{resizetikz}
&
\begin{tikzpicture}
\begin{axis}[
styleglobal,
repere,
width=0.45\linewidth,
xmin=-1,xmax=4,
ymin=-2,ymax=5,
x post scale=2
]
\node[label={-135:{O}}] (O) at (0,0) {};
\draw[->,>=latex,line width=1pt,cap=round,color=black!30!red] (0,0) -- (1,0) node[pos=0.5,below] {$\vec i$};
\draw[->,>=latex,line width=1pt,cap=round,color=black!30!red] (0,0) -- (0,1) node[pos=0.5,left] {$\vec j$};

% Contenu
\node[stylepoint,fill=blue,label={-5:$M$}] (A) at (2,-1) {};
\end{axis}
\end{tikzpicture}
&
\begin{resizetikz}{0.9\linewidth} % Repère quelconque à la main
\begin{tikzpicture}
\coordinate (liminf) at (-5,-2);
\coordinate (limsup) at (5,5);
\draw[line width=0.4pt,draw=black!30,densely dashed] (liminf) rectangle (limsup);
\clip (liminf)+(-0.01,-0.01) rectangle (limsup)+(0.01,0.01);
\begin{scope}[setupRepereTransformation={(1.5,-5),(1.5,70)}]
\draw[step=1,line width=0.4pt,draw=black!30,densely dashed]($5*(-2,-2)$)+(-5.01,-5.01) grid ($5*(8,5)$)+(5.01,5.01);
\tkzInit[xmin=5*(-2),xmax=5*10,ymin=5*(-2),ymax=5*5]
\tkzDrawX[label={},tickup=3pt,tickdn=3pt]
\tkzDrawY[label={},ticklt=2pt,tickrt=2pt]

\node[label={-135:{O}}] (O) at (0,0) {};
\draw[->,>=latex,line width=1pt,cap=round,color=black!30!red] (0,0) -- (1,0) node[pos=0.5,below] {$\vec i$};
\draw[->,>=latex,line width=1pt,cap=round,color=black!30!red] (0,0) -- (0,1) node[pos=0.5,left] {$\vec j$};
\node[stylepoint,label=45:{$M$},fill=blue,semithick] (M) at (-3,2) {}; 
\end{scope}
\end{tikzpicture}
\end{resizetikz}
\\
$M( \ldots ; \ldots )$&$\makebox[2cm]{\dotfill}$&$\makebox[2cm]{\dotfill}$
\end{tabularx}
\end{exs}

\vfill

\begin{defin}
Soient $O,I,J$ trois points du plan non alignés. On pose $\vec i = \vec{OI}$ et $\vec j = \vec{OJ}$. Un repère du plan est un triplet \makebox[2cm]{\dotfill} On dit alors que :
\begin{itemize}
\item $O$ est \dotfill\hspace{6cm}\strut
\item $(OI)$ est \dotfill\hspace{6cm}\strut
\item $(OJ)$ est \dotfill\hspace{6cm}\strut
\end{itemize}
\end{defin}

\begin{propr}
Tout point $M$ du plan est repéré par un unique couple de coordonnées $(x;y)$. $x$ est \makebox[2cm]{\dotfill} de $M$ et $y$ est \makebox[2cm]{\dotfill} de $M$.
\end{propr}

\begin{exs}

\begin{tabularx}{\linewidth}{YYY}
\begin{resizetikz}{0.9\linewidth} % Repère quelconque à la main
\begin{tikzpicture}
% Cadre
\coordinate (liminf) at (-2,-2);
\coordinate (limsup) at (8,5);
\draw[line width=0.4pt,draw=black!30,densely dashed] (liminf) rectangle (limsup);
\clip (liminf)+(-0.01,-0.01) rectangle (limsup)+(0.01,0.01);

% Sous-image, changement de coordonnées
\begin{scope}[setupRepereTransformation={(2,20),(1,80)}]
% Initialisation des axes, de la grille
\draw[step=1,line width=0.4pt,draw=black!30,densely dashed] ($5*(liminf)+(-5,-5)$)+(-0.01,-0.01) grid ($5*(limsup)+(5,5)$)+(0.01,0.01);
\tkzInit[xmin=5*(-2),xmax=5*10,ymin=5*(-2),ymax=5*5]
\tkzDrawX[label={},tickup=3pt,tickdn=3pt]
\tkzDrawY[label={},ticklt=1.5pt,tickrt=1.5pt]

% Initialisation du repère
\node[label={-135:{O}}] (O) at (0,0) {};
\draw[->,>=latex,line width=1pt,cap=round,color=black!30!red] (0,0) -- (1,0) node[pos=0.5,below] {$\vec i$};
\draw[->,>=latex,line width=1pt,cap=round,color=black!30!red] (0,0) -- (0,1) node[pos=0.5,left] {$\vec j$};

%Contenu
\node[stylepoint,label=45:{$M$},fill=blue,semithick] (M) at (1,3) {}; 
\end{scope}
\end{tikzpicture}
\end{resizetikz}
&
\begin{tikzpicture}
\begin{axis}[
styleglobal,
repere,
width=0.45\linewidth,
xmin=-1,xmax=4,
ymin=-2,ymax=5,
x post scale=2
]
\node[label={-135:{O}}] (O) at (0,0) {};
\draw[->,>=latex,line width=1pt,cap=round,color=black!30!red] (0,0) -- (1,0) node[pos=0.5,below] {$\vec i$};
\draw[->,>=latex,line width=1pt,cap=round,color=black!30!red] (0,0) -- (0,1) node[pos=0.5,left] {$\vec j$};

% Contenu
\node[stylepoint,fill=blue,label={-5:$M$}] (A) at (2,-1) {};
\end{axis}
\end{tikzpicture}
&
\begin{resizetikz}{0.9\linewidth} % Repère quelconque à la main
\begin{tikzpicture}
\coordinate (liminf) at (-5,-2);
\coordinate (limsup) at (5,5);
\draw[line width=0.4pt,draw=black!30,densely dashed] (liminf) rectangle (limsup);
\clip (liminf)+(-0.01,-0.01) rectangle (limsup)+(0.01,0.01);
\begin{scope}[setupRepereTransformation={(1.5,-5),(1.5,70)}]
\draw[step=1,line width=0.4pt,draw=black!30,densely dashed]($5*(-2,-2)$)+(-5.01,-5.01) grid ($5*(8,5)$)+(5.01,5.01);
\tkzInit[xmin=5*(-2),xmax=5*10,ymin=5*(-2),ymax=5*5]
\tkzDrawX[label={},tickup=3pt,tickdn=3pt]
\tkzDrawY[label={},ticklt=2pt,tickrt=2pt]

\node[label={-135:{O}}] (O) at (0,0) {};
\draw[->,>=latex,line width=1pt,cap=round,color=black!30!red] (0,0) -- (1,0) node[pos=0.5,below] {$\vec i$};
\draw[->,>=latex,line width=1pt,cap=round,color=black!30!red] (0,0) -- (0,1) node[pos=0.5,left] {$\vec j$};
\node[stylepoint,label=45:{$M$},fill=blue,semithick] (M) at (-3,2) {}; 
\end{scope}
\end{tikzpicture}
\end{resizetikz}
\\
$M( \ldots ; \ldots )$&$\makebox[2cm]{\dotfill}$&$\makebox[2cm]{\dotfill}$
\end{tabularx}
\end{exs}

\end{document}

\begin{defin}
\compo[0.75]
{
On dit que le repère \Oij est orthonormé si:
\begin{itemize}
\item Ses axes sont perpendiculaires: $(OI) \ldots (OJ)$
\item Les vecteurs unité $\vec i$ et $\vec j$ sont de même longueur: \makebox[2cm]{\dotfill}
\end{itemize}
}
{
\vspace{-2em}
\begin{tikzpicture}
\begin{axis}[
styleglobal,
repere,
width=0.9*\linewidth,
xmin=-2,xmax=5,
ymin=-2,ymax=3,
]
\node[label={-135:{O}}] (O) at (0,0) {};
\draw[->,>=latex,line width=1pt,cap=round,color=black!30!red] (0,0) -- (1,0) node[pos=0.5,below] {$\vec i$};
\draw[->,>=latex,line width=1pt,cap=round,color=black!30!red] (0,0) -- (0,1) node[pos=0.5,left] {$\vec j$};

% Contenu
\node[stylepoint,fill=blue,label=-60:M] (M) at (3,2) {};
\end{axis}
\end{tikzpicture}
\centering $\makebox[2cm]{\dotfill}$
}
\end{defin}

\end{document}