\documentclass[a4paper,12pt,french]{article}

\usepackage[TD]{../../Style}

\setlength{\columnsep}{2cm}
\setlength{\composep}{1cm}
\geometry{margin=1cm}

\pagestyle{empty}

%%% Contenu

% But: Travailler la formule des distances. Méthode libre (tests avec ou sans inconnue...) Tests directs (plutôt facile mais peut être long), Introduction d'une inconnue et tests, ou résolution d'équation.

\newcommand{\activitelecteurbase}{
\titre{Un lecteur éclairé?}
Dans la pièce rectangulaire de dix mètres par six représentée ci-dessous, un lampadaire est placé en $L$ et un fauteuil en $F$. Le lampadaire donne un éclairage satisfaisant pour la lecture dans un rayon de $3,5$ mètres. L'éclairage est-il suffisant si on lit un livre dans un fauteuil placé en $F$?

\begin{center}
\begin{resizetikz}{0.7\linewidth}
\begin{tikzpicture}
\node[stylepoint,fill=blue,label=above right:$F$] (F) at (7,1) {};
\node[stylepoint,fill=blue,label=below left:$L$] (L) at (9,4) {};
\draw[very thick] (0,0) rectangle (10,6);
%\draw[shift={(0,-0.3)},<->] (0,0) -- (10,0) node[below,pos=0.5]{10m};
%\draw[shift={(-0.3,0)},<->,overlay] (0,0) -- (0,6) node[left,pos=0.5]{6m};
\draw[semithick,<->] (0,1) -- (F.center) node[above,pos=0.5] {7m};
\draw[semithick,<->] (7,0) -- (F.center) node[right,pos=0.5] {1m};
\draw[semithick,<->] (9,6) -- (L.center) node[left,pos=0.5] {2m};
\draw[semithick,<->] (10,4) -- (L.center) node[below,pos=0.5] {1m};

%\coordinate[label={[overlay]-135:{\scriptsize O}}] (O) at (0,0) {};	%Origine
%\draw[stylevecteur,color=black!30!red] (0,0) -- (1,0) node[pos=0.5,below] {$\scriptstyle \vec i$};
%\draw[stylevecteur,color=black!30!red] (0,0) -- (0,1) node[overlay,pos=0.5,left] {$\scriptstyle \vec j$};
\end{tikzpicture}
\end{resizetikz}
\end{center}
}

\newcommand{\activitelecteur}{\compo{\activitelecteurbase}{\activitelecteurbase}}

\newcommand{\indicabase}{
\titre{Indication}
\centering On veut trouver l'ordonnée de T telle que $AT=AE$. Comment calculer la distance AE?
}

\newcommand{\indica}{\compo{\indicabase}{\indicabase}}

\newcommand{\indicbbase}{
\titre{Indication}
\centering Est-ce qu'une profondeur de 20m conviendrait?
}

\newcommand{\indicb}{\compo{\indicbbase}{\indicbbase}}

\newcommand{\indiccbase}{
\titre{Indication}
\centering Déterminer les coordonnées de $F$ et $L$.
}

\newcommand{\indicc}{\compo{\indiccbase}{\indiccbase}}

%\newcommand{\indicbbase}{
%\titre{Indication}
%\centering L'ordonnée de T est inconnue, mais on peut tout de même déterminer la longueur de $AT$ en fonction de cette ordonnée. Comment faire?
%}
%
%\newcommand{\indicb}{\compo{\indicbbase}{\indicbbase}}
%
%\newcommand{\indiccbase}{
%\titre{Indication}
%\centering Les coordonnées de T sont $(87;y)$ où $y$ est une inconnue. Déterminer $AT$ en fonction de $y$.
%}
%
%\newcommand{\indicc}{\compo{\indiccbase}{\indiccbase}}
%
%\newcommand{\indicdbase}{
%\titre{Indication}
%\centering En connaissant $AE$ et $AT$ en fonction de $y$, trouver la bonne valeur de $y$ qui permettrait d'avoir $AT=AE$. Est-ce que $y=-20$ convient?% Equation un peu difficile (carrés/racines) donc tests me vont. (L'aide est déjà orientée pour les élèves en difficulté donc autnat les aiguller sur les tests.)
%}
%
%\newcommand{\indicd}{\compo{\indicdbase}{\indicdbase}}
% Début du document
%%%%%%%%%%%%%%%%%%%
\begin{document}

\noindent\includegraphics[width=\linewidth]{Exo+.png} \vfill \noindent\includegraphics[width=\linewidth]{Exo+.png}

\newpage

\compo
{
\titre{Indication}

\centering Faire apparaitre le repère mentionné sur le schéma.
}
{
\titre{Indication}

\centering Faire apparaitre le repère mentionné sur le schéma.
}

\vfill

\compo
{
\titre{Indication}

\centering Quelles sont les coordonnées de $A$ et $E$?
}
{
\titre{Indication}

\centering Quelles sont les coordonnées de $A$ et $E$?
}

\vfill

\compo
{
\titre{Indication}

\centering Par exemple, les coordonnées de $B$ sont $(0;0)$.
}
{
\titre{Indication}

\centering Par exemple, les coordonnées de $B$ sont $(0;0)$.
}

\vfill

\compo
{
\titre{Indication}

\centering Les coordonnées de $E$ sont $(-25;-10)$. Quelles sont celles de $A$?
}
{
\titre{Indication}

\centering Les coordonnées de $E$ sont $(-25;-10)$. Quelles sont celles de $A$?
}

\vfill

\compo
{
\titre{Indication}

\centering Quelle est la distance entre $A$ et $E$?
}
{
\titre{Indication}

\centering Quelle est la distance entre $A$ et $E$?
}

\newpage

\compo
{
\titre{Indication}

\centering Formule de la distance entre $A$ et $E$: $AE=\sqrt{(x_E-x_A)^2+(y_E-y_A)^2}$
}
{
\titre{Indication}

\centering Formule de la distance entre $A$ et $E$: $AE=\sqrt{(x_E-x_A)^2+(y_E-y_A)^2}$
}

\vfill

\compo
{
\titre{Indication}

\centering Si le trésor se situait à 20m de profondeur, quelle serait la distance entre  $T$ et $A$? Cela conviendrait-il?
}
{
\titre{Indication}

\centering Si le trésor se situait à 20m de profondeur, quelle serait la distance entre  $T$ et $A$? Cela conviendrait-il?
}

\vfill

\compo
{
\titre{Indication}

\centering Et si le trésor se situait à 30m? 25m? ...
}
{
\titre{Indication}

\centering Et si le trésor se situait à 30m? 25m? ...
}

\vfill

\compo
{
\titre{Indication}

\centering Comment peut-on trouver la profondeur de $T$ de façon encore plus précise?
}
{
\titre{Indication}

\centering Comment peut-on trouver la profondeur de $T$ de façon encore plus précise?
}

\vfill

\compo
{
\titre{Indication}

\centering Déterminer la distance $AT$ en fonction de la profondeur de $T$.
}
{
\titre{Indication}

\centering Déterminer la distance $AT$ en fonction de la profondeur de $T$.
}

\end{document}

%\indica \vfill \indica \vfill \indica \vfill \indicb \vfill \indicb \vfill \indicb
%\vfill \indicd \vfill \indicd \vfill \indicd
\newpage
\activitelecteur \vfill \activitelecteur \vfill \activitelecteur
\newpage
\indicc \vfill \indicc \vfill \indicc \vfill \indicc \vfill \indicc \vfill \indicc%


\end{document}