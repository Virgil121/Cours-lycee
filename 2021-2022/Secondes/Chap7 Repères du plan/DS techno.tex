
\documentclass[a4paper,12pt,french] {article}

\usepackage[sujet]{../../Style}

\fancyhead[L]{16/05/2022}
\fancyhead[C]{\textbf{DS6 : Repères du plan et coordonnées}}
\fancyhead[R]{\seconde 12}

\begin{document}

\rem{L'usage de la calculatrice est interdit. La propreté et l'orthographe seront prises en compte. Tout le devoir peut être fait sur le sujet.}

Nom: \hfill Prénom: \hfill \

\setstretch{1.3}

\begin{exercice} \

\noindent Factoriser $A=x^2+12x+36$ \hfill Développer $B=(2x+7)(x+1)$ \hfill Résoudre $5x-7=3$
\vspace{5cm}

\end{exercice}

\vfill

\begin{exercice}
On donne dans un repère quelconque les points $A(0;2)$, $B(9;8)$ et $C(-6;-2)$.

\begin{enumerate}
\item Déterminer les coordonnées des vecteurs $\vv{AB}$ et $\vv{AC}$.

\points 4

\item On donne $\vv{BC} \dbinom {-15}{-10}$. Déterminer les coordonnées du vecteur $-3 \vv{BC}$.

\points 4

\item Les points $A$, $B$ et $C$ sont-ils alignés?

\points 4
\end{enumerate}
\end{exercice}

\begin{exercice} \

\compo[0.7]
{
\strut En physique, un système est équilibré lorsque la \textbf{\underline{somme}} des forces qui s'exercent sur lui est nulle.
\begin{enumerate}
\item Donner les coordonnées des vecteurs $\vv{F_1}$, $\vv{F_2}$ et $\vv{F_3}$ \textit{(ceux-ci représentent les forces en question)}:

\points 4

\item Le système représenté par les trois forces ci-contre est-il équilibré?

\points 4
\end{enumerate}
}
{
\begin{center}
%\vspace{-2.3cm}
\includegraphics[width=\linewidth, trim=0 0 0 -1cm]{Forces DS NB.png}
\end{center}
}
\end{exercice}

\begin{exercice} \

\compo[0.73]
{
\strut On se donne ci-contre un repère orthonormé \Oij et les points $A(-2;3)$, $B(4;1)$ et $C(3;-2)$.
\begin{enumerate}
\item Placer le triangle $ABC$ dans le repère donné.
\item Quelle est la nature du triangle $ABC$? On donne $AC=\sqrt{50}=5\sqrt 2$.
\end{enumerate}
}
{
\vspace{-11mm}
\begin{resizetikz}{\linewidth}
\begin{tikzpicture}
\repereOij[xmin=-3,xmax=5,ymin=-3,ymax=4];
\end{tikzpicture}
\end{resizetikz}
}

\

\points 8

\begin{enumerate}[start=3]
\item Soit $I$ le milieu du segment $[AC]$. Calculer ses coordonnées, puis placer ce point sur le repère.

\points 4
\end{enumerate}
\end{exercice}

\newpage

\begin{exercice}
On se donne un repère orthonormé et quatre points P, Q, R, S. On sait que:

\hfill
$P(-2;1)$ et $Q(6;2)$;
\hfill
$\vv{SR} \dbinom 8 1$;
\hfill
$\begin{cases} PS=\sqrt{65}=5\sqrt{13}\simeq 8,1 \\ QS=\sqrt{80}=4\sqrt 5\simeq 8,9 \end{cases}$
\hfill\strut

\noindent\strut Déterminer la nature \textbf{précise} du quadrilatère $PQRS$. \textit{Toute trace de recherche sera valorisée.}

\begin{center}
\begin{resizetikz}{0.8\linewidth}
\begin{tikzpicture}
\repereOij[xmin=-6,xmax=12,ymin=-7,ymax=3];
%\node[stylepoint,fill=blue,label=-135:$S$] (S) at (2,-6) {};
\end{tikzpicture}
\end{resizetikz}
\end{center}

\points {19}

\end{exercice}

\end{document}

\begin{exercice}
Compléter le tableau suivant, en cochant la case adaptée.

\begin{center}
\begin{tabularx}{\linewidth}{|X|c|c|} \hline
\centering Proposition & \makebox[2cm]{Vrai} & \makebox[2cm]{Faux} \\ \hline
Soient $\vec u \tbinom 2 0$ et $ \vv v \tbinom {-5} 0$. Alors $\vv u$ et $\vv v$ sont colinéaires. & & \\ \hline
Pour trouver la distance entre deux points $A(x_A;y_A)$ et $B(x_B;y_B)$, on calcule $\sqrt{(y_A-x_A)^2+(y_B-x_B)^2}$. & & \\ \hline
\end{tabularx}
\end{center}
\end{exercice}