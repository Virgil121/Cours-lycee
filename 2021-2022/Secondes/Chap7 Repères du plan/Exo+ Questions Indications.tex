\documentclass[a4paper,12pt,french]{article}

\usepackage[TD]{../../Style}

\setlength{\columnsep}{2cm}
\setlength{\composep}{1cm}
\geometry{margin=1cm}

\pagestyle{empty}

\newcommand{\contenua}{\begin{enumerate}[start=4]
\item Déterminer les coordonnées du vecteur $\vv{BD}$.
\item Déterminer les coordonnées de M, du milieu du segment $[GI]$.
\item Quelle est la distance entre M et A?
\end{enumerate}}
\newcommand{\contenu}{\compo{\contenua}{\contenua}}

%%% \vv

%%%%%%%%%%%%%%%%%%%
\begin{document}

\compo
{
\titre{Indication}

\centering Faire apparaitre le repère mentionné sur le schéma.
}
{
\titre{Indication}

\centering Faire apparaitre le repère mentionné sur le schéma.
}

\vfill

\compo
{
\titre{Indication}

\centering Faire apparaitre le repère mentionné sur le schéma.
}
{
\titre{Indication}

\centering Faire apparaitre le repère mentionné sur le schéma.
}

\vfill

\compo
{
\titre{Indication}

\centering Faire apparaitre le repère mentionné sur le schéma.
}
{
\titre{Indication}

\centering Faire apparaitre le repère mentionné sur le schéma.
}

\vfill

\compo
{
\titre{Indication}

\centering Les coordonnées de $B$ sont $(0;0)$ puisque c'est l'origine du repère.
}
{
\titre{Indication}

\centering Les coordonnées de $B$ sont $(0;0)$ puisque c'est l'origine du repère.
}

\vfill

\compo
{
\titre{Indication}

\centering Les coordonnées de $B$ sont $(0;0)$ puisque c'est l'origine du repère.
}
{
\titre{Indication}

\centering Les coordonnées de $B$ sont $(0;0)$ puisque c'est l'origine du repère.
}


\vfill

\compo
{
\titre{Indication}

\centering Les coordonnées de $B$ sont $(0;0)$ puisque c'est l'origine du repère.
}
{
\titre{Indication}

\centering Les coordonnées de $B$ sont $(0;0)$ puisque c'est l'origine du repère.
}

\newpage

\contenu

\vfill

\contenu

\vfill

\contenu

\vfill

\contenu

\vfill

\contenu

\vfill

\contenu

\vfill
\end{document}

%\indica \vfill \indica \vfill \indica \vfill \indicb \vfill \indicb \vfill \indicb
%\vfill \indicd \vfill \indicd \vfill \indicd
\newpage
\activitelecteur \vfill \activitelecteur \vfill \activitelecteur
\newpage
\indicc \vfill \indicc \vfill \indicc \vfill \indicc \vfill \indicc \vfill \indicc%


\end{document}