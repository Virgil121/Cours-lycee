\documentclass[a4paper,12pt,french]{article}

\usepackage{../../../Style}

\pagestyle{empty}

\geometry{margin=1cm}

\renewcommand{\composep}{1cm}

\newcommand{\contenua}{
\compo
{
Soient $A(6;1)$ et $B(4;0)$.Déterminer par le calcul les coordonnées du milieu $I$ du segment $[AB]$.
}
{
Soient $A(6;1)$ et $B(4;0)$.Déterminer par le calcul les coordonnées du milieu $I$ du segment $[AB]$.
}
}

\newcommand{\contenub}{
\compo
{
Déterminer par le calcul la norme de $\vec u \dbinom 8 6$.
}
{
Déterminer par le calcul la norme de $\vec u \dbinom 8 6$.
}
}
% Début du document
%%%%%%%%%%%%%%%%%%%

\begin{document}

\contenua \foreach \i in {2,...,8} {\vfill \contenua}%
%
\newpage
%
\contenub \foreach \i in {2,...,8} {\vfill \contenub}%
%
\newpage
%
On se place dans un repère orthonormé $\Oij$. Soient $A(3;2)$, $B(9;5)$, $C(7;1,5)$ et $D(5;5,5)$ des points du plan. Que dire des droites $(AB)$ et $(CD)$? % Mq losange ou médiatrice

%\vfill \contenuc \vfill \contenuc \vfill \contenuc \vfill \contenuc \vfill \contenuc \vfill
%
%\newpage
%
%\contenud \vfill \contenud \vfill \contenud \vfill \contenud \vfill \contenud
%
%\newpage
%
%\contenue \vfill \contenue \vfill \contenue
%
%\newpage
%
%\contenuf \vfill \contenuf \vfill \contenuf

%\contenua \contenub \contenuc \contenud \contenuee \contenuf

\end{document}
