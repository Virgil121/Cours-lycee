\documentclass[twocolumn,landscape,a4paper,12pt,french]{article}

\usepackage[TD]{../../Style}

\setlength{\columnsep}{2cm}
\geometry{margin=1cm}
% Début du document
%%%%%%%%%%%%%%%%%%%
\begin{document}

\newcommand{\contenu}{
\titre{Repérage de bateaux}

\compo[0.5]
{
Sur l'océan, trois bateaux sont initialement situés à des points A,B et C. On modélise une carte géographique par un repère (1 unité = 1 km sur chaque axe). De plus chaque axe représente l'un des points cardinaux Nord, Sud,Est, Ouest.
}
{
%\includegraphics[width=0.8\linewidth]{bateau.png}
%
\begin{center}
\begin{resizetikz}{\linewidth}
    \begin{tikzpicture}
    %Grille
	\draw[stylegrille] (-7,-3) grid (4,4);
	%Création des axes avec tkz-base
	\tkzInit[xmin=-7,xmax=4,ymin=-3,ymax=4]
	%On peut utiliser \tkzGrid pour la grille, mais la personnalisation est mal faite	
	%\tkzSetUpAxis[ticka=3.5pt,tickb=3.5pt]	%Taille graduations
	\tkzDrawX[label={},right space=0pt]	%Axe X,pas de nombres,pas de flèche
	\tkzDrawY[label={},up space=0pt]	%Axe Y
	% Origine, vecteurs unité
	%\coordinate[label={-135:{\scriptsize O}}] (O) at (0,0) {};	%Origine
	%\draw[stylevecteur,color=black!30!red] (0,0) -- (1,0) node[pos=0.5,below] {$\scriptstyle \vec i$};
	%\draw[stylevecteur,color=black!30!red] (0,0) -- (0,1) node[pos=0.5,left] {$\scriptstyle \vec j$};
	%\clip (\repere@xmin,\repere@ymin) rectangle (\repere@xmax,\repere@ymax);
    \node[stylepoint,cross out,label=above left:$B$] (B) at (-5,1) {};
    \node[stylepoint,cross out,label=below left:$A$] (A) at (-4,-1) {};
	\node[stylepoint,cross out,label=below right:$C$] (B) at (-1,-2) {};    
    \draw (0,0) node[below]{$0$};
    \draw (1,0) node[below]{$1$};
    \draw (3.5,0) node[below]{Est};
    \draw (-6.3,0) node[below]{Ouest};
    \draw (0,3.5) node[right]{Nord};
    \draw (0,-2.5) node[right]{Sud};
    \end{tikzpicture}
    \end{resizetikz}
\end{center}
}
\begin{enumerate}
    \item Placer les points $A,B,C$ sur géogébra.
    \item Le bateau A se déplace de $5$km vers l'est et de $2$km vers le nord. 
    \begin{enumerate}
        \item Placer sa nouvelle position $A'$.
        \item Quelles sont les coordonnées de $A'$?
    \end{enumerate}
    \item Tracer le vecteur $\vv{AA'}$. On pourra soit utiliser l'outil Vecteur, soit écrire dans la fenêtre de saisie "vecteur(A,A')". Comment est alors décrit ce vecteur dans la fenêtre algèbre?

Les nombres $5,2$ sont appelés les coordonnées du vecteur $\vv{AA'}$.
    \item 
        \begin{enumerate}
            \item Le bateau $B$ décrit le même déplacement que le bateau $A$. Placer la nouvelle position $B'$ du bateau $B$.
            \item Donner les coordonnées du vecteur $\vv{BB'}$.
            \item Que remarque-t-on par rapport au vecteur $\vv{AA'}$?
        \end{enumerate}
    \item Le bateau $C$ est plus lent. Il effectue un déplacement de vecteur $\vv{u}\binom{-1}{3}$
    \begin{enumerate}
        \item Décrire le déplacement de ce bateau en utilisant les points cardinaux (nord/sud ...).
        \item Placer le point $C'$.
        \item Exhiber une relation entre les coordonnées de $C$, de $C'$ et de $\vv u$. On pourra regarder les abscisses et les ordonnées séparément.
        \item Comment peut-on trouver les coordonnées du vecteur $\vec{CC'}$ à l'aide des points $C$ et $C'$?
    \end{enumerate}
    \item Pour la suite, rendez-vous sur capytale et entrer le code suivant: c1e9-403698.
\end{enumerate}}

\contenu

\newpage

\contenu

\end{document}