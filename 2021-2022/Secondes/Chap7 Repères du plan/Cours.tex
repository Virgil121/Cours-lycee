\documentclass[a4paper,12pt,french]{article}

\usepackage[cours]{../../../Style}

\pgfplotsset{
setupRepere/.style args={(#1,#2),(#3,#4)}{ % (Longueur x, angle x par rap à hori),(Longueur y,angle x->y)
x={(#1*cos(#2),#1*sin(#2))},
y={(#3*cos(#2+#4),#3*sin(#2+#4))},
},
reperesimple/.style={
overlay,
xticklabels={},
yticklabels={},
xlabel={},
ylabel={},
axis line style={-},
grid=none,
},
}

\def\defaultticksize{1pt}
\def\ticksize{\defaultticksize}
\def\xticksize{\ticksize}
\def\yticksize{\ticksize}

\tikzset{
stylepoint/.style={
color=black,circle,minimum size=1pt,fill,inner sep=2pt,fill opacity=1,
},
ytickscale/.style args={#1}{execute at begin scope={\def\yticksize{#1*\ticksize}},execute at end scope={\def\yticksize{\ticksize}}},
xtickscale/.style args={#1}{execute at begin scope={\def\xticksize{#1*\ticksize}},execute at end scope={\def\xticksize{\ticksize}}},
ticksize/.style args={#1}{execute at begin scope={\def\ticksize{#1}},execute at end scope={\def\xticksize{\defaultticksize}}},
setupRepere/.style args={(#1,#2),(#3,#4)}{
cm={#1*cos(#2),#1*sin(#2),#3*cos(#2+#4),#3*sin(#2+#4),(0,0)},
execute at begin scope={\pgfmathsetmacro\taillex{max(1,#1/#3)}\pgfmathsetmacro\tailley{max(1,#3/#1)}},
xtickscale=\taillex,
ytickscale=\tailley,
},
}

\usetikzlibrary{tikzmark}
\usepackage{tkz-base}
\usetikzlibrary{calc}

\makeatletter
\newsavebox{\measure@tikzpicture}
\NewEnviron{scaletikzpicturetowidth}[1]{%
  \def\tikz@width{#1}%
  \def\tikzscale{1}\begin{lrbox}{\measure@tikzpicture}%
  \BODY
  \end{lrbox}%
  \pgfmathparse{#1/\wd\measure@tikzpicture}%
  \edef\tikzscale{\pgfmathresult}%
  \BODY
}
\makeatother

\NewDocumentEnvironment{axispenche}{mmmmmb}]{
\draw[opacity=0.4,dashed] (#2,#3) rectangle (#4,#5);
\clip (#2,#3)+(-0.01,-0.01) rectangle (#4,#5)+(0.01,0.01);
\begin{scope}[#1]
\draw[step=0.5,line width=0.4pt,densely dashed,opacity=0.2] (5*#2-5,5*#3-5)+(-0.01,-0.01) grid (5*#4+5,5*#5+5)+(0.01,0.01);
\tkzInit[xmin=5*#2,xmax=5*#4,ymin=5*#3,ymax=5*#5]
\tkzDrawX[label={},tickup=\xticksize,tickdn=\xticksize]
\tkzDrawY[label={},ticklt=\yticksize,tickrt=\yticksize]
%\tkzGrid[gray,line width=0.4pt,style=densely dashed,sub,subxstep=0.5,subystep=0.5]
%\tkzGrid[gray,line width=0.4pt,style=densely dashed]
%\draw (10*#2,0) -- (10*#4,0);
%\draw (0,10*#3) -- (0,10*#5);
#6
\end{scope}
}{}

% Début du document
%%%%%%%%%%%%%%%%%%%
\begin{document}

\title{Repères du plan et coordonnées}
\maketitle

\begin{programme}
\item Base orthonormée. Coordonnées d'un vecteur. Expression de la norme d'un vecteur.
\item Expression des coordonnées de $\vec{AB}$ en fonction de celles de $A$ et $B$.
\item Déterminant de deux vecteurs dans une base orthonormée, critère de colinéarité. Application à l'alignement, au parallélisme.
\item Capacités:
\begin{itemize}
\item Représenter un vecteur dont on connait les coordonnées. Lire les coordonnées d'un vecteur.
\item Calculer les coordonnées d'une somme de vecteurs, d'un produit de vecteur par un réel.
\item Calculer la distance entre deux points. Calculer les coordonnées du milieu d'un segment.
\item Caractériser alignement et parallélisme par la colinéarité de vecteurs.
\item Résoudre des problèmes en utilisant la représentation la plus adaptée des vecteurs.
\end{itemize}
\item Démonstration: Colinéarité $\Leftrightarrow$ déterminant nul
\end{programme}

%\begin{FlushLeft}

\section{Généralités}

\begin{defin}
Soient $O,I,J$ trois points du plan non alignés. On pose $\vec i = \vec{OI}$ et $\vec j = \vec{OJ}$. Un repère du plan est un triplet $\left( O;\vec i; \vec j \right)$. On dit alors que :
\begin{itemize}
\item $O$ est l'origine du repère
\item $(OI)$ est l'axe des abscisses
\item $OJ$ est l'axe des ordonnées
\end{itemize}
\end{defin}

\begin{propr}
Tout point $M$ du plan est repéré par un unique couple de coordonnées $(x;y)$. $x$ est l'abscisse de $M$ et $y$ est l'ordonnée de $M$ 
\end{propr}


\compo[0.5]
{
\begin{tikzpicture}[scale=\echellepgf]
%\clip (Bord1) rectangle (Bord2);
\begin{axis}[
%clip=false
styleglobal,
reperesimple,
width=0.9*\echellepgfinv*\linewidth,
xmin=-2,xmax=5,
ymin=-2,ymax=5,
xtick distance=1,
ytick distance=1,
%axis line style={shorten >=-1cm, shorten <=-1cm},
%xslant=0.5,
setupRepere={(1,40),(1,90)},
]
\draw[->,>=latex,thick] (0,0) -- (1,0) node[pos=0.5,below] {$\vec i$};
\draw[->,>=latex,thick] (0,0) -- (0,1) node[pos=0.5,left] {$\vec j$};
\node[draw=none] (lim) at (\pgfkeysvalueof{/pgfplots/xmin},\pgfkeysvalueof{/pgfplots/ymin}) {};
\node[draw=none] (O) at (0,0) {};
\node[draw=none] (I) at (\echellepgf,0) {};
\node[draw=none] (J) at (0,\echellepgf) {};
\node[draw=none] (lima) at (\pgfkeysvalueof{/pgfplots/xmin},0) {};
\node[draw=none] (limb) at (\pgfkeysvalueof{/pgfplots/xmax},0) {};
\node[draw=none] (limc) at (0,\pgfkeysvalueof{/pgfplots/ymin}) {};
\node[draw=none] (limd) at (0,\pgfkeysvalueof{/pgfplots/ymax}) {};
%\addplot[styleplot] {cos(deg(x))};
\def\xmaxfirst{\pgfkeysvalueof{/pgfplots/xmax}}
\edef\xmax{\xmaxfirst}
\end{axis}
\draw[draw=none] let \p1=($(lim)+\echellepgf*(lima)-\echellepgf*(lim)$) in (\x1,\y1) -- (0,0);
\draw[draw=none] let \p2=($(lim)+\echellepgf*(limb)-\echellepgf*(lim)$) in (\x2,\y2) -- (0,0);
\draw[draw=none] let \p3=($(lim)+\echellepgf*(limc)-\echellepgf*(lim)$) in (\x3,\y3) -- (0,0);
\draw[draw=none] let \p4=($(lim)+\echellepgf*(limd)-\echellepgf*(lim)$) in (\x4,\y4) -- (0,0);
\path[remember picture] let
	\p1=($(lim)+\echellepgf*(lima)-\echellepgf*(lim)$),
	\p2=($(lim)+\echellepgf*(limb)-\echellepgf*(lim)$),
	\p3=($(lim)+\echellepgf*(limc)-\echellepgf*(lim)$),
	\p4=($(lim)+\echellepgf*(limd)-\echellepgf*(lim)$),
	in coordinate (Bord1) at ({min(\x1,\x2,\x3,\x4)},{min(\y1,\y2,\y3,\y4)});
\path[remember picture] let
	\p1=($(lim)+\echellepgf*(lima)-\echellepgf*(lim)$),
	\p2=($(lim)+\echellepgf*(limb)-\echellepgf*(lim)$),
	\p3=($(lim)+\echellepgf*(limc)-\echellepgf*(lim)$),
	\p4=($(lim)+\echellepgf*(limd)-\echellepgf*(lim)$),
	in coordinate (Bord2) at ({max(\x1,\x2,\x3,\x4)},{max(\y1,\y2,\y3,\y4)});
	
\draw (Bord1) rectangle (Bord2);
\clip (Bord1) rectangle (Bord2);

\draw[line width=0.4pt,shorten >=-10cm, shorten <=-10cm] ($(lim.center)+\echellepgf*(lima.center)-\echellepgf*(lim.center)$) -- ($(lim.center)+\echellepgf*(limb.center)-\echellepgf*(lim.center)$);

\draw[line width=0.4pt,shorten >=-10cm, shorten <=-10cm] ($(lim.center)+\echellepgf*(limc.center)-\echellepgf*(lim.center)$) -- ($(lim.center)+\echellepgf*(limd.center)-\echellepgf*(lim.center)$);
	
%Grille

%\draw (lim.center) -- ($(lim.center)+0.8*(lima.center)-0.8*(lim.center)$);
\foreach \y in {-14,-13,...,28}{
\draw[line width=0.4pt,densely dashed,opacity=0.2,shorten <= -10cm, shorten >= -10cm] ($(lim)+\y*0.5*(J.center)-\y*0.5*(O.center)$) -- ($(lim)+\y*0.5*(J.center)-\y*0.5*(O.center)+10*(I.center)-10*(O.center)$);
}
\foreach \x in {-14,-13,...,28}{
\draw[line width=0.4pt,densely dashed,opacity=0.2,shorten <= -10cm, shorten >= -10cm] ($(lim)+\x*0.5*(I.center)-\x*0.5*(O.center)$) -- ($(lim)+\x*0.5*(I.center)-\x*0.5*(O.center)+10*(J.center)-10*(O.center)$);
}
\end{tikzpicture}
}
{
\begin{scaletikzpicturetowidth}{0.9\linewidth}
\begin{tikzpicture}[scale=\tikzscale]
\begin{axispenche}{setupRepere={(2,20),(1,80)}}{-2}{-2}{10}{5} %ticksize=1pt
\node[label={-135:{$O$}}] (O) at (0,0) {};
\draw[->,>=latex,thick] (0,0) -- (1,0) node[pos=0.5,below] {$\vec i$};
\draw[->,>=latex,thick] (0,0) -- (0,1) node[pos=0.5,left] {$\vec j$};
\end{axispenche}
\end{tikzpicture}
\end{scaletikzpicturetowidth}
}
\end{document}
