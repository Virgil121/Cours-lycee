\documentclass[a4paper,12pt,french]{article}

\usepackage[cours]{../../../Style}

%
%\tcbset{
%	styleboitedeux/.style={
%		% Cadre
%		%% Pour cacher les bords
%			frame hidden,
%			boxrule=0pt,
%			boxsep=0pt,
%		enhanced, % Rajoute des possibilités, à laisser par défaut
%		sharp corners,	% Pas d'arrondis aux coins
%		breakable,	% Boites cassables (sauts de page possibles)
%		attach boxed title to top left, % Placement titre
%		boxed title style={ % Style titre (texte blanc par défaut)
%			boxrule=0pt,	% Pas de bordures
%			sharp corners,	% Coins non arrondis
%			boxsep=3pt,		% Bordures additionnelles
%			%height=20pt,	% Hauteur boite titre
%			%valign=center,	% Centrer verticalement titre
%		},
%		fonttitle=\bfseries,	% Titre en gras
%		% Marges intérieures
%		top=10pt,
%		right=10pt,
%		bottom=10pt,
%		left=10pt,
%		%
%	},
%}
%
%\newcommand{\redeftcbUn}[6][10]{
%\renewtcolorbox{#2}[1][]{
%	styleboitedeux,
%	colback=#2,
%	colbacktitle=#2-titre,
%	borderline west={1pt}{0pt}{#2-titre},
%	attach boxed title to top left={yshift=-21.6pt}, % décalage vers le bas du titre
%	title=\MakeUppercase{\rule[-3pt]{0pt}{14.3pt}#3},
%	before upper={\vspace{21.8pt} \colorlet{bg}{#2} \definecolor{bfcolor}{Hsb}{#4,0.75,0.15}},
%	after upper={\colorlet{bg}{white} \colorlet{bfcolor}{black}},
%	%before upper={\tcolorboxsubtitle{#1}{##1}},
%}
%}
%
%\newcommand{\redeftcb}[6][10]{ % Pour définir tous les types de thm en une ligne: voir options
%\redeftcbUn[#1]{#2}{#3}{#4}{#5}{#6}
%\redeftcbUn[#1]{#2s}{#3s}{#4}{#5}{#6}
%}
%
%	\redeftcb{defin}{Définition}{211}{0.8}{0.75}
%	\redeftcb{thm}{Théorème}{0}{0.8}{0.75}
%	\redeftcb{prop}{Proposition}{15}{0.8}{0.75}
%	\redeftcb[8]{ex}{Exemple}{120}{0.8}{0.58}
%	\redeftcb[8]{cex}{Contre-exemple}{120}{0.8}{0.58}
%	\redeftcb{rmq}{Remarque}{211}{0.8}{0.75}
%	\redeftcb{caspart}{Cas particulier}{211}{0.8}{0.75}
%	\redeftcb{cor}{Corollaire}{211}{0.8}{0.75}
%	\redeftcb{app}{Application}{211}{0.8}{0.75}
%	\redeftcb{propr}{Propriété}{211}{0.8}{0.75}
%	\redeftcb{methode}{Méthode}{211}{0.8}{0.75}
%	\redeftcb{exercice}{Exercice}{210}{0.6}{0.7}

% Début du document
%%%%%%%%%%%%%%%%%%%
\begin{document}

\title{Repères du plan et coordonnées}
\maketitle

\begin{programme}
\item Base orthonormée. Coordonnées d'un vecteur. Expression de la norme d'un vecteur.
\item Expression des coordonnées de $\vec{AB}$ en fonction de celles de $A$ et $B$.
\item Déterminant de deux vecteurs dans une base orthonormée, critère de colinéarité. Application à l'alignement, au parallélisme.
\item Capacités:
\begin{itemize}
\item Représenter un vecteur dont on connait les coordonnées. Lire les coordonnées d'un vecteur.
\item Calculer les coordonnées d'une somme de vecteurs, d'un produit de vecteur par un réel.
\item Calculer la distance entre deux points. Calculer les coordonnées du milieu d'un segment.
\item Caractériser alignement et parallélisme par la colinéarité de vecteurs.
\item Résoudre des problèmes en utilisant la représentation la plus adaptée des vecteurs.
\end{itemize}
\item Démonstration: Colinéarité $\Leftrightarrow$ déterminant nul
\end{programme}

\section{Généralités sur les repères}

\begin{defin}
Soient $O,I,J$ trois points du plan non alignés. On pose $\vec i = \vec{OI}$ et $\vec j = \vec{OJ}$. Un repère du plan est un triplet \Oij. On dit alors que :
\begin{itemize}
\item $O$ est l'origine du repère
\item $(OI)$ est l'axe des abscisses
\item $(OJ)$ est l'axe des ordonnées
\end{itemize}
\end{defin}

\begin{fait}
On se donne maintenant un repère \Oij pour la suite du cours.
\end{fait}

\begin{propr}
Tout point $M$ du plan est repéré par un unique couple de coordonnées $(x;y)$. $x$ est l'abscisse de $M$ et $y$ est l'ordonnée de $M$.
\end{propr}

\begin{exs}

\begin{tabularx}{\linewidth}{YYY}
\begin{resizetikz}{0.9\linewidth} % Repère quelconque à la main
\begin{tikzpicture}
% Cadre
\coordinate (liminf) at (-2,-2);
\coordinate (limsup) at (8,5);
\draw[stylegrille] (liminf) rectangle (limsup);
\clip (liminf)+(-0.01,-0.01) rectangle (limsup)+(0.01,0.01);

% Sous-image, changement de coordonnées
\begin{scope}[setupRepereTransformation={(2,20),(1,80)}]
% Initialisation des axes, de la grille
\draw[stylegrille] ($5*(liminf)+(-5,-5)$)+(-0.01,-0.01) grid ($5*(limsup)+(5,5)$)+(0.01,0.01);
\tkzInit[xmin=5*(-2),xmax=5*10,ymin=5*(-2),ymax=5*5]
\tkzDrawX[label={},tickup=3pt,tickdn=3pt,thin]
\tkzDrawY[label={},ticklt=1.5pt,tickrt=1.5pt,thin]

% Initialisation du repère
\node[label={[anchor=60,label distance=-3pt]-120:{\scriptsize O}}] (O) at (0,0) {};
\draw[stylevecteur,color=black!30!red] (0,0) -- (1,0) node[pos=0.5,below] {$\scriptstyle \vec i$};
\draw[stylevecteur,color=black!30!red] (0,0) -- (0,1) node[pos=0.5,left] {$\scriptstyle \vec j$};

%Contenu
\node[stylepoint,label=above right:{$M$},fill=blue,semithick] (M) at (1,3) {}; 
\end{scope}
\end{tikzpicture}
\end{resizetikz}
&
\begin{resizetikz}{0.9\linewidth} % Repère quelconque à la main
\begin{tikzpicture}
% Cadre
\coordinate (liminf) at (-2,-2);
\coordinate (limsup) at (8,5);
\clip (liminf)+(-0.01,-0.01) rectangle (limsup)+(0.01,0.01);

% Sous-image, changement de coordonnées
\begin{scope}[setupRepereTransformation={(2,0),(1,90)}]
% Initialisation des axes, de la grille
\draw[stylegrille] ($5*(liminf)+(-5,-5)$)+(-0.01,-0.01) grid ($5*(limsup)+(5,5)$)+(0.01,0.01);
\tkzInit[xmin=5*(-2),xmax=5*10,ymin=5*(-2),ymax=5*5]
\tkzDrawX[label={},tickup=3pt,tickdn=3pt,thin]
\tkzDrawY[label={},ticklt=1.5pt,tickrt=1.5pt,thin]

% Initialisation du repère
\node[label={[label distance=-3pt]-135:{\scriptsize O}}] (O) at (0,0) {};
\draw[stylevecteur,color=black!30!red] (0,0) -- (1,0) node[pos=0.5,below] {$\scriptstyle \vec i$};
\draw[stylevecteur,color=black!30!red] (0,0) -- (0,1) node[pos=0.5,left] {$\scriptstyle \vec j$};

%Contenu
\node[stylepoint,fill=blue,label={[anchor=145]-35:$M$}] (A) at (2,-1) {};
\end{scope}
\end{tikzpicture}
\end{resizetikz}
&
\begin{resizetikz}{0.9\linewidth} % Repère quelconque à la main
\begin{tikzpicture}
\coordinate (liminf) at (-5,-2);
\coordinate (limsup) at (5,5);
\draw[stylegrille] (liminf) rectangle (limsup);
\clip (liminf)+(-0.01,-0.01) rectangle (limsup)+(0.01,0.01);
\begin{scope}[setupRepereTransformation={(1.5,-5),(1.5,70)}]
\draw[stylegrille]($5*(-2,-2)$)+(-5.01,-5.01) grid ($5*(8,5)$)+(5.01,5.01);
\tkzInit[xmin=5*(-2),xmax=5*10,ymin=5*(-2),ymax=5*5]
\tkzDrawX[label={},tickup=3pt,tickdn=3pt,thin]
\tkzDrawY[label={},ticklt=2pt,tickrt=2pt,thin]

\node[label={[anchor=30,label distance=-3pt]-135:{\scriptsize O}}] (O) at (0,0) {};
\draw[stylevecteur,color=black!30!red] (0,0) -- (1,0) node[pos=0.5,below] {$\scriptstyle \vec i$};
\draw[stylevecteur,color=black!30!red] (0,0) -- (0,1) node[pos=0.5,left] {$\scriptstyle \vec j$};
\node[stylepoint,label=above right:{$M$},fill=blue,semithick] (M) at (-3,2) {}; 
\end{scope}
\end{tikzpicture}
\end{resizetikz}
\\
$M(1;3)$&$M(2;-1)$&$M(-3;2)$
\end{tabularx}
\end{exs}

\rem[eleve]{On écrit $M(1;3)$ et pas $M=(1;3)$!}

\begin{defin}
\compo[0.75]
{
On dit que le repère \Oij est orthonormé si:
\begin{itemize}
\item Ses axes sont perpendiculaires: $(OI) \perp (OJ)$
\item Les vecteurs unité $\vec i$ et $\vec j$ ont pour longueur 1:  $\Vert \vec i \Vert = \Vert \vec j \Vert=1$
\end{itemize}
}
{
\vspace{-2em}
\begin{center}
\begin{resizetikz}{0.9\linewidth}
\begin{tikzpicture}
\repereOij[xmin=-2,xmax=5,ymin=-2,ymax=3];
\node[stylepoint,fill=blue,label=-60:M] (M) at (3,2) {};
\end{tikzpicture}
\end{resizetikz}

\centering $M(3;2)$
\end{center}
}
\end{defin}

\section{Coordonnées d'un vecteur}

\begin{defin}
\compo[0.75]
{
Soit $\vec u$ un vecteur du plan. On se donne le point $M(x;y)$ tel que $\vec{OM} = \vec u$. Les coordonnées de $\vec u$ sont celles de $M$, et l'on note $\vec u \dbinom x y$.
}
{\vspace{-2em}
\begin{center}
\begin{resizetikz}{0.9\linewidth}
\begin{tikzpicture}
\repereOij[xmin=-2,xmax=5,ymin=-2,ymax=3,];

\node[stylepoint,fill=blue,label=-60:M] (M) at (3,2) {};
\draw[stylevecteur] (O.center) -- (M.center) node[pos=0.5,above] {$\vv u$};
\end{tikzpicture}
\end{resizetikz}

\centering $M(3;2)$ donc $\vv u \binom 3 2$
\end{center}

}
\end{defin}

\rem[eleve]{On écrit $\vec u \binom 3 2$ et pas $\vec u = \binom 3 2$ !}

\begin{ex}
\compo[0.6]
{
On a représenté ci-contre deux vecteurs $\vec u$ et $\vec v$.

On a $\vec u \dbinom 5 {-2}$, et $\vec v \dbinom {-2} {0}$.
}
{\vspace{-1.5em}
\begin{center}
\begin{resizetikz}{0.9\linewidth}
\begin{tikzpicture}
\repereOij[xmin=-5,xmax=3,ymin=-1,ymax=4];

\draw[stylevecteur] (-3,3) -- (2,1) node[pos=0.5,above] {$\vec u$};
\draw[stylevecteur] (-2,1) -- (-4,1) node[pos=0.5,above] {$\vec v$};
\end{tikzpicture}
\end{resizetikz}
\end{center}
}
\end{ex}

\rem{Exo 1}

\begin{prop}
Soient $A(x_A;y_A)$ et $B(x_B;y_B)$ deux points du plan. Alors on a $\vec{AB} \dbinom {x_B-x_A} {y_B-y_A}$.
\end{prop}

\begin{ex}
\compo[0.6]
{
\setstretch{2.5}
On se donne $A(1;2)$, $B(5;-1)$ et $C(6,3)$.

Alors on a $\vec{AB} \dbinom {5-1}{-1-2}$ d'où $\vec{AB} \dbinom {4} {-3}$.

De même, on a $\vec{BC} \dbinom {6-5}{3-(-1)}$ d'où $\vec{BC} \dbinom 1 4$.
}
{\vspace{-1.5em}
\begin{center}
\begin{resizetikz}{0.9\linewidth}
\begin{tikzpicture}
\repereOij[xmin=-2,xmax=8,ymin=-2,ymax=5];

\node[stylepoint,fill=blue,label=-60:B] (B) at (5,-1) {};
\node[stylepoint,fill=blue,label=120:A] (A) at (1,2) {};
\node[stylepoint,fill=blue,label=60:C] (C) at (6,3) {};
\draw[stylevecteur] (A.center) -- (B.center);
\draw[stylevecteur] (B.center) -- (C.center);
\end{tikzpicture}
\end{resizetikz}
\end{center}
}
\end{ex}

\rem{Exos 2,3,4 (par deux),(5),6}

\begin{proprs}
Soient $\vec u \dbinom {x_{\vec u}} {y_{\vec u}}$ et $\vec v \dbinom {x_{\vec v}} {y_{\vec v}}$ deux vecteurs du plan, et $\lambda \in \R$. Alors:
\begin{itemize}
\item Les coordonnées de $\vec u + \vec v$ sont $\dbinom {x_{\vec u}+x_{\vec v}} {y_{\vec u}+y_{\vec v}}$.
\item Les coordonnées de $\lambda \vec u$ sont $\dbinom {\lambda x_{\vec u}} {\lambda y_{\vec u}}$.
\end{itemize}
\end{proprs}
\begin{ex}
Soient $\vec u \dbinom 1 3$ et $\vec v \dbinom 2 {-1}$. Alors on a:
\begin{itemize}
\item $\left( \vec u + \vec v  \right) \dbinom {1+2} {3+(-1)}$ d'où $\left( \vec u + \vec v  \right) \dbinom 3 2$.
\item $3\vec v \dbinom {3 \times 2} {3 \times (-1)}$ d'où $3\vec v \dbinom {6} {-3}$.
\item $\left( 2 \vec u - 5\vec v \right) \dbinom {2 \times 1 -5 \times 2} {2 \times 3 -5 \times (-1)}$ d'où $\left( 2 \vec u - 5\vec v \right) \dbinom {-8}{11}$.
\end{itemize}
\end{ex}

\rem{Exo 7,(8)}

\section{Calculs de distances et de milieux}

\subsection{Milieu d'un segment}

\begin{prop}
Soient $A(x_A;y_A)$ et $B(x_B;y_B)$ deux points du plan. On se donne de plus $I$ le milieu de $[AB]$. Alors les coordonnées de $I$ sont $\left( \dfrac{x_A+x_B} 2 ; \dfrac{y_A+y_B} 2 \right)$.
\end{prop}

\begin{rmq}
On fait en fait la moyenne des coordonnées des deux points.
\end{rmq}

\begin{ex}
Soient $A(1;7)$, $B (6;-5)$ et $I$ le milieu de $[AB]$. Alors on a $I\left( \frac{1+6} 2;\frac {7-5} 2 \right)$ d'où $I\left( \frac 7 2 ; \frac 2 2 \right)$, et enfin $I(3,5 ; 1)$.
\end{ex}

\rem{Exo 9,(14,15)}

\subsection{Normes et distances}

\begin{fait}
Dans toute cette partie, on se place dans un repère orthonormé.
\end{fait}

\begin{prop}
Soit $\vec u \dbinom x y$ un vecteur du plan. Alors la norme de $\vec u$ est donnée par:
$$\Vert \vec u \Vert = \sqrt{x^2+y^2}$$
\end{prop}

\begin{cor}
Soient $A(x_A;y_A)$ et $B(x_B;y_B)$ deux points du plan. Alors la distance entre $A$ et $B$ vaut:
$$AB=\sqrt{(x_B-x_A)^2+(y_B-y_A)^2}$$
\end{cor}

\begin{ex}
Soient $A (1;7)$ et $B (6;-5)$.
Alors $\begin{aligned}[t]AB&=\sqrt{(6-1)^2+(-5-7)^2}\\
							&=\sqrt{5^2+(-12)^2}\\
							&=\sqrt{25+144}\\
							&=\sqrt{169}\\
							&=13 \end{aligned}$
\end{ex}

\begin{apps}
On peut alors montrer l'appartenance d'un point à un cercle, ou déterminer la nature d'un polygone en utilisant des coordonnées.
\end{apps}

\begin{ex}
Soit $\mathcal C$ le cercle de centre $A(3;5)$ de rayon 3, et le point $M(5;7)$. On a:
$$\begin{aligned}[t] AM &= \sqrt{(5-3)^2+(7-5)^2}\\
&=\sqrt{2^2+2^2}\\
&=\sqrt{4+4}\\
&=\sqrt 8 \simeq 2,83\end{aligned}$$
Comme $AM \neq 3$, $M$ n'appartient pas au cercle $\mathcal C$.
\end{ex}

\rem{Exos 10,11,12,(13)}

\section{Colinéarité}
\subsection{Définitions et caractérisations}

\begin{defin}
Deux vecteurs $\vec u$ et $\vec v$ sont dits colinéaires s'il existe $\lambda \in \R$ tel que $\vec u = \lambda \vec v$. Dans le cas où ils sont non nuls, cela revient à dire qu'ils ont la même direction.
\end{defin}

\begin{rmq}
Deux vecteurs sont colinéaire si et seulement si leurs coordonnées sont proportionnelles.
\end{rmq}

\begin{exs}
On se donne $\vec u \dbinom 2 5$ et $\vec v \dbinom 4 {10}$. On remarque que $\vec v = 2 \vec u$. Alors $\vec u$ et $\vec v$ sont colinéaires.

\begin{center}
\begin{tabularx}{0.95\linewidth}{|c|c|Y|} \hline
	Coordonnées de $\vv u$ & Coordonnées de $\vv v$ & $\vv u$ et $\vv v$ sont-ils colinéaires? \\ \hline
	$\dbinom 2 5$ & $\dbinom 4 {10}$ & $\vv v = 2 \vv u$ donc oui \\ \hline
	$\dbinom 1 1$ & $\dbinom 7 7$ & $\vv v = 7 \vv u$ donc oui \\ \hline
	$\dbinom 2 5$ & $\dbinom {-6} {-15}$ & $\vv v = -3 \vv u$ donc oui \\ \hline
	$\dbinom 4 {11}$ & $\dbinom {-7} {-20}$ & $\left.\begin{matrix} 4 \times (-20) = -80 \\ -7 \times 11 = -77 \end{matrix}\right\}$ et $-80 \neq -77$ donc non \\ \hline
\end{tabularx}
\end{center}

%\item On prend maintenant $\vv a \dbinom 4 {11}$ et $\vv b \dbinom {-7} {-20}$. On a $11 \times (-7) = -77$ et $4 \times (-20) = -80$. Comme $-77 \neq -80$, les coordonnées de $\vv a$ et $\vv b$ ne sont pas proportionnelles donc $\vv a$ et $\vv b$ ne sont pas colinéaires.
\end{exs}

\subsection{Déterminant de deux vecteurs}

\begin{defin}
Soient $\vec u \dbinom {x} {y}$ et $\vec v \dbinom {x'} {y'}$ deux vecteurs du plan. On appelle déterminant de $\vec u$ et $\vec v$ le réel: $$\det(\vec u;\vec v)=xy'-x'y$$
\end{defin}

\begin{ex}
Soient $\vec u \dbinom {4}{11}$ et $\vec v \dbinom {-7}{-20}$.
On a $\begin{aligned}[t] \det (\vec u ; \vec v) &= 4 \times (-20) - (-7) \times 11\\&= -80 +77 \\&=-3\end{aligned}$
\end{ex}

\rem{Exo 16}

\begin{prop}
Deux vecteurs sont colinéaires si et seulement si leur déterminant est nul.
\end{prop}

\begin{ex}
Les vecteurs $\vv u$ et $\vv v$ de l'exemple précédent ne sont donc pas colinéaires.

Soient $\vec a \dbinom {-15}{40}$ et $\vec b \dbinom {9}{-24}$.
On a $\begin{aligned}[t] \det (\vec a ; \vec b) &= -15 \times (-24) -9 \times 40\\&= 360 - 360\\&=0\end{aligned}$

Les vecteurs $\vec a$ et $\vec b$ sont donc colinéaires. (On a en fait $\vec u = -\frac 5 3 \vec v$).
\end{ex}

\rem{Exos 17,18}

\subsection{Applications en géométrie}

\begin{proprs}
\begin{itemize}
\item Deux droites $(AB)$ et $(CD)$ sont parallèles si et seulement si les vecteurs $\vv{AB}$ et $\vv{CD}$ sont colinéaires.
\item Trois points $A$,$B$,$C$ sont alignés si et seulement si les vecteurs $\vv{AB}$ et $\vv{AC}$ sont colinéaires.
\end{itemize}
\end{proprs}

\begin{ex}
\compo[0.6]
{\setstretch{2.5}
On se donne les points $A(2;4)$, $B(11;10)$ et $C(-4;0)$.

Alors on a $\vec{AB} \dbinom 9 6$ et $\vec{AC} \dbinom {-6}{-4}$.

On a $\det (\vec{AB};\vec{AC})=9 \times (-4) - (-6) \times 6 = -36+36=0$.
}
{
\begin{center}
\begin{resizetikz}{0.9\linewidth}
\begin{tikzpicture}
\repereOij[xmin=-5,xmax=12,ymin=-2,ymax=11];

\node[stylepoint,fill=blue,label=-90:B] (B) at (11,10) {};
\node[stylepoint,fill=blue,label=120:A] (A) at (2,4) {};
\node[stylepoint,fill=blue,label=-45:C] (C) at (-4,0) {};
\draw[stylevecteur] (A) -- (B.center);
\draw[stylevecteur] (A) -- (C.center);
\end{tikzpicture}
\end{resizetikz}
\end{center}
}

On en déduit que les vecteurs $\vec{AB}$ et $\vec{AC}$ sont colinéaires, d'où $(AB) \parallel (AC)$, et les points $A$, $B$ et $C$ sont alignés.
\end{ex}

\rem{Exos 19,20}

\end{document}

%\begin{tikzpicture}
%\begin{axis}[
%styleglobal,
%repere,
%width=0.45\linewidth,
%xmin=-1,xmax=4,
%ymin=-2,ymax=5,
%x post scale=2
%]
%\node[label={[label distance=-3pt]-135:{\scriptsize O}}] (O) at (0,0) {};
%\draw[stylevecteur,color=black!30!red] (0,0) -- (1,0) node[pos=0.5,below] {$\scriptstyle \vec i$};
%\draw[stylevecteur,color=black!30!red] (0,0) -- (0,1) node[pos=0.5,left] {$\scriptstyle \vec j$};
%
%% Contenu
%\node[stylepoint,fill=blue,label={[anchor=145]-35:$M$}] (A) at (2,-1) {};
%\end{axis}
%\end{tikzpicture}

%\NewDocumentEnvironment{axispenche}{mmmmmb}]{
%\draw[line width=0.4pt,draw=black!30,densely dashed] (#2,#3) rectangle (#4,#5);
%\clip (#2,#3)+(-0.01,-0.01) rectangle (#4,#5)+(0.01,0.01);
%\begin{scope}[#1]
%\draw[step=1,line width=0.4pt,draw=black!30,densely dashed] (5*#2-5,5*#3-5)+(-0.01,-0.01) grid (5*#4+5,5*#5+5)+(0.01,0.01);
%\tkzInit[xmin=5*#2,xmax=5*#4,ymin=5*#3,ymax=5*#5]
%\tkzDrawX[label={},tickup=\xticksize,tickdn=\xticksize]
%\tkzDrawY[label={},ticklt=\yticksize,tickrt=\yticksize]
%%\tkzGrid[gray,line width=0.4pt,style=densely dashed,sub,subxstep=0.5,subystep=0.5]
%%\tkzGrid[gray,line width=0.4pt,style=densely dashed]
%%\draw (10*#2,0) -- (10*#4,0);
%%\draw (0,10*#3) -- (0,10*#5);
%#6
%\end{scope}
%}{}

% Quelques tests
%
%\def\defaultticksize{2pt}
%\def\ticksize{\defaultticksize}
%\def\xticksize{\ticksize}
%\def\yticksize{\ticksize}
%
%\tikzset{
%ytickscale/.style args={#1}{execute at begin scope={\def\yticksize{#1*\ticksize}},execute at end scope={\def\yticksize{\ticksize}}},
%xtickscale/.style args={#1}{execute at begin scope={\def\xticksize{#1*\ticksize}},execute at end scope={\def\xticksize{\ticksize}}},
%ticksize/.style args={#1}{execute at begin scope={\def\ticksize{#1}},execute at end scope={\def\xticksize{\defaultticksize}}},
%setupRepere/.style args={(#1,#2),(#3,#4)}{
%cm={#1*cos(#2),#1*sin(#2),#3*cos(#2+#4),#3*sin(#2+#4),(0,0)},
%%execute at begin scope={\pgfmathsetmacro\taillex{max(1,#1/#3)}\pgfmathsetmacro\tailley{max(1,#3/#1)}},
%xtickscale=1/#3,
%ytickscale=1/#1,
%},
%}

%\tikzstyle{every label}=[font=\large];

%\begin{scaletikzpicturetowidth}{0.9\linewidth}
%\begin{tikzpicture}[scale=\tikzscale]
%\coordinate (liminf) at (-2,-2);
%\coordinate (limsup) at (8,5);
%%\draw[line width=0.4pt,draw=black!30,densely dashed] (liminf) rectangle (limsup);
%\clip (liminf)+(-0.01,-0.01) rectangle (limsup)+(0.01,0.01);
%\begin{scope}[scale=1]
%\draw[step=1,line width=0.4pt,draw=black!30,densely dashed] ($5*(liminf)+(-5,-5)$)+(-0.01,-0.01) grid ($5*(limsup)+(5,5)$)+(0.01,0.01);
%\tkzInit[xmin=5*(-2),xmax=5*10,ymin=5*(-2),ymax=5*5]
%\tkzDrawX[label={},tickup=3pt,tickdn=3pt]
%\tkzDrawY[label={},ticklt=3pt,tickrt=3pt]
%
%\node[label={-135:{O}}] (O) at (0,0) {};
%\draw[->,>=latex,thick] (0,0) -- (1,0) node[pos=0.5,below] {$\vec i$};
%\draw[->,>=latex,thick] (0,0) -- (0,1) node[pos=0.5,left] {$\vec j$};
%\node[stylepoint,label=-5:{$M$},fill=blue,semithick] (M) at (5,-1) {}; 
%\end{scope}
%\end{tikzpicture}
%\end{scaletikzpicturetowidth}

%\begin{scaletikzpicturetowidth}{0.5\linewidth}
%\begin{tikzpicture}[scale=1]
%% Limites et grille
%\node[draw=none] (liminf) at (-2,-2) {};
%\node[draw=none] (limsup) at (13,11) {};
%\draw[gray] (liminf) rectangle (limsup);
%\clip (liminf) rectangle (limsup);
%\draw[step=1,line width=0.5pt,densely dashed,gray,opacity=0.5] (liminf)+(-0.01,-0.01) grid (limsup)+(0.01,0.01);
%
%% Repère
%\tkzInit[xmin=5*(-2),xmax=5*10,ymin=5*(-2),ymax=5*5]
%\tkzDrawX[label={},tickup=3pt,tickdn=3pt]
%\tkzDrawY[label={},ticklt=2pt,tickrt=2pt]
%\node[label={-135:{O}}] (O) at (0,0) {};
%\draw[->,>=latex,thick] (0,0) -- (1,0) node[pos=0.5,below] {$\vec i$};
%\draw[->,>=latex,thick] (0,0) -- (0,1) node[pos=0.5,left] {$\vec j$};
%
%% Contenu
%\node[stylepoint,fill=blue,label={180:\textbf{\large A}}] (A) at (3,6) {};
%\node[stylepoint,fill=blue,label={180:\textbf{\large B}}] (B) at (6,7) {};
%\node[stylepoint,fill=blue,label={-2:\textbf{\large C}}] (C) at (7,-1) {};
%\node[stylepoint,fill=blue,label={2:\textbf{\large D}}] (D) at (2,1) {};
%\draw[line width=1pt,->,>=latex] (9,-8) -- ++(2,3) node[pos=0.5,above left] {$\vec u$};
%%\node[color=black,circle,minimum size=1pt,fill,inner sep=2pt,fill opacity=1,label={90:\textbf{\large E}}] (E) at (4,-4) {};
%%\draw[line width=1pt, shorten <= -10cm, shorten >= -10cm] (A) -- (B);
%\end{tikzpicture}
%\end{scaletikzpicturetowidth}

% Pour aller plus vite:

%\begin{scaletikzpicturetowidth}{0.9\linewidth}
%\begin{tikzpicture}[scale=\tikzscale]
%\begin{axispenche}{setupRepere={(2,20),(1,80)}}{-2}{-2}{10}{5} %ticksize=1pt
%\node[label={-135:{O}}] (O) at (0,0) {};
%\draw[->,>=latex,thick] (0,0) -- (1,0) node[pos=0.5,below] {$\vec i$};
%\draw[->,>=latex,thick] (0,0) -- (0,1) node[pos=0.5,left] {$\vec j$};
%\end{axispenche}
%\end{tikzpicture}
%\end{scaletikzpicturetowidth}

%$$\setlength{\fboxsep}{2\fboxsep}\boxed{...}$$