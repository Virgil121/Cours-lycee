\documentclass[a4paper,12pt,french]{article}

\usepackage[cours]{../../../Style}

\tikzset{
setupRepereTransformation/.style args={(#1,#2),(#3,#4)}{
cm={#1*cos(#2),#1*sin(#2),#3*cos(#2+#4),#3*sin(#2+#4),(0,0)},
},
}

% Début du document
%%%%%%%%%%%%%%%%%%%
\begin{document}

\title{Repères du plan et coordonnées}
\maketitle

\begin{programme}
\item Base orthonormée. Coordonnées d'un vecteur. Expression de la norme d'un vecteur.
\item Expression des coordonnées de $\vec{AB}$ en fonction de celles de $A$ et $B$.
\item Déterminant de deux vecteurs dans une base orthonormée, critère de colinéarité. Application à l'alignement, au parallélisme.
\item Capacités:
\begin{itemize}
\item Représenter un vecteur dont on connait les coordonnées. Lire les coordonnées d'un vecteur.
\item Calculer les coordonnées d'une somme de vecteurs, d'un produit de vecteur par un réel.
\item Calculer la distance entre deux points. Calculer les coordonnées du milieu d'un segment.
\item Caractériser alignement et parallélisme par la colinéarité de vecteurs.
\item Résoudre des problèmes en utilisant la représentation la plus adaptée des vecteurs.
\end{itemize}
\item Démonstration: Colinéarité $\Leftrightarrow$ déterminant nul
\end{programme}

\section{Généralités sur les repères}

\begin{defin}
Soient $O,I,J$ trois points du plan non alignés. On pose $\vec i = \vec{OI}$ et $\vec j = \vec{OJ}$. Un repère du plan est un triplet \Oij. On dit alors que :
\begin{itemize}
\item $O$ est l'origine du repère
\item $(OI)$ est l'axe des abscisses
\item $OJ$ est l'axe des ordonnées
\end{itemize}
\end{defin}

\begin{fait}
On se donne maintenant un repère \Oij pour la suite du cours.
\end{fait}

\begin{propr}
Tout point $M$ du plan est repéré par un unique couple de coordonnées $(x;y)$. $x$ est l'abscisse de $M$ et $y$ est l'ordonnée de $M$.
\end{propr}

\begin{exs}

\begin{tabularx}{\linewidth}{YYY}
\begin{scaletikzpicturetowidth}{0.9\linewidth} % Repère quelconque à la main
\begin{tikzpicture}[scale=\tikzscale]
% Cadre
\coordinate (liminf) at (-2,-2);
\coordinate (limsup) at (8,5);
\draw[line width=0.4pt,draw=black!30,densely dashed] (liminf) rectangle (limsup);
\clip (liminf)+(-0.01,-0.01) rectangle (limsup)+(0.01,0.01);

% Sous-image, changement de coordonnées
\begin{scope}[setupRepereTransformation={(2,20),(1,80)}]
% Initialisation des axes, de la grille
\draw[step=1,line width=0.4pt,draw=black!30,densely dashed] ($5*(liminf)+(-5,-5)$)+(-0.01,-0.01) grid ($5*(limsup)+(5,5)$)+(0.01,0.01);
\tkzInit[xmin=5*(-2),xmax=5*10,ymin=5*(-2),ymax=5*5]
\tkzDrawX[label={},tickup=3pt,tickdn=3pt]
\tkzDrawY[label={},ticklt=1.5pt,tickrt=1.5pt]

% Initialisation du repère
\node[label={-135:{O}}] (O) at (0,0) {};
\draw[->,>=latex,line width=1pt,cap=round,color=black!30!red] (0,0) -- (1,0) node[pos=0.5,below] {$\vec i$};
\draw[->,>=latex,line width=1pt,cap=round,color=black!30!red] (0,0) -- (0,1) node[pos=0.5,left] {$\vec j$};

%Contenu
\node[stylepoint,label=45:{$M$},fill=blue,semithick] (M) at (1,3) {}; 
\end{scope}
\end{tikzpicture}
\end{scaletikzpicturetowidth}
&
\begin{tikzpicture}
\begin{axis}[
styleglobal,
repere,
width=0.45\linewidth,
xmin=-1,xmax=4,
ymin=-2,ymax=5,
x post scale=2
]
\node[label={-135:{O}}] (O) at (0,0) {};
\draw[->,>=latex,line width=1pt,cap=round,color=black!30!red] (0,0) -- (1,0) node[pos=0.5,below] {$\vec i$};
\draw[->,>=latex,line width=1pt,cap=round,color=black!30!red] (0,0) -- (0,1) node[pos=0.5,left] {$\vec j$};

% Contenu
\node[stylepoint,fill=blue,label={-5:$M$}] (A) at (2,-1) {};
\end{axis}
\end{tikzpicture}
&
\begin{scaletikzpicturetowidth}{0.9\linewidth} % Repère quelconque à la main
\begin{tikzpicture}[scale=\tikzscale]
\coordinate (liminf) at (-5,-2);
\coordinate (limsup) at (5,5);
\draw[line width=0.4pt,draw=black!30,densely dashed] (liminf) rectangle (limsup);
\clip (liminf)+(-0.01,-0.01) rectangle (limsup)+(0.01,0.01);
\begin{scope}[setupRepereTransformation={(1.5,20),(1,80)}]
\draw[step=1,line width=0.4pt,draw=black!30,densely dashed]($5*(-2,-2)$)+(-5.01,-5.01) grid ($5*(8,5)$)+(5.01,5.01);
\tkzInit[xmin=5*(-2),xmax=5*10,ymin=5*(-2),ymax=5*5]
\tkzDrawX[label={},tickup=3pt,tickdn=3pt]
\tkzDrawY[label={},ticklt=2pt,tickrt=2pt]

\node[label={-135:{O}}] (O) at (0,0) {};
\draw[->,>=latex,line width=1pt,cap=round,color=black!30!red] (0,0) -- (1,0) node[pos=0.5,below] {$\vec i$};
\draw[->,>=latex,line width=1pt,cap=round,color=black!30!red] (0,0) -- (0,1) node[pos=0.5,left] {$\vec j$};
\node[stylepoint,label=45:{$M$},fill=blue,semithick] (M) at (-2,4) {}; 
\end{scope}
\end{tikzpicture}
\end{scaletikzpicturetowidth}
\\
$M(1;3)$&$M(2;-1)$&$M(-2;4)$
\end{tabularx}
\end{exs}

\section{Coordonnées d'un vecteur}

\begin{defin}
Soit $\vec u$ un vecteur du plan. On se donne le point $M(x;y)$ tel que $\vec{OM} = \vec u$. Les coordonnées de $\vec u$ sont celles de $M$, et l'on note $\vec u \displaystyle\binom x y$.
\end{defin}

\begin{prop}
Soient $A(x_A;y_A)$ et $B(x_B;y_B)$ deux points du plan. Alors on a $\vec{AB} \displaystyle\binom {x_B-x_A} {y_B-y_A}$.
\end{prop}

\begin{ex}
\compo[0.6]
{
\setstretch{2.5}
On se donne $A(1;2)$, $B(5;-1)$ et $C(6,3)$.

Alors on a $\vec{AB} \displaystyle\binom {5-1}{-1-2}$ d'où $\vec{AB} \displaystyle\binom {4} {-3}$.

De même, on a $\vec{BC} \displaystyle\binom {6-5}{3-(-1)}$ d'où $\vec{BC} \displaystyle\binom 1 4$.
}
{\vspace{-1.5em}
\begin{tikzpicture}
\begin{axis}[
styleglobal,
repere,
width=0.9\linewidth,
xmin=-2,xmax=8,
ymin=-2,ymax=5,
]
\node[label={-135:{O}}] (O) at (0,0) {};
\draw[->,>=latex,line width=1pt,cap=round,color=black!30!red] (0,0) -- (1,0) node[pos=0.5,below] {$\vec i$};
\draw[->,>=latex,line width=1pt,cap=round,color=black!30!red] (0,0) -- (0,1) node[pos=0.5,left] {$\vec j$};

% Contenu
\node[stylepoint,fill=blue,label=-60:B] (B) at (5,-1) {};
\node[stylepoint,fill=blue,label=120:A] (A) at (1,2) {};
\node[stylepoint,fill=blue,label=60:C] (C) at (6,3) {};
\draw[line width=1pt,->,cap=round] (A.center) -- (B.center);
\draw[line width=1pt,->,cap=round] (B.center) -- (C.center);
\end{axis}
\end{tikzpicture}
}
\end{ex}

\begin{proprs}
Soient $\vec u \displaystyle\binom {x_{\vec u}} {y_{\vec u}}$ et $\vec v \displaystyle\binom {x_{\vec v}} {y_{\vec v}}$ deux vecteurs du plan, et $\lambda \in \R$. Alors:
\begin{itemize}
\item Les coordonnées de $\vec u + \vec v$ sont $\displaystyle\binom {x_{\vec u}+x_{\vec v}} {y_{\vec u}+y_{\vec v}}$.
\item Les coordonnées de $\lambda \vec u$ sont $\displaystyle\binom {\lambda x_{\vec u}} {\lambda y_{\vec u}}$.
\end{itemize}
\end{proprs}

Alors $\vec u=x \vec i + y \vec j$. $x$ et $y$ sont les coordonnées de $\vec u$ et o

\section{Norme/milieu}

Repère orthonormé à définir

\begin{prop}
Soit $\vec u \displaystyle\binom x y$ un vecteur du plan, dont les coordonnées sont exprimées dans un repère orthonormé. Alors la norme de $\vec u$ est donnée par:
$$\Vert \vec u \Vert = \sqrt{x^2+y^2}$$
\end{prop}

\begin{cor}
Soient $A(x_A;y_A)$ et $B(x_B;y_B)$ deux points du plan, dont les coordonnées sont exprimées dans un repère orthonormé. Alors la distance entre $A$ et $B$ vaut:
$$AB=\sqrt{(x_B-x_A)^2+(y_B-y_A)^2}$$
\end{cor}

\section{Colinéarité}

\end{document}

%\NewDocumentEnvironment{axispenche}{mmmmmb}]{
%\draw[line width=0.4pt,draw=black!30,densely dashed] (#2,#3) rectangle (#4,#5);
%\clip (#2,#3)+(-0.01,-0.01) rectangle (#4,#5)+(0.01,0.01);
%\begin{scope}[#1]
%\draw[step=1,line width=0.4pt,draw=black!30,densely dashed] (5*#2-5,5*#3-5)+(-0.01,-0.01) grid (5*#4+5,5*#5+5)+(0.01,0.01);
%\tkzInit[xmin=5*#2,xmax=5*#4,ymin=5*#3,ymax=5*#5]
%\tkzDrawX[label={},tickup=\xticksize,tickdn=\xticksize]
%\tkzDrawY[label={},ticklt=\yticksize,tickrt=\yticksize]
%%\tkzGrid[gray,line width=0.4pt,style=densely dashed,sub,subxstep=0.5,subystep=0.5]
%%\tkzGrid[gray,line width=0.4pt,style=densely dashed]
%%\draw (10*#2,0) -- (10*#4,0);
%%\draw (0,10*#3) -- (0,10*#5);
%#6
%\end{scope}
%}{}

% Quelques tests
%
%\def\defaultticksize{2pt}
%\def\ticksize{\defaultticksize}
%\def\xticksize{\ticksize}
%\def\yticksize{\ticksize}
%
%\tikzset{
%ytickscale/.style args={#1}{execute at begin scope={\def\yticksize{#1*\ticksize}},execute at end scope={\def\yticksize{\ticksize}}},
%xtickscale/.style args={#1}{execute at begin scope={\def\xticksize{#1*\ticksize}},execute at end scope={\def\xticksize{\ticksize}}},
%ticksize/.style args={#1}{execute at begin scope={\def\ticksize{#1}},execute at end scope={\def\xticksize{\defaultticksize}}},
%setupRepere/.style args={(#1,#2),(#3,#4)}{
%cm={#1*cos(#2),#1*sin(#2),#3*cos(#2+#4),#3*sin(#2+#4),(0,0)},
%%execute at begin scope={\pgfmathsetmacro\taillex{max(1,#1/#3)}\pgfmathsetmacro\tailley{max(1,#3/#1)}},
%xtickscale=1/#3,
%ytickscale=1/#1,
%},
%}

%\tikzstyle{every label}=[font=\large];

%\begin{scaletikzpicturetowidth}{0.9\linewidth}
%\begin{tikzpicture}[scale=\tikzscale]
%\coordinate (liminf) at (-2,-2);
%\coordinate (limsup) at (8,5);
%%\draw[line width=0.4pt,draw=black!30,densely dashed] (liminf) rectangle (limsup);
%\clip (liminf)+(-0.01,-0.01) rectangle (limsup)+(0.01,0.01);
%\begin{scope}[scale=1]
%\draw[step=1,line width=0.4pt,draw=black!30,densely dashed] ($5*(liminf)+(-5,-5)$)+(-0.01,-0.01) grid ($5*(limsup)+(5,5)$)+(0.01,0.01);
%\tkzInit[xmin=5*(-2),xmax=5*10,ymin=5*(-2),ymax=5*5]
%\tkzDrawX[label={},tickup=3pt,tickdn=3pt]
%\tkzDrawY[label={},ticklt=3pt,tickrt=3pt]
%
%\node[label={-135:{O}}] (O) at (0,0) {};
%\draw[->,>=latex,thick] (0,0) -- (1,0) node[pos=0.5,below] {$\vec i$};
%\draw[->,>=latex,thick] (0,0) -- (0,1) node[pos=0.5,left] {$\vec j$};
%\node[stylepoint,label=-5:{$M$},fill=blue,semithick] (M) at (5,-1) {}; 
%\end{scope}
%\end{tikzpicture}
%\end{scaletikzpicturetowidth}

%\begin{scaletikzpicturetowidth}{0.5\linewidth}
%\begin{tikzpicture}[scale=1]
%% Limites et grille
%\node[draw=none] (liminf) at (-2,-2) {};
%\node[draw=none] (limsup) at (13,11) {};
%\draw[gray] (liminf) rectangle (limsup);
%\clip (liminf) rectangle (limsup);
%\draw[step=1,line width=0.5pt,densely dashed,gray,opacity=0.5] (liminf)+(-0.01,-0.01) grid (limsup)+(0.01,0.01);
%
%% Repère
%\tkzInit[xmin=5*(-2),xmax=5*10,ymin=5*(-2),ymax=5*5]
%\tkzDrawX[label={},tickup=3pt,tickdn=3pt]
%\tkzDrawY[label={},ticklt=2pt,tickrt=2pt]
%\node[label={-135:{O}}] (O) at (0,0) {};
%\draw[->,>=latex,thick] (0,0) -- (1,0) node[pos=0.5,below] {$\vec i$};
%\draw[->,>=latex,thick] (0,0) -- (0,1) node[pos=0.5,left] {$\vec j$};
%
%% Contenu
%\node[stylepoint,fill=blue,label={180:\textbf{\large A}}] (A) at (3,6) {};
%\node[stylepoint,fill=blue,label={180:\textbf{\large B}}] (B) at (6,7) {};
%\node[stylepoint,fill=blue,label={-2:\textbf{\large C}}] (C) at (7,-1) {};
%\node[stylepoint,fill=blue,label={2:\textbf{\large D}}] (D) at (2,1) {};
%\draw[line width=1pt,->,>=latex] (9,-8) -- ++(2,3) node[pos=0.5,above left] {$\vec u$};
%%\node[color=black,circle,minimum size=1pt,fill,inner sep=2pt,fill opacity=1,label={90:\textbf{\large E}}] (E) at (4,-4) {};
%%\draw[line width=1pt, shorten <= -10cm, shorten >= -10cm] (A) -- (B);
%\end{tikzpicture}
%\end{scaletikzpicturetowidth}

% Pour aller plus vite:

%\begin{scaletikzpicturetowidth}{0.9\linewidth}
%\begin{tikzpicture}[scale=\tikzscale]
%\begin{axispenche}{setupRepere={(2,20),(1,80)}}{-2}{-2}{10}{5} %ticksize=1pt
%\node[label={-135:{O}}] (O) at (0,0) {};
%\draw[->,>=latex,thick] (0,0) -- (1,0) node[pos=0.5,below] {$\vec i$};
%\draw[->,>=latex,thick] (0,0) -- (0,1) node[pos=0.5,left] {$\vec j$};
%\end{axispenche}
%\end{tikzpicture}
%\end{scaletikzpicturetowidth}

%$$\setlength{\fboxsep}{2\fboxsep}\boxed{...}$$