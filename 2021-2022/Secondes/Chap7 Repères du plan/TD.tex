\documentclass[,a4paper,12pt,french]{article}

\usepackage[TD]{../../../Style}

%Quelques tests

%\NewDocumentEnvironment{scaletikzpicture}{mO{}b}{
%\begin{resizetikz}{#1} % Repère quelconque à la main
%\begin{tikzpicture}[scale=\tikzscale,#2]
%#3
%\end{tikzpicture}
%\end{resizetikz}}{}

%\usepackage{xkeyval}

%\NewDocumentCommand{\repereOij}{>{\SplitArgument{4}{,}}m}{\repereOijbase #1} % Pour faire {xmin,xmax,ymin,ymax} au lieu de {xmin}{xmax}{ymin}{ymax} dans la fonction repereOijbase -- j'ai mieux maitenant

%Pour la géométrie, mieux vaut utiliser tkz-base // tkz-euclide plutôt que pgfplots... Gestion des angles et coordonnées en ++ moins bonnes

%\renewcommand\tkzGrid{%
%\begingroup
%	 \edef\tkz@gxa{\fpeval{(\tkz@tmp@xa-\tkz@init@xorigine)/\tkz@init@xstep}}
%	 \edef\tkz@gxb{\fpeval{(\tkz@tmp@xb-\tkz@init@xorigine)/\tkz@init@xstep}}
%	 \edef\tkz@gya{\fpeval{(\tkz@tmp@ya-\tkz@init@yorigine)/\tkz@init@ystep}}
%	 \edef\tkz@gyb{\fpeval{(\tkz@tmp@yb-\tkz@init@yorigine)/\tkz@init@ystep}}
%\draw [color= black!30,line width = 0.4pt,densely dashed]%
%           (\tkz@gxa,\tkz@gya) grid (\tkz@gxb,\tkz@gyb);
%\endgroup}

% Début du document
%%%%%%%%%%%%%%%%%%%
\begin{document}
\titre{Repères et coordonnées - Exercices}

\begin{multicols*}{2}

\section{Coordonnées de vecteurs}

\begin{exercice}
Par lecture graphique, donner les coordonnées des vecteurs représentés ci-dessous:

\begin{centrer}
\begin{resizetikz}{0.9\linewidth}
\begin{tikzpicture}
\repereOij[
xmin=-5,
xmax=9,
ymin=-2,
ymax=5
]
\draw[stylevecteur] (-4,3) -- ++(5,-1) node[pos=0.5,above] {$\vec u$};
\draw[stylevecteur] (4,4) -- ++(-2,-3) node[pos=0.5,above left] {$\vec v$};
\draw[stylevecteur] (-3,-1) -- ++(11,2) node[pos=0.5,above] {$\vec w$};
\draw[stylevecteur] (8,0) -- ++(-3,3) node[pos=0.5,above right] {$\vec \alpha$};
\end{tikzpicture}
\end{resizetikz}
\end{centrer}
\end{exercice}

\begin{exercice} On se donne le repère ci-dessous:

\begin{centrer}
\begin{resizetikz}{0.9\linewidth}
\begin{tikzpicture}
\repereOij[
xmin=-5,
xmax=9,
ymin=-2,
ymax=5
];
\end{tikzpicture}
\end{resizetikz}
\end{centrer}

\begin{enumerate}
\item Sur le repère ci-dessous, placer les points $A(6;2)$, $B(3;-1)$ et $C(-2;4)$.
\item En déduire les coordonnées des vecteurs $\vec{AB}$, $\vec{AC}$ et $\vec{BC}$.
\item Retrouver par le calcul les résultats précédents.
\end{enumerate}
\end{exercice}

\begin{exercice}
Soient $A(3;3)$, $B(6;-1)$ et $C(-4;7)$. Donner les coordonnées des vecteurs $\vv{AB}, \vv{AC}, \vv{BC}$.
\end{exercice}

\begin{exercice}
Soient $A(-2;-1)$, $B(3;1)$, $C(1;3)$ et $D(-4;1)$. Montrer que $ABCD$ est un parallélogramme.
\end{exercice}

\begin{exercice}
On donne les points: $A(0;5)$, $B(-2;1)$ et $C(5;4)$. 

\begin{enumerate}
\item Calculer les coordonnées du vecteur $\vv{AB}$.
\item Déterminer les coordonnées du point $D$ tel que $ABCD$ soit un parallélogramme.
\end{enumerate}
\end{exercice}

\begin{exercice}
On se donne les vecteurs $\vv{u}\binom{1}{2}$, $\vv{v}\binom{-1}{3}$ et $\vv{w}\binom{\frac{1}{2}}{-7}$. Calculer les coordonnées de:\\
$\vv{u}+\vv{v} \hfill 3\vv v \hfill \vv{u}-\vv{w} \hfill 4\vv{w}+\vv{v} \hfill \dfrac{3}{4}\vv{u}-\dfrac{1}{3}\vv{v}$
\end{exercice}

\section{Normes, distances, milieux}

\begin{exercice}
Soit $A(2;3)$, $B(-1,3)$ et $C(4,-5)$. Donner les coordonnées des milieux $I$ de $[AB]$, $J$ de $[AC]$ et $K$ de $[BC]$.
\end{exercice}

On suppose à partir de maintenant que tous les repères utilisés sont orthonormés.

\begin{exercice}
On considère les points $A(2;3)$, $B(-1;4)$, $C(2,1)$. Calculer les normes suivantes:\\
$\strut\hfill \left\Vert \vv{AB} \right\Vert
\hfill \left\Vert \vv{BC} \right\Vert
\hfill \left\Vert 2\vv{AB} \right\Vert
\hfill \left\Vert \vv{AC} \right\Vert \hfill$
\end{exercice}

\begin{exercice}
Soient $\mathcal{C}$ le cercle de centre $I(1;2)$ et de rayon $5$ et $M(5;5)$.
\begin{enumerate}
    \item Calculer la distance $IM$.
    \item Le point $M$ appartient-il au cercle $\mathcal{C}$?
\end{enumerate}
\end{exercice}

\begin{exercice}
On a représenté trois points $A$,$B$ et $C$. Quelle est la nature du triangle $ABC$?
\begin{centrer}
	\begin{resizetikz}{0.9\linewidth}
    \begin{tikzpicture}
    \repereOij[xmin=-7,xmax=7,ymin=-3,ymax=5]
    \coordinate[label=above right:$A$] (A) at (5,4) {};
    \coordinate[label=below:$B$] (B) at (2,-2) {};
    \coordinate[label=above left:$C$] (C) at (-4,1) {};
    \draw[line width=1pt] (A) -- (B) -- (C) -- (A) -- cycle; %join=round?
    \end{tikzpicture}
    \end{resizetikz}
\end{centrer}
\end{exercice}

\section{Colinéarité}

\begin{exercice}
Déterminer les déterminants des couples de vecteurs suivants:
\begin{enumerate}
    \item $\vv{u}\binom{1}{4}$ et $\vv{v}\binom{7}{3}$
    \item $\vv{u}\binom{-1}{2}$ et $\vv{v}\binom{-3}{5}$
    \item $\vv{u}\binom{4}{2}$ et $\vv{v}\binom{-2}{1}$
\end{enumerate}
\end{exercice}

\begin{exercice}
Les vecteurs $\vv{u}\binom{3}{9}$ et $\vv{v}\binom{6}{18}$ sont-ils colinéaires?
\end{exercice}

\begin{exercice}
Prenons $\vv{u}\binom{3}{-1}$ et $\vv{v}\binom{4}{m}$ où $m\in \R$. Pour quelle valeur de $m$ les vecteurs $\vv{u}$ et $\vv{v}$ sont-ils colinéaires?
\end{exercice}

\begin{exercice}
Dans un repère orthonormé on considère les points suivants:\\
$A(-1;4) \hfill B \left(\frac{1}{2};1 \right) \hfill C(3;-4) \hfill D(-1;0) \hfill E(0;-2)$
\begin{enumerate}
    \item Placer les points dans le repère ci-dessous
    \begin{center}
    	\begin{resizetikz}{0.9\linewidth}
        \begin{tikzpicture}
        \repereOij
        \end{tikzpicture}
        \end{resizetikz}
    \end{center}
    \item Montrer que $A$,$B$ et $C$ sont alignés.
    \item Démontrer que $(DE) \parallel (AB)$.
\end{enumerate}
\end{exercice}

\begin{exercice}
Considérons les points $A(2;3)$, $B(5;7)$ et $C(-7;-9)$. Sont-ils alignés?
\end{exercice}

\section{Problèmes avancés}

\begin{exercice}[*]
On considère les points $A(2;3)$ et $B(-1;4)$.
%\begin{center}
%	\begin{resizetikz}{0.9\linewidth}
%    \begin{tikzpicture}
%    \repereOij[xmin=-2,xmax=7,ymin=-2,ymax=7]
%    \draw (2,3) node{$\times$};
%    \draw (2,3) node[above]{$A$};
%    \draw (-1,4) node{$\times$};
%    \draw (-1,4) node[above]{$B$};
%    \end{tikzpicture}
%    \end{resizetikz}
%\end{center}
Donner l'ensemble des points $C(x;y)$ tel que $ABC$ soit isocèle en $C$.
\end{exercice}

\begin{exercice}[*]
Prenons les points $A(1;-2)$ et $B(2;5)$.
Déterminer l'intersection de la médiatrice du segment $[AB]$ avec l'axe des abscisses.
\end{exercice}

\begin{exercice}[*]
$A$ est un point et $k\in \R^{*}$. On appelle homothétie de centre $A$ et rapport $k$ la fonction qui transforme pour tout point $M$ du plan le vecteur $\vv{AM}$ en $k\vv{AM}$. On note alors $M'$ le point tel que $\vv{AM'}=k\vv{AM}$.\\
Considérons une homothétie de rapport $k$.
\begin{enumerate}
    \item Montrer que pour $N,M$, $\vv{M'N'}=k\vv{MN}$.
    \item En déduire que les droites $(MN)$ et $(M'N')$ sont parallèles.
    \item Démontrer que $M'N'=\vert k\vert MN$.
    \item Construire l'image de la figure $f$ par une homothétie de rapport $2$ de centre $A$.
    \item Quel est le rapport d'aire entre celui de $f$ et celui de $f'$?
\end{enumerate}
\begin{centrer}
    \begin{resizetikz}{0.9\linewidth}
    \begin{tikzpicture}
    \repereOij
    \draw[fill=white] (-3,2) rectangle (-1,3);
    \draw (-2,2.5) node{$f$};
    \draw (-4,4) node{$\times$};
    \draw (-4,4) node[above]{$A$};
    \end{tikzpicture}
    \end{resizetikz}
\end{centrer}
\end{exercice}

\section{Exos en + hyperbole}

\begin{exercice}
On reprend les points de l'exercice 4. Calculer les coordonnées du milieu de $ABCD$.
\end{exercice}

\begin{exercice}
Soient $A(-2;3)$ et $\vec u \binom 4 {-2}$. Calculer les coordonnées du point $B$ image de $A$ par la translation de vecteur $\vec u$.
\end{exercice}

\begin{exercice}[*]
Soient $A(-2;-1)$, $B(4;1)$ et $I(0;1)$. Calculer les coordonnées des points $C$ et $D$ tels que $ABCD$ soit un parallélogramme de centre $I$.
\end{exercice}

\begin{exercice}
Soit $\mathcal C$ le cercle de centre $I(1,8;-3,5)$ passant par $A(-1,8;1,3)$. Calculer le rayon du cercle $\mathcal C$.
\end{exercice}

\begin{exercice}
Dans un repère orthonormé on considère les points suivants:\\
$\strut\hfill A \left(-1;-\frac 3 2 \right) \hfill B \left(0;\frac{1}{2}\right) \hfill C \left(-2;\frac 3 2 \right) \hfill$\\
$\strut\hfill D \left(\frac 3 2;\frac 7 2 \right) \hfill E(1;-2) \hfill F(3;-1) \hfill$
\begin{enumerate}
    \item Placer ces points dans un repère.
    \item Conjecturer des alignements de trois points.
    \item Démontrer ces conjectures par le calcul.
\end{enumerate}
\end{exercice}

\end{multicols*}
\end{document}








\begin{comment}

\begin{center}
        \begin{resizetikz}{0.9\linewidth}
        \begin{tikzpicture}
        \repereOij[xmin=-10,xmax=8,ymin=-10,ymax=10]
        \node[stylepoint,fill=blue,label=south east:$A$] (A) at (2,3) {};
        \node[stylepoint,fill=blue,label=south east:$B$] (B) at (5,7) {};
        \node[stylepoint,fill=blue,label=north west:$C$] (C) at (-7,-9) {};
        \end{tikzpicture}
        \end{resizetikz}
\end{center}


\begin{resizetikz}{0.9\linewidth} % Repère quelconque à la main
\begin{tikzpicture}
\binominate (liminf) at (-5,-2);
\binominate (limsup) at (9,5);
%\draw[line width=0.4pt,draw=black!30,densely dashed] (liminf) rectangle (limsup);
%\clip (liminf)+(-0.01,-0.01) rectangle (limsup)+(0.01,0.01);
%\draw[step=1,line width=0.4pt,draw=black!30,densely dashed](liminf) grid (limsup);
\tkzInit[xmin=-5,xmax=9,ymin=-2,ymax=5]
\tkzGrid
\tkzDrawX[label={},tickup=4pt,tickdn=4pt]
\tkzDrawY[label={},ticklt=4pt,tickrt=4pt]
\node[label={-135:{O}}] (O) at (0,0) {};
\draw[->,>=latex,line width=1pt,cap=round,color=black!30!red] (0,0) -- (1,0) node[pos=0.5,below] {$\vec i$};
\draw[->,>=latex,line width=1pt,cap=round,color=black!30!red] (0,0) -- (0,1) node[pos=0.5,left] {$\vec j$};
% Contenu
\draw[stylevecteur] (-4,3) -- ++(5,-1) node[pos=0.5,above] {$\vec u$};
\draw[stylevecteur] (4,4) -- ++(-2,-3) node[pos=0.5,above left] {$\vec v$};
\draw[stylevecteur] (-3,-1) -- ++(11,2) node[pos=0.5,above] {$\vec w$};
\draw[stylevecteur] (8,0) -- ++(-3,3) node[pos=0.5,above right] {$\vec \alpha$};
\end{tikzpicture}
\end{resizetikz}

\begin{tikzpicture}
\begin{axis}[
styleglobal,
repere,
width=0.9*\linewidth,
xmin=-5,xmax=9,
ymin=-2,ymax=5,
]
\node[label={-135:{O}}] (O) at (0,0) {};
\draw[->,>=latex,line width=1pt,cap=round,color=black!30!red] (0,0) -- (1,0) node[pos=0.5,below] {$\vec i$};
\draw[->,>=latex,line width=1pt,cap=round,color=black!30!red] (0,0) -- (0,1) node[pos=0.5,left] {$\vec j$};

% Contenu
\draw[stylevecteur] (-4,3) -- ++(axis direction cs:5,-1) node[pos=0.5,above] {$\vec u$};
\draw[stylevecteur] (4,4) -- ++(axis direction cs:-2,-3) node[pos=0.5,above left] {$\vec v$};
\draw[stylevecteur] (-3,-1) -- ++(axis direction cs:11,2) node[pos=0.5,above] {$\vec w$};
\draw[stylevecteur] (8,0) -- ++(axis direction cs:-3,3) node[pos=0.5,above right] {$\vec \alpha$};
\end{axis}
\end{tikzpicture}
\begin{resizetikz}{0.9\linewidth} % Repère quelconque à la main
\begin{tikzpicture}
\binominate (liminf) at (-5,-2);
\binominate (limsup) at (9,5);
%\draw[line width=0.4pt,draw=black!30,densely dashed] (liminf) rectangle (limsup);
%\clip (liminf)+(-0.01,-0.01) rectangle (limsup)+(0.01,0.01);
\draw[step=1,line width=0.4pt,draw=black!30,densely dashed](liminf) grid (limsup);
\tkzInit[xmin=-5,xmax=9,ymin=-2,ymax=5]
\tkzDrawX[label={},tickup=4pt,tickdn=4pt]
\tkzDrawY[label={},ticklt=4pt,tickrt=4pt]
\node[label={-135:{O}}] (O) at (0,0) {};
\draw[->,>=latex,line width=1pt,cap=round,color=black!30!red] (0,0) -- (1,0) node[pos=0.5,below] {$\vec i$};
\draw[->,>=latex,line width=1pt,cap=round,color=black!30!red] (0,0) -- (0,1) node[pos=0.5,left] {$\vec j$};
% Contenu
\draw[stylevecteur] (-4,3) -- ++(5,-1) node[pos=0.5,above] {$\vec u$};
\draw[stylevecteur] (4,4) -- ++(-2,-3) node[pos=0.5,above left] {$\vec v$};
\draw[stylevecteur] (-3,-1) -- ++(11,2) node[pos=0.5,above] {$\vec w$};
\draw[stylevecteur] (8,0) -- ++(-3,3) node[pos=0.5,above right] {$\vec \alpha$};
\end{tikzpicture}
\end{resizetikz}
\end{comment}