%%% Option progres
%%%%%%%%%%%%%%%%%%%%%%%%%%%%%%%%%%%%%%%%%%%%%%%%%%%%%%%%

% Le but est d'ignorer tout ce qui vient après la commande dans le fichier cours pour pouvoir avoir des cours partiels. La commande prend son sens avec l'option eleve.
\newcommand{\progres}{}


% Options
%%%%%%%%%%%%%%%%%%%%%%%%%%%%%%%%%%%%%%%%%%%%%%%%%%%%%%%%

% C'est un peu bordélique ici:
%Si on a l'option cours, tous les thm apparaissent dans des tcolorbox. Sinon on utilise ntheorem
%Si on a l'option élève, on doit désactiver la commande rem sauf si option eleve dans la remarque( remarques perso dans le cours ), on a besoin de tester si elle est présente pour faire un si-sinon

%
%	Si cours alors boites tcolorbox
%	Sinon
%	Définir styles ntheorem classiques
%		Si option TD alors style exo mixte et marges modifiées
%		Si sujet alors pareil sans style exo mixte
%	Si eleve alors désactiver les rmq perso sauf cas particuliers
%

\newif\ifoptioncours
\optioncoursfalse

\newif\ifoptioneleve
\optionelevefalse

\newif\ifoptionTD
\optionTDfalse

\newif\ifoptionsujet
\optionsujetfalse

\newif\ifoptionminted
\optionmintedfalse

\DeclareOption{cours}{\optioncourstrue}

\DeclareOption{eleve}{\optionelevetrue}

\DeclareOption{TD}{\optionTDtrue}

\DeclareOption{sujet}{\optionsujettrue}

\DeclareOption{minted}{
	\optionmintedtrue
}

\DeclareOption{coursInline}{
	\renewcommand{\baselinestretch}{1.1}
	\tcbset{
		rounded corners=southwest,
		arc=5pt,
		auto outer arc,
		styletitre/.style={
			sharp corners,
			boxrule=0pt,
			%boxsep=0pt,
			enlarge bottom by=2pt,
			enhanced jigsaw,
			on line,
			top=0pt,
			bottom=0pt,
			right=0mm,
			left=3pt,
		},
	}
	\renewcommand{\deftcbUn}[6][10]{
	\definecolor{#2-titre}{Hsb}{#4,#5,#6}
	\colorlet{#2}{#2-titre!#1}
	\newtcolorbox{#2}[1][]{
		colback=#2,
		title={},
		colbacktitle=#2,
		top=0pt,
		bottom=5pt,
		enlarge bottom by=0pt,
		%borderline west={1pt}{0pt}{#2-titre},
		before upper={\colorlet{bg}{#2}\definecolor{bfcolor}{Hsb}{#4,0.75,0.15}\hspace{-10pt}\tcbox[styletitre,colback=#2-titre]{\rule[-3pt]{0pt}{14pt}{\color{white}\bfseries\MakeUppercase{#3}\ifstrempty{##1}{}{ (##1)} :}}\ },
		after upper={\colorlet{bg}{white} \colorlet{bfcolor}{black}},
	}
	}
}

\ProcessOptions

\ifoptioncours
	
	% Formats titres, sections
	
	\definecolor{titre}{HTML}{912c21}
	\definecolor{section}{HTML}{1c567d}
	\definecolor{subsection}{HTML}{2673a6}
	\definecolor{subsubsection}{HTML}{2f90d0}

	\renewcommand{\thesection}{\Roman{section} - }
	\renewcommand{\thesubsection}{\arabic{subsection}. }
	\renewcommand{\thesubsubsection}{\alph{subsubsection}) }

	\newcommand{\sectionstyle}{\normalfont\Large\bfseries\color{section}}
	\titleformat{\section}{\sectionstyle}{\thesection #1}{0pt}{}[{\color{gray}\hrule}]
	\titleformat{name=\section, numberless}{\sectionstyle}{#1}{0pt}{}[{\color{gray}\hrule}]

	\newcommand{\subsectionstyle}{\normalfont\large\bfseries\color{subsection}}
	\titleformat{\subsection}{\subsectionstyle}{\hspace{0.1\linewidth}\thesubsection #1}{0pt}{}
	\titleformat{name=\subsection, numberless}{\subsectionstyle}{\hspace{0.1\linewidth}#1}{0pt}{}
	
	\newcommand{\subsubsectionstyle}{\normalfont\large\bfseries\color{subsubsection}}
	\titleformat{\subsubsection}{\subsubsectionstyle}{\hspace{0.2\linewidth}\thesubsubsection #1}{0pt}{}
	\titleformat{name=\subsubsection, numberless}{\subsubsectionstyle}{\hspace{0.2\linewidth}#1}{0pt}{}
	
	\def\@maketitle{\centering  \let \footnote \thanks {\fontfamily{cmr}\fontseries{bx}\selectfont\huge\scshape\color{titre} \@title \par}%
	\textcolor{darkgray}{\hrule}%
	\vspace{2pt}%
	\textcolor{darkgray}{\hrule height 2pt depth 0pt \relax}%
  \par
  \vskip 2em}

	\renewcommand{\titre}[1]{\title{#1} \maketitle}
	
	% Template: \deftcb{identifiant}{Nom}{Teinte}{Saturation}{Luminosité} ( Modèle HSB )
	\deftcb{defin}{Définition}{211}{0.8}{0.75}
	\deftcb{thm}{Théorème}{0}{0.8}{0.75}
	\deftcb{prop}{Proposition}{15}{0.8}{0.75}
	\deftcb[8]{ex}{Exemple}{120}{0.8}{0.58}
	\deftcb[8]{cex}{Contre-exemple}{120}{0.8}{0.58}
	\deftcb{rmq}{Remarque}{211}{0.8}{0.75}
	\deftcb{caspart}{Cas particulier}{211}{0.8}{0.75}
	\deftcb{cor}{Corollaire}{211}{0.8}{0.75}
	\deftcb{app}{Application}{211}{0.8}{0.75}
	\deftcb{propr}{Propriété}{211}{0.8}{0.75}
	\deftcb{methode}{Méthode}{211}{0.8}{0.75}
	\deftcb{exercice}{Exercice}{210}{0.6}{0.7}
	

	\newenvironment{enonce}[1]{
	\deftcbUn{enonce\thecompteurenonce}{#1}{211}{0.8}{0.75}
	\begin{enonce\thecompteurenonce}}
	{\end{enonce\thecompteurenonce} \refstepcounter{compteurenonce}}
	
	\deftcbUn{tcolor-programme}{Programme}{0}{0.8}{0.75}	
	\deftcbUn{tcolor-deroule}{Déroulé}{0}{0.8}{0.75}
	
	\NewDocumentEnvironment{programme}{b}{\vspace{-2em}\begin{tcolor-programme}\saut\setstretch{0.75}{\footnotesize{\begin{itemize}[align=left,leftmargin=8pt,labelwidth=10pt,itemindent=10pt]#1\end{itemize}}}\end{tcolor-programme}}{\par}
	\NewDocumentEnvironment{deroule}{b}{\begin{tcolor-deroule}\saut\setstretch{0.75}{\footnotesize{\begin{itemize}[align=left,leftmargin=8pt,labelwidth=10pt,itemindent=10pt]#1\end{itemize}}}\end{tcolor-deroule}}{\par}
	
	%Les tcolorbox avec titre font un saut automatique mais pas les thm donc saut fait le lien
	\newcommand{\saut}{}
	
	\renewcommand{\textbf}[1]{{\color{bfcolor}\TextOrMath{{\bfseries #1}}{\mathbf{#1}}}}
	
	\renewcommand{\rembase}[1]{\begin{remq} \begin{tabulary}{0.5\linewidth}{>{$\blacktriangleright$ \ }L} #1 \end{tabulary} \end{remq}}
	\NewDocumentEnvironment{remsbase}{b}{\begin{remq} \begin{tabulary}{0.5\linewidth}{>{$\blacktriangleright$ \ }L} #1 \end{tabulary} \end{remq}}{\par}	
	
	\renewcommand{\rem}[2][]{\rembase{#2}}
	\NewDocumentEnvironment{rems}{ob}{\begin{remsbase}#2\end{remsbase}}{\par}
			

\else % Style +- à l'ancienne

	% Style des sections

	\renewcommand{\thesection}{\Roman{section} - \hspace{-2ex}}
	\renewcommand{\thesubsection}{\arabic{subsection}) \hspace{-2ex}}
	\renewcommand{\thesubsubsection}{\alph{subsubsection}) \hspace{-2ex}}

	\titlespacing*{\section}{0mm}{0mm}{1mm}
	\titlespacing*{\subsection}{10mm}{0mm}{3mm}
	\titlespacing*{\subsubsection}{20mm}{0mm}{3mm}

	\def\@maketitle{\centering  \let \footnote \thanks {\huge \@title \par}\par\vskip 2em}
	\NewDocumentEnvironment{programme}{b}{\section*{Programme}\setstretch{0.75}{\footnotesize{\begin{itemize}#1\end{itemize}}}}{\par}
	\NewDocumentEnvironment{deroule}{b}{\section*{Déroulé}\setstretch{0.75}{\footnotesize{\begin{itemize}#1\end{itemize}}}}{\par}
	
	\renewenvironment{fait}[1][]{}{}
	
	\everymath{\displaystyle}
	% Style général
	
	\theorembodyfont{\normalfont}
	\theoremframepostskip{0mm}
	\theoremframepreskip{0mm}
	\theoreminframepostskip{0mm}
	\theoreminframepreskip{0mm}
	\theoremnumbering{arabic}
	
	\ifoptionTD
	
		\setlist[itemize]{align=parleft,left=5pt..15pt}
		\setlist[enumerate]{align=left,left=-3pt..15pt,leftmargin=0pt,itemindent=15pt}
		\setlist[enumerate,1]{align=left,left=5pt..20pt,itemindent=*}
		\setlist[enumerate,2]{align=left,left=5pt..20pt,itemindent=*}
		\geometry{margin=10mm, bottom=17mm}
		\setlength\columnsep{10pt}
		\colorlet{couleurrem}{black}
		
		\renewcommand{\titre}[1]{\begin{center}{\fontfamily{cmr}\fontseries{bx}\selectfont\huge{#1}}\end{center}}
		%\color[HTML]{682018}
		
		% Style exos				
		
		\definecolor{couleurExo}{Hsb}{211,0.8,0.65}
		\newtheoremstyle{stylexo}	%Base plain
		{\item[\theorem@headerfont\hskip\labelsep\colorbox{couleurExo}{\color{white}\theorem@headerfont ##1\ ##2\theorem@separator}]}%
		{\item[\theorem@headerfont\hskip \labelsep\colorbox{couleurExo}{\color{white}\theorem@headerfont ##1\ ##2\IfBeginWith{##3}{*}{##3}{\ (##3)}\theorem@separator}]}
		
		\theoremstyle{stylexo}
		\theoremseparator{.}
		\theorempreskip{\topsep}
		\theorempostskip{\topsep}
		\newtheorem{exercice}{Exercice}
		
		\definecolor{couleurDefinTD}{Hsb}{15,0.8,0.65}
		\newtheoremstyle{styledefintd}	%Base plain
		{\item[\theorem@headerfont\hskip\labelsep\colorbox{couleurDefinTD}{\color{white}\theorem@headerfont ##1\theorem@separator}]}%
		{\item[\theorem@headerfont\hskip \labelsep\colorbox{couleurDefinTD}{\color{white}\theorem@headerfont ##1\ (##3) \theorem@separator}]}
		
	\theoremstyle{styledefintd}
		
	\else
	
		\theoremstyle{plain}
		\theoremseparator{.}
		\theorempreskip{\topsep}
		\theorempostskip{\topsep}
		\newtheorem{exercice}{Exercice}
		
		\theoremstyle{nonumberplain}
		
	\fi
	
	\ifoptionsujet
	
		\setlist[itemize]{align=parleft,left=5pt..15pt}
		\setlist[enumerate]{align=left,left=-3pt..15pt,leftmargin=0pt,itemindent=15pt}
		\setlist[enumerate,1]{align=left,left=5pt..20pt,itemindent=*}
		\setlist[enumerate,2]{align=left,left=5pt..20pt,itemindent=*}
		\geometry{includeheadfoot,bottom=12.5mm,top=2mm}
		\pagestyle{fancy}
		\setlength{\headheight}{10mm}
		\fancyfoot[C]{Page \thepage}
		\setlength\columnsep{10pt}
		\colorlet{couleurrem}{black}
	
	\fi	
	
	% Style théorèmes
	
	\theorembodyfont{\normalfont}
	%\theorempostskip{2mm}
	%\theorempreskip{2mm}
	\theoremseparator{ :\par}
	%\theoremsymbol{\rule{1.5mm}{1.5mm}}
	
	\newtheorem{defin}{Définition}
	\newtheorem{thm}[defin]{Théorème}
	\newtheorem{prop}[defin]{Proposition}
	\newtheorem{lemme}[defin]{Lemme}
	\newtheorem{ex}[defin]{Exemple}
	\newtheorem{exs}[defin]{Exemples}
	\newtheorem{cex}[defin]{Contre-exemple}
	\newtheorem{rmq}[defin]{Remarque}
	\newtheorem{rmqs}[defin]{Remarques}
	\newtheorem{cor}[defin]{Corollaire}
	\newtheorem{propr}[defin]{Propriété}
	\newtheorem{proprs}[defin]{Propriétés}
	\newtheorem{app}[defin]{Application}
	\newtheorem{defprop}[defin]{Def-prop}
	\newtheorem{algo}[defin]{Algorithme}
	\newtheorem{rappel}[defin]{Rappel}
	\newtheorem{methode}[defin]{Méthode}
	\newtheorem{caspart}[defin]{Cas particulier}
	\newtheorem{casparts}[defin]{Cas particuliers}
	\newframedtheorem{thmframed}[defin]{Théorème}
	\newframedtheorem{exframed}[defin]{Exemple}
	
	\newenvironment{enonce}[1]{
	\newtheorem{enonce\thecompteurenonce}[defin]{#1}
	\begin{enonce\thecompteurenonce}}
	{\end{enonce\thecompteurenonce} \refstepcounter{compteurenonce}}
	
	\newcommand{\saut}{\
	
	}
	
	\NewDocumentEnvironment{rems}{ob}{\rem{#2}}{\par}
	
\fi

\ifoptioneleve
	
	\RenewDocumentEnvironment{programme}{b}{}{\par}
	\RenewDocumentEnvironment{deroule}{b}{}{\par}

	\RenewDocumentEnvironment{rems}{ob}{\ifthenelse{\equal{#1}{eleve}}{\begin{remsbase}#2\end{remsbase}}}{\par}

	\renewcommand{\rem}[2][]{\ifthenelse{\equal{#1}{eleve}}{\rembase{#2}}{}}
	
	% Commande progres pour arrêter la compilation (uniquement pour les élèves)	
	
	\renewcommand{\progres}{\end{document}}

\fi

\ifoptionminted

	
	\RequirePackage[cachedir=\detokenize{_}minted]{minted}
	\usemintedstyle{rainbow_dash} % Styles sympa: default,manni,monokai,pastie,native,igor,lovelace,rainbow_dash,inkpot,
	\setminted{xleftmargin=\parindent,linenos,
    	gobble=0,
    	breaklines=true,
    	breakafter=,,
    	tabsize=4,
    	numbersep=8pt}
	
\fi

	

%%% Ajout option élève
%%%%%%%%%%%%%%%%%%%%%%%%%%%%%%%%%%%%%%%%%%%%%%%%%%%%%%%%

\begin{comment}
\newcounter{compteurexostcb}

\newcommand{\deftcbUnExoTest}[6][10]{
\definecolor{#2-titre}{Hsb}{#4,#5,#6}
\colorlet{#2}{#2-titre!#1}
\newtcolorbox{#2}[1][]{
	colback=#2,
	title={},
	colbacktitle=#2,
	top=0pt,
	breakable,
	bottom=4pt,
	left=4pt,
	right=4pt,
	enlarge bottom by=0pt,
	%borderline west={1pt}{0pt}{#2-titre},
	before upper={\stepcounter{compteurexostcb}\colorlet{bg}{#2} \definecolor{bfcolor}{Hsb}{#4,0.75,0.15} \hspace{-4pt}\tcbox[arc=0pt,boxrule=0pt,outer arc=0pt,enlarge bottom by=0pt,enhanced jigsaw,tcbox raise base,after={},before={},top=0pt,bottom=1pt,right=0mm,left=5pt,colback=#2-titre]{\color{white}\bfseries\MakeUppercase{#3}\ifstrempty{##1}{}{ (##1)} \thecompteurexostcb :}\ },
	after upper={\colorlet{bg}{white} \colorlet{bfcolor}{black}},
	%before upper={\tcolorboxsubtitle{#1}{##1}},
}
}

\deftcbUn{exotcbbase}{Exercice \thecompteurexostcb}{30}{0.8}{0.75} % Ne pas utiliser, compteur pas incrémenté ( utiliser exercice)

%\newenvironment{exercice}[1][]{\stepcounter{compteurexostcb}\begin{exotcbbase}[#1]}{\end{exotcbbase}}
\end{comment}