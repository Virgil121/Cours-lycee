% Abréviations displaystyle 
\newcommand{\Int} {\displaystyle \int}

% Abréviations maths
\newcommand{\C} {\mathbb{C}}
\newcommand{\R} {\mathbb{R}}
\newcommand{\Z} {\mathbb{Z}}
\newcommand{\N} {\mathbb{N}}
\newcommand{\K} {\mathbb K}
\newcommand{\U} {\mathbb U}
\newcommand{\D} {\mathbb D}
\newcommand{\Q} {\mathbb Q}
\newcommand{\Pro} {\mathbb P}
\newcommand{\Esp} {\mathbb E}

\newcommand{\fonction}[5]{\begin{array}[t]{ccccc}
#1: & #2 & \longrightarrow & #3 \\
    & #4 & \longmapsto & #5 \end{array}}
\newcommand{\fonctionb}[4]{\begin{array}[t]{ccccc}
#1 & \longrightarrow & #2 \\
#3 & \longmapsto & #4 \end{array}}
\newcommand{\fonct}[3]{#1: #2 \longmapsto #3}

\newcommand{\abs}[1]{\left\vert #1 \right\vert}

\let\vec\vv
\let\vecc\vv
%\newcommand{\vecc}[1] {\overrightarrow{#1}}

% Gras dans titres en mode math

\g@addto@macro\bfseries{\boldmath}
\SetSymbolFont{stmry}{bold}{U}{stmry}{m}{n}

% Commandes cours
%%%%%%%%%%%%%%%%%%%%%%%%%%%%%%%%%%%%%%%%%%%%%%%%%%%%%%%%

\author{}
\date{}

\newcommand{\titre}[1]{\begin{center}{\fontfamily{cmr}\fontseries{bx}\selectfont \huge{#1}}\end{center}}

\newcommand{\seconde}{2\textsuperscript{nde}}
\newcommand{\premiere}{1\textsuperscript{ère}}

\newcommand{\fakeitem}{\textemdash \ }

% Compo

\newlength{\colG}\newlength{\colD}

\newcommand{\compobase}[5]{
\setlength{\colG}{#3\linewidth} \setlength{\colD}{\linewidth}
\addtolength{\colD}{-\colG} \addtolength{\colD}{-#1}
\addtolength{\colG}{-#1}%
\par\noindent%
\begin{minipage}[#2]{\colG}\vspace{0pt}#4\end{minipage}\hfill%	%vspace=hack pour des problèmes d'alignement qui aparaissent avec compohaut
\begin{minipage}[#2]{\colD}\vspace{0pt}#5\end{minipage}\par}

\newcommand{\compo}[3][0.5]{\compobase{10pt}{c}{#1}{#2}{#3}}
\newcommand{\compohaut}[3][0.5]{\compobase{10pt}{t}{#1}{#2}{#3}}

% Compo avec ligne séparatrice

\newlength{\hauteurGauche} \newlength{\hauteurDroite} \newlength{\decalageLigne}
\newsavebox{\partieGauche} \newsavebox{\partieDroite}

\newcommand{\compolignebase}[5]{
\setlength{\colG}{#3\linewidth} \setlength{\colD}{\linewidth}
\addtolength{\colD}{-\colG} \addtolength{\colD}{-#1}
\addtolength{\colG}{-#1}%
\sbox{\partieGauche}{\begin{minipage}[#2]{\colG}\vspace{0pt}#4\end{minipage}}
\sbox{\partieDroite}{\begin{minipage}[#2]{\colD}\vspace{0pt}#5\end{minipage}}
\setlength{\hauteurGauche}{\ht\partieGauche} \setlength{\hauteurDroite}{\ht\partieDroite}
\addtolength{\hauteurGauche}{\dp\partieGauche} \addtolength{\hauteurGauche}{14pt}
\addtolength{\hauteurDroite}{\dp\partieDroite} \addtolength{\hauteurDroite}{14pt}
\par\noindent
\usebox{\partieGauche}%
\setlength{\decalageLigne}{7pt}
\ifthenelse{\lengthtest{\hauteurGauche>\hauteurDroite}}
{\addtolength{\decalageLigne}{\dp\partieGauche} \hfill \rule[-\decalageLigne]{0.5pt}{\hauteurGauche} \hfill}
{\addtolength{\decalageLigne}{\dp\partieDroite} \hfill \rule[-\decalageLigne]{0.5pt}{\hauteurDroite} \hfill}
\usebox{\partieDroite}
\par
}

\newcommand{\compoligne}[3][0.5]{\compolignebase{10pt}{c}{#1}{#2}{#3}}
\newcommand{\compolignehaut}[3][0.5]{\compolignebase{10pt}{t}{#1}{#2}{#3}}

% Environnement centrer sans passage à la ligne

\newenvironment{centrer}{\par \centering }{ \par}

% Remarques (avec compatibilité)
% A refaire entièrement: problème avec options définies avant d'où les ifthenelse à foison

\colorlet{couleurrem}{DarkGreen}

\newcommand{\rembase}[1] {\begin{center} \color{couleurrem}\textit{#1} \end{center}}
\newcommand{\remmbase}[1] {\rem{#1}} % A supprimer quand je serai sûr que ca n'est plus utilisé

\newcommand{\rem}[2][]{\ifthenelse{\equal{#1}{black}}{\colorlet{couleurrem}{black}}{}\rembase{#2}\colorlet{couleurrem}{DarkGreen}} % Pour pouvoir utiliser l'option eleve sur les remarques plus tard. Option If pour compatibilité ...

% Lignes de points pour sujets écrits

\newcommand{\points}[2][1.5]{ \setstretch{#1}
\foreach \n in{1,...,#2}
{\noindent \dotfill

}
}

% Entete pour interros ( quand il en faut plusieurs en une page, imitation fancyhdr )

\newcommand{\entete}[3]{
\begin{center}
\par\smallskip\noindent\parbox[t]{.25\textwidth}{\raggedright #1}%
\parbox[t]{.499\textwidth}{\centering #2}%
\parbox[t]{.25\textwidth}{\raggedleft #3}%

\rule[3.3mm]{\textwidth}{0.4pt}
\end{center}
}

% Centrer tabularx de base et ligne centrée Y

\renewcommand\tabularxcolumn[1]{m{#1}}
\newcommand{\tabularxgauche}{\renewcommand\tabularxcolumn[1]{p{##1}}}
\newcommand{\tabularxcentrer}{\renewcommand\tabularxcolumn[1]{m{##1}}}
\newcolumntype{Y}{>{\centering\arraybackslash}X}