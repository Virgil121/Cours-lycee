% Initialisation pgfplots, tikz et styles
%%%%%%%%%%%%%%%%%%%%%%%%%%%%%%%%%%%%%%%%%%%%%%%%%%%%%%%%

\newcommand{\echellepgf} {4/5}
\newcommand{\echellepgfinv} {5/4}

\pgfplotsset{ % Pas besoin de "styleglobal" en fait ... Important: ne pas faire xscale mais x post scale pour conserver la taille des ticks
compat=1.18,
styleglobal/.style={
height=\pgfkeysvalueof{/pgfplots/width}*(((\pgfkeysvalueof{/pgfplots/ymax})+((-1)*(\pgfkeysvalueof{/pgfplots/ymin})))/((\pgfkeysvalueof{/pgfplots/xmax})+((-1)*(\pgfkeysvalueof{/pgfplots/xmin}))))+0.1pt, %Hauteur auto en fonction des coordonnées et de l'échelle
axis x line=bottom,
axis y line = left,
axis lines=middle,
axis line style={-{Stealth}}, %Plus grosse flèche
tick label style = {font=\footnotesize},
scale only axis, %Pour bien tomber sur la grille
xlabel={$x$},
ylabel={$y$},
label style= {font=\scriptsize},
minor x tick num=1, %Sous graduations
minor y tick num=1,
%xtick distance=1, %Graduations
%ytick distance=1,
enlargelimits={0.001},
grid=both,
grid style={densely dashed,line width=0.4pt},
minor grid style={line width=0.1pt},
axis equal,
legend pos=north east
},
styleplot/.style={
samples=101,
smooth,
line width=1.3pt,
mark=none
},
labelgros/.style={
label style= {font=\Large},
tick label style = {font=\Large}
},
hauteurproptick/.style={
height=\pgfkeysvalueof{/pgfplots/width}*((\pgfkeysvalueof{/pgfplots/xtick distance}/\pgfkeysvalueof{/pgfplots/ytick distance})*((\pgfkeysvalueof{/pgfplots/ymax})+((-1)*(\pgfkeysvalueof{/pgfplots/ymin})))/((\pgfkeysvalueof{/pgfplots/xmax})+((-1)*(\pgfkeysvalueof{/pgfplots/xmin})))),
axis equal=false,
enlargelimits=0.001,
}
}

\pgfkeys{/pgf/number format/use comma,/pgf/number format/.cd,1000 sep={\,}}

\newcommand{\pointsextremites}{node[pos=0,circle,minimum size=1pt,fill,inner sep=1pt,fill opacity=1] {} node[pos=1,circle,minimum size=1pt,fill,inner sep=1pt,fill opacity=1] {}}

\tikzset{
stylepoint/.style={
	draw=black,
	color=black,
	circle,
	minimum size=1pt,
	fill,
	inner sep=2pt,
	fill opacity=1
},
}

% Redéfinitions tableaux de signes
%%%%%%%%%%%%%%%%%%%%%%%%%%%%%%%%%%%%%%%%%%%%%%%%%%%%

% Note: makeatletter / makeatother inutile dans un sty ou cls

\def\tkzTabDefaultBackgroundColor{bg}

% Changement style ligne pour zéros, prendre tout le bloc avec xpatch ne marche pas donc solution un peu moche mais fonctionnelle

\xpatchcmd{\tkzTabLine}
{
\ifthenelse{\equal{\expandafter\stripspaces\expandafter{\x}}{z}}%
}
{
\ifthenelse{\equal{\expandafter\stripspaces\expandafter{\x}}{z}}{\tikzset{t style/.append style = { style = solid }}}{}
\ifthenelse{\equal{\expandafter\stripspaces\expandafter{\x}}{z}}
}
{}{}

%\tikzset{t style/.style={style=densely dotted,draw=\tkzTabDefaultWritingColor}}

\xpatchcmd{\tkzTabLine}
{
\node at (Z\thetkz@cnt@impair\thetkz@cnt@lg){$0$}; 
} % search
{
\node[fill=black,line width=0.5pt,draw=black,circle,minimum size=7pt,inner sep=2pt,xscale=0.93,yscale=1.3] at (Z\thetkz@cnt@impair\thetkz@cnt@lg){};
\node[fill=\tkzTabDefaultBackgroundColor,line width=0.5pt,draw=\tkzTabDefaultBackgroundColor,circle,minimum size=7pt,inner sep=2pt,xscale=0.7,yscale=1.18] at (Z\thetkz@cnt@impair\thetkz@cnt@lg){};
\tikzset{t style/.style = {style      = \cmdTAB@TTS@tstyle,
                           draw       = \cmdTAB@TTS@tcolor,
                           line width = \cmdTAB@TTS@twidth}} % Bizaremment, ça remet à zéro tout seul le style donc ligne inutile ...
}  % replace
{}{}

% Augmentation espacement pour double ligne

\tikzset{double style/.append style = {double distance=1pt}}