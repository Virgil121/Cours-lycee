% Test tcolorbox
%%%%%%%%%%%%%%%%%%%%%%%%%%%%%%%%%%%%%%%%%%%%%%%%%%%%%%%%

%Mise à jour dynamique de la couleur de fond
%Au lieu d'utiliser des options de transparence on pourra utiliser des couleurs comme red!10!bg ( 10% de rouge sur un fond de couleur bg )
%Voir cours ensembles de nombres pour un exemple d'utilisation

\colorlet{bg}{white}

% Mise à jour dynamique de la couleur de textbf

\colorlet{bfcolor}{black}
\newcommand{\colorfadebg}{15}
\tcbset{
	frame hidden,
	boxrule=0pt,
	boxsep=0pt,
	enlarge bottom by=0pt,
	enhanced jigsaw,
	boxed title style={sharp corners,boxrule=0pt,coltitle={white},titlerule=0pt},
	fonttitle=\bfseries\boldmath,
	top=10pt,
	right=10pt,
	bottom=10pt,
	left=10pt,
	sharp corners,
}

\newcommand{\deftcbUn}[6][10]{
\definecolor{#2-titre}{Hsb}{#4,#5,#6}
\colorlet{#2}{#2-titre!#1}
\newtcolorbox{#2}[1][]{
	colback=#2,
	breakable,
	borderline west={1pt}{0pt}{#2-titre},
	title={\hspace{-10pt}\colorbox{#2-titre}{\rule[-3pt]{0pt}{14pt} \hspace{5pt}\MakeUppercase{#3}\ifstrempty{##1}{}{ (##1) }\hspace{10pt}}},
	colbacktitle=#2,
	before upper={\colorlet{bg}{#2} \definecolor{bfcolor}{Hsb}{#4,0.75,0.15}},
	after upper={\colorlet{bg}{white} \colorlet{bfcolor}{black}},
	%before upper={\tcolorboxsubtitle{#1}{##1}},
}
}

\newcommand{\deftcb}[6][10]{ % Pour définir tous les types de thm en une ligne: voir options
\deftcbUn[#1]{#2}{#3}{#4}{#5}{#6}
\deftcbUn[#1]{#2s}{#3s}{#4}{#5}{#6}
}

\definecolor{remq-titre}{Hsb}{50,0.8,0.8}
\colorlet{remq}{remq-titre!10}
\newtcolorbox{remq}{
	colback=remq,
	colframe=remq-titre,
	hbox,
	flush right,
	right=4pt,
	left=4pt,
	top=7pt,
	bottom=7pt,
	borderline west={2pt}{-2pt}{remq-titre},
	%title={\hspace{-10pt}\colorbox{rmq-perso-titre}{\rule[-3pt]{0pt}{14pt} \hspace{5pt}\MakeUppercase{#3}\ifstrempty{##1}{}{ (##1) }\hspace{10pt}}},
	%colbacktitle=#2,
	before upper={\colorlet{bg}{remq} \definecolor{bfcolor}{Hsb}{50,0.75,0.15}},
	after upper={\colorlet{bg}{white} \colorlet{bfcolor}{black}},
	%before upper={\tcolorboxsubtitle{#1}{##1}},
}

%\definecolor{programme-titre}{Hsb}{0,0.8,0.75}
%\colorlet{programme}{programme-titre!10}
%\newtcolorbox{tcolor-programme}{
%	colback=programme,
%	borderline west={1pt}{0pt}{programme-titre},
%	title={\hspace{-10pt}\colorbox{programme-titre}{\rule[-3pt]{0pt}{14pt} \hspace{5pt}\MakeUppercase{Programme}\hspace{10pt}}},
%	colbacktitle=programme,
%	before upper={\colorlet{bg}{programme} \definecolor{bfcolor}{Hsb}{0,0.75,0.15}},
%	after upper={\colorlet{bg}{white} \colorlet{bfcolor}{black}},
%	%before upper={\tcolorboxsubtitle{#1}{##1}},
%}
%
%\definecolor{deroule-titre}{Hsb}{0,0.8,0.75}
%\colorlet{deroule}{deroule-titre!10}
%\newtcolorbox{tcolor-deroule}{
%	colback=deroule,
%	borderline west={1pt}{0pt}{deroule-titre},
%	title={\hspace{-10pt}\colorbox{deroule-titre}{\rule[-3pt]{0pt}{14pt} \hspace{5pt}\MakeUppercase{Déroulé}\hspace{10pt}}},
%	colbacktitle=deroule,
%	before upper={\colorlet{bg}{deroule} \definecolor{bfcolor}{Hsb}{0,0.75,0.15}},
%	after upper={\colorlet{bg}{white} \colorlet{bfcolor}{black}},
%	%before upper={\tcolorboxsubtitle{#1}{##1}},
%}

\definecolor{fait-titre}{Hsb}{0,0,0.6} % Quand un résultat n'est pas vraiment catégorisé: intro d'une partie...
\colorlet{fait}{fait-titre!10}
\newtcolorbox{fait}[1][fait]{ % Option = couleur à prendre ( mettre nom de thm )
    top=5pt,
    bottom=5pt,
	colback={#1},
	borderline west={1pt}{0pt}{#1-titre},
}

\newcounter{compteurenonce}   % Pour des noms d'énoncés rares, a voir selon option cours