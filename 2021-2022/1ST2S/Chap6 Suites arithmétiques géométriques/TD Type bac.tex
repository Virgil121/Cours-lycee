\documentclass[a4paper,12pt,french]{article}

\usepackage[TD]{../../../Style}

\setstretch{1.2}

% Début du document
%%%%%%%%%%%%%%%%%%%
\begin{document}

%Exos récupérés sur la banque, niveau terminale donc adaptés

\begin{exercice}
Le dioxyde de carbone ou $CO_2$ est un des gaz à effet de serre.

En 1960, les émissions de $CO_2$ dans le monde ont été estimées à 15,4 milliards de tonnes. Depuis, on estime que ces émissions augmentent chaque année de $1,8\%$ par rapport à l'année précédente.

Pour tout entier naturel $n$, le nombre $u_n$ désigne les émissions de $CO_2$, exprimées en milliards de tonnes, pendant l'année $1960+n$. On a ainsi $u_0=15,4$.

\begin{enumerate}
\item Vérifier que $u_1=15,6772$.
\item Justifier que la suite $u$ est une suite géométrique dont on précisera la raison.
\item Exprimer $u_{n+1}$ en fonction de l'entier naturel $u_n$.
\item Selon ce modèle, déterminer l'année à partir de laquelle les émissions annuelles de $CO_2$ émises dans le monde dépasseront les 75 milliards de tonnes.
\item Un journaliste prétend que les émissions totales de $CO_2$ émises dans le monde depuis 1960 dépasseront les 2000 milliards de tonnes en 2030. A-t-il raison?
\end{enumerate}
\end{exercice}

\begin{exercice}
Une entreprise produit 500 tonnes de déchets en 2008.La production de déchets augmente de 12\% par an, depuis l'année 2008.

Pour tout entier naturel $n$, $d_n$ représente la quantité produite de déchets, exprimée en tonne, en l'année $2008+n$.

\begin{enumerate}
\item Déterminer la nature de la suite $d$ en précisant son premier terme et sa raison.
\item Selon ce modèle, vérifier qu'en 2018 l'entreprise a produit environ 1553 tonnes de déchets.
\end{enumerate}

En 2018, la nouvelle stratégie commerciale de l'entreprise change les procédés de fabrication afin de diminuer la masse produite de déchets. À compter de l'année 2018, la production des déchets baisse de 4\% par an au cours des deux années suivantes.
\begin{enumerate}[resume]
\item Selon ce nouveau modèle, estimer la quantité produite de déchets, exprimée en tonne, en 2020.On arrondira le résultat à l'unité.
%\item Résoudre sur $\N$ l’inéquation $1553 \times 0,96n<500$.
%\item En considérant que cette baisse de production de déchets de4\% par an se prolonge au-delà de l’année 2020, interpréter le résultat de la question précédente dans le contexte de l’exercice.
\end{enumerate}
\end{exercice}

\begin{exercice}
Le directeur d'un cinéma de centre-ville a vu le nombre d'entrées diminuer de 5\% par an depuis l'ouverture en 2000, année au cours de laquelle il avait comptabilisé 200 000 entrées.

Pour tout entier naturel $n$, on modélise par $u_n$ le nombre d'entrées dans ce cinéma l'année $2000+n$. On définit ainsi la suite $u$ sur $\N$. On a $u_0=200000$

\begin{enumerate}
    \item Quelle est la nature de la suite $u$? Justifier et donner la valeur de la raison.
    \item Exprimer $u_n$ en fonction de $n$, où $n$ est un entier naturel.
	\item Selon ce modèle, combien d'entrées le directeur a-t-il comptabilisé en 2010? Arrondir le résultat à l'unité.
	\item A l'aide de la calculatrice, déterminer au bout de combien d'années le nombre d'entrées dans ce cinéma aura été divisé par deux par rapport à celui de l'année d'ouverture du cinéma.
\end{enumerate}
\end{exercice}

\begin{exercice}
Un client dépose $880$\euro{} sur un compte bancaire, en janvier 2002. Cette somme est placée à intérêts composés au taux annuel de $2 \%$.

Pour tout entier naturel $n$, le nombre $u_n$ représente la somme d'argent, exprimée en euros, sur ce compte en janvier de l'année $2020+n$.

On définit ainsi la suite $u$ de premier terme $u_0$ où $u_0$ est la somme d'argent, exprimée en euros, se trouvant sur ce compte en janvier 2020. On a donc $u_0=880$.

\begin{enumerate}
\item Justifier que $u_1$ est égal à $897,6$.
\item Justifier que $u$ est une suite géométrique dont on précisera la raison.
\item Exprimer $u_{n+1}$ en fonction de $u_n$.
\item A l'aide de la calculatrice:
\begin{enumerate}
\item Calculer $u_{10}$ et interpréter sa valeur dans le contexte de l'exercice. On pourra donner une valeur approchée du résultat au dixième près.
\item Selon ce modèle, à partir de quelle année la somme d'argent sur le compte sera-t-elle supérieure à 2000\euro{} ? Justifier votre réponse.
\end{enumerate}
\end{enumerate}
\end{exercice}

\begin{exercice}
Lors de son embauche dans une société pour un contrat à durée indéterminée, Harry se voit proposer un salaire mensuel net de 1590\euro{}, plus une augmentation chaque mois de 15\euro{} par rapport au salaire mensuel net précédent.
\begin{enumerate}
\item Calculer le salaire mensuel net d'Harry le deuxième mois de son contrat ainsi que celui le troisième mois de son contrat.
\item On note $u_n$ le montant, exprimé en euros, du salaire mensuel net d'Harry au bout de $n$ mois, où $n$ est un entier naturel. Ainsi, on a $u_0=1590$.
\begin{enumerate}
\item Justifier que $u$ est une suite arithmétique dont on précisera le premier terme et la raison.
\item Exprimer, pour tout entier naturel $n$ non nul, $u_{n+1}$ en fonction de $u_n$.
\end{enumerate}
\item Au bout de combien de temps son salaire net dépassera-il 2000\euro{}? On pourra utiliser la calculatrice.
\end{enumerate}
\end{exercice}

\begin{exercice}
Deux salariés d'une entreprise, Paul et Pierre, ont perçu chacun, en fin d'année 2021, une prime suite aux bénéfices réalisés. Pour l'année 2021, Pierre a reçu 100 euros et Paul a reçu 120 euros.

L'entreprise étant prospère, Paul espère une augmentation annuelle de sa prime de 10 \euro{} par an et Pierre espère une augmentation de sa prime de $5 \%$ par an.

On modélise les primes perçues, exprimées en euros, par Paul et Pierre à l'aide de deux suites $u$ et $v$. Pour tout entier naturel $n$, $u_n$ et $v_n$ représentent donc la prime perçue respectivement par Paul et Pierre l'année $2021+n$. On a donc $u_0=100$ et $v_0=120$.

\begin{enumerate}
\item Calculer les primes que percevront Paul et Pierre en 2022 et en 2023.
\item Donner la nature de chacune des suites $u$ et $v$. Préciser, pour chacune, sa raison.
\item Exprimer alors $u_{n+1}$ en fonction de $u_n$ et $v_{n+1}$ en fonction de $v_n$.
\item Expliquer qui de Pierre et Paul percevra la plus haute prime en 2040, puis en 2060. On utilisera la calculatrice.
\end{enumerate}
\end{exercice}
\end{document}
