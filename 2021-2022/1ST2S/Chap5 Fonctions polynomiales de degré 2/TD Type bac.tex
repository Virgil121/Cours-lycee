\documentclass[a4paper,12pt,french]{article}

\usepackage[TD]{../../Style}

% Début du document
%%%%%%%%%%%%%%%%%%%
\begin{document}

\begin{exercice}
Un mobile se déplace sur une droite graduée en mètre.\\
Son abscisse p(t) sur cette droite graduée (exprimée en mètre) en fonction du temps
écoulé t (exprimé en minute) depuis le départ est donnée par :
$$p(t)=0,25t^2-t-3$$

\begin{enumerate}
    \item Quelle est la position du mobile à l'instant t = 0 min (c'est-à-dire au début du mouvement), puis à l'instant t = 2 min ?
    \item La courbe représentative de la fonction p est tracée ci-dessous.
    \begin{center}
        \includegraphics[scale=0.4]{TD Type bac images/fig1.png}
    \end{center}
    À l'aide de cette courbe, répondre aux questions suivantes :
    \begin{enumerate}
        \item Déterminer à quel(s) instant(s) le mobile est à la position $-3$
    \end{enumerate}
    \item
    \begin{enumerate}
        \item Montrer que, pour tout réel $t\geq 0$, $p(t)=0,25(t-6)(t+2)$
        \item À l'aide du tableau de signes de $p$ sur $[0~;~+ \infty [$ , déterminer à quels instants le
mobile a une abscisse positive ou nulle.
    \end{enumerate}
\end{enumerate}
\end{exercice}

\begin{exercice}
Une entreprise fabrique des lampes solaires. Elle ne peut pas produire plus de 5000 lampes
par mois.
Le résultat qu'elle peut réaliser en un mois, exprimé en centaines d'euros, est modélisé par
une fonction $b$ dont la représentation graphique est donnée ci-dessous. On rappelle que
lorsque le résultat est positif, on l'appelle bénéfice. L'axe des abscisses indique le nombre de
lampes produites et vendues exprimé en centaines.

En utilisant le graphique :

\begin{enumerate}
    

\compo[0.5]{\item Lire $b(10)$ et interpréter ce résultat
dans le contexte de l'exercice.
\item Déterminer avec la précision que
la lecture graphique permet, le
bénéfice maximal que peut réaliser
l'entreprise et les quantités de
lampes à fabriquer correspondantes.}{\includegraphics[scale=0.5]{TD Type bac images/fig2.png}}

\item La fonction $b$ définie sur l'intervalle $[0~;~+\infty[$ est définie par l'expression suivante :
$$b(x)=-3x^2+160x-1600$$
\begin{enumerate}
    \item Montrer $b(x)=-3(x-40)(x-\frac {40} 3)$
    \item Résoudre $b(x)=0$
    \item Donner la valeur exacte du maximum de la fonction $b$ et en quel nombre il est atteint.
\end{enumerate}
\end{enumerate}
\end{exercice}

\begin{exercice}
On considère la fonction du second degré $f$ définie sur $\R$ dont la représentation graphique est donnée ci-dessous dans un repère.
    \begin{center}
        \includegraphics[scale=0.4]{TD Type bac images/fig3.png}
    \end{center}

Par lecture graphique, répondre aux questions suivantes.

\begin{enumerate}
    \item Résoudre dans $\R$ l'équation $f(x)=0$
    \item Dresser le tableau de signes de $f$ sur $\R$
    \item Tracer l'axe de symétrie de la courbe représentative de la fonction $f$.
    \item Dresser le tableau de variations de la fonction $f$.
    \item Résoudre dans $\R$ l'inéquation $f(x)\geq 28$
\end{enumerate}
\end{exercice}

\begin{exercice}
Soit $f$ la fonction définie sur $[0~;~60]$ par $f(x)=-0,1x^2+6x-50$. La fonction
$f$ représente le résultat (en million d'euros) que réalise une entreprise pour la
fabrication de $x$ millions de jouets (on suppose que tous les jouets fabriqués sont
vendus). La représentation graphique $C$ de la fonction f est tracée ci-dessous.

  \begin{center}
        \includegraphics[scale=0.4]{TD Type bac images/fig4.png}
    \end{center}

\begin{enumerate}
    \item 
    \begin{enumerate}
        \item Déterminer graphiquement le bénéfice maximal et le nombre de jouets fabriqués pour lequel ce maximum est atteint.
        \item Résoudre graphiquement $f(x) > 35$. Interpréter votre réponse.
    \end{enumerate}
    \item On sait que cette fonction peut s'écrire sous la forme $f(x)=a(x-x_1 )(x-x_2 )$. Expliquer comment, à partir du graphique, on peut conjecturer que $f(x)=a(x-10 )(x-50 )$.
    \item \textbf{Démontrer} que, pour tout $x$ de $[0~;~60]$, $f(x)=-0,1(x-10 )(x-50 )$.
    \item Résoudre sur $[0~;~60]$, l'inéquation $f(x)<0$. Interpréter votre réponse.
\end{enumerate}
\end{exercice}

\begin{exercice} \

\compo[0.4]{Durant une balade en forêt,
un enfant se fabrique un arc
et des flèches. Il s'intéresse à
la trajectoire d'une de ses
flèches.
L'enfant décide de tirer sa
flèche par-dessus un hangar
désaffecté.
La trajectoire est une portion
de la courbe représentative
de la fonction $f$ située dans le
quart de plan rapporté au
repère $(O;\vv i;\vv j)$ ci-contre et
définie pour tout réel x, par
$f(x) = -0,2(x-5)^2 +6,5$.}{\includegraphics[width=\linewidth]{TD Type bac images/fig5.png}}

Une unité graphique correspond à 1 mètre dans la réalité.

\begin{enumerate}
    \item
    \begin{enumerate}
        \item De quelle hauteur, en mètre, la flèche est-elle tirée ? Justifier la réponse.
        \item Quelle hauteur maximale, en mètre, atteint-elle ? Justifier la réponse.
    \end{enumerate}
    \item On s'intéresse au pan du toit représenté par le segment $[AB]$, où $A(10;2)$ et $B(6;5.6)$ dans le repère $(O;\vv i; \vv j)$.\\
    Démontrer qu'une équation de la droite $(AB)$ est $y=-0,9x+11$
    \item Démontrer que pour tout réel $x$, $f(x)-g(x)=-0,2(x-5)(x-9,5)$.
    \item Quelles sont les coordonnées exactes du point d'impact sur le toit ?
\end{enumerate}
\end{exercice}

\begin{exercice} \

\compo[0.6]{La Casa Batllo est l'une des réalisations de l'architecte Antoni Gaudí à Barcelone.\\
Le grenier abrite une succession d'arcs en forme de paraboles
évoquant la cage thoracique d'un grand animal.
Le but de cet exercice est de déterminer une équation de l'un
de ces arcs (celui situé au fond sur la photo).}{\includegraphics[scale=0.5]{TD Type bac images/fig6.png}}

\begin{center}
    \includegraphics[width=0.5\linewidth]{TD Type bac images/fig7.png}
\end{center}

On modélise cet arc à l'aide d'une fonction polynôme du second degré $f$, qui a pour expression $$f(x)=ax^2+bx+c$$
où $a$, $b$ et $c$ sont des réels qui seront déterminés
dans cet exercice.\\
La parabole de sommet $S$,  qui représente graphiquement la fonction $f$ , est tracée ci-dessus
dans un repère orthonormé d'origine $O$ dont l'axe des abscisses est la droite $(OB)$ et l'axe
des ordonnées la droite $(OS)$ . \\
\textbf{L'unité est le mètre sur chacun des axes}.\\

La largeur $AB$ de l'arc au sol est égale à $6$ mètres. On a donc $A(-3;0)$ et $B(3;0)$

\begin{enumerate}
    \item Déterminer $f(-3)$ et $f(3)$
    \item En déduire que l'on a: $9a-3b+c=0$ et $9a+3b+c=0$
    \item A l'aide de la question 2, prouver que $b=0$ et $9a+c=0$
    \item Une personne de taille $1,70$m représentée par le segment $[HC]$ sur le graphique passe exactement sous l'arc en se plaçant à $1$m du point $B$ de l'arc.\\
    On a donc $C(2;1,7)$. Montrer que $4a+c=1,7$
    \item A l'aide des questions précédentes, déterminer l'expression de $f$.
\end{enumerate}
\end{exercice}

\begin{exercice}
Une entreprise produit mensuellement entre 200 et 3 000 panneaux solaires.\\
On modélise le résultat de l'entreprise réalisé sur la vente de $x$ centaines de panneaux
solaires par la fonction f définie sur l'intervalle $[2~; ~30]$ par : $$f(x)=-2x^2+90x-400$$

\begin{enumerate}
    \item On admet que, pour tout $x$ de l'intervalle  $[2~; ~30]$, on a $f(x)=-2(x-40)(x-5)$.\\
    Donner le tableau de signes de la fonction $f$ sur l'intervalle $[2~; ~30]$.
    \item A partir de quel volume de production de panneaux solaires le résultat réalisé par
l'entreprise est-il positif ?
    \item Donner le tableau de variations de la fonction f sur l'intervalle $[2~; ~30]$.
    \item Déterminer la valeur du bénéfice maximal et le volume de production correspondant.
\end{enumerate}
\end{exercice}
\end{document}
