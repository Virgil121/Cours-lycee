\documentclass[a4paper,12pt,french] {article}

\usepackage[sujet]{../../../Style}

\fancyhead[L]{02/02/2022}
\fancyhead[C]{\textbf{DS5 : Fonctions polynomiales de degré 2}}
\fancyhead[R]{\premiere ST2S 2}

\renewcommand{\baselinestretch}{1.2}

% Passage en A3
%\geometry{a3paper,landscape,bottom=7mm,twocolumn}
%\setlength{\headwidth}{39cm} %42cm-2*margin pour fancyhdr

\begin{document}

\rem[black]{L'usage de la calculatrice est interdit. La propreté et l'orthographe seront prises en compte. Tout le devoir peut être fait sur le sujet.}

Nom: \hfill Prénom: \hfill \

\begin{exercice} \

\compo[0.7]
{\fixcompo
On se donne la parabole $\mathcal P$ ci-contre, et $f:x \mapsto ax^2+c$ la fonction de degré deux associée.
\begin{enumerate}
\item Quel est le signe de $a$? \dotfill
\item Donner la valeur de $c$: \dotfill
\item Placer le sommet de $\mathcal P$ et préciser ses coordonnées: \dotfill
\item Quel est l'axe de symétrie de $f$? \dotfill
\item Donner les deux racines de $f$: \dotfill
\item Dresser le tableau de variations de $\mathcal P$:

\begin{center}
\begin{tikzpicture}
\tkzTabInit[lgt=1.2]{ /1, /2}{,,}
\end{tikzpicture}
\end{center}
\end{enumerate}
}
{
\begin{centrer}
\begin{tikzpicture}[scale=\echellepgf]
\begin{axis}[
styleglobal,
width=0.95*\echellepgfinv*\linewidth,
xmin=-4, xmax=4,
ymin=-5, ymax=5,
xtick distance=1,
ytick distance=1,
]
\addplot[styleplot,domain=(-4:4)]{(-0.5*x^2+2} node[pos=0.6,above right]{$\mathcal P$};;
\end{axis}
\end{tikzpicture}
\end{centrer}
}
\

\begin{enumerate}[start=7]
\item Déterminer $a$, et en déduire l'équation de $\mathcal P$, sous forme développée puis factorisée.
\end{enumerate}
\points 4
\end{exercice}

\begin{exercice}
Une entreprise produit mensuellement entre 200 et 3 000 panneaux solaires.\\
On modélise le résultat de l'entreprise réalisé sur la vente de $x$ centaines de panneaux
solaires par la fonction f définie sur l'intervalle $[2~; ~30]$ par : $$f(x)=-2x^2+90x-400$$

\begin{enumerate}
\item On admet que, pour tout $x$ de l'intervalle  $[2~; ~30]$, on a $f(x)=-2(x-40)(x-5)$.\\
Dresser le tableau de signes de la fonction $f$ sur l'intervalle $[2~; ~30]$:
\compo
{
\points 4
}
{
\begin{center}
\begin{tikzpicture}
\tkzTabInit[lgt=1.2]{ /1, /4}{,,}
\end{tikzpicture}
\end{center}
}
\item A partir de quel volume de production de panneaux solaires le résultat réalisé par
l'entreprise est-il positif ?
\item Donner le tableau de variations de la fonction f sur l'intervalle $[2~; ~30]$.
\item Déterminer la valeur du bénéfice maximal et le volume de production correspondant.
\end{enumerate}
\end{exercice}

\begin{exercice} \

\compo[0.5]
{\fixcompo
On s'intéresse à la trajectoire d'un ballon de
basketball lancé par un joueur faisant face
au panneau. Cette trajectoire est modélisée
dans le repère ci-contre.

Dans ce repère, l'axe des abscisses
correspond à la droite passant par les pieds
du joueur et la base du panneau.

On suppose que la position initiale du
ballon se trouve au point $J$ et que le segment
$[AB]$ représente le panneau sur lequel il
faut que le ballon rebondisse pour atteindre
le panier.

La trajectoire du ballon est assimilée à la courbe $C_f$ représentant une fonction $f$.
}
{
\begin{centrer}
\begin{tikzpicture}[scale=\echellepgf]
\begin{axis}[
styleglobal,
hauteurproptick,
width=0.9*\echellepgfinv*\linewidth,
xmin=-0.5, xmax=6,
ymin=-0.5, ymax=5.5,
xtick distance=1,
ytick distance=1,
%major grid style={line width=1pt},
]
\addplot[styleplot,domain=(0:4)]{-0.4*x^2+2.2*x+2} node[pos=0.9,above right] {$\mathscr C_f$};
\node[stylepoint,inner sep=1pt,fill=blue,label={-60:A}] (A) at (5.3,2.9) {};
\node[stylepoint,inner sep=1pt,fill=blue,label={60:B}] (B) at (5.3,3.5) {};
\draw[very thick] (A) -- (B);
\end{axis}
\end{tikzpicture}
\end{centrer}
}

\begin{enumerate}
\item \textbf{ Étude graphique}:\\
En exploitant la figure, répondre aux questions suivantes:
\begin{enumerate}
\item De quelle hauteur le ballon est-il lancé ?
\item Quelle est la hauteur du ballon lorsque $x = 0,5$ m ?
\item Quelle semble être la hauteur maximale atteinte par le ballon ?
    
\end{enumerate}
      \item \textbf{Étude de la fonction $f$:}\\
      La fonction $f$ est définie sur l'intervalle $[0; 6]$ par $f(x)= -0,4x^2+2,2x+2$:
      
\begin{enumerate}
\item Calculer $f(2)$ et $f(3.5)$. Interpréter ces résultats par une phrase.
\item En utilisant la symétrie de la courbe $C_f$, calculer la hauteur maximale atteinte par le ballon.
      
 \end{enumerate}     
\item \textbf{Le joueur a-t-il marqué ?}\\
Le panneau, représenté par le segment $[AB]$, se trouve à une distance de $5,3$ m du joueur.
Le point $A$ est à une hauteur de $2,9$ m et le point $B$ est à une hauteur de $3,5$ m
Le joueur a-t-il marqué ? Justifier par un calcul.
\end{enumerate}
\end{exercice}

%\detokenize{%}

\end{document}