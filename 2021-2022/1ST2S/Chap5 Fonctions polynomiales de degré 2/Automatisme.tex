\documentclass[french,12pt,a4paper]{article}

%\makeatletter
%\let\th@plain\relax
%\makeatother

\usepackage{../../../Style}

\renewcommand\tabularxcolumn[1]{m{#1}}

\setlist[itemize]{align=parleft,left=5pt..15pt}
\setlist[enumerate]{align=parleft,left=5pt..20pt}

\pagestyle{empty}

\geometry{paperwidth=16cm,paperheight=12cm}

\setlength{\columnsep}{2cm}

% Début du document
%%%%%%%%%%%%%%%%%%%
\begin{document}

\vfill

\begin{itemize}
\item Déterminer le signe de $a$.
\item Déterminer $c$.
%\item Identifier le sommet et préciser ses coordonnées.
%\item Tracer l'axe de symétrie sur le repère.
\end{itemize}

\begin{tikzpicture}
\begin{axis}[
styleglobal,
width=0.6*\linewidth,
xmin=-4, xmax=6,
ymin=-4, ymax=4,
xtick distance=1,
ytick distance=1,
]
\addplot[styleplot,domain=(-9:9)]{(x-1)^2-2};
\end{axis}
\end{tikzpicture}

\vfill

\newpage

\vfill

\begin{itemize}
\item Déterminer le signe de $a$.
\item Déterminer $c$ et les coordonnées du sommet.
%\item Donner les coordonnées du sommet.
%\item Tracer l'axe de symétrie sur le repère.
\end{itemize}

\begin{tikzpicture}
\begin{axis}[
styleglobal,
width=0.6*\linewidth,
xmin=-5, xmax=5,
ymin=-4, ymax=4,
xtick distance=1,
ytick distance=1,
]
\addplot[styleplot,domain=(-9:9)]{-0.15*x^2+2};
\end{axis}
\end{tikzpicture}

\vfill

\newpage

\vfill

\begin{itemize}
\item Déterminer le signe de $a$.
\item Déterminer $c$ et les coordonnées du sommet.
\item Déterminer les racines de la fonction associée.
%\item Donner les coordonnées du sommet.
%\item Tracer l'axe de symétrie sur le repère.
\end{itemize}

\begin{tikzpicture}
\begin{axis}[
styleglobal,
width=0.6*\linewidth,
xmin=-5, xmax=5,
ymin=-4, ymax=4,
xtick distance=1,
ytick distance=1,
labelgros,
]
\addplot[styleplot,domain=(-9:9)]{0.15*(x-3)*(x+3)};
\end{axis}
\end{tikzpicture}

\vfill

\end{document}
