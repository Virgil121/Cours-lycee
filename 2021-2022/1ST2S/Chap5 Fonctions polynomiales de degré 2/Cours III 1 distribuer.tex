\documentclass[a4paper,12pt,french]{article}

\usepackage[eleve,cours]{../../../Style}

\pagestyle{empty}

%\usepackage[theorems]{tcolorbox}

%\newtcbtheorem
%{defi}{Définition}
%{colback=red!5,colframe=red!60!black!80,fonttitle=
%\sffamily\bfseries}{th}

% Début du document
%%%%%%%%%%%%%%%%%%%
\begin{document}

\begin{prop}
Les fonctions du type $f:x \mapsto a(x-x_1)(x-x_2)$ sont des fonctions de degré 2. On dit que cette forme est l'écriture factorisée de $f$ (lorsqu'elle existe).
\end{prop}

\begin{rems}
Exos 79,80,(81->83) p125\\
Remarque par rapport à la suite?
\end{rems}

\begin{defin}
Soit $f$ une fonction de degré 2. On appelle \textbf{racines} de $f$ les solutions de l'équation $f(x)=0$. Ce sont donc les abscisses des points d'intersection entre la courbe représentative de $f$ et l'axe des abscisses.
\begin{centrer}
\begin{tikzpicture}
\begin{axis}[
styleglobal,
hauteurproptick,
width=0.7*\linewidth,
xmin=-2, xmax=6,
ymin=-1, ymax=2,
xtick distance=1,
ytick distance=1,
domain=(-6:6),
]
\addplot[styleplot]{0.5*(x-1)*(x-3)} node[pos=0.52,right] {$\mathscr C_f$};
\node[stylepoint,fill=red] (S) at (2,-0.5) {};
\node (sommet) at (3,-0.7) {Sommet};
\draw[->,>=latex,thick] (sommet.west) to[bend left=10] (S);
\node[stylepoint,fill=blue] (x1) at (1,0) {};
\node[stylepoint,fill=blue] (x2) at (3,0) {};
\node (racines) at (2,1) {Racines};
\draw[->,>=latex,thick] (racines) to[bend right=10] (x1);
\draw[->,>=latex,thick] (racines) to[bend left=10] (x2);
\end{axis}
\end{tikzpicture}
\end{centrer}
\end{defin}

\begin{rmq}
Si $f(x)=a(x-x_1)(x-x_2)$ $(a \neq 0)$, les racines de $f$ sont $x_1$ et $x_2$.
\end{rmq}

\begin{ex}
Les racines de la fonction $f:x \mapsto -2(x-2)(x+4)$ sont 2 et $-4$.
\begin{centrer}
\begin{tikzpicture}
\begin{axis}[
styleglobal,
%hauteurproptick,
width=0.7*\linewidth,
xmin=-8, xmax=6,
ymin=-5, ymax=20,
xtick distance=1,
ytick distance=5,
domain=(-6:6),
y post scale=0.2,
]
\addplot[styleplot]{-2*(x-2)*(x+4)} node[pos=0.52,right] {$\mathscr C_f$};
\node[stylepoint,fill=blue] (x1) at (2,0) {};
\node[stylepoint,fill=blue] (x2) at (-4,0) {};
\end{axis}
\end{tikzpicture}
\end{centrer}
On voit alors que la parabole associée par $f$ coupe l'axe des abscisses en $M(2;0)$ et $N(0;-4)$.
\end{ex}

\rem{Exos 84,85 p125}

\begin{propr}
Soit $f:x \mapsto a(x-x_1)(x-x_2)$ avec $a \neq 0$. On pose $s=\frac{x_1+x_2} 2$. Alors le sommet de la parabole associée a pour coordonnées $(s,f(s))$.
\end{propr}

\begin{ex}
\compobase{0pt}{t}{0.55}
{
Pour la parabole précédente, on a $s=\frac{2-4} 2=-1$, et

Les coordonnées du sommet de la parabole sont donc $(-1;18)$.
}
{
$\begin{aligned}[t]
f(s)&=f(-1)\\
&=-2(-1-2)(-1+4)\\
&=-2 \times (-3) \times 3\\
&=18
\end{aligned}
$
}
\end{ex}

\rem{Exos 26,27,29 p121}

\end{document}
