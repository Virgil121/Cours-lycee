\documentclass[twocolumn,landscape,a4paper,12pt,french]{article}

\usepackage[TD]{../../Style}

\renewcommand\tabularxcolumn[1]{m{#1}}

\setlist[itemize]{align=parleft,left=5pt..15pt}
\setlist[enumerate]{align=parleft,left=5pt..20pt}

\pagestyle{empty}

\setlength{\columnsep}{2cm}

% Début du document
%%%%%%%%%%%%%%%%%%%
\begin{document}

\newcommand{\contenu}{
\titre{Fonctions polynômiales de degré 2 - Exercices}

\begin{exercice}
Parmi les fonctions suivantes, lesquelles sont des fonctions polynômiales de degré 2? Si oui, préciser les valeurs de $a,b,c$.

\noindent $ f:x \mapsto x^2+5x-1 \hfill g:x \mapsto x^2 \hfill h:x \mapsto 1$

\noindent $ \hfill i:x \mapsto 2+5x-7x^3 \hfill j:x \mapsto (x-2)^2+3x-1 \hfill$
\end{exercice}

\begin{exercice} \

\compo[0.45]
{
Pour chaque parabole:

\begin{itemize}
\item Déterminer le signe de $a$.
\item Déterminer $c$.
\item Identifier le sommet et préciser ses coordonnées.
\item Tracer l'axe de symétrie sur le repère.
\end{itemize}
}
{
\vspace{-5mm}
\begin{centrer}
\begin{tikzpicture}
\begin{axis}[
styleglobal,
width=0.95*\linewidth,
xmin=-4, xmax=9,
ymin=-4, ymax=4,
xtick distance=1,
ytick distance=1,
]
\addplot[styleplot,domain=(-9:9)]{(x+1)^2};
\addplot[color=red,styleplot,densely dashed,domain=(-9:9)]{-0.05*(x-3.5)^2+1.5};
\addplot[color=blue,styleplot,densely dotted,domain=(4:8)]{-10*(x-6)^2+3};
\end{axis}
\end{tikzpicture}
\end{centrer}
}
\end{exercice}

\begin{exercice} \

\compo[0.5]
{
Déterminer l'équation des trois paraboles suivantes (sous la forme $y=ax^2+c$):
}
{
\vspace{-5mm}
\begin{centrer}
\begin{tikzpicture}
\begin{axis}[
styleglobal,
width=0.95*\linewidth,
xmin=-4, xmax=4,
ymin=-2, ymax=6,
xtick distance=1,
ytick distance=1,
]
\addplot[styleplot,domain=(-9:9)]{(x^2+2};
\addplot[color=red,styleplot,densely dashed,domain=(-9:9)]{4*x^2-1};
\addplot[color=blue,styleplot,densely dotted,domain=(-9:9)]{-0.5*x^2+3};
\end{axis}
\end{tikzpicture}
\end{centrer}
}
\end{exercice}

\begin{exercice}
Identifier $a$, $x_1$ et $x_2$ pour les fonctions suivantes:

\noindent $\hfill f:x \mapsto 3(x-5)(x-9) \hfill g:x \mapsto 4(x-1)(x+12) \hfill$
\center{$h:x \mapsto -2\left(x+ \frac 5 2 \right)(x-0.25)$}
\end{exercice}
}

\contenu

\newpage

\setcounter{exercice}{0}

\contenu
\end{document}
