\documentclass[landscape,twocolumn,a4paper,12pt,french]{article}

\usepackage[TD]{../../Style}

\pagestyle{empty}

\setlength{\columnsep}{2cm}

% Début du document
%%%%%%%%%%%%%%%%%%%
\begin{document}

\setcounter{exercice}{2}

\begin{exercice} \

\compo[0.5]
{
Déterminer l'équation des trois paraboles suivantes (sous la forme $y=ax^2+c$):
}
{
\vspace{-5mm}
\begin{centrer}
\begin{tikzpicture}
\begin{axis}[
styleglobal,
width=0.95*\linewidth,
xmin=-4, xmax=4,
ymin=-2, ymax=6,
xtick distance=1,
ytick distance=1,
]
\addplot[styleplot,domain=(-4:4)]{(x^2+2};
\addplot[color=red,styleplot,densely dashed,domain=(-4:4)]{4*x^2-1};
\addplot[color=blue,styleplot,densely dotted,domain=(-4:4)]{-0.5*x^2+3};
\end{axis}
\end{tikzpicture}
\end{centrer}
}
\end{exercice}

\begin{correction} \

\vspace{1mm}
\compo[0.5]
{
Pour la parabole bleue:
\begin{itemize}
\item On a $c=3$ (ordonnée à l'origine) donc $f:x \mapsto ax^2+3$.
\item On prend le point $A(2;1)$.
\item On a alors:
$$\begin{aligned}
f(2)&=1\\
a \times 2^2 +3&=1\\
4a+3&=1\\
4a&=-2\\
a&=\frac{-2}4=-0.5
\end{aligned}$$
Il suit alors qu'on a $f:x \mapsto -0.5x^2+3$ donc l'équation de la parabole est: $$y=-0.5x^2+3$$
\end{itemize}
}
{
\begin{centrer}
\begin{tikzpicture}
\begin{axis}[
styleglobal,
width=0.95*\linewidth,
xmin=-4, xmax=4,
ymin=-2, ymax=6,
xtick distance=1,
ytick distance=1,
]
\addplot[color=blue,styleplot,densely dotted,domain=(-4:4)]{-0.5*x^2+3};
\node[stylepoint,fill=blue,label={30:A}] at (2,1) {};
\end{axis}
\end{tikzpicture}
\end{centrer}
}

\vspace{1mm}
\compo[0.5]
{
Pour la parabole rouge:
\begin{itemize}
\item On a $c=-1$ (ordonnée à l'origine) donc $f:x \mapsto ax^2-1$.
\item On prend le point $B(1;3)$.
\item On a alors:
$$\begin{aligned}
f(1)&=3\\
a \times 1^2 -1&=3\\
1a-1&=3\\
a&=4
\end{aligned}$$
Il suit alors qu'on a $f:x \mapsto 4x^2-1$ donc l'équation de la parabole est: $$y=4x^2-1$$
\end{itemize}
}
{
\begin{centrer}
\begin{tikzpicture}
\begin{axis}[
styleglobal,
width=0.95*\linewidth,
xmin=-4, xmax=4,
ymin=-2, ymax=6,
xtick distance=1,
ytick distance=1,
]
\addplot[color=red,styleplot,densely dashed,domain=(-4:4)]{4*x^2-1};
\node[stylepoint,fill=red,label={30:B}] at (1,3) {};
\end{axis}
\end{tikzpicture}
\end{centrer}
}

\vspace{5mm}
\compo[0.5]
{
Pour la parabole noire:
\begin{itemize}
\item On a $c=2$ (ordonnée à l'origine) donc $f:x \mapsto ax^2+2$.
\item On prend le point $C(1;3)$.
\item On a alors:
$$\begin{aligned}
f(1)&=3\\
a \times 1^2 +2&=3\\
1a+2&=3\\
a&=1
\end{aligned}$$
Il suit alors qu'on a $f:x \mapsto x^2+2$ donc l'équation de la parabole est: $$y=x^2+2$$
\end{itemize}
}
{
\begin{centrer}
\begin{tikzpicture}
\begin{axis}[
styleglobal,
width=0.95*\linewidth,
xmin=-4, xmax=4,
ymin=-2, ymax=6,
xtick distance=1,
ytick distance=1,
]
\addplot[styleplot,domain=(-4:4)]{(x^2+2};
\node[stylepoint,fill=black,label={-10:C}] at (1,3) {};
\end{axis}
\end{tikzpicture}
\end{centrer}
}

\end{correction}

\end{document}
