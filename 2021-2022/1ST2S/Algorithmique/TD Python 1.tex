\documentclass[a4paper,12pt,french]{article}

\usepackage[minted,TD]{../../../Style}

% Début du document
%%%%%%%%%%%%%%%%%%%
\begin{document}

\titre{TP Python}

\section{Python: La Boucle For}

En python, une boucle permet de répéter plusieurs fois de suite un même traitement.\\
Lorsque le nombre n d'itérations est connu à l'avance, on parle de \textbf{boucle bornée} et on utilise un compteur qui s'incrémente de 1 (généralement ) à chaque itération jusqu'à la valeur n.

\begin{minted}{python} 
for i in range(n): 
	bloc instructions # sera répétée n fois
	#la variable i prenant successivement 
	les valeurs 0, 1, ..., n-1
\end{minted}

     

\begin{exercice}
Pour calculer un terme de la suite $\left(u_n\right)$
définie sur $ \mathbb{N}$, on utilise la fonction Python ci-dessous:


\begin{minted}{python}
def u(n):
    u=12
    for i in range(n): 
        u=2*u-3
    return u
\end{minted}


\begin{enumerate}
    \item Quelle la valeur de la variable u à la fin du programme pour $n=3$?
    \item Proposer une forme récurrente de la suite $\left(u_n\right)$
    \item Réaliser le programme sur le notebook de Basthon et vérifier votre résultat pour $n=3$
\end{enumerate}
\end{exercice}

\begin{exercice}
Pour calculer un terme de la suite $\left(v_n\right)$
définie sur $ \mathbb{N}$, on utilise la fonction Python ci-dessous:

\begin{minted}{python} 
def v(n):
    v=4
    for i in range(n):
        v=2*v-5
    return v
\end{minted}

\begin{enumerate}
    \item Déterminer $v_0$
    \item Proposer une forme récurrente de la suite $\left(v_n\right)$
    \item Déterminer la valeur de $v_7$ en utilisant le notebook de Basthon
\end{enumerate}
\end{exercice}




\begin{exercice}
Compléter la fonction \mintinline{python}|somme(n)| pour qu'elle renvoie la somme $1+2+3+\ldots n$

\begin{minted}{python} 
def somme(n):
	S=0
    for i in range(n+1) :  # Pour i allant de 0 à n faire
        S=......................
    return(S)

\end{minted}  

\end{exercice}

\begin{exercice}
On considère la suite définie pour tout entier naturel $n$ par: $u_n=2n^2-n+1$

\begin{enumerate}
    \item Ecrire une fonction python \mintinline{python}|rang(n)| qui permet de retourner le terme de rang n de cette suite pour une valeur de n donnée
    \item Que renvoie l'instruction \mintinline{python}|rang(5)| ? Quelle valeur renvoie-t-elle ?
\end{enumerate}
\end{exercice}

\end{document}
