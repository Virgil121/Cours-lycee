\documentclass[a4paper,12pt,french]{article}

\usepackage[minted,TD]{../../../Style}

% Début du document
%%%%%%%%%%%%%%%%%%%
\begin{document}

\titre{Tableur et Python}

\section{Utilisation du tableur}

\begin{exercice}
On utilise un tableur pour calculer les termes de deux suites $u$ et $v$.

\begin{center}
    \includegraphics[scale=0.5]{tab1.png}
\end{center}

\begin{enumerate}
    \item D'après la formule saisie en \textbf{B4}, et recopiée vers le bas, écrire un procédé pour passer du terme $u_1$ à $u_2$
    \item Calculer $u_4$
    \item Généraliser ce procédé entre $u_{n+1}$ et $u_n$
    \item On garde le même procédé pour passer de  $v_{n+1}$ à $v_n$. Pourquoi n'obtient-on pas les mêmes résultats ? Argumenter.
\end{enumerate}
\end{exercice}

\begin{exercice}
On considère la relation de récurrence:

$$u_{n+1}=\dfrac{u_n}{2} +5$$

\compo[0.7]{\begin{enumerate}
    \item Interpréter par une phrase cette relation entre deux termes consécutifs de la suite $u$
    \item Le terme initial $u_1$ n'est pas donné mais sera saisi en \textbf{B2}.
    Indiquer la formule à saisir en \textbf{B3, et recopiée vers le }, pour obtenir les termes de la suite.
    \item Calculer les premiers termes pour :
    \begin{enumerate}
        \item $u_1=2$
        \item $u_1=10$
        \item $u_1=6$
    \end{enumerate}
\end{enumerate}}{\includegraphics[scale=0.5]{tab2.png}}
\end{exercice}

\begin{exercice}
Soit $\left(v_n\right)$ la suite définie par $v_0=1$ et, pour tout entier naturel $n$, par $v_{n+1}=1,02 v_n+100$

\begin{enumerate}
    \item La suite $\left(v_n\right)$ est-elle définie par récurrence ou de manière explicite ?
    \item Calculer $v_1$ et $v_2$
    \item A l'aide de la calculatrice puis du tableur, déterminer une valeur approchée au centième de $v_{100}$
\end{enumerate}
\end{exercice}

\begin{exercice} \\
\begin{enumerate}
    \item Calculer les cinq premiers termes de chaque suite
    \begin{enumerate}
        \item $\left(u_n\right)$ définie par $u_n=\dfrac{1}{n^2}-7n$, avec $n\in \mathbb{N}^{*}$
        \item $\left(v_n\right)$ définie par $v_n=5(n+2)^2-3$, avec $n \in \mathbb{N}$
    \end{enumerate}
   \item Calculer le dix-huitième terme des deux suites précédentes
\end{enumerate}
\end{exercice}

\begin{exercice} \\

\begin{enumerate}
    \item Représenter graphiquement à l'aide du tableur les 5 premiers termes de la suite $\left(u_n\right)$ définie pour tout $n\in \mathbb{N}$
    par $u_n=n^2-2n$
    \item Représenter graphiquement à l'aide du tableur les 5 premiers termes de la suite $\left(u_n\right)$ définie par $u_0=8$ et pour tout $n\in \mathbb{N}$, $u_{n+1}=-\dfrac{1}{2}u_n+1$
\end{enumerate}
\end{exercice}



\section{Python: La Boucle For}

En python, une boucle permet de répéter plusieurs fois de suite un même traitement.\\
Lorsque le nombre n d'itérations est connu à l'avance, on parle de \textbf{boucle bornée} et on utilise un compteur qui s'incrémente de 1 (généralement ) à chaque itération jusqu'à la valeur n.

\begin{minted}{python} 
for i in range(n): 
	bloc instructions # sera répétée n fois
	#la variable i prenant successivement 
	les valeurs 0, 1, ..., n-1
\end{minted}

     

\begin{exercice}
Pour calculer un terme de la suite $\left(u_n\right)$
définie sur $ \mathbb{N}$, on utilise la fonction Python ci-dessous:


\begin{minted}{python}
def u(n):
    u=12
    for i in range(n): 
        u=2*u-3
    return u
\end{minted}


\begin{enumerate}
    \item Quelle la valeur de la variable u à la fin du programme pour $n=3$?
    \item Proposer une forme récurrente de la suite $\left(u_n\right)$
    \item Réaliser le programme sur le notebook de Basthon et vérifier votre résultat pour $n=3$
\end{enumerate}
\end{exercice}

\begin{exercice}
Pour calculer un terme de la suite $\left(v_n\right)$
définie sur $ \mathbb{N}$, on utilise la fonction Python ci-dessous:

\begin{minted}{python} 
def v(n):
    v=4
    for i in range(n):
        v=2*v-5
    return v
\end{minted}

\begin{enumerate}
    \item Déterminer $v_0$
    \item Proposer une forme récurrente de la suite $\left(v_n\right)$
    \item Déterminer la valeur de $v_7$ en utilisant le notebook de Basthon
\end{enumerate}
\end{exercice}




\begin{exercice}
Compléter la fonction \mintinline{python}|somme(n)| pour qu'elle renvoie la somme $1+2+3+\ldots n$

\begin{minted}{python} 
def somme(n):
	S=0
    for i in range(n+1) :  # Pour i allant de 0 à n faire
        S=......................
    return(S)

\end{minted}  

\end{exercice}

\begin{exercice}
On considère la suite définie pour tout entier naturel $n$ par: $u_n=2n^2-n+1$

\begin{enumerate}
    \item Ecrire une fonction python \mintinline{python}|rang(n)| qui permet de retourner le terme de rang n de cette suite pour une valeur de n donnée
    \item Que renvoie l'instruction \mintinline{python}|rang(5)| ? Quelle valeur renvoie-t-elle ?
\end{enumerate}
\end{exercice}

\end{document}
