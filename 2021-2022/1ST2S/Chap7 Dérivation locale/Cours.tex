\documentclass[a4paper,12pt,french]{article}

\usepackage[cours]{../../../Style}
%\usepackage{wrapfig}
%\makeatletter
%\setlength{\parskip}{1ex}
%\newcommand{\@minipagerestore}{\setlength{\parskip}{1ex}}
%\frenchsetup{ItemLabels=\textendash}
% Début du document
%%%%%%%%%%%%%%%%%%%
\begin{document}

% Compilation avec lualatex nécessaire pour utiliser declare function dans tikz, à cause d'un conflit avec babel ( statut du ; )

\title{Dérivation - Point de vue local}
\maketitle

\begin{programme}
\item Sécantes à une courbe passant par un point donné, taux de variation en un point
\item Tangente à une courbe en un point, définie comme position limite des sécantes
\item Nombre dérivé en un point défini comme limite du taux de variation
\item Equation réduite de la tangente en un point.
\item Capacités
\begin{itemize}
\item Interprétation du nombre dérivé comme coeff directeur de la tangente
\item Construire la tangente à une courbe en un point
\item Déterminer l'équation réduite de la tangente à une courbe en un point
\end{itemize}
\end{programme}

\section{Sécantes et tangentes}
%Tangente, lim des sécantes. Lecture graphique de la dérivée
\begin{defin}
\compo[0.6]
{
Soit $f$ une fonction, avec $A$ et $M$ deux points sur la courbe de $f$.

La droite $(AM)$ est appelée \textbf{sécante} de la courbe de $f$.
}
{
\vspace{-1.5em}
\begin{center}
\begin{tikzpicture}
\begin{axis}[
styleglobal,
width=0.9*\linewidth,
xmin=-1, xmax=4,
ymin=-1, ymax=2,
xtick distance=1,
ytick distance=1,
ticks=none,
declare function={f(\x)=(0.3*(\x-1)*(\x-1)-0.5);}, % Erreur de compilation avec pdflatex, à cause du ;. Pour y remédier, passage à luatex ou ne pas déclarer la fonction
]
\addplot[styleplot]{f(x)} node [pos=0.9,left] {$\mathscr C_f$};
\node[stylepoint,fill=blue,label=below:$A$] (A) at (0.5,{f(0.5)}) {};
\node[stylepoint,fill=blue,label=above left:$M$] (B) at (3,{f(3)}) {};

\draw[line width=1.3pt,shorten <= -20cm,shorten >= -20cm,densely dotted] (A) -- (B);
\end{axis}
\end{tikzpicture}
\end{center}
}
\end{defin}

\begin{propr}
\compo[0.6]
{
A mesure que $M$ se rapproche du point $A$, la sécante $(AM)$ se rapproche d'une autre droite, appelée \textbf{tangente de $\mathscr C_f$ en $A$}, qui épouse la courbe de $f$ près de $A$.
}
{
\vspace{-1.5em}
\begin{center}
\begin{tikzpicture}
\begin{axis}[
styleglobal,
width=0.9*\linewidth,
xmin=-3, xmax=4,
ymin=-1, ymax=3,
xtick distance=1,
ytick distance=1,
ticks=none,
declare function={f(\x)=e^(0.75*(\x-1)-0.5);}
]
\addplot[styleplot]{f(x)} node [pos=0.65,above left] {$\mathscr C_f$};
\node[stylepoint,fill=blue,label=above:$A$] (A) at (0.5,{f(0.5)}) {};
\node[stylepoint,fill=blue,inner sep=1.3pt] (B) at (2.5,{f(2.5)}) {};
\node[stylepoint,fill=blue,inner sep=1.4pt] (C) at (1.5,{f(1.5)}) {};
\node[stylepoint,fill=blue,inner sep=1.5pt] (D) at (0.75,{f(0.75)}) {};
\node[stylepoint,fill=blue,inner sep=1.6pt] (E) at (0.625,{f(0.625)}) {};
\draw[thick,shorten <= -20cm,shorten >= -20cm,densely dotted] (A) -- (B);
\draw[thick,shorten <= -20cm,shorten >= -20cm,densely dotted] (A) -- (C);
\draw[thick,shorten <= -20cm,shorten >= -20cm,densely dotted] (A) -- (D);
\draw[line width=1.3pt,shorten <= -20cm,shorten >= -20cm,densely dotted] (A) -- (E);
\addplot[styleplot,color=Crimson,ultra thick]{0.75*e^(0.75*(-0.5)-0.5)*(x-0.5)+f(0.5)};
\end{axis}
\end{tikzpicture}
\end{center}
}
\end{propr}

\rem{Voir fichier géogebra}

\section{Lecture du nombre dérivé}

\begin{defin}
On appelle nombre dérivé de $f$ en $a$, noté $f'(a)$, le coefficient directeur de la tangente à la courbe de $f$ au point d'abscisse $a$.
\end{defin}

\begin{ex}
\compo[0.5]
{\setlength{\parskip}{1ex}
On a représenté une fonction $f$ ci-contre.

Pour obtenir $f'(2)$, on place $A$ défini comme le point de $\mathscr C_f$ d'abscisse $2$, puis on détermine le coefficient directeur de la tangente de $\mathscr C_f$ passant par $A$. On a alors $f'(2)=\frac{-2}{2} = -1$.

Pour obtenir $f'(-1)$, on place $B$, et on détermine le coefficient directeur de la tangente associée: $f'(-1)=\frac 2 1 = 2$. En utilisant l'ordonnée à l'origine, on en déduit que l'équation de cette tangente est $y=2x+3$.
}
{
\begin{center}
\begin{tikzpicture}
\begin{axis}[
styleglobal,
width=0.9*\linewidth,
xmin=-3, xmax=5,
ymin=-1, ymax=4,
xtick distance=1,
ytick distance=1,
minor x tick num=1,
minor y tick num=1,
declare function={f(\x)=-0.5*(\x-1)*(\x-1)+3;}%0.2*\x*(\x+2)*(\x-3);}
]
\addplot[styleplot] {f(x)} node[pos=0.76,left] {$\mathscr C_f$};
\node[stylepoint,fill=blue,label=above right:$A$] (A) at (2,{f(2)}) {};
\node[stylepoint,fill=blue,label=above left:$B$] (B) at (-1,{f(-1)}) {};

\draw[color=blue,dashed,very thick] (2, 0) -- (A);%node [pos=0,below] {$a$};
\draw[color=blue,dashed,very thick] (-1, 0) -- (B);% node [pos=0,below] {$a+h$};

\addplot[styleplot,densely dashed,color=DarkBlue] {{-1*(x-2)+f(2)}};
\addplot[styleplot,densely dashed,color=DarkRed] {{2*(x+1)+f(-1)}};

\draw[->,thick,color=DarkBlue] (A) -- (4,{f(2)});
\draw[->,thick,color=DarkBlue] (4,{f(2)}) -- (4,0.5);
\draw[->,thick,color=DarkRed] (B) -- (-1,3);
\draw[->,thick,color=DarkRed] (-1,3) -- (0,3);

\end{axis}
\end{tikzpicture}
\end{center}
}
\end{ex}

\rem{Lectures de dérivées: Exos 17,18 p148\\Exos 35,36,37 p150\\
	 Constructions de tangentes: Exos 43,44,45 p151\\
	 Equations de tangentes: Exos 48,49,50,51,(52) p151\\
	 Interprétation de la dérivée: Exos 42, (46), (53) }

\section{Lien avec le taux de variation}

\begin{defin}
\compo[0.55]
{
Soit $f$ une fonction, avec $A$ et $M$ deux points sur la courbe de $f$ d'abscisses respectives $a$ et $a+h$ ($h \neq 0$).

Le taux de variation de $f$ entre $a$ et $a+h$ (autrement dit le coefficient directeur de la sécante associée) vaut: $$\frac{\text{déplacement vertical}}{\text{déplacement horizontal}}=\frac{f(a+h)-f(a)}{h}$$
}
{
\vspace{-1.5em}
\begin{center}
\begin{tikzpicture}
\begin{axis}[
styleglobal,
width=1*\linewidth,
xmin=-3, xmax=4,
ymin=-1, ymax=4,
xtick distance=1,
ytick distance=1,
ticks=none,
declare function={f(\x)=e^(0.75*(\x-1)-0.5);}
]
\addplot[styleplot]{f(x)} node [pos=0.8,left] {$\mathscr C_f$};
\node[stylepoint,fill=blue,label=above:$A$] (A) at (0.5,{f(0.5)}) {};
\node[stylepoint,fill=blue,label=above left:$M$] (B) at (3,{f(3)}) {};
%
\draw[color=blue,dashed,very thick] (0.5, 0) -- (A) node [pos=0,below] {$a$};% -- (0,{f(0.5)}) node [pos=1,left] {$f(x)$};
\draw[color=blue,dashed,very thick] (3, 0) -- (B) node [pos=0,below] {$a+h$};% -- (0,{f(1.2)}) node [pos=1,left] {$f(y)$};
\draw[<->,color=DarkGreen,densely dashed,ultra thick] (0.5,0) -- (3,0) node[pos=0.5,below] {$h$};
%
\draw[color=blue,dashed,very thick] (0,{f(0.5)}) -- (A) node [pos=0,left] {$f(a)$};% -- (0,{f(0.5)}) node [pos=1,left] {$f(x)$};
\draw[color=blue,dashed,very thick] (0,{f(3)}) -- (B) node [pos=0,left] {$f(a+h)$};% -- (0,{f(1.2)}) node [pos=1,left] {$f(y)$};
\draw[<->,color=DarkGreen,densely dashed,ultra thick] (0,{f(0.5)}) -- (0,{f(3)}) node[pos=0.5,left] {$f(a+h)-f(a)$};
%
\draw[line width=1.3pt,shorten <= -20cm,shorten >= -20cm,densely dotted] (A) -- (B);
\end{axis}
\end{tikzpicture}
\end{center}
}
\end{defin}

\begin{propr}
Si le taux de variation $\frac{\text{déplacement vertical}}{\text{déplacement horizontal}}=\frac{f(a+h)-f(a)}{h}$  se rapproche d'un nombre réel quand $h$ \textit{tend} vers $0$, on dit que $f$ est \textbf{dérivable} en $a$, et le nombre en question est noté $f'(a)$.
\end{propr}

\begin{ex}
\compo[0.65]
{
On se donne la fonction $f:x \mapsto x^2$, et le point $A(1;1)$ sur $\mathscr C_f$. Soit $h \in \R$, $h \neq 0$.
On a:
\begin{itemize}
\item $f(1)=1^2=1$
\item $f(1+h)=(1+h)^2=1+2h+h^2$
\end{itemize}
Alors:

$$\begin{aligned}\frac{f(1+h)-f(1)}{h}&=\frac{1+2h+h^2-1}{h}\\
									 &=\frac{2h+h^2}{h}\\
									 &=2+h \end{aligned}$$
Cette quantité se rapproche de $2$ lorsque $h$ tend vers $0$.

Alors $f$ est dérivable en $1$ et $f'(1)=2$.
}
{
\begin{tikzpicture}
\begin{axis}[
styleglobal,
width=0.9*\linewidth,
xmin=-1, xmax=3,
ymin=-1, ymax=4,
xtick distance=1,
ytick distance=1,
minor x tick num=1,
minor y tick num=1,
declare function={f(\x)=\x^2;}
]
\addplot[styleplot] {f(x)} node[pos=0.76,left] {$\mathscr C_f$};
\node[stylepoint,fill=blue,label=above right:$A$] (A) at (1,{f(1)}) {};

\addplot[styleplot,densely dashed,color=DarkBlue] {{2*(x-1)+f(1)}};
\end{axis}
\end{tikzpicture}
}
\end{ex}
%Formule générale avec schéma

\end{document}
