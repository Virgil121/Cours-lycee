\documentclass[12pt,a4paper,french] {article}

\pagestyle{empty}

\usepackage{../../../Style}

\renewcommand{\baselinestretch}{1.25}

\begin{document}
 
\titre{Direction instantanée}

\renewcommand\tabularxcolumn[1]{p{#1}}

\noindent\begin{tabularx}{\linewidth}{X|X}

Trois voitures $A$, $B$ et $C$ sont en train de tourner sur un rond-point. Imaginons que chaque conducteur redresse d'un coup son volant pour le remettre droit. Tracer alors une \textbf{demi-droite} représentant la direction que prendra chacune des trois voitures.
\begin{center}
\includegraphics[width=0.9\linewidth]{Voiture_NB.jpg}
\end{center}
&
On représente maintenant le rond-point par un cercle et les voitures par des points $A$ et $B$. Tracer cette fois-ci une \textbf{droite} représentant la direction que prendra chacune des voitures si elles redressent leur volant d'un coup.

\begin{center}
\begin{tikzpicture}
\draw[thick] (-4,-4) rectangle (4,4);
\draw[thick] (0,0) circle (2.5);
\draw[thick,densely dashed,->] ({1.6*cos(135)},{1.6*sin(135)}) arc[start angle=135,end angle=405,radius=1.6cm];
\node[stylepoint,fill=blue,label=90:$A$] (A) at ({2.5*cos(100)},{2.5*sin(100)}) {};
\node[stylepoint,fill=blue,label=200:$B$] (A) at ({2.5*cos(190)},{2.5*sin(190)}) {};
\end{tikzpicture}
\end{center}
\\ \hline
On suppose maintenant que les voitures $A$ et $B$ se déplacent le long de la courbe d'une fonction $f$. Tracer les droites associées comme précédemment.

\begin{center}
\begin{tikzpicture}[scale=\echellepgf]
\begin{axis}[
styleglobal,
width=0.9*\echellepgfinv*\linewidth,
xmin=-3.5, xmax=6.5,
ymin=-4.5, ymax=4.5,
xtick distance=1,
ytick distance=1,
minor x tick num=1,
minor y tick num=1,
]
\addplot[styleplot] {0.1*x*(x+2)*(x-4)} node[pos=0.6,right] {$\mathscr C_f$};
\node[stylepoint,fill=blue,label=135:$A$] at (4,0) {};
\node[stylepoint,fill=blue,label=-90:$B$] at (1,-0.9) {};
\node[stylepoint,fill=blue,label=90:$C$] at ({(2-2*sqrt(7))/3},{8/135*(7*sqrt(7)-10)}) {};
%\addplot[styleplot] {-0.25*(x+2)*(x-4)}; %Dérivée en 3,2,-2 pas trop compliquée (-1,-0.5,0.5)
\end{axis}
\end{tikzpicture}
\end{center}
&
Cette fois-ci, les droites associées à $A$ et $B$ sont déjà tracées. Donner le coefficient directeur de chaque droite.

\begin{center}
\begin{tikzpicture}[scale=\echellepgf]
\begin{axis}[
styleglobal,
width=0.9*\echellepgfinv*\linewidth,
xmin=-4, xmax=7,
ymin=-4, ymax=10,
xtick distance=1,
ytick distance=1,
minor x tick num=0,
minor y tick num=0,
declare function={f(\x)=0.5*(\x-1)*(\x-1)-2;}%0.2*\x*(\x+2)*(\x-3);}
]
\addplot[styleplot] {f(x)} node[pos=0.6,right] {$\mathscr C_f$};
\node[stylepoint,fill=blue,label=135:$A$] at (3,{f(3)}) {};
\node[stylepoint,fill=blue,label=20:$B$] at (-3,{f(-3)}) {};
\node[stylepoint,fill=blue,label=90:$C$] at (1,{f(1)}) {};
\addplot[styleplot,densely dashed,color=DarkBlue] {{2*(x-3)+f(3)}};
\addplot[styleplot,densely dashed,color=DarkRed] {{-4*(x+3)+f(-3)}};
\addplot[styleplot,densely dashed,color=DarkGreen] {{f(1)}};
%\addplot[styleplot] {-0.25*(x+2)*(x-4)}; %Dérivée en 3,2,-2 pas trop compliquée (-1,-0.5,0.5)
\end{axis}
\end{tikzpicture}
\end{center}
\end{tabularx}

\end{document}
