\documentclass[a4paper,12pt,french] {article}

\usepackage[sujet]{../../../Style}

\fancyhead[L]{27/04/2022}
\fancyhead[C]{\textbf{DS7 : Dérivation 1}}
\fancyhead[R]{\premiere ST2S 2}

\begin{document}

\rem{L'usage de la calculatrice est interdit. La propreté et l'orthographe seront prises en compte. Tout le devoir peut être fait sur le sujet.}

Nom: \hfill Prénom: \hfill \

\setstretch{1.2}

\begin{exercice} \ 
On se donne la fonction $f$ représentée ci-dessous. Déterminer:
\begin{itemize}
\item $f'(0)= \dotfill \quad f'(0,6)= \dotfill \quad f'(-2,5)=\dotfill$
\item Le coefficient directeur de la tangente de $\mathscr C_f$ en $C$: $f'(\ldots \ldots)=\dotfill$
\item L'équation de la tangente $\mathcal T_1$: \dotfill
\item L'équation de la tangente $\mathcal T_2$: \dotfill
\end{itemize}

\begin{center}
\begin{tikzpicture}
\begin{axis}[
styleglobal,
width=0.7*\linewidth,
xmin=-4, xmax=4,
ymin=-3, ymax=4,
xtick distance=1,
ytick distance=1,
minor x tick num=1,
minor y tick num=1,
declare function={f(\x)=0.5*\x^2+2*sin(deg(\x)*pi)/pi;}
]
\addplot[styleplot] {f(x)} node[pos=0.7,below right] {$\mathscr C_f$};
%\addplot[styleplot,dotted,red] {x+2*cos(pi*deg(x))};

% Je triche un peu sur les dérivées...

%\node[stylepoint,fill=blue,label=above right:$ $] (A) at (0.6,{f(0.6)}) {};
\addplot[styleplot,densely dashed,color=DarkBlue] {{0*(x-0.6)+f(0.6)}};

%\node[stylepoint,fill=blue,label=above right:$ $] (A) at (0,{f(0)}) {};
\addplot[styleplot,densely dotted,color=DarkBlue] {{2*(x-0)+f(0)}} node[pos=0.65,above left] {$\mathcal T_1$};;

\node[stylepoint,fill=red,label=below right:$C$] (A) at (1.5,{f(1.5)}) {};
\addplot[styleplot,densely dashed,color=DarkRed] {{1.5*(x-1.5)+f(1.5)}} node[pos=0.9,below right] {$\mathcal T_2$};

%\node[stylepoint,fill=red,label=above right:$ $] (A) at (-2.5,{f(-2.5)}) {};
\addplot[styleplot,densely dotted,color=DarkRed] {{-2.5*(x+2.5)+f(-2.5)}};

\addplot[styleplot,dashed,color=DarkRed] {{-3*(x+1)+f(-1)}};
\end{axis}
\end{tikzpicture}
\end{center}
\end{exercice}

\compo[0.5]
{
\begin{exercice}
On se donne la fonction $g$ représentée ci-contre. Tracer les tangentes associées aux nombres dérivés ci-dessous:
\begin{itemize}
\item $f'(2)=-1$
\item $f'(-2,5)= \frac 5 2$
\item $f'(0,4)= 0$
\end{itemize}
\end{exercice}
}
{
\begin{tikzpicture}
\begin{axis}[
styleglobal,
width=0.9*\linewidth,
xmin=-3, xmax=4,
ymin=-2, ymax=5,
xtick distance=1,
ytick distance=1,
minor x tick num=1,
minor y tick num=1,
declare function={f(\x)=-0.5*\x^2+sin(deg(\x)*pi)/pi+3;}
]
\addplot[styleplot] {f(x)} node[pos=0.7,above right] {$\mathscr C_f$};
%\addplot[styleplot,dotted,red] {-x+cos(pi*deg(x))};

% Je triche un peu sur les dérivées...

%\node[stylepoint,fill=blue,label=above right:$ $] (A) at (0.6,{f(0.6)}) {};
%\addplot[styleplot,densely dashed,color=DarkBlue] {{0.5*(x+0.5)+f(-0.5)}};

%\node[stylepoint,fill=blue,label=above right:$ $] (A) at (0,{f(0)}) {};
%\addplot[styleplot,densely dotted,color=DarkBlue] {{2/3*(x+0.4)+f(-0.4)}};

%\node[stylepoint,fill=red,label=below right:$C$] (A) at (1.5,{f(1.5)}) {};
%\addplot[styleplot,densely dashed,color=DarkRed] {{1.5*(x-1.5)+f(1.5)}};

%\node[stylepoint,fill=red,label=above right:$ $] (A) at (-2.5,{f(-2.5)}) {};
%\addplot[styleplot,densely dotted,color=DarkRed] {{-2.5*(x+2.5)+f(-2.5)}};
\end{axis}
\end{tikzpicture}
}

\compo
{
\begin{exercice}
On se donne une fonction $g$ représentée ci-contre. Tracer de manière approximative les tangentes de $\mathcal C_g$ en $A$, $B$ et $C$.
\end{exercice}
}
{
\begin{tikzpicture}
\begin{axis}[
styleglobal,
width=0.9*\linewidth,
xmin=-4, xmax=4,
ymin=-1.5, ymax=2,
xtick distance=1,
ytick distance=1,
minor x tick num=1,
minor y tick num=1,
]
\addplot[styleplot] coordinates {(-3.5,-0.5) (-1.5,1.5) (0.5,-1) (2.5,1) (3.5,1)} node[pos=0.98,above] {$\mathscr C_f$} \pointsextremites;

\node[stylepoint,fill=blue,label=above:$A$] (A) at (-1.5,1.48) {};
\node[stylepoint,fill=blue,label=below left:$B$] (B) at (-0.66,0.5) {};\\
\node[stylepoint,fill=blue,label=above left:$C$] (C) at (1.63,0) {};

\end{axis}
\end{tikzpicture}
}

\begin{exercice} \

\compo[0.4]
{
\begin{center}
\vspace{3mm}
Questions 2 et 3
\begin{tikzpicture}
\begin{axis}[
styleglobal,
width=\linewidth,
xmin=-2.5, xmax= 6.5,
ymin=-1.5, ymax=4.5,
xtick distance=1,
ytick distance=1,
]
\addplot[styleplot,tension=0.3] plot coordinates {(-2,2) (-1,4) (2,-1) (4.5,2) (6,0.5)} node[pos=0.9,above right] {$\mathscr C_f$}  \pointsextremites;
\end{axis}
\end{tikzpicture}

\vspace{3mm}
Questions 4 et 5
\begin{tikzpicture}
\begin{axis}[
styleglobal,
width=\linewidth,
xmin=-2.5, xmax= 6.5,
ymin=-1.5, ymax=4.5,
xtick distance=1,
ytick distance=1,
]
\addplot[styleplot,tension=0.3] plot coordinates {(-2,2) (-1,4) (2,-1) (4.5,2) (6,0.5)} node[pos=0.9,above right] {$\mathscr C_f$}  \pointsextremites;
\end{axis}
\end{tikzpicture}
\end{center}
}
{
On a représenté une fonction $f$ sur le repère ci-contre. Des constructions sont demandées pour les questions indiquées.
\begin{enumerate}
\item L'ensemble de définition de $f$ est $\dotfill$
\item L'image de -1 est \dotfill
\item L'image de 1 est \dotfill
\item $1,5$ a pour antécédent(s) \dotfill
\item Dresser le tableau de signes de $f$.
\end{enumerate}
%
\begin{centrer}
\begin{tikzpicture}
\tkzTabInit[espcl=1.7,lgt=1.4]{$ $/1, $ $/1}{,,,}
\end{tikzpicture}
\end{centrer}
%
\begin{enumerate}[resume]
\item Dresser le tableau de variations de $f$.
\end{enumerate}
%
\begin{centrer}
\begin{tikzpicture}
\tkzTabInit[espcl=1.7,lgt=1.4]{$ $/1, $ $/1.7}{,,,}
\end{tikzpicture}
\end{centrer}
}
\end{exercice}

\begin{exercice}[Hors barème]
Soit $f:x \mapsto x^2+2$ et $h \in \R$.
\begin{enumerate}
\item Calculer $f(1)$, puis déterminer $f(1+h)$ en fonction de $h$.

\points 3

\item Développer l'expression $\dfrac {f(1+h)-f(1)}{h}$.

\points 5

\item Que dire de cette quantité lorsque $h$ se rapproche de 0? Que peut-on en déduire?

\points 2
\end{enumerate}
\end{exercice}

\end{document}