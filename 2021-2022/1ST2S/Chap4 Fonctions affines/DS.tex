\documentclass[a4paper,12pt,french] {article}

\usepackage[sujet]{../../../Style}

\fancyhead[L]{05/01/2022}
\fancyhead[C]{\textbf{DS4 : Fonctions affines}}
\fancyhead[R]{\premiere ST2S 2}

\begin{document}

\rem{L'usage de la calculatrice est interdit. La propreté et l'orthographe seront prises en compte. Tout le devoir peut être fait sur le sujet.}

Nom: \hfill Prénom: \hfill \

\ \renewcommand{\baselinestretch}{1.3}

\compo
{
\begin{exercice} \
Sur le repère ci-contre, tracer les droite $d_1$ et $d_2$ d'équation respective $y=-2x+3$ et $y=\frac 3 2 x -2$.
\end{exercice}

\begin{exercice} \
Déterminer algébriquement (par le calcul) l'équation réduite de la droite passant par les points $A(-4;-9)$ et $B(2;9)$.

\points 7
\end{exercice}
}
{
\begin{centrer}
\begin{tikzpicture}[scale=\echellepgf]
\begin{axis}[
styleglobal,
width=0.9*\echellepgfinv*\linewidth,
xmin=-3, xmax=5,
ymin=-4, ymax=6,
xtick distance=1,
ytick distance=1,
%major grid style={line width=1pt},
]
\end{axis}
\end{tikzpicture}
\end{centrer}
}

\compo[0.6]
{
\begin{exercice}
Déterminer l'équation réduite des droites $d_3$ et $d_4$ représentées ci-contre.

\points {10}
\end{exercice}
}
{
\begin{centrer}
\begin{tikzpicture}[scale=\echellepgf]
\begin{axis}[
styleglobal,
width=0.9*\echellepgfinv*\linewidth,
xmin=-2, xmax=6,
ymin=-6, ymax=6,
xtick distance=1,
ytick distance=1,
%major grid style={line width=1pt},
]
\addplot[styleplot,domain=(-5:5)]{4*x-3} node[pos=0.7,right] {$d_3$};
\addplot[styleplot,domain=(-6:6)]{-2/5*x+3} node[pos=0.9,above right] {$d_4$};
\end{axis}
\end{tikzpicture}
\end{centrer}
}

\begin{exercice} \ \vspace{3mm}
\compo[0.45]
{
Dresser le tableau de signes de la fonction

\centering $f:x \mapsto 12-4x$

\points 2
}
{
\begin{centrer}
\begin{tikzpicture}
\tkzTabInit[lgt=1.4,espcl=2]{/1,/1}{,,,}
\end{tikzpicture}
\end{centrer}
}
\end{exercice}

\end{document}