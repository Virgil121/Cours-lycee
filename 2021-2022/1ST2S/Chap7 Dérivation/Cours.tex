\documentclass[a4paper,12pt,french]{article}

\usepackage[cours]{../../../Style}

%\selectcolormodel{cmyk}

%\usepackage{wrapfig}

% Début du document
%%%%%%%%%%%%%%%%%%%
\begin{document}

\title{Dérivation}
\maketitle

\begin{programme}
\item Point de vue local:
\begin{itemize}
\item Sécantes à une courbe passant par un point donné, taux de variation en un point
\item Tangente à une courbe en un point, définie comme position limite des sécantes
\item Nombre dérivé en un point défini comme limite du taux de variation
\item Equation réduite de la tangente en un point.
\end{itemize}
\item Point de vue global
\begin{itemize}
\item Fonction dérivée
\item Dérivées de $x \mapsto x^2$, $x \mapsto x^3$, de combinaisons linéaires, de polynômes de degré $\leq 3$
\item Sens de variation d'une fonction, lien avec le signe de la dérivée
\item Tableau de variations, extremums
\end{itemize}
\item Capacités
\begin{itemize}
\item Interprétation du nombre dérivé comme coeff directeur de la tangente
\item Construire la tangente à une courbe en un point
\item Déterminer l'équation réduite de la tangente à une courbe en un point
\item Calculer la dérivée d'un polynome de degré $\leq 3$
\item Déterminer le sens de variation et les extremums d'une fonction polynome de degré $\leq 3$
\end{itemize}
\end{programme}

\section{Nombre dérivé}

\subsection{Tangentes}

Tangente, lim des sécantes. Lecture graphique de la dérivée

\subsection{Détermination algébrique de la dérivée}

\compo
{
\begin{center}
\begin{tikzpicture}[scale=\echellepgf]
\begin{axis}[
styleglobal,
width=0.8*\echellepgfinv*\linewidth,
xmin=-1, xmax=4,
ymin=-1, ymax=4,
xtick distance=1,
ytick distance=1,
ticks=none,
declare function={f(\x)=e^(0.75*(\x-1)-0.5);}
]
\addplot[styleplot]{f(x)} node [pos=0.75,below right] {$\mathscr C_f$};
\node[stylepoint,fill=blue] (A) at (0.5,{f(0.5)}) {};
\node[stylepoint,fill=blue] (B) at (2.5,{f(2.5)}) {};
\draw[color=blue,dashed,very thick] (0.5, 0) -- (A) node [pos=0,below] {$a$};% -- (0,{f(0.5)}) node [pos=1,left] {$f(x)$};
\draw[color=blue,dashed,very thick] (2.5, 0) -- (B) node [pos=0,below] {$a+h$};% -- (0,{f(1.2)}) node [pos=1,left] {$f(y)$};
\draw[line width=1.3pt,shorten <= -20cm,shorten >= -20cm,densely dotted] (A) -- (B);
\end{axis}
\end{tikzpicture}
\end{center}
}
{
\begin{center}
\begin{tikzpicture}[scale=\echellepgf]
\begin{axis}[
styleglobal,
width=0.8*\echellepgfinv*\linewidth,
xmin=-1, xmax=4,
ymin=-1, ymax=4,
xtick distance=1,
ytick distance=1,
ticks=none,
declare function={f(\x)=e^(0.75*(\x-1)-0.5);}
]
\addplot[styleplot]{f(x)} node [pos=0.75,below right] {$\mathscr C_f$};
\node[stylepoint,fill=blue] (A) at (0.5,{f(0.5)}) {};
\node[stylepoint,fill=blue,inner sep=1.3pt] (B) at (2.5,{f(2.5)}) {};
\node[stylepoint,fill=blue,inner sep=1.5pt] (C) at (1.5,{f(1.5)}) {};
\node[stylepoint,fill=blue,inner sep=1.7pt] (D) at (0.75,{f(0.75)}) {};
\node[stylepoint,fill=blue,inner sep=1.9pt] (E) at (0.625,{f(0.625)}) {};
\draw[color=blue,dashed,thick] (0.5, 0) -- (A) node [pos=0,below] {$a$};% -- (0,{f(0.5)}) node [pos=1,left] {$f(x)$};
\draw[color=blue,dashed,thick] (2.5, 0) -- (B) node [pos=0,below] {$a+h$};% -- (0,{f(1.2)}) node [pos=1,left] {$f(y)$};
\draw[color=blue,dashed,thick] (1.5, 0) -- (C);
\draw[color=blue,dashed,thick] (0.75, 0) -- (D);
\draw[color=blue,dashed,thick] (0.625, 0) -- (E);
\draw[thick,shorten <= -20cm,shorten >= -20cm,densely dotted] (A) -- (B);
\draw[thick,shorten <= -20cm,shorten >= -20cm,densely dotted] (A) -- (C);
\draw[thick,shorten <= -20cm,shorten >= -20cm,densely dotted] (A) -- (D);
\draw[line width=1.3pt,shorten <= -20cm,shorten >= -20cm,densely dotted] (A) -- (E);
\addplot[styleplot,color=Crimson,ultra thick]{0.75*e^(0.75*(-0.5)-0.5)*(x-0.5)+f(0.5)};
\end{axis}
\end{tikzpicture}
\end{center}
}

\begin{prop}
Soit $a \in \R$ et $h \in \R$. Alors le taux de variation $\dfrac{f(a+h)-f(a)}{h}$ se rapproche de $f'(a)$ à mesure que $h$ se rapproche de $0$.
\end{prop}

Formule générale avec schéma

\section{Fonction dérivée}

\subsection{Généralités}

\begin{defin}
On définit la fonction dérivée de $f$, notée $f'$, qui à $x$ associe le coefficient directeur de la tangente à la courbe de $f$ au point d'abscisse $x$.
\end{defin}

\subsection{Dérivées usuelles et règles de calcul}

\begin{prop}
On a les dérivées usuelles suivantes:
\begin{center}\renewcommand{\arraystretch}{1.5}
\begin{tabularx}{5cm}{|Y|Y|} \hline
$f(x)$ & $f'(x)$ \\ \hline
$k \in \R$ & $0$ \\ \hline
$x$ & $1$ \\ \hline
$x^2$ & $2x$ \\ \hline
$x^3$ & $3x^2$ \\ \hline
\end{tabularx}
\end{center}
\end{prop}


\begin{prop}
Soient $f$ et $g$ deux fonctions, et $\lambda \in \R$. Alors:
\begin{itemize}
\item $(f+g)'=f'+g'$
\item $(\lambda f)'=\lambda f'$
\end{itemize}
\end{prop}

\begin{ex}
Soit $f:x \mapsto {\color{DarkRed}x^2}+{\color{DarkBlue}2x}+{\color{black}1}$. Alors $f'(x)={\color{DarkRed}2x}+{\color{DarkBlue}2 \times 1} + {\color{black}0} =2x+2$.
\end{ex}

$f(x)=\tikzmark{cube} \ 2x^{\tikzmark{expcube} 3}-\tikzmark{carre}3x^{\tikzmark{expcarre} 2}+5\cancel{x} \cancel{-1}$

\tikz[remember picture]{\draw[overlay,->] (pic cs:expcube) to[bend right=80] (pic cs:cube);}
\tikz[remember picture]{\draw[overlay,->] (pic cs:expcarre) to[bend right=10] (pic cs:carre);}

somme, combinaisons linéaires

\subsection{Lien avec les variations}

\begin{thm}
Soit $f$ une fonction et $a \in \R$, $b \in \R$.
\begin{itemize}
\item Si $f'$ est positive sur $[a;b]$, alors $f$ est croissante sur $[a;b]$.
\item Si $f'$ est négative sur $[a;b]$, alors $f$ est décroissante sur $[a;b]$.
\item Si $f'$ est nulle sur $[a;b]$, alors $f$ est constante sur $[a;b]$.
\end{itemize}
\end{thm}

\end{document}
