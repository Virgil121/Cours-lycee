\documentclass[a4paper,12pt,french]{article}

\usepackage[TD]{../../Style}

\geometry{margin=10mm}
\setstretch{1.2}
\renewcommand{\arraystretch}{1.5}
% Début du document
%%%%%%%%%%%%%%%%%%%
\begin{document}

Une entreprise de pièces automobiles emploie des cadres et des ouvriers. Cette entreprise compte 1000 salariés dont 40\% sont des femmes. Les autres salariés sont des hommes. On sait aussi que parmi les femmes, 15 \% sont des cadres, et que 525 hommes sont des ouvriers.

\begin{enumerate}
\item Compléter le tableau d'effectifs qui traduit la situation:

\begin{center}
    \begin{tabularx}{\linewidth}{|c|Y|Y|Y|}
    \cline{2-4} \multicolumn{1}{c|}{}
         &Hommes&Femmes&Total  \\ \hline
         Cadres &&&  \\ \hline 
         Ouvriers&&& \\ \hline 
         Total&&&1000 \\ \hline
    \end{tabularx}
\end{center}

\item Justine affirme: \og La proportion de cadres parmi les hommes est plus élevée que la proportion de cadres parmi les femmes. \fg

A-t-elle raison? Justifier.

\item On choisit au hasard un salarié de l'entreprise. On admet que chaque salarié a la même probabilité d'être choisi. Soient les évènements F: \og Le salarié est une femme \fg et C: \og Le salarié est un cadre \fg.

\begin{enumerate}
\item Définir par une phrase les évènements $\overline C$ et F $\cap$ C.
\item Calculer la probabilité de ces évènements.
\item Calculer $\Pro_F(\overline C)$. Interpréter ce résultat dans le contexte de l'énoncé.
\end{enumerate}
\end{enumerate}

\vspace{2cm}

Une entreprise de pièces automobiles emploie des cadres et des ouvriers. Cette entreprise compte 1000 salariés dont 40\% sont des femmes. Les autres salariés sont des hommes. On sait aussi que parmi les femmes, 15 \% sont des cadres, et que 525 hommes sont des ouvriers.

\begin{enumerate}
\item Compléter le tableau d'effectifs qui traduit la situation:

\begin{center}
    \begin{tabularx}{\linewidth}{|c|Y|Y|Y|}
    \cline{2-4} \multicolumn{1}{c|}{}
         &Hommes&Femmes&Total  \\ \hline
         Cadres &&&  \\ \hline 
         Ouvriers&&& \\ \hline 
         Total&&&1000 \\ \hline
    \end{tabularx}
\end{center}

\item Justine affirme: \og La proportion de cadres parmi les hommes est plus élevée que la proportion de cadres parmi les femmes. \fg

A-t-elle raison? Justifier.

\item On choisit au hasard un salarié de l'entreprise. On admet que chaque salarié a la même probabilité d'être choisi. Soient les évènements F: \og Le salarié est une femme \fg et C: \og Le salarié est un cadre \fg.

\begin{enumerate}
\item Définir par une phrase les évènements $\overline C$ et F $\cap$ C.
\item Calculer la probabilité de ces évènements.
\item Calculer $\Pro_F(\overline C)$. Interpréter ce résultat dans le contexte de l'énoncé.
\end{enumerate}
\end{enumerate}

\end{document}
