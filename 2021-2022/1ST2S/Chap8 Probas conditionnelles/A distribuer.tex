\documentclass[a4paper,12pt,french]{article}

\usepackage[cours,NB]{../../Style}

%\selectcolormodel{cmyk}

%\usepackage{wrapfig}
%\makeatletter
%\setlength{\parskip}{1ex}
%\newcommand{\@minipagerestore}{\setlength{\parskip}{1ex}}

\newcommand{\card}{\textrm{Card}}
\setstretch{1.15}
\pagestyle{empty}
\geometry{margin=8mm}
% Début du document
%%%%%%%%%%%%%%%%%%%
\begin{document}

\begin{ex}
Au sein d'une classe de 1ST2S de 35 élèves, il y a 23 filles. Les élèves ont le choix entre l'allemand ou l'espagnol. On sait que 7 garçons ont choisi l'allemand contre seulement 3 filles. On peut alors dresser le tableau suivant:
\begin{center}\renewcommand{\arraystretch}{1.2}
    \begin{tabular}{|c|c|c|c|}
    	\cline{2-4} \multicolumn{1}{c|}{}
         & Espagnol & Allemand & Total  \\ \hline 
         Filles & & & \\ \hline
         Garçons & & & \\ \hline
         Total & & & 35\\ \hline 
    \end{tabular}
\end{center}
On note $F$ l'ensemble des filles, $G$ l'ensemble des garçons et $A$ l'ensemble des élèves ayant choisi l'allemand.
\begin{itemize}
    \item Le nombre d'éléments de $F \cap A$ (c'est-à-dire le nombre de filles ayant choisi l'allemand) est \dotfill
    \item Le nombre d'éléments de \makebox[3cm]{\dotfill} (c'est à dire le nombre de garçons qui n'ont pas choisi l'allemand, donc qui ont choisi l'espagnol) est \ldots .
    \item La fréquence des filles dans cette classe est \dotfill \\
    On parle de \makebox[4cm]{\dotfill}.
    \item La fréquence des garçons ayant choisi l'allemand dans cette classe est \dotfill
    \item La fréquence des filles \underline{parmi les élèves ayant choisi l'espagnol} est \dotfill \\
    On parle de \makebox[5cm]{\dotfill}.
    \item La fréquence des garçons ayant choisi l'espagnol \underline{parmi les garçons} est \dotfill
\end{itemize}
\end{ex}

\vfill

\begin{ex}
Au sein d'une classe de 1ST2S de 35 élèves, il y a 23 filles. Les élèves ont le choix entre l'allemand ou l'espagnol. On sait que 7 garçons ont choisi l'allemand contre seulement 3 filles. On peut alors dresser le tableau suivant:
\begin{center}\renewcommand{\arraystretch}{1.2}
    \begin{tabular}{|c|c|c|c|}
    	\cline{2-4} \multicolumn{1}{c|}{}
         & Espagnol & Allemand & Total  \\ \hline 
         Filles & & & \\ \hline
         Garçons & & & \\ \hline
         Total & & & 35\\ \hline 
    \end{tabular}
\end{center}
On note $F$ l'ensemble des filles, $G$ l'ensemble des garçons et $A$ l'ensemble des élèves ayant choisi l'allemand.
\begin{itemize}
    \item Le nombre d'éléments de $F \cap A$ (c'est-à-dire le nombre des filles ayant choisi l'allemand) est \dotfill
    \item Le nombre d'éléments de \makebox[3cm]{\dotfill} (c'est à dire le nombre de garçons qui n'ont pas choisi l'allemand, donc qui ont choisi l'espagnol) est \ldots .
    \item La fréquence des filles dans cette classe est \dotfill \\
    On parle de \makebox[4cm]{\dotfill}.
    \item La fréquence des garçons ayant choisi l'allemand dans cette classe est \dotfill
    \item La fréquence des filles \underline{parmi les élèves ayant choisi l'espagnol} est \dotfill \\
    On parle de \makebox[5cm]{\dotfill}.
    \item La fréquence des garçons ayant choisi l'espagnol \underline{parmi les garçons} est \dotfill
\end{itemize}
\end{ex}

\newpage\setstretch{1}

\begin{defins}
\begin{itemize}
\item On appelle expérience aléatoire une expérience dont on ne peut pas prévoir le résultat.
\item On appelle \dotfill, noté \ldots, l'ensemble de toutes les issues possibles de cette expérience.
\item Un évènement A est un ensemble d'issues, autrement dit \makebox[5cm]{\dotfill}. On appelle évènement élémentaire tout évènement ne contenant qu'un seul élément de $\Omega$ (Ce sont les singletons).
\item L'ensemble vide, noté \ldots, est \makebox[5cm]{\dotfill}: Il ne se réalise jamais.
\item L'ensemble $\Omega$ est \makebox[5cm]{\dotfill}: Il est toujours réalisé.
\item On dit qu'on est en situation \makebox[3cm]{\dotfill} lorsque toutes les issues ont la même probabilité.
\item On dit que deux événements A et B sont \makebox[3cm]{\dotfill} lorsqu'ils ne peuvent se produire en même temps, c'est-à-dire lorsque \makebox[3cm]{\dotfill}
\item Pour tout évènement A d'une expérience aléatoire d'univers $\Omega$, on a:
\end{itemize}
$$\ldots \leq \Pro(A) \leq \ldots \hspace{2cm} \Pro(\varnothing)=\ldots \hspace{2cm} \Pro(\Omega)=\ldots$$
\end{defins}

\vfill

\begin{defins}
\begin{itemize}
\item On appelle expérience aléatoire une expérience dont on ne peut pas prévoir le résultat.
\item On appelle \dotfill, noté \ldots, l'ensemble de toutes les issues possibles de cette expérience.
\item Un évènement A est un ensemble d'issues, autrement dit \makebox[5cm]{\dotfill}. On appelle évènement élémentaire tout évènement ne contenant qu'un seul élément de $\Omega$ (Ce sont les singletons).
\item L'ensemble vide, noté \ldots, est \makebox[5cm]{\dotfill}: Il ne se réalise jamais.
\item L'ensemble $\Omega$ est \makebox[5cm]{\dotfill}: Il est toujours réalisé.
\item On dit qu'on est en situation \makebox[3cm]{\dotfill} lorsque toutes les issues ont la même probabilité.
\item On dit que deux événements A et B sont \makebox[3cm]{\dotfill} lorsqu'ils ne peuvent se produire en même temps, c'est-à-dire lorsque \makebox[3cm]{\dotfill}
\item Pour tout évènement A d'une expérience aléatoire d'univers $\Omega$, on a:
\end{itemize}
$$\ldots \leq \Pro(A) \leq \ldots \hspace{2cm} \Pro(\varnothing)=\ldots \hspace{2cm} \Pro(\Omega)=\ldots$$
\end{defins}

\vfill

\begin{defins}
\begin{itemize}
\item On appelle expérience aléatoire une expérience dont on ne peut pas prévoir le résultat.
\item On appelle \dotfill, noté \ldots, l'ensemble de toutes les issues possibles de cette expérience.
\item Un évènement A est un ensemble d'issues, autrement dit \makebox[5cm]{\dotfill}. On appelle évènement élémentaire tout évènement ne contenant qu'un seul élément de $\Omega$ (Ce sont les singletons).
\item L'ensemble vide, noté \ldots, est \makebox[5cm]{\dotfill}: Il ne se réalise jamais.
\item L'ensemble $\Omega$ est \makebox[5cm]{\dotfill}: Il est toujours réalisé.
\item On dit qu'on est en situation \makebox[3cm]{\dotfill} lorsque toutes les issues ont la même probabilité.
\item On dit que deux événements A et B sont \makebox[3cm]{\dotfill} lorsqu'ils ne peuvent se produire en même temps, c'est-à-dire lorsque \makebox[3cm]{\dotfill}
\item Pour tout évènement A d'une expérience aléatoire d'univers $\Omega$, on a:
\end{itemize}
$$\ldots \leq \Pro(A) \leq \ldots \hspace{2cm} \Pro(\varnothing)=\ldots \hspace{2cm} \Pro(\Omega)=\ldots$$
\end{defins}

\newpage\setstretch{1.15}


\begin{ex}
On a interrogé $1500$ élèves d'un lycée sur la nature de leurs loisirs. On considère alors les évènements C: \og L'élève pratique une activité culturelle \fg et S: \og L'élève pratique une activité sportive \fg . On a obtenu les résultats suivants:
\begin{center}\renewcommand{\arraystretch}{1.2}
    \begin{tabularx}{\linewidth}{|c|c|c|Y|}
    \cline{2-4} \multicolumn{1}{c|}{}
         &Activité sportive $(S)$ & Pas d'activité sportive ($\overline{S}$) & Total  \\ \hline
         Activité culturelle $(C)$ & $402$ & $591$ &   \\ \hline 
         Pas d'activité culturelle ($\overline{C}$) & $315$ & $192$ &  \\ \hline
         Total & & & \\ \hline
    \end{tabularx}
\end{center}
On choisit un élève au hasard dans le lycée.
\begin{enumerate}
    \item La probabilité qu'un élève pratique une activité culturelle est:
    \item La probabilité qu'un élève pratique les deux types d'activité est:
    \item La probabilité qu'un élève fasse du sport en sachant qu'il pratique une activité culturelle est: \vspace{1.5cm}
    \item La probabilité qu'un élève \dotfill est: $$\Pro_{\overline S}(C)=\hspace{10cm}\vspace{5mm}$$

\end{enumerate}
\end{ex}

\vfill

\begin{ex}
On a interrogé $1500$ élèves d'un lycée sur la nature de leurs loisirs. On considère alors les évènements C: \og L'élève pratique une activité culturelle \fg et S: \og L'élève pratique une activité sportive \fg . On a obtenu les résultats suivants:
\begin{center}\renewcommand{\arraystretch}{1.2}
    \begin{tabularx}{\linewidth}{|c|c|c|Y|}
    \cline{2-4} \multicolumn{1}{c|}{}
         &Activité sportive $(S)$ & Pas d'activité sportive ($\overline{S}$) & Total  \\ \hline
         Activité culturelle $(C)$ & $402$ & $591$ &   \\ \hline 
         Pas d'activité culturelle ($\overline{C}$) & $315$ & $192$ &  \\ \hline
         Total & & & \\ \hline
    \end{tabularx}
\end{center}
On choisit un élève au hasard dans le lycée.
\begin{enumerate}
    \item La probabilité qu'un élève pratique une activité culturelle est:
    \item La probabilité qu'un élève pratique les deux types d'activité est:
    \item La probabilité qu'un élève fasse du sport en sachant qu'il pratique une activité culturelle est: \vspace{1.5cm}
    \item La probabilité qu'un élève \dotfill est: $$\Pro_{\overline S}(C)=\hspace{10cm}\vspace{5mm}$$

\end{enumerate}
\end{ex}
\end{document}