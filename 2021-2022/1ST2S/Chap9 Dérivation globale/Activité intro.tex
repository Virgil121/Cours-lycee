\documentclass[twocolumn,landscape,12pt,a4paper,french] {article}

\pagestyle{empty}

\usepackage[TD]{../../Style}

\renewcommand{\baselinestretch}{1.25}
\renewcommand{\arraystretch}{1.2}
\setlength{\columnsep}{2cm}

\begin{document}

\titre{Des dérivées, encore et encore...}

\compo
{
\begin{center}
\begin{tikzpicture}
\begin{axis}[
styleglobal,
width=0.9*\linewidth,
xmin=-3, xmax=3,
ymin=-2, ymax=7,
xtick distance=1,
ytick distance=1,
minor x tick num=0,
minor y tick num=0,
]
\addplot[styleplot] {x^2} node[pos=0.8,right] {$\mathscr C_f$};
\pgfplotsinvokeforeach {-1,0,1} {%
	\node[stylepoint,fill=blue] at (#1,#1*#1) {};
	\addplot[styleplot,ultra thick,densely dashed,color=DarkBlue] {2*#1*(x-#1)+#1*#1};}
\end{axis}
\end{tikzpicture}
\end{center}
}
{
\begin{center}
\begin{tikzpicture}
\begin{axis}[
styleglobal,
width=0.9*\linewidth,
xmin=-3, xmax=3,
ymin=-2, ymax=7,
xtick distance=1,
ytick distance=1,
minor x tick num=0,
minor y tick num=0,
]
\addplot[styleplot] {x^2} node[pos=0.8,right] {$\mathscr C_f$};
\pgfplotsinvokeforeach {-2,2} {%
	\node[stylepoint,fill=blue] at (#1,#1*#1) {};
	\addplot[styleplot,ultra thick,densely dashed,color=DarkBlue] {2*#1*(x-#1)+#1*#1};}
\end{axis}
\end{tikzpicture}
\end{center}
}

On a représenté deux fois la fonction $f:x \mapsto x^2$ ci-dessus et plusieurs tangentes. Compléter le tableau suivant:

\begin{center}
\begin{tabularx}{\linewidth}{|*6{Y|}} \hline
$x$  & $-2$ & $-1$  & $0$ & $1$ & $2$ \\ \hline
$f'(x)$ & & & & & \\ \hline
\end{tabularx}
\end{center}

Etablir une conjecture concernant la valeur de $f'(100)$, de $f'(-1000)$ puis de $f'(x)$ pour $x \in \R$.

\newpage

\titre{Des dérivées, encore et encore...}

\begin{center}
\begin{tikzpicture}
\begin{axis}[
styleglobal,
width=0.9*\linewidth,
xmin=-3, xmax=3,
ymin=-1.5, ymax=5,
xtick distance=1,
ytick distance=1,
minor x tick num=1,
minor y tick num=3,
]
\addplot[styleplot] {x^2} node[pos=0.8,right] {$\mathscr C_f$};
\pgfplotsinvokeforeach {-2,-0.5,0,0.5,2} {%
	\node[stylepoint,fill=blue] at (#1,#1*#1) {};
	\addplot[styleplot,ultra thick,densely dashed,color=DarkBlue] {2*#1*(x-#1)+#1*#1};}
\end{axis}
\end{tikzpicture}
\end{center}


On a représenté deux fois la fonction $f:x \mapsto x^2$ ci-dessus et plusieurs tangentes. Compléter le tableau suivant:

\begin{center}
\begin{tabularx}{\linewidth}{|*6{Y|}} \hline
$x$ & $-2$ & $-0,5$ & $0$ & $0,5$ & $2$ \\ \hline
$f'(x)$ & & & & & \\ \hline
\end{tabularx}
\end{center}

Etablir une conjecture concernant la valeur de $f'(100)$, de $f'(-1000)$ puis de $f'(x)$ pour $x \in \R$.

\end{document}
