\documentclass[a4paper,12pt,french]{article}

\usepackage[cours,NB]{../../../Style}

%\selectcolormodel{cmyk}

%\usepackage{wrapfig}
%\makeatletter
%\setlength{\parskip}{1ex}
%\newcommand{\@minipagerestore}{\setlength{\parskip}{1ex}}

\setstretch{1.15}
\pagestyle{empty}
\geometry{margin=8mm}
\renewcommand{\baselinestretch}{1.2}
% Début du document
%%%%%%%%%%%%%%%%%%%
\begin{document}

\begin{ex}
\compo
{%\setstretch{1.2}
Soit $f:x \mapsto x$. C'est une fonction affine avec $a=1$ et $b=0$.

\

Sa représentation graphique est une droite $d$. De plus, toutes les tangentes à la courbe de $f$ sont cette même droite $d$.

\

Alors quelque soit $x \in \R, f'(x)=a=1$. $f'$ est donc la fonction constante valant toujours 1.
}
{
\begin{center}
\begin{tikzpicture}
\begin{axis}[
styleglobal,
width=0.9*\linewidth,
xmin=-4, xmax=5,
ymin=-1, ymax=4,
xtick distance=1,
ytick distance=1,
minor x tick num=1,
minor y tick num=1,
]
\addplot[styleplot] {x} node[pos=0.76,below right] {$\mathscr C_f$};
\addplot[styleplot,DarkRed] {1} node[pos=0.76,below] {$\mathscr C_{f'}$};

\end{axis}
\end{tikzpicture}
\end{center}
}
\end{ex}

\vfill

\begin{ex}
\compo
{%\setstretch{1.2}
Soit $f:x \mapsto x$. C'est une fonction affine avec $a=1$ et $b=0$.

\

Sa représentation graphique est une droite $d$. De plus, toutes les tangentes à la courbe de $f$ sont cette même droite $d$.

\

Alors quelque soit $x \in \R, f'(x)=a=1$. $f'$ est donc la fonction constante valant toujours 1.
}
{
\begin{center}
\begin{tikzpicture}
\begin{axis}[
styleglobal,
width=0.9*\linewidth,
xmin=-4, xmax=5,
ymin=-1, ymax=4,
xtick distance=1,
ytick distance=1,
minor x tick num=1,
minor y tick num=1,
]
\addplot[styleplot] {x} node[pos=0.76,below right] {$\mathscr C_f$};
\addplot[styleplot,DarkRed] {1} node[pos=0.76,below] {$\mathscr C_{f'}$};

\end{axis}
\end{tikzpicture}
\end{center}
}
\end{ex}

\vfill

\begin{ex}
\compo
{%\setstretch{1.2}
Soit $f:x \mapsto x$. C'est une fonction affine avec $a=1$ et $b=0$.

\

Sa représentation graphique est une droite $d$. De plus, toutes les tangentes à la courbe de $f$ sont cette même droite $d$.

\

Alors quelque soit $x \in \R, f'(x)=a=1$. $f'$ est donc la fonction constante valant toujours 1.
}
{
\begin{center}
\begin{tikzpicture}
\begin{axis}[
styleglobal,
width=0.9*\linewidth,
xmin=-4, xmax=5,
ymin=-1, ymax=4,
xtick distance=1,
ytick distance=1,
minor x tick num=1,
minor y tick num=1,
]
\addplot[styleplot] {x} node[pos=0.76,below right] {$\mathscr C_f$};
\addplot[styleplot,DarkRed] {1} node[pos=0.76,below] {$\mathscr C_{f'}$};

\end{axis}
\end{tikzpicture}
\end{center}
}
\end{ex}

\end{document}