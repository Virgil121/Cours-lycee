\documentclass[a4paper,12pt,french]{article}

\usepackage[sujet]{../../Style}

\fancyhead[L]{06/2022}
\fancyhead[C]{\textbf{DS: Rattrapage}}
\fancyhead[R]{\premiere ST2S 2}

\renewcommand{\arraystretch}{1.8}
% Début du document
%%%%%%%%%%%%%%%%%%%
\begin{document}

\rem{L'usage de la calculatrice est autorisé. La propreté et l'orthographe seront prises en compte.\\ \textbf{Des phrases de réponse sont exigées dans l'exercice 2!}}

\newpage

\begin{exercice}
Une enquête a été menée dans un établissement de 1550 élèves afin de connaitre leur groupe sanguin. De plus, on sait que:
\begin{itemize}
\item Parmi les 800 garçons, 47 sont du groupe B.
\item 970 élèves sont du groupe O dont 434 filles.
\item 295 filles sont du groupe A.
\end{itemize}

\begin{enumerate}

\item Compléter le tableau croisé d'effectifs ci-dessous:
\begin{center}
    \begin{tabularx}{\linewidth}{|c|Y|Y|Y|Y|}
    \cline{2-5} \multicolumn{1}{c|}{}
         &Groupe A&Groupe B&Groupe O&Total  \\ \hline
         Garçons&&47&&800  \\ \hline 
         Filles&295&&434& \\ \hline 
         Total&&&970&1550 \\ \hline
    \end{tabularx}
\end{center}

On choisit un élève au hasard et on l'interroge sur son groupe sanguin.

\item Déterminer la probabilité de sélectionner une fille du groupe A.

\points 3

\item Calculer la probabilité que l'élève interrogé soit du groupe B en sachant que c'est une fille.

\points 3

\item On considère les évènements suivants:
\begin{center}
\hfill
\textendash \ G: \og L'élève est un garçon \fg; \hfill
\textendash \ O: \og L'élève est du groupe O \fg.
\hfill \
\end{center}

\begin{enumerate}
\item Calculer $\Pro(O)$ et interpréter ce résultat dans le contexte de l'énoncé.

\points 3

\item Calculer $\Pro_G(O)$ et $\Pro_{\overline G}(O)$, puis interpréter ces résultats dans le contexte de l'énoncé.

\points 7

\end{enumerate}
\end{enumerate}
\end{exercice}

\begin{exercice} \ 
On se donne la fonction $f$ représentée ci-dessous.
\begin{enumerate}
\item Déterminer:
\begin{itemize}
\item $f'(0)= \dotfill \quad f'(0,6)= \dotfill \quad f'(-2,5)=\dotfill$
\item L'équation de la tangente $\mathcal T_1$: \dotfill
\end{itemize}
\item On sait que $f'(-2,5)=\frac{-5}{2}$. Tracer la tangente associée.
\item Tracer approximativement la tangente de $\mathscr C_f$ en $C$.
\end{enumerate}

\begin{center}
\begin{tikzpicture}
\begin{axis}[
styleglobal,
width=0.6*\linewidth,
xmin=-4, xmax=4,
ymin=-3, ymax=4,
xtick distance=1,
ytick distance=1,
minor x tick num=1,
minor y tick num=1,
declare function={f(\x)=0.5*\x^2+2*sin(deg(\x)*pi)/pi;}
]
\addplot[styleplot] {f(x)} node[pos=0.7,below right] {$\mathscr C_f$};
%\addplot[styleplot,dotted,red] {x+2*cos(pi*deg(x))};

% Je triche un peu sur les dérivées...

%\node[stylepoint,fill=blue,label=above right:$ $] (A) at (0.6,{f(0.6)}) {};
\addplot[styleplot,densely dashed,color=DarkBlue] {{0*(x-0.6)+f(0.6)}};

%\node[stylepoint,fill=blue,label=above right:$ $] (A) at (0,{f(0)}) {};
\addplot[styleplot,densely dotted,color=DarkBlue] {{2*(x-0)+f(0)}} node[pos=0.65,above left] {$\mathcal T_1$};;

\node[stylepoint,fill=red,label=below right:$C$] (A) at (1.5,{f(1.5)}) {};
%\addplot[styleplot,densely dashed,color=DarkRed] {{1.5*(x-1.5)+f(1.5)}} node[pos=0.9,below right] {$\mathcal T_2$};

%\node[stylepoint,fill=red,label=above right:$ $] (A) at (-2.5,{f(-2.5)}) {};
\addplot[styleplot,densely dotted,color=DarkRed] {{-2.5*(x+2.5)+f(-2.5)}};

\addplot[styleplot,dashed,color=DarkRed] {{-3*(x+1)+f(-1)}};
\end{axis}
\end{tikzpicture}
\end{center}
\end{exercice}

\begin{exercice} \

\compolignehaut[0.45]
{
\begin{enumerate}
\item Compléter le tableau suivant, donnant les dérivées des fonctions usuelles:

\begin{center}\renewcommand{\arraystretch}{1.5}
\begin{tabularx}{5cm}{|Y|Y|} \hline
$f(x)$ & $f'(x)$ \\ \hline
$k \in \R$ &  \\ \hline
$x$ &  \\ \hline
$x^2$ & \\ \hline
$x^3$ &  \\ \hline
\end{tabularx}
\end{center}
\end{enumerate}
}
{
\begin{enumerate}[start=2]
\item Calculer la dérivée des fonctions suivantes:

\begin{enumerate}
\item $f:x \mapsto x^3-7$:

\points 3

\item $g:x \mapsto x^2+8x$:

\points 3

\item $h:x \mapsto 5x^3-2x^2+x$:

\points 3
\end{enumerate}
\end{enumerate}
}
\end{exercice}
\end{document}

\end{document}

