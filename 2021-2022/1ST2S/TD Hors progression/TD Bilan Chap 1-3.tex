\documentclass[,a4paper,11pt,french]{article}

\usepackage{../../Style}

\renewcommand\tabularxcolumn[1]{m{#1}}

\setlist[itemize]{align=parleft,left=5pt..15pt}
\setlist[enumerate]{align=parleft,left=5pt..20pt}

\geometry{centering,vcentering,left=13mm, right=13mm, top=10mm, bottom=10mm, marginparwidth=15mm}

\pagestyle{empty}

% Début du document
%%%%%%%%%%%%%%%%%%%
\begin{document}

\titre{Exercices - Chapitres 1 à 3}

\begin{multicols*}{2}

\section*{Proportions, évolutions}

\begin{exercice}
En septembre 2000, la superficie minimum de la banquise arctique était de $6,32$ millions de km$^2$. Elle n'était plus que de $4,59$ millions de km$^2$ en septembre 2018. De quel pourcentage la superficie de la banquise arctique a-t-elle diminué entre septembre 2000 et septembre 2018?
\end{exercice}

\begin{exercice}
Une boulangerie propose une baguette de pain à $90$ centimes. Son prix augmente de $10 \%$. Par quel nombre a-t-il été multiplié? Quel est alors son nouveau prix?
\end{exercice}

\begin{exercice}
Le prix d'un litre d'essence a augmenté de $15 \%$ entre janvier et juillet, puis de $10 \%$ entre juillet et décembre.
\begin{enumerate}
\item Calculer le coefficient multiplicateur de ces deux évolutions.
\item Justifier que le coefficient multiplicateur global est $1,265$.
\end{enumerate}
\end{exercice}

\begin{exercice}
La moitié de la carte d'un restaurant est composée de plats végétariens. Parmi ceux-ci, 20 $\%$ contiennent des tomates. Déterminer la proportion de plats végétariens contenant des tomates dans la carte du restaurant.
\end{exercice}

\section*{Fonctions, généralités}

\begin{exercice}
Soit $f$ la fonction définie par la courbe ci-dessous:

\Centering{
\begin{tikzpicture}
\begin{axis}[
styleglobal,
width=0.9*\linewidth,
xmin=-7, xmax= 7,
ymin=-3, ymax=5,
xtick distance=1,
ytick distance=1,
minor x tick num=0,
minor y tick num=0,
tick label style = {font=\scriptsize},
]
\addplot[styleplot] plot coordinates {(-6,3) (-5,1) (-3,-1) (-1,4) (1,3) (3,4) (4,2) (5,-1) (6,-2)};
\end{axis}
\end{tikzpicture}}
\begin{enumerate}
\item Quel est l'ensemble de définition de $f$?
\item Déterminer les images par $f$ de $-5 \ , \ 3 \ , \ -1 \ , \ 1 \ , \ 4$.
\item Quel est le nombre d'antécédents par $f$ de $4 \ , \ 2 \ , \ 0 \ , \ -1 \ , \ -2 \ , \ 3$?
\end{enumerate}
\end{exercice}

\begin{exercice}
On se donne la fonction $f$ définie sur $[-2;7]$ qui à $x$ associe $\frac {2x^3} {x^2+5x+15}$.
\begin{enumerate}
\item Réaliser le tableau de valeurs de $f$ entre $-2$ et $7$ par pas de $1$. On pourra s'aider de la calculatrice. On arrondira au dixième près.
\item A l'aide de ce tableau de valeurs, tracer dans un repère la courbe représentative de $f$ sur $[-2;7]$.
\end{enumerate}
\end{exercice}

\begin{exercice}
La courbe dans le repère ci-dessous représente la fonction $f$ qui à un instant $t$ exprimé en heures de l'intervalle $[0;24]$ associe la température $T$ en degrés Celsius, en un lieu.

\Centering{
\begin{tikzpicture}
\begin{axis}[
styleglobal,
hauteurproptick,
width=0.8*\linewidth,
xmin=0, xmax= 25,
ymin=-12, ymax=16,
xtick distance=2,
ytick distance=2,
minor x tick num=1,
minor y tick num=0,
ylabel={Température ($^{\circ}$C)},
xlabel={Temps (h)},
label style={font=\normalsize},
yscale=0.5,
]
\addplot[styleplot] plot coordinates {(0,10) (1,6) (2,2) (5,-4) (6,-8) (8,-10) (10,-8) (13,2) (15,8) (16,12) (18,14) (20,2) (24,-4)};
\end{axis}
\end{tikzpicture}}
\begin{enumerate}
\item Résoudre graphiquement l'équation $f(t)=2$. Interpréter le résultat.
\item Résoudre graphiquement l'inéquation $f(t) \geq -8$. Interpréter le résultat.
\end{enumerate}
\end{exercice}

\begin{exercice} \label{schemafonctions}
On se donne trois fonctions $f$, $g$ et $h$ représentées sur le graphe ci-dessous:
\begin{centrer}
\begin{tikzpicture}
\begin{axis}[
styleglobal,
width=0.8*\linewidth,
xmin=-0.5, xmax= 8.5,
ymin=-2.5, ymax=3.5,
xtick distance=1,
ytick distance=1,
minor x tick num=1,
minor y tick num=1,
]
\addplot[styleplot,domain=(0:8)] plot {0.5*x-1.5)} node[pos=0.75,above] {$\mathscr C_f$};
\addplot[styleplot,color=DarkRed,densely dashed,domain=(0:8)] plot coordinates{(0,-2) (1,-1) (2,2) (3,3) (5,1) (6,0) (8,-2)} node[pos=0.6,above right] {$\mathscr C_g$};
\addplot[styleplot,densely dotted,color=DarkGreen,domain=(0:8)] plot coordinates{(0,3) (2,2) (3,0) (4,-2) (6,0) (7,2) (8,3)} node[pos=0.1,above right] {$\mathscr C_h$};
\end{axis}
\end{tikzpicture}
\end{centrer}
Dresser le tableau de variation de $g$ et le tableau de signes de $h$.
\end{exercice}

\begin{exercice}
On reprend les fonctions définies à l'exercice précédent. Dans chaque cas, résoudre:
\begin{enumerate}
\item $g(x)=h(x)$
\item $f(x) \leq h(x)$
\item $h(x) > g(x)$
\end{enumerate}
\end{exercice}

\section*{Suites, généralités}

\begin{exercice} \
\begin{enumerate}
\item Soit $u$ une suite de premier terme $u_{11}$. Déterminer l'indice du troisième et du septième terme.
\item Soit $v$ la suite définie sur $\N$ par $u_n=3n-2$. Déterminer $v_0, v_1, v_2, v_{10}$.
\item Soit $w$ la suite telle que $w_0=1$ et pour $n \in N, w_{n+1}=5-2w_n$. Déterminer $w_1, w_2, w_3$.
\end{enumerate}
\end{exercice}

\begin{exercice}
Représenter la suite $u$ définie sur $\N$ par $u_n=\frac {n \times \sqrt n} 3$, et établir une conjecture sur son sens de variation.
\end{exercice}

\begin{exercice}
Soit $u$ la suite définie sur $\N$ par $u_n=5-n$. Que dire de son sens de variation? Prouver cette conjecture.
\end{exercice}
\end{multicols*}
\end{document}
