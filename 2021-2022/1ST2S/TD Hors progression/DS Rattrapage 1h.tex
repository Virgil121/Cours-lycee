\documentclass[a4paper,12pt,french]{article}

\usepackage[sujet]{../../Style}

\fancyhead[L]{06/2022}
\fancyhead[C]{\textbf{DS: Rattrapage}}
\fancyhead[R]{\premiere ST2S 2}

\renewcommand{\arraystretch}{1.6}
\renewcommand{\baselinestretch}{1.2}
% Début du document
%%%%%%%%%%%%%%%%%%%
\begin{document}

\rem{L'usage de la calculatrice est autorisé. La propreté et l'orthographe seront prises en compte.\\ \textbf{Des phrases de réponse sont exigées dans l'exercice 2!}}

\begin{exercice}
Lors de son embauche dans une société pour un contrat à durée indéterminée, Harry se voit proposer deux propositions de salaire:
\begin{itemize}
\item 1600\euro{} par mois, plus une augmentation chaque mois de 30\euro{} par rapport au salaire mensuel précédent.
\item 1400\euro{} par mois, plus une augmentation chaque mois de $2 \%$ par rapport au salaire mensuel précédent.
\end{itemize}

\begin{enumerate}
\item Dans cette question, Harry choisit la première proposition. Il touche donc 1600\euro{} le premier mois.
\begin{enumerate}
\item Calculer le salaire mensuel net d'Harry le deuxième mois de son contrat ainsi que celui le troisième mois de son contrat.

\points 4

\item On note $u_n$ le montant, exprimé en euros, du salaire mensuel net d'Harry au bout de $n$ mois, où $n$ est un entier naturel. Justifier que $u$ est une suite arithmétique dont on précisera le premier terme et la raison.

\points 5
\end{enumerate}

\item Dans cette question, Harry choisit la seconde proposition. Il touche donc 1400\euro{} le premier mois. On note $v_n$ le montant, exprimé en euros, du salaire mensuel net d'Harry au bout de $n$ mois, où $n$ est un entier naturel.
\begin{enumerate}

\item Justifier qu'augmenter une valeur de $2\%$ revient à la multiplier par $1,02$, puis calculer $v_1$ et interpréter cette valeur dans le contexte de l'énoncé.

\points 5

\item Quelle est la nature de $v$? On précisera le premier terme et la raison de cette suite.

\points 4

\item Déterminer $v_{n+1}$ en fonction de $v_n$.

\points 3
\end{enumerate}

\item Quelle proposition serait la plus souhaitable à long terme? \textit{On pourra s'aider de la calculatrice}

\points 5
\end{enumerate}
\end{exercice}

\begin{exercice}
Une enquête a été menée dans un établissement de 1550 élèves afin de connaitre leur groupe sanguin. On sait que parmi les 800 garçons, 47 sont du groupe B. De plus, 970 élèves sont du groupe O dont 434 filles. Enfin, 295 filles sont du groupe A.

\begin{enumerate}

\item Compléter le tableau croisé d'effectifs ci-dessous:
\begin{center}
    \begin{tabularx}{\linewidth}{|c|Y|Y|Y|Y|}
    \cline{2-5} \multicolumn{1}{c|}{}
         &Groupe A&Groupe B&Groupe O&Total  \\ \hline
         Garçons&&47&&800  \\ \hline 
         Filles&295&&434& \\ \hline 
         Total&&&970&1550 \\ \hline
    \end{tabularx}
\end{center}

On choisit un élève au hasard et on l'interroge sur son groupe sanguin.

\item Déterminer la probabilité de sélectionner une fille du groupe A.

\points 4

\item Calculer la probabilité que l'élève interrogé soit du groupe B en sachant que c'est une fille.

\points 4

\item On considère les évènements suivants:
\begin{center}
\hfill
\textendash \ G: \og L'élève est un garçon \fg; \hfill
\textendash \ O: \og L'élève est du groupe O \fg.
\hfill \
\end{center}

\begin{enumerate}
\item Calculer $\Pro(O)$ et interpréter ce résultat dans le contexte de l'énoncé.

\points 4

\item Calculer $\Pro_G(O)$ et $\Pro_{\overline G}(O)$, puis interpréter ces résultats dans le contexte de l'énoncé.

\points 8

\end{enumerate}
\end{enumerate}
\end{exercice}

\begin{exercice} \

\compolignehaut[0.45]
{
\begin{enumerate}
\item Compléter le tableau suivant, donnant les dérivées des fonctions usuelles:

\begin{center}\renewcommand{\arraystretch}{1.8}
\begin{tabularx}{5cm}{|Y|Y|} \hline
$f(x)$ & $f'(x)$ \\ \hline
$k \in \R$ &  \\ \hline
$x$ &  \\ \hline
$x^2$ & \\ \hline
$x^3$ &  \\ \hline
\end{tabularx}
\end{center}
\end{enumerate}
}
{
\begin{enumerate}[start=2]
\item Calculer la dérivée des fonctions suivantes:

\begin{enumerate}
\item $f:x \mapsto x^3-7$:

\points 3

\item $g:x \mapsto x^2+8x$:

\points 3

\item $h:x \mapsto 5x^3-2x^2+x$:

\points 3
\end{enumerate}
\end{enumerate}
}
\end{exercice}

\end{document}