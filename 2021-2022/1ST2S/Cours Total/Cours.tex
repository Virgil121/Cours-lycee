\documentclass[a4paper,french,12pt]{article}
\usepackage{titling}
\usepackage{sectsty}
%\usepackage[toctitles]{titlesec}
\usepackage[eleve,cours]{../../Style}
\usepackage[subpreambles=false]{standalone}
\usepackage{anyfontsize}
%\PassOptionsToPackage{toctitles}{titlesec}
\usepackage{titletoc}
\hypersetup{linktoc=all,hypertexnames=false,pdftitle={Mathématiques: 1ST2S 2021-2022}}


\renewcommand{\progres}{} % Commande progrès termine le document en version élève, on en veut pas ici


% CODE TRES MOCHE A REVOIR - toc
%%%%%%%%%%%%%%%%%

% Lien vers toc sur numéro page

\pagestyle{fancy}
\fancyhf{}
\fancyfoot[C]{\hyperlink{page.2}{\thepage}}
\renewcommand{\headrulewidth}{0pt}
\renewcommand{\footrulewidth}{0pt}

\setcounter{tocdepth}{1} % Voir chapitres et sections, pas plus loin
\newcounter{chapitre}\setcounter{chapitre}{-1} % Chapitre initialisé à -1, on le remet à 0 après le toc (comme ca on peut tester si on affiche le premier chapitre dans le toc)

\makeatletter

\renewcommand\tableofcontents{%
	\titre{Table des matières} % Titre
	\thispagestyle{empty} % Enlever numéro page
    \@starttoc{toc}%
    \setcounter{page}{0} % Remettre page à 0 ( commencera à 1 )
    % A chaque titre après le toc, on crée un marqueur et on ajoute le chapitre dans le toc
	\xpatchcmd\@maketitle{\thanks}{\thanks% Patch de la commande nécessaire, si on met cela après \centering ... \thanks, on a un espace moche
		\refstepcounter{part}%
    	\tikz[overlay,remember picture]{\node at (current page.north) {\phantomsection};}% tikz pour mettre le lien tout en haut de la page
    	\addcontentsline{toc}{part}{Chapitre \Roman{chapitre}.\ \@title}%Modifie commande titre après pour mettre une ref. Chapitre X dedans pour que l'hyperlien le prenne en compte
    }{}{}
}
\makeatother

% Styles toc
\def\stylepart{\fontfamily{cmr}\fontseries{bx}\selectfont\large\scshape\color{section}}

\titlecontents{part}[0pt]%
{\ifthenelse{\equal{\value{chapitre}}{-1}}{\setcounter{chapitre}{0}}{\vfill}\stylepart}%{\vspace{5mm}\normalfont\large\bfseries\color{section}}%
{}%
{}%{Chapitre \stepcounter{chapitre}\Roman{chapitre}.\ }%
%{\makebox[\longueurchap][l]{Chapitre \stepcounter{chapitre}\Roman{chapitre}.}}%
{\dotfill \contentspage}%[{\color{gray}\hrule}] % Gros hack pour afficher le chapitre... % Tester \ref{page.\thecontentspage}

\titlecontents{section}[0.1\linewidth]%
{\normalfont\bfseries\color{subsection}}%
{\thecontentslabel\ - }%
{}%
{\hfill \contentspage}%[{\color{gray}\hrule}]

%%%%%%%%%%%%%%%%%%%%

\newcommand{\chap}[1]{%
%\addcontentsline{toc}{part}{#1}
\stepcounter{chapitre}%
\input{#1}%
\setcounter{section}{0}\newpage
}

\begin{document}
\thispagestyle{empty}
%%%%%%%%%%%%%%%%%%%%%%%%%%%%%%%%%%%%%%%%%%%%%%%%%%%%%%%%%%%%%%%
%
% Welcome to Overleaf --- just edit your LaTeX on the left,
% and we'll compile it for you on the right. If you open the
% 'Share' menu, you can invite other users to edit at the same
% time. See www.overleaf.com/learn for more info. Enjoy!
%
%%%%%%%%%%%%%%%%%%%%%%%%%%%%%%%%%%%%%%%%%%%%%%%%%%%%%%%%%%%%%%%
% How to create a beautiful cover page in LaTeX using TikZ
% Latexdraw.com
% 14:15, 03:03:2020 

\documentclass[a4paper,french,12pt]{article}
\usepackage[dvipsnames,svgnames]{xcolor}
\usepackage{tikz}
\usetikzlibrary{calc}
\usepackage{anyfontsize}
\usepackage{sectsty}
\usepackage[a4paper,margin=15mm,marginparwidth=0mm,marginparsep=0mm]{geometry}

\begin{document}

\pagestyle{empty}

\tikzset{
imgmaths/.style={
path picture={
      \node[opacity=0.5,anchor=west,rotate=-20] at (path picture bounding box.north west) {
        \includegraphics[scale=0.6]{Fondmoitie}};}
}
}

\begin{tikzpicture}[overlay,remember picture]

% Background color
\path[
top color=black!4,
bottom color=black!2,
path picture={
      \node[opacity=0.03,rotate=10] at ($(-3,-3)+(path picture bounding box.west)$) {
        \includegraphics[scale=1.2]{FondNegatifmoitie}};}
]
(current page.south west) rectangle (current page.north east);

% Rectangles
\path[
left color=Dandelion, 
right color=Dandelion!40,
imgmaths,
transform canvas ={rotate around ={45:($(current page.north west)+(0,-6)$)}}] 
($(current page.north west)+(0,-6)$) rectangle ++(9,1.5);

\path[
imgmaths,
left color=lightgray,
right color=lightgray!50,
rounded corners=0.75cm,
transform canvas ={rotate around ={45:($(current page.north west)+(.5,-10)$)}}]
($(current page.north west)+(0.5,-10)$) rectangle ++(15,1.5);

\shade[
left color=lightgray,
rounded corners=0.3cm,
transform canvas ={rotate around ={45:($(current page.north west)+(.5,-10)$)}}] ($(current page.north west)+(1.5,-9.55)$) rectangle ++(7,.6);

\shade[
imgmaths,
left color=orange!80,
right color=orange!60,
rounded corners=0.4cm,
transform canvas ={rotate around ={45:($(current page.north)+(-1.5,-3)$)}}]
($(current page.north)+(-1.5,-3)$) rectangle ++(9,0.8);

\shade[
imgmaths,
left color=red!80,
right color=red!80,
rounded corners=0.9cm,
transform canvas ={rotate around ={45:($(current page.north)+(-3,-8)$)}}] ($(current page.north)+(-3,-8)$) rectangle ++(15,1.8);

\shade[
imgmaths,
left color=orange,
right color=Dandelion,
rounded corners=0.9cm,
transform canvas ={rotate around ={45:($(current page.north west)+(4,-15.5)$)}}]
($(current page.north west)+(4,-15.5)$) rectangle ++(30,1.8);

\shade[
imgmaths,
left color=DodgerBlue,
right color=Emerald,
rounded corners=0.75cm,
transform canvas ={rotate around ={45:($(current page.north west)+(13,-10)$)}}]
($(current page.north west)+(13,-10)$) rectangle ++(15,1.5);

\shade[
imgmaths,
left color=lightgray,
rounded corners=0.3cm,
transform canvas ={rotate around ={45:($(current page.north west)+(18,-8)$)}}]
($(current page.north west)+(18,-8)$) rectangle ++(15,0.6);

\shade[
imgmaths,
left color=lightgray,
rounded corners=0.4cm,
transform canvas ={rotate around ={45:($(current page.north west)+(19,-5.65)$)}}]
($(current page.north west)+(19,-5.65)$) rectangle ++(15,0.8);

\shade[
imgmaths,
left color=DeepPink,
right color=red!80,
rounded corners=0.6cm,
transform canvas ={rotate around ={45:($(current page.north west)+(20,-9)$)}}] 
($(current page.north west)+(20,-9)$) rectangle ++(14,1.2);

\node[ultra thick,gray,
black!75,
inner sep=0pt,
anchor=east,
] (maths) at ($(current page.east)+(-1,0)$)
{
{\fontsize{30}{36} \selectfont \bfseries LGT Jean Rostand}
};

\node[ultra thick,gray,
orange,
inner sep=0pt,
anchor=north east,
] (annee) at ($(maths.south east)+(0,-0.4)$)
{
{\fontsize{25}{30}\selectfont\bfseries 1\textsuperscript{ère} ST2S}
};

\draw[ultra thick,gray] ($(annee.north west)+(0,0.2)$) -- ($(annee.north east)+(0,0.2)$);

% Title
\node[align=center] (titre) at ($(current page.center)+(0,-7)$) 
{
{\fontsize{55}{66} \selectfont\scshape\textcolor{black!90} {Mathématiques}} \\[10mm]
{\fontsize{18}{21.6} \selectfont \textcolor{orange}{ \bfseries {M. DELAUNEY}}}% \\[10mm]
%{\fontsize{14}{16.8} \selectfont \textcolor{black!75}{ \bfseries {2021 - 2022}}}
};
\end{tikzpicture}

%\vspace{16cm}
%\begin{center}
%{\fontsize{50}{60} \selectfont\scshape {Mathématiques}} \\[1cm]
%{\fontsize{16}{19.2} \selectfont \textcolor{orange}{ \bf {M. DELAUNEY}}}
%\end{center}
%
%\tableofcontents
%\selectfont\scshape\color[HTML]{912c21}{{

\end{document}
%
\tableofcontents
%
\chap{../Chap1 Proportions Variations/cours}
\chap{../Chap2 Généralités fonctions/cours v2}
\chap{../Chap3 Généralités Suites/cours}
\chap{../Chap4 Fonctions affines/cours}
\chap{../Chap5 Fonctions polynomiales de degré 2/cours v4}
\chap{../Chap6 Suites arithmétiques géométriques/cours}
\chap{../Chap7 Dérivation locale/cours}
\chap{../Chap8 Probas conditionnelles/cours}
\chap{../Chap9 Dérivation globale/cours}
\chap{../Chap10 Variables aléatoires/cours}
\end{document}