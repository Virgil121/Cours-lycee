\documentclass[,a4paper,12pt,french]{article}

\usepackage{../../../Style}

\renewcommand\tabularxcolumn[1]{m{#1}}

\pagestyle{empty}

% Début du document
%%%%%%%%%%%%%%%%%%%
\begin{document}

\titre{Pour s'entrainer sur les suites}

\begin{exercice} \
\begin{enumerate}
\item Soit $u$ une suite de premier terme $u_3$.
\begin{enumerate}
\item Déterminer l'indice du troisième terme.
\item Déterminer l'indice du septième terme.
\end{enumerate}
\item Soit $v$ une suite de premier terme $v_7$.
\begin{enumerate}
\item Déterminer l'indice du cinquième terme.
\item Déterminer l'indice du huitième terme.
\end{enumerate}
\end{enumerate}
\end{exercice}

\begin{exercice}
On donne les dix premiers termes d'une suite $w$ dont le premier terme est $w_2$:

$$1;1;2;3;5;8;13;21;34;55$$

\begin{enumerate}
\item Donner les valeurs de $w_2$, de $w_5$, de $w_7$.
\item Quel est l'indice de $5$? De 21?
\end{enumerate}
\end{exercice}

\begin{exercice} \
\begin{enumerate}
\item Soit $u$ la suite définie sur $\N$ par $u_n=3n-2$. Déterminer $u_0, u_1, u_2, u_3, u_{10}$.
\item Soit $v$ la suite définie sur $\N$ par $v_n=\frac{1}{n^2+1}$. Déterminer $v_1, v_2, v_3$.
\item Soit $w$ la suite définie sur $\N$ par $w_n=u_n \times v_n$. A l'aide des questions précédentes, déterminer $w_0, w_1, w_2, w_3$.
\end{enumerate}
\end{exercice}

\vfill

\setcounter{exercice}{0}
\titre{Pour s'entrainer sur les suites}

\begin{exercice} \
\begin{enumerate}
\item Soit $u$ une suite de premier terme $u_3$.
\begin{enumerate}
\item Déterminer l'indice du troisième terme.
\item Déterminer l'indice du septième terme.
\end{enumerate}
\item Soit $v$ une suite de premier terme $v_7$.
\begin{enumerate}
\item Déterminer l'indice du cinquième terme.
\item Déterminer l'indice du huitième terme.
\end{enumerate}
\end{enumerate}
\end{exercice}

\begin{exercice}
On donne les dix premiers termes d'une suite $w$ dont le premier terme est $w_2$:

$$1;1;2;3;5;8;13;21;34;55$$

\begin{enumerate}
\item Donner les valeurs de $w_2$, de $w_5$, de $w_7$.
\item Quel est l'indice de $5$? De 21?
\end{enumerate}
\end{exercice}

\begin{exercice} \
\begin{enumerate}
\item Soit $u$ la suite définie sur $\N$ par $u_n=3n-2$. Déterminer $u_0, u_1, u_2, u_{10}$.
\item Soit $v$ la suite définie sur $\N$ par $v_n=\frac{1}{n^2+1}$. Déterminer $v_0, v_1, v_2$.
\item Soit $w$ la suite définie sur $\N$ par $w_n=u_n \times v_n$. A l'aide des questions précédentes, déterminer $w_0, w_1, w_2$.
\end{enumerate}
\end{exercice}

\end{document}
