\documentclass[a4paper,12pt,french] {article}

\usepackage{../../../Style}

\renewcommand\tabularxcolumn[1]{m{#1}}

\renewcommand{\baselinestretch}{1.25}

\pagestyle{empty}

\begin{document}

\newcommand{\contenu}{
\compo[0.65]{
\begin{enumerate}
\item Soit $u$ une suite dont le premier terme est $u_{24}$. Donner son cinquième terme: \dotfill
\item On se donne la suite $v$ définie pour tout $n \in \N$ par $v_n=3-2n$. Compléter:
\begin{itemize}
\item $v_0=$ \dotfill
\item $v_1=$ \dotfill
\item $v_{11}=$ \dotfill
\end{itemize}
\item Représenter la suite $v$ dans le repère ci-contre:
\end{enumerate}
}
{
\begin{center}
\begin{tikzpicture}
\begin{axis}[
styleglobal,
hauteurproptick,
width=0.8*\linewidth,
xmin=0, xmax= 5.5,
ymin=-12, ymax=4,
xlabel={$n$},
ylabel={$u_n$},
minor x tick num=1,
minor y tick num=1,
xtick distance=1,
ytick distance=2,
label style= {font=\normalsize}
]
\end{axis}
\end{tikzpicture}
\end{center}
}
\begin{enumerate}[start=4]
\item Soit $w$ une suite telle que $w_0=7$ et pour $n \in \N$, $w_{n+1}=3-2w_n$. Calculer:
\begin{itemize}
\item $w_1=$ \dotfill
\item $w_2=$ \dotfill
\item $w_3=$ \dotfill
\end{itemize}
\end{enumerate}}

\vfill

\contenu

\vfill

\contenu

\vfill

\end{document}