\documentclass[a4paper,12pt,french] {article}

\usepackage[sujet]{../../Style}

\fancyhead[L]{17/11/2021}
\fancyhead[C]{\textbf{DS2 : Fonctions - Généralités}}
\fancyhead[R]{\premiere ST2S 2}

\renewcommand{\baselinestretch}{1.25}

\begin{document}

\rem{L'usage de la calculatrice est autorisé. La propreté et l'orthographe seront prises en compte. Tout le devoir peut être fait sur le sujet.}

Nom: \hfill Prénom: \hfill \

\begin{exercice} \

\compo[0.4]
{
\begin{center}
\vspace{3mm}
Questions 2 et 3
\begin{tikzpicture}
\begin{axis}[
styleglobal,
width=\linewidth,
xmin=-2.5, xmax= 6.5,
ymin=-1.5, ymax=4.5,
xtick distance=1,
ytick distance=1,
]
\addplot[samples=101,smooth,line width=1.3pt,domain=(0:4),mark=none] plot coordinates {(-2,2) (-1,4) (0.5,0) (1.5,-1) (3,-0.5) (4.5,2) (6,0.5)} node[pos=0.9,above right] {$\mathscr C_f$}  \pointsextremites;
\end{axis}
\end{tikzpicture}

\vspace{3mm}
Questions 4 et 5
\begin{tikzpicture}
\begin{axis}[
styleglobal,
width=\linewidth,
xmin=-2.5, xmax= 6.5,
ymin=-1.5, ymax=4.5,
xtick distance=1,
ytick distance=1,
]
\addplot[samples=101,smooth,line width=1.3pt,domain=(0:4),mark=none] plot coordinates {(-2,2) (-1,4) (0.5,0) (1.5,-1) (3,-0.5) (4.5,2) (6,0.5)} node[pos=0.9,above right] {$\mathscr C_f$} \pointsextremites;
\end{axis}
\end{tikzpicture}
\end{center}
}
{
On a représenté une fonction $f$ sur le repère ci-contre. Des constructions sont demandées pour les questions indiquées.
\begin{enumerate}
\item L'ensemble de définition de $f$ est $\dotfill$
\item L'image de 2 est \dotfill
\item L'image de -1 est \dotfill
\item $1,5$ a pour antécédent(s) \dotfill
\item $3$ possède \makebox[3cm]{\dotfill} antécédent(s).
\item Dresser un tableau de signes de la fonction $f$.
\end{enumerate}
\begin{center}
\begin{tabularx}{0.9\linewidth}{|c|X|} \hline
\rule{0pt}{25pt} & \\ \hline
\phantom{f(x)} & \rule{0pt}{25pt}\\ \hline
\end{tabularx}
\end{center}
}
\end{exercice}

\begin{exercice} \

\compo[0.6]
{
On a représenté deux fonctions $f$ et $g$ sur le repère ci-contre. Résoudre graphiquement les (in)équations suivantes:
\begin{enumerate}
\item $f(x) \leq 0$: \dotfill
\item $f(x)=g(x)$: \dotfill
\item $f(x) > g(x)$: \dotfill
\end{enumerate}
}
{
\begin{center}
\begin{tikzpicture}
\begin{axis}[
styleglobal,
width=\linewidth,
xmin=-4, xmax= 4,
ymin=-2, ymax=3,
xtick distance=1,
ytick distance=1,
]
\addplot[samples=101,smooth,line width=1.3pt,domain=(0:4),mark=none] plot coordinates {(-3,2) (-2.5,0) (-1,-1.5) (1,1) (3,2)} node[pos=0.9,above left] {$\mathscr C_f$}  \pointsextremites;
\addplot[color=blue,densely dashed,samples=101,smooth,line width=1.3pt,domain=(-3:3),mark=none] plot {(2-0.3*x^2)*(1.2-e^(0.8*x-1.5))-0.5} node[pos=0.85,below] {$\mathscr C_g$}  \pointsextremites;
\end{axis}
\end{tikzpicture}
\end{center}
}

\end{exercice}

\begin{exercice}
Soit $f:[0;5] \rightarrow \R$ la fonction qui a $x$ associe $\frac {2x-3}{x+6}$.

\begin{enumerate}
\item Compléter le tableau de valeurs suivant:

\begin{center}
\begin{tabularx}{0.8\linewidth}{|c|
    >{\centering\arraybackslash}X|
    >{\centering\arraybackslash}X|
    >{\centering\arraybackslash}X|
    >{\centering\arraybackslash}X|
    >{\centering\arraybackslash}X|
    >{\centering\arraybackslash}X|} \hline
$x$ & $0$ & $1$ & $2$ & $3$ & $4$ & $5$ \\ \hline
$f(x)$ & & & & & & \rule{0pt}{8mm} \\ \hline
\end{tabularx}
\end{center}
\item Calculer le taux de variation de $f$ entre $3$ et $4$, puis entre $0$ et $5$.

\points 3
\end{enumerate}

\end{exercice}
\renewcommand{\baselinestretch}{1.1}
\vspace{-3mm}
\begin{exercice} \
\vspace{3mm}
\compo[0.5]
{
\quad Un hangar a une forme rectangulaire ABCD avec $AB=50$m et $BC=100$m. Pour surveiller ce hangar, on place une caméra au point M, milieu de $[AD]$. Son angle de vision est de $30^{\circ}$.

\quad On note $x$ l'angle, en degrés, balayé par son axe de vision $(MN)$ lors de la rotation de la caméra.

\quad On pose alors $f$ la fonction qui à un angle $x \in [0;180]$ associe l'aire du hangar observable par la caméra.
}
{
\includegraphics[width=\linewidth]{Schéma contrôle.png}
}
\vspace{-2mm}
\begin{center}
\begin{tikzpicture}
\begin{axis}[
styleglobal,
hauteurproptick,
width=0.85*\linewidth,
xmin=0, xmax= 180,
ymin=0, ymax=1200,
xlabel={Angle (degrés)},
ylabel={Aire ($m^2$)},
minor x tick num=1,
minor y tick num=1,
ytick distance=200,
xtick distance=10,
label style= {font=\normalsize}
]
\addplot[styleplot,domain=(0:15)] plot {1250*tan(x+15)};
\addplot[styleplot,domain=(15:30)] plot {1250*(tan(x+15)-tan(x-15))};
\addplot[styleplot,domain=(30:60)] plot {2500-1250*(tan(75-x)+tan(x-15))};
\addplot[styleplot,domain=(60:90)] plot {2500-1250*sqrt(2)*(sin(x-60)/sin(195-x)+sin(120-x)/sin(15+x))};
\addplot[styleplot,domain=(90:120)] plot {2500-1250*sqrt(2)*(sin(180-x-60)/sin(195+x-180)+sin(120+x-180)/sin(15+180-x))};
\addplot[styleplot,domain=(120:150)] plot {2500-1250*(tan(75+x-180)+tan(180-x-15))} node[pos=1,above right] {$\mathscr C_f$};
\addplot[styleplot,domain=(150:165)] plot {1250*(tan(180-x+15)-tan(180-x-15))};
\addplot[styleplot,domain=(165:180)] plot {1250*tan(180-x+15)};

\end{axis}
\end{tikzpicture}
\end{center}
\vspace{-5mm}
\begin{enumerate}
\item Quelle est la surface observable lorsque l'angle de la caméra est de $60^{\circ}$?

\points 1
\item Réaliser un tableau de variations de la fonction $f$.

\begin{center}
\begin{tabularx}{0.9\linewidth}{|c|X|} \hline
\rule{0pt}{20pt} & \\ \hline
\phantom{f(x)} & \rule{0pt}{50pt}\\ \hline
\end{tabularx}
\end{center}

\item Quels sont les angles de vision permettant de couvrir la plus grande surface?

\points 2

\item Justifier que l'aire du hangar est de $5000$m$^2$.

\points 1

\item Pour quels angles la caméra balaye-t-elle plus de $20\%$ de l'aire du hangar?

\points 2

\item Moins de $16\%$?

\points 2

\end{enumerate}

\end{exercice}

\begin{comment}

\begin{exercice}
Problème avec lecture graphique image, antécédent, résolution équation, tableaux signe variation, tx variation

Par exemple production de deux entreprises avec deux fonctions, à mettre sur le même graphe?

Bénéfices de deux entreprises pendant 

\begin{center}
\begin{tikzpicture}[scale=0.8]
\begin{axis}[
styleglobal,
width=\linewidth,
xmin=0, xmax= 30,
ymin=-1, ymax=4.5,
xlabel={Temps (en jours)},
ylabel={Population},
minor x tick num=1,
minor y tick num=1,
ytick distance=2,
xtick distance=1,
yscale=0.5
]
\addplot[samples=101,smooth,ultra thick,mark=none] plot coordinates {(0,0) (0.5,4) (1,0) (2,-1) (4,0) (5,2)};

\end{axis}
\end{tikzpicture}
\end{center}

\end{exercice}

\begin{exercice}
Tableau de variations et de signes
\end{exercice}

\end{comment}

\end{document}