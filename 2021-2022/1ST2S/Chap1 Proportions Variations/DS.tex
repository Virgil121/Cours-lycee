\documentclass[a4paper,12pt,french] {article}

\usepackage{../../../Style}

\renewcommand\tabularxcolumn[1]{m{#1}}

\pagestyle{fancy}
\setlength{\headheight}{10mm}
\fancyhead[L]{29/09/2021}
\fancyhead[C]{\textbf{DS1 : Proportions, variations, pourcentages - 1h}}
\fancyhead[R]{1ST2S 2}
\fancyfoot{}

\begin{document}

\rem{L'usage de la calculatrice est autorisée. L'usage de schémas est encouragé. La propreté et l'orthographe seront prises en compte. Des phrases de réponse sont demandées.}

\begin{comment}
\begin{exercice}[2 pts]
En 2016, en France, 1 523 442 candidats se sont présentés aux épreuves théoriques du permis de conduire et 1 053 460 ont été reçus. En 2017, il y avait 1 544 546 candidats dont 1 021 557 reçus. Killian affirme que la proportion de reçus est meilleure en 2016 qu'en 2017. A-t-il raison?
\end{exercice}
\end{comment}

\begin{exercice}[4 pts]
En France métropolitaine, en 2018, la forêt couvre environ $34 \%$ du territoire. Les trois quarts de ces forêts appartiennent à des propriétaires privés.
\begin{enumerate}
\item Quel pourcentage de la superficie de la France métropolitaine occupent les forêts privées?
\item On estime que la superficie de la France métropolitaine est $551695$ km$^2$. Quelle est, en km$^2$, la superficie des forêts privées? \textit{Arrondir à l'unité.}
\end{enumerate}
\end{exercice}

\begin{exercice}[4 pts]
Vrai ou faux? Justifier.
\begin{enumerate}
\item Le nombre de spectateurs d'une représentation a augmenté de 35 $\%$ en un an. Ce nombre a donc été multiplié par $1.35$.
\item Un prix est passé de 90\euro{} à 100\euro{}. Alors il a augmenté de 10$\%$.
\end{enumerate}
\end{exercice}

\begin{exercice}[4 pts]
Un sac à dos est en soldes. Son étiquette indique que son prix a subi une première réduction de $40\%$, puis une deuxième de $10\%$. Son prix final est affiché à $29.99$\euro{}.
\begin{enumerate}
\item Calculer le taux de réduction global du prix de ce sac.
\item En déduire le prix du sac avant les réductions. On arrondira au centime près.
\end{enumerate}
\end{exercice}

\begin{exercice}[3 pts]
Le tableau suivant donne la répartition des salariés d'une start-up.

\compo[0.46]
{
\
\begin{enumerate}
\item Quelle est la proportion des femmes dans cette startup?
\item Quelle est celle des commerciaux dans la startup?
\item Quelle est celle des commerciales parmi les commerciaux?
\end{enumerate}
}
{
\begin{center}
\begin{tabularx}{\linewidth}{
|>{\centering\arraybackslash}c
|>{\centering\arraybackslash}X
|>{\centering\arraybackslash}X
|>{\centering\arraybackslash}X|} \cline{2-4} \multicolumn{1}{c|}{}
& Hommes & Femmes & Total \\ \hline
Commerciaux & 15 & 9 & 24 \\ \hline
Développeurs web & 7 & 9 & 16 \\ \hline
Total & 22 & 18 & 40 \\ \hline
\end{tabularx}
\end{center}
}

\end{exercice}

\begin{exercice}[4 pts]
Le tableau ci-dessous donne les indices des bénéfices d'une société de 2014 à 2018.

\compo[0.7]
{
\
\begin{enumerate}
\item Quelle est l'année de référence de ce tableau?
\item \begin{enumerate}
\item En quelle année la société a-t-elle réalisé le meilleur bénéfice?
\item Quel était alors le taux d'évolution de ce bénéfice depuis 2014?
\item Depuis 2016?
\end{enumerate}
%\item Une autre société, qui prend également pour base 100 son bénéfice de 2014, a atteint en 2018 un bénéfice correspondant à l'indice 105. Peut-on en déduire que cette société a réalisé en 2018 un bénéfice plus important que la première société?
\end{enumerate}
}
{
\begin{center}
\begin{tabularx}{\linewidth}{
|>{\centering\arraybackslash}X
|>{\centering\arraybackslash}X|} \hline
Année & Indice \\ \hline
2014 & 100 \\ \hline
2015 & 101.5 \\ \hline
2016 & 98.3 \\ \hline
2017 & 103 \\ \hline
2018 & 102.9 \\ \hline
\end{tabularx}
\end{center}
}
\end{exercice}

\begin{exercice}[Difficile - 1 pt]
\

\compo[0.4]
{
Une candidate à la présidentielle 2022 a proposé de doubler le salaire des enseignants. Cette augmentation sera échelonnée sur cinq années. Voici le compte rendu qu'en a fait TF1. Que dire de ce graphique? Expliquer leur démarche.
}
{
\includegraphics[trim=35mm 45mm 15mm 3mm, clip, width=\linewidth]{"Erreur JT NB".png}
}
\end{exercice}

\end{document}
