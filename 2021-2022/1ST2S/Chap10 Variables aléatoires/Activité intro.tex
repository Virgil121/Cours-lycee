\documentclass[12pt,a4paper,french] {article}

\pagestyle{empty}

\usepackage[TD]{../../Style}

\renewcommand{\baselinestretch}{1.25}
\renewcommand{\arraystretch}{1.2}
\setlength{\columnsep}{2cm}

\begin{document}

\titre{Le jeu d'Albert}

%Vidéo : https://youtu.be/poABWE2kNdo (Chat sceptique : La loi des grands nombres)

Albert propose un jeu à ses amis: Chaque joueur lance deux dés équilibrés à quatre faces et additionne les résultats. Ensuite:
\begin{itemize}
    \item Si la somme est paire, le joueur remporte la valeurs des dés en euros.
    \item Si la somme est impaire, le joueur perd 5 euros.
\end{itemize}

On veut déterminer si le jeu vaut la peine d'être joué.

\begin{enumerate}
\item \begin{enumerate}
\item Remplir le tableau suivant avec les résultats de l'expérience:

\begin{center} 
\begin{tabularx}{0.9\linewidth}{|*5{Y|}} \hline
    Somme & 1 & 2 & 3 & 4 \\ \hline 
    1 & & & & \\ \hline 
    2 & & & & \\ \hline 
    3 & & & & \\ \hline
    4 & & & & \\ \hline
\end{tabularx}
\end{center}

\item On note $X$ la somme obtenue. Remplir le tableau suivant.
\begin{center}
    \begin{tabularx}{0.9\linewidth}{|>{\centering\arraybackslash}m{3cm}|*{7}{Y|}} \hline 
       $X= \ldots $ &2&3&4&5&6&7&8 \\ \hline
       Nombre de possibilités &&&&&&& \\ \hline
       Probabilité que $X = \ldots$ &&&&&&&\\ \hline
    \end{tabularx}
\end{center}

\item Quel est la probabilité de gagner des jetons au jeu d'Albert ?

\end{enumerate}

\item L'espérance mathématique d'une variable aléatoire se calcule en multipliant la valeur de chaque issue (Gains) par sa probabilité (Probabilité) et d'additionner les résultats. Elle correspond à la valeur moyenne prise par la variable aléatoire.

\begin{enumerate}
\item Remplir (encore) le tableau.

\begin{center}
    \begin{tabular}{|p{3cm}|p{0.5cm}|p{0.5cm}|p{0.5cm}|p{0.5cm}|p{0.5cm}|p{0.5cm}|p{0.5cm}|p{0.5cm}|p{0.5cm}|p{0.5cm}|p{0.5cm}|p{0.5cm}|p{0.5cm}|}
    \hline 
       Nombre obtenu &2&3&4&5&6&7&8&9&10&11&12 \\ 
    \hline 
       Probabilité$\times$gain &\rule{0pt}{20pt}  &&&&&&&&&&\\
    \hline 
    
    \end{tabular}
\end{center}

\item Calculer l'espérance de cette expérience.

\vspace{2cm}
\item Répondre à la question initiale.
\end{enumerate}
\end{enumerate}
\end{document}
