\documentclass[a4paper,12pt,french,twocolumn,landscape]{article}

\usepackage[cours,NB]{../../Style}

\setlength{\columnsep}{1cm}
\renewcommand{\arraystretch}{1.8}

\pagestyle{empty}
\geometry{margin=8mm,left=5mm,right=5mm}
% Début du document
%%%%%%%%%%%%%%%%%%%
\begin{document}

\begin{ex}
On lance une pièce équilibrée. Si on tombe sur pile, on gagne 2\euro, sinon on perd 1\euro. On note $X$ les gains après un lancer. Ainsi $X$ peut valoir soit $2$, soit $-1$. La probabilité que $X$ vaille 2, notée $\Pro(X=2)$, vaut $\frac 1 2 = 0,5$. On dit que $X$ est une \textbf{variable aléatoire réelle}.
\end{ex}

\begin{rmqs}
\begin{itemize}
\item Une variable aléatoire réelle est en fait une fonction $X:\Omega \rightarrow \R$.

\item En pratique, une variable aléatoire permet de raccourcir les notations: L'évènement \og On gagne 2\euro{} après le lancer \fg devient $X=2$.
\end{itemize}
\end{rmqs}

\begin{ex}
\compo[0.6]
{
On reprend l'exemple ci-dessus, mais $X$ représente cette fois les gains après \textbf{deux} lancers successifs. On représente les possibilités grâce à l'arbre ci-contre. On a alors:

\begin{itemize}
\item \
\item \
\item \
\end{itemize}

On peut alors donner la \textbf{loi de probabilité} de $X$, c'est-à-dire donner la probabilité de chaque issue:

\begin{center}
\begin{tabularx}{0.9\linewidth}{|c|Y|Y|Y|} \hline
$x$ & & & \\ \hline
$\Pro(X=x)$ & & & \\ \hline
\end{tabularx}
\end{center}

On peut aussi déterminer d'autres probabilités:
\begin{itemize}
\item $\Pro(X \leq 1)=\dotfill$;
\item $\Pro(X \geq 0) = \dotfill$
\end{itemize}
}
{

}
\end{ex}

\newpage

\begin{ex}
On lance une pièce équilibrée. Si on tombe sur pile, on gagne 2\euro, sinon on perd 1\euro. On note $X$ les gains après un lancer. Ainsi $X$ peut valoir soit $2$, soit $-1$. La probabilité que $X$ vaille 2, notée $\Pro(X=2)$, vaut $\frac 1 2 = 0,5$. On dit que $X$ est une \textbf{variable aléatoire réelle}.
\end{ex}

\begin{rmqs}
\begin{itemize}
\item Une variable aléatoire réelle est en fait une fonction $X:\Omega \rightarrow \R$.

\item En pratique, une variable aléatoire permet de raccourcir les notations: L'évènement \og On gagne 2\euro{} après le lancer \fg devient $X=2$.
\end{itemize}
\end{rmqs}

\begin{ex}
\compo[0.6]
{
On reprend l'exemple ci-dessus, mais $X$ représente cette fois les gains après \textbf{deux} lancers successifs. On représente les possibilités grâce à l'arbre ci-contre. On a alors:

\begin{itemize}
\item \
\item \
\item \
\end{itemize}

On peut alors donner la \textbf{loi de probabilité} de $X$, c'est-à-dire donner la probabilité de chaque issue:

\begin{center}
\begin{tabularx}{0.9\linewidth}{|c|Y|Y|Y|} \hline
$x$ & & & \\ \hline
$\Pro(X=x)$ & & & \\ \hline
\end{tabularx}
\end{center}

On peut aussi déterminer d'autres probabilités:
\begin{itemize}
\item $\Pro(X \leq 1)=\dotfill$;
\item $\Pro(X \geq 0) = \dotfill$
\end{itemize}
}
{

}
\end{ex}

\end{document}