\documentclass[a4paper,12pt,french]{article}

\usepackage[cours]{../../../Style}

%\selectcolormodel{cmyk}

%\usepackage{wrapfig}
%\makeatletter
%\setlength{\parskip}{1ex}
%\newcommand{\@minipagerestore}{\setlength{\parskip}{1ex}}

% Début du document
%%%%%%%%%%%%%%%%%%%
\begin{document}

\title{Conditionnement}
\maketitle

\begin{programme}
\item Croisement de deux variables catégorielles
\begin{itemize}
\item Tableau croisé d'effectifs
\item Fréquence conditionnelle, fréquence marginale
\item Capacités
\begin{itemize}
\item Calculer des fréquences conditionnelles et des fréquences marginales
\item Compléter un tableau croisé par des raisonnements sur les effectifs ou en utilisant des fréquences conditionnelles
\end{itemize}
\end{itemize}
\item Probabilités conditionnelles
\begin{itemize}
\item Notation $\Pro_A(B)$
\item Capacité: Calculer des probabilités conditionnelles lorsque les évènements sont présentés sous forme de tableau croisé d'effectifs
\end{itemize}
\end{programme}


\end{document}
